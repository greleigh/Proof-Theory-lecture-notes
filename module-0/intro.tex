%
\chapter{Lend me thy proof}
%

What does a proof tell about a theorem beyond its truth?
If the theorem states the existence of an object to what extent does the proof isolate the object in mind?
The reader will be familiar with the classical logic and the method of ‘proof by contradiction’ --- also known by the Latin phrase \emph{reductio ad absurdum} --- whereby an existential claim can be established by showing the negative \emph{universal} claim to be contradictory.
The mere statement of a theorem does not determine whether such method of proof is used or necessary.
One proof of a theorem may directly construct a witness.
Another may invoke only indirect reasoning but, perhaps, rely on fewer assumptions.
A third proof might be too complex to determine; it might, for instance, appeal to lemmas whose proofs you do not have access to.
And only a characterisation of the mathematical theories in which the theorem holds can answer the \emph{real} question: Can the theorem be proved \emph{only} by indirect methods?

With logic in mind, other questions also stand out.
How \emph{complex} is logic? 
For that matter, what does it mean to say that one logic --- or even one \emph{proof} --- is more complex than another?
Neither question can be given a definite answer, but we can get a handle on them by studying, comparing and manipulating proofs.
In these lecture notes
I will show, for example, that every classically valid formula can be given a proof in which only subformulas of the conclusion are used.
Such a proof will not, in general, be the shortest such proof nor the most concise.
But it is the \emph{simplest} in one concrete sense: it does not reference any concepts more complex than the one being proved.
%Is there an algorithm that given a formula returns a proof of the formula if one exists?

The reader will also be shown situations of the opposite kind: an example of a mathematical theorem admitting an elementary proof but for which every proof necessarily refers to concepts \emph{more} complex than the conclusion.
No doubt you will have encountered such cases before although you may not have realised at the time: the scenario is arithmetic and the theorem one of many examples whose proofs (in the language of arithmetic) necessitate a stronger induction invariant than the theorem itself.

%Arithmetic, of course, 

On the topic of arithmetic, I assume you won't deny me the consistency of \emph{Peano} arithmetic, the first-order theory axiomatised by the defining equations for functions of successor, addition and multiplication, plus the axiom schema of induction.
One need only observe that each axiom is a true statement about the natural numbers, that is, that the structure of the natural numbers and its elementary functions forms a model of the Peano axioms.
But the standard model of arithmetic is overkill for the purpose of consistency of the Peano axioms.
Gödel's incompleteness theorem presents statements in the language of arithmetic that are true yet \emph{not} provable from the Peano axioms.
So what mathematical assumptions truly underpin the consistency of Peano arithmetic and, for that matter, other mathematical theories?
And thinking of \emph{different} theories, 
can the deductive \emph{power} of a theory be measured, so that one theory can be directly compared to another?

This, in a nutshell, is
\emph{Proof Theory}: the mathematical theory of formal proofs and, by extension, the mathematical theory of mathematical proofs.
%
And through the course of this text you, dear reader, will see  for yourself the delights and delicacies that only a proof conceals.
Together we will taste the sweetness of the topping, break through its smooth crust and sample the richness beneath.
%Some will be sitting on top for all to see, others

But the proof of the pudding is in the eating.
%
I hope you are hungry.

