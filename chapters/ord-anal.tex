%
\chapter{Arithmetic and Proof Theory}\label{c-oa}
%
We now turn our attention to the theory of arithmetic.

%---------------------------------
\section{Peano and Heyting arithmetic}\label{s-oa-arithmetic}
%---------------------------------
%
\begin{definition}
	The \emph{language of arithmetic} is the first-order language \( \La \) with equality comprising of the nonlogical symbols:
	\begin{enumerate}
		\item function symbols: \( \0^0 \), \( \suc^1 \), \( +^2 \), \( ×^2 \).
		\item predicates: \( P^1 \).
	\end{enumerate}
\end{definition}
%
\begin{definition}
	The theory of Peano arithmetic is axiomatised by …
\end{definition}
%
As well as some basics of the theory of arithmetic, we recall the primitive recursive representation theorem.
%
\begin{definition}
	A formula is \( Σ_1 \) if it can be constructed from atomic formulas excluding \( P \) using the monotone propositional connectives, bounded universal quantifier and (unbounded) existential quantifier. That is, the \( Σ_1 \) formulas forms the smallest collection of \( \La \)-formulas satisfying:
	\begin{itemize}
		\item \( s = t , ⊥ ∈ Σ_1 \) for all (arithmetic) terms \( s \) and \( t \),
		\item if \( A \) and \( B \) are \( Σ_0 \), then so is \( A ∨ B \), \( A ∧ B \), and \( ∃ x\, A \),
		\item if \( A \) is \( Σ_1 \) and \( s \) is any (arithmetic) term not containing \( x \), then \( ∀x< s\, A \) is \( Σ_1 \), where this formula is shorthand for \( ∀x ( x< s → A ) \),
		\item if \( A \) and \( B \) are \( Σ_1 \) and, in addition, \( A \) does not contain the existential quantifier, then \( A → B \) is \( Σ_1 \).
	\end{itemize}
\end{definition}
%
\begin{theorem}
	Let \( R ⊆ \Nat^n \) be an \( n \)-ary relation on natural numbers. If \( R \) is primitive recursive then there exists a \( Σ_1 \) formula \( \Frml{R}(x_1, …, x_k ) \) of \( \La \) with at most the displayed variables occurring free such that for all \( n_1, …, n_k ∈ \Nat \),
	\[
		\HA ⊢ \Frml{R}(n_1 , …, n_k ) \quad\text{iff}\quad (n_1 , …, n_k ) ∈ R.
	\]
\end{theorem}
%

%---------------------------------
\section{Infinitary sequent calculi}\label{s-oa-omega-logic}
%---------------------------------

Onwards into infinitary proofs.
\begin{definition}\label{d-PAomega}
	\( \PAo \). Sequents/rules
	
	\( \HAo \). Likewise
\end{definition}
%
A look at induction.
%
\begin{namedlemma}[Embedding lemma]
	If \( \PA ⊢ Γ ⇒ Δ \) and \( Γ ⇒ Δ \) is a closed sequent.
	Then \( \PAo ⊢ Γ ⇒ Δ \).
\end{namedlemma}
%
To prove the lemma it is necessary to generalise its statement to include also
\( Γ^* ⇒ Δ^* \) is any closed instantiation of \( Γ ⇒ Δ \), then \( \PAo ⊢ Γ^* ⇒ Δ^* \).
%
%\begin{proof}
%	asd
%\end{proof}

Currently we have no means to measure the size of \( \PAo \)-proofs.
For this we will use ordinals.


