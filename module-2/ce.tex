%---------------------------------
\chapter{Cut elimination}\label{c-cut-elim}
%---------------------------------
Here we present cut elimination for the calculi.

Cut rank,
Inversion lemma and the like

\begin{lemma}[First inversion lemma]\label{ce-inversion-lemma}
	Let \( ⊢ \) denoted provability in either \( \Gc \) or \( \Gi \).
	The following hold for all sequents and all \( n \), \( k \):
	\begin{enumerate}
		\item If \( \prv n k Γ ⇒ Δ , ⊥ \) then \( \prv n k Γ ⇒ Δ \).
		\item If \( \prv n k Γ ⇒ Δ , F ∧ G \) then \( \prv n k Γ ⇒ Δ , F \) and \( \prv n k Γ ⇒ Δ , G \).
		\item If \( \prv n k F ∧ G , Γ ⇒ Δ \) then \( \prv n k F , G , Γ ⇒ Δ \).
		\item If \( \prv n k F ∨ G , Γ ⇒ Δ \) then \( \prv n k F , Γ ⇒ Δ \) and \( \prv n k G , Γ ⇒ Δ \).
		\item If \( \prv n k Γ ⇒ Δ , F → G \) then \( \prv n k F , Γ ⇒ Δ , G \).
		\item If \( \prv n k Γ ⇒ Δ , ∀x F(x) \) then \( \prv n k Γ ⇒ Δ , F(s) \) for every term \( s \).
		\item If \( \prv n k ∃x F(x) , Γ ⇒ Δ \) then \( \prv n k F(s) , Γ ⇒ Δ \) for every term \( s \).
	\end{enumerate}
\end{lemma}

\begin{lemma}[Second inversion lemma]\ 
	\begin{enumerate}
		\item If \( \Gc \prv n k Γ ⇒ Δ , F ∨ G \) then \( \Gc \prv n k Γ ⇒ Δ , F , G \).
		\item If \( \Gc \prv n k F → G , Γ ⇒ Δ \) then \( \Gc \prv n k G ,Γ ⇒ Δ \) and \( \Gc \prv n k Γ ⇒ Δ , F \).
	\end{enumerate}
\end{lemma}

\begin{lemma}[Third inversion lemma]
	If \( \Gi \prv n k F → G , Γ ⇒ Δ \) then \( \Gi \prv n k G , Γ ⇒ Δ \).
\end{lemma}

% ---
\section{For classical logic}

\begin{lemma}[Reduction lemma for \( \Gc \)]
	\label{ce-red-lem-C}
	Suppose \( \prv m k Γ ⇒ Δ , C \) and \( \prv n k C , Σ ⇒ Λ  \).
	If \( \rk C = k \) then \( \prv {m+n}k Γ,Σ⇒ D,Λ  \).
\end{lemma}

\begin{exercise}
	\label{ex-red-lem-special}
	In this exercise you will prove a strengthening of the reduction lemma and, as a consequence, obtain more precise bounds on the cost of cut elimination in classical logic.
	
	See Canvas assign 4.
\end{exercise}


% ---
\section{For intuitionistic logic}