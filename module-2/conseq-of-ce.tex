%
\chapter{Consequences of cut elimination}
\label{c-ce-conseq}
%
Now we are getting somewhere

\bigskip

\begin{itemize}
	\item Non-classical restrictions
	\item Subformula and conservativity
	\item Interpolation theorem -- Exercise.
	\item Harrop's theorem
	\item Herbrand's theorem
\end{itemize}

The separation of classical and intuitionistic logic is usually established by semantic means.
One show that there certain classical validities are not intuitionistic validities, from which a soundness theorem for intuitionistic logic completes the separation.
The first application of cut elimination I present is purely syntactic (that is, without recourse to any semantic notions)  counterpart of these theorems.
%
\begin{proposition}
%	The law of excluded middle is not derivable in \( \IL \).
	There is a formula \( F \) such that 
	\begin{enumerate}
		\item \( \CL ⊢ { } ⇒ F \),
		\item \( \IL ⊬ {} ⇒ F  \).
	\end{enumerate}
\end{proposition}
%
\begin{proof}
	A natural choice for \( F \) is an instance of the law of excluded middle. Indeed, let \( P \) be a propositional letter.
	That \( \CL ⊢ {} ⇒ P ∨ \lnot P \) has already been shown (\cref{CL-exc-mid}).
	Now suppose, to the contrary, that 
	\( \IL ⊢ {} ⇒ P ∨ \lnot P  \).
	By cut elimination (\cref{ce-theorem}), there exists  a cut-free intuitionistic proof, i.e., 
	\( \IL \prv n 0 {} ⇒ P ∨ \lnot P  \) for some \( n \).
	As the proof is cut-free, the only rule that can derive this sequent is one of the two \( \disjR \), but then there must be an intuitionistic cut-free proof of either \( ⇒ P \) or \( P ⇒ ⊥ \), neither of which exists as no rule (except cut) can have either sequent as its conclusion.
\end{proof}

\begin{itemize}
	\item Peirce
\end{itemize}
