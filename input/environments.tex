
%% 
%% Theorems et al
\usepackage[amsmath,thmmarks,hyperref]{ntheorem}
%% LTeX: enabled=false
%%%%%%%%%%%%%%%%%%%%%%%%%%%%%%%%%%%%%%%%%
%% BEGIN: §7. Environments
%%%%%%%%%%%%%%%%%%%%%%%%%%%%%%%%%%%
%%% §7.2. Default (plain) style
%% Our Theorem styles:
\makeatletter
%%%
%% plain:
%%   Theorem 6.13 · Löb's theorem  
\renewtheoremstyle{plain}
    {\item[\hskip\labelsep \theorem@headerfont ##1\ ##2\theorem@separator]}%
    {\item[\hskip\labelsep \theorem@headerfont ##1\ ##2\ \textperiodcentered\ ##3\theorem@separator]}
% plain w/o number:
\renewtheoremstyle{nonumberplain}
    {\item[\hskip\labelsep \theorem@headerfont ##1\theorem@separator]}%
    {\item[\hskip\labelsep \theorem@headerfont ##1\ \textperiodcentered\ ##3\theorem@separator]}
%%% 
% implicit: (uses optional argument for name)
%   9.20 Löb's Theorem
%
\newtheoremstyle{implicit}
    {\item[\hskip\labelsep \theorem@headerfont ##1\ ##2\theorem@separator]}%
    {\item[\hskip\labelsep \theorem@headerfont ##3\ ##2\theorem@separator]}
%
% no-number version  
\newtheoremstyle{nonumberimplicit}
    {\item[\hskip\labelsep \theorem@headerfont ##1\theorem@separator]}%
    {\item[\hskip\labelsep \theorem@headerfont ##3\theorem@separator]}
%%%
%%% remark:
%%%    Remark on A6
%%
%%\newtheoremstyle{remark}
%%  {\item[\hskip\labelsep \theorem@headerfont ##2\ ##1\theorem@separator]}%
%%  {\item[\hskip\labelsep \theorem@headerfont ##2\ ##1 on\ ##3\theorem@separator]}
%%% no-number version:
%%\newtheoremstyle{nonumberremark}
%%  {\item[\hskip\labelsep \theorem@headerfont ##1\theorem@separator]}%
%%  {\item[\hskip\labelsep \theorem@headerfont ##1\ on\ ##3\theorem@separator]}
%%%
%% PROOF
%%   Proof [of theorem 8.71]
\newtheoremstyle{proof}
  {\item[\hskip\labelsep \theorem@headerfont ##1\theorem@separator]}%
  {\item[\hskip\labelsep \theorem@headerfont ##1\ ##3\theorem@separator]}
\makeatother
%%%
\theoremstyle{plain}
%\theoremheaderfont{\upshape\scshape}
\newtheorem {theorem}           {Theorem} [chapter]
\newtheorem {lemma}[theorem]    {Lemma}
\newtheorem {proposition}[theorem] {Proposition}
\newtheorem {claim}[theorem]    {Claim}
\newtheorem {corollary}[theorem]{Corollary}
\newtheorem {example}[theorem]	{Example}
%%
\theoremstyle{implicit}
%\newtheorem*{thing}{thing}
%% \newtheorem*{covind}    {Course of values induction}
%% \newtheorem*{natind}    {Principle of induction for natural numbers}
%%
%\newtheorem{namedparadox}[theorem]{paradox}
\newtheorem{namedtheorem}[theorem]{Theorem}
\newtheorem{namedlemma}[theorem]{Lemma}
\newtheorem*{nameonly}{Thing}
%%%
%%%%%%%%%%%%%%%%%%%%%%%%%%%%%%%%%%%%
%%%% §7.3. Upright style
\theoremstyle{plain}
\theorembodyfont{\upshape}
\newtheorem{definition}[theorem] {Definition}
%%\newtheorem{assumption}[theorem]  {assumption}
%%%%%%%%%%%%%%%%%%%%%%%%%%%%%%%%%%%%
%%%% §7.4. Proof style (no number)
\theoremstyle{proof}
\theorembodyfont{\upshape}
%% \theoremprework{\arabicenum}
\theoremsymbol{\ensuremath{\dashv}}
\newtheorem {proof} {Proof}
%%%%%%%%%%%%%%%%%%%%%%%%%%%%%%%%%%%%
%%%% §7.5. Exercise (remove?)
\theoremstyle{plain}
\theorembodyfont{\upshape}
\newtheorem {exercise}[theorem]   {Exercise}
%%%%%%%%%%%%%%%%%%%%%%%%%%%%%%%%%%%%
%%%% §7.6. Remark
%%% Uses special remark formatting
\theoremstyle{remark}
\theoremsymbol{} % no end of remark symbol
\newtheorem*{remark}{remark}
%\newtheorem*{convention}{convention}
%
%% END
