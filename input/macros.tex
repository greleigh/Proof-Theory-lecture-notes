
%---------------------------------
% Note-making
\NewDocumentCommand{\marginnote}{m}{\marginpar{\scriptsize #1}}
\providecommand{\note}[1]{\marginnote{\textcolor{magenta}{#1}}}
\providecommand{\tbw}{\textcolor{magenta}{To be written.}}
%---------------------------------
% LOGICS & THEORIES
\newcommand{\NDc}{\Logic{Nc}}
\newcommand{\NDi}{\Logic{Ni}}
\newcommand{\NDceq}{\Logic{Nc_=}}
\newcommand{\NDieq}{\Logic{Ni_=}}

\newcommand{\Gip}{\Logic{IP}}
\newcommand{\Gcp}{\Logic{CP}}
\newcommand{\Gmp}{\Logic{M}}
\newcommand{\Gi}{\Logic{I}}
\newcommand{\Gc}{\Logic{C}}
\newcommand{\Gieq}{\Logic{I}_=}
\newcommand{\Gceq}{\Logic{C}_=}

\newcommand{\PA}{\Theory{PA}}
\newcommand{\HA}{\Theory{HA}}
%
\newcommand{\IS}{\Theory{I}\Sigma}
%\newcommand{\HA}{\Theory{HA}}

\newcommand{\PAo}{\Theory{PAω}}
\newcommand{\HAo}{\Theory{HAω}}
\newcommand{\PAop}[1][\prec]{\Theory{PAω}+({#1})}
%---------------------------------
% SEQUENTS
\newcommand{\sa}{\Rightarrow}

\newcommand{\prv}[2]{\vdash^{\smash{#1}}_{\smash{#2}}}
%---------------------------------
% INFERENCES
% \Rule, \LeftRule, \RightRule
% are defined in styling.tex
%
\newcommand{\Cut}{\Rule{cut}}
%
\newcommand{\idRule}{\Rule{id}}
\newcommand{\botL}{\LeftRule{⊥}}
%
\newcommand{\faR}{\RightRule{∀}}
\newcommand{\faL}{\LeftRule{∀}}
\newcommand{\exR}{\RightRule{∃}}
\newcommand{\exL}{\LeftRule{∃}}
%
\newcommand{\impR}{\RightRule{→}}
\newcommand{\impL}{\LeftRule{→}}
%
\newcommand{\disjR}{\RightRule{∨}}
\newcommand{\disjL}{\LeftRule{∨}}
%
\newcommand{\conjR}{\RightRule{∧}}
\newcommand{\conjL}{\LeftRule{∧}}
%
% for equality:
\newcommand{\refRule}{\Rule{ref}}
\newcommand{\compRule}{\Rule{comp}}
%
% for arithmetic
\newcommand{\IRule}{\Rule{ir}}
%
% for ω-logic:
\newcommand{\omR}{\RightRule{ω}}
\newcommand{\omL}{\LeftRule{ω}}
\newcommand{\eqR}{\RightRule{=}}
\newcommand{\eqL}{\LeftRule{=}}

%
%---------------------------------
% TERMS & FORMULAS
\newcommand{\rk}[1]{\lvert{#1}\rvert}
\newcommand{\nrk}[1]{\lvert{#1}\rvert_*}
\newcommand{\qrk}[1]{\lvert{#1}\rvert_q}

\newcommand{\nm}[1]{\underline{#1}}
\newcommand{\cd}[1]{\overline{#1}}
%
\newcommand{\0}{\Symbol{0}}
\newcommand{\suc}{\Symbol{s}}
%\newcommand{\num}[1]{\underline{#1}}

% Shorthands
\newcommand{\Prog}[2][\prec]{\Frml{Prog}_{#1}{#2}}
\newcommand{\TI}[2][\prec]{\Frml{TI}_{#1}({#2})}
%---------------------------------
\NewDocumentCommand{\setof}{mo}{\{\,{#1}\IfValueT{#2}{\mid {#2}}\,\}}
\NewDocumentCommand{\Setof}{mo}{\bigl\{\,{#1}\IfValueT{#2}{\bigm| {#2}}\,\bigr\}}
\NewDocumentCommand{\supof}{mo}{\sup\{\,{#1}\IfValueT{#2}{\mid {#2}}\,\}}
\newcommand{\supseq}[1][i]{\sup_i}
%---------------------------------
% LANGUAGES
\newcommand{\La}{\Lang{L}_{\mathrm{A}}}

%---------------------------------
% SETS
\newcommand{\Nat}{\mathbb{N}}

\newcommand{\card}[1]{\lvert{#1}\rvert}

%---------------------------------
% Ordinals
\newcommand{\Ord}{\mathbb{O}}
\newcommand{\AP}{\Set{AP}}
\newcommand{\dom}{\mathop{\mathrm{dom}}}
\newcommand{\NF}{\textsc{nf}}

\newcommand{\nsum}{\mathbin{\#}}
%---------------------------------
% Proof-theoretic tools
\NewDocumentCommand{\pto}{m}{\lVert{#1}\rVert}
\NewDocumentCommand{\ot}{O{\prec}}{\lVert{#1}\rVert}
\NewDocumentCommand{\otin}{ O{\prec} m }{\lvert{#2}\rvert_{#1}}

%---------------------------------
% Proof-tree shortcuts
\NewDocumentCommand{\Infer}{omO{}m}{\begin{prooftree}\hypo{#2}\infer1[#3]{#4}\end{prooftree}}

