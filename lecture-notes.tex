\documentclass[%
	paper=174mm:247mm,
	11pt,
	DIV=10,
	leqno,
	titlepage,
	headsepline,
	%headings=twolinechapter,
	headsepline=false,
	toc=bib,
	toc=sectionentrywithoutdots,
	toc=chapterentrywithoutdots
	% enabledeprecatedfontcommands,
	% final,
	unicode
	]%
	{scrbook}
%
%\usepackage{polyglossia}
\usepackage[utf8]{inputenc}
\usepackage[british]{babel}
\usepackage{xparse}
%
\usepackage[T1]{fontenc}
\usepackage{amsmath,amssymb}
\usepackage[lining]{libertine}
%\usepackage{pifont}
\usepackage[libertine,nosymbolsc]{newtxmath}
\usepackage[cal=rsfso]{mathalfa}
\useosf
%
\let\emptyset\emptysetAlt
\let\forall\forallAlt
\let\exists\existsAlt
%
% Packages
\usepackage{graphicx}
%
% Proofs
\usepackage{ebproof}
% File styling.tex
% !TEX root = ../lecture-notes.tex
% LTeX: enabled=false
% Packages and default settings regarding typesetting and format
%---------------------------------
% Page layout assistance
\usepackage{scrlayer-scrpage}
\usepackage[marginparsep=2mm]{geometry}
%---------------------------------
% Theorem environments
\usepackage{mathtools}
%---------------------------------
% Styling for theorems
%
%---------------------------------
% BibLaTeX
%\usepackage[%
%  style=authoryear,% author-year style with extensions
%  backend=biber,%
%  articlein=false,% remove 'in' for journals
%  innamebeforetitle=true,% "in: Editor, Book"
%  dashed=false,% repeated authorship is not dashed
%  maxnames=1000,% don't cut listings short unless we want them too.
%  language=autobib,autolang=hyphen,% Can help with language
%  doi=false,isbn=false,eprint=false,% do not print doi, etc. 
%  ]{biblatex}
%
%---------------------------------
% ELEMENTS
\NewDocumentCommand{\Logic}{m}{\mathsf{#1}}
\NewDocumentCommand{\Theory}{m}{\mathsf{#1}}
\NewDocumentCommand{\Set}{m}{\mathrm{#1}}
\NewDocumentCommand{\Frml}{m}{\mathsf{#1}} % Why not working?
\NewDocumentCommand{\Symbol}{m}{\mathsf{#1}}
\NewDocumentCommand{\Lang}{m}{\mathcal{#1}}
% Rule names
\NewDocumentCommand{\Rule}{m}{\ensuremath{\mathsf{#1}}}
\NewDocumentCommand{\LeftRule}{m}{\Rule{L{#1}}}
\NewDocumentCommand{\RightRule}{m}{\Rule{R{#1}}}
%---------------------------------
% Disposition
% Part:
\renewcommand*{\partname}{Module} % Does not achieve anything!
\renewcommand*{\partformat}{Module~\thepart\autodot}
%\renewcommand*{\addparttocentry}[2]{%
%  \addtocentrydefault{part}{Module\nobreakspace #1}{#2}%
%	}%
\newcommand\partentrynumberformat[1]{Module~\ #1}
\RedeclareSectionCommand[
  tocentrynumberformat=\partentrynumberformat,
  tocnumwidth=6em
]{part}
% Chapter:
\renewcommand{\raggedchapter}{\centering}
\setkomafont{chapter}{\sffamily\mdseries\LARGE}
\addtokomafont{chapterentry}{\rmfamily\mdseries}
\addtokomafont{chapterprefix}{\normalsize}
% (Sub)section
\renewcommand{\raggedsection}{\centering}
\setkomafont{section}{\rmfamily\mdseries\sffamily\large}
\setkomafont{subsection}{\centering\rmfamily\mdseries\itshape}
% Paragraph
% - Normal (par-)spacing with emphasis leading
\setkomafont{paragraph}{\rmfamily\mdseries\itshape}
\def\ParIndent{\the\parindent}
\RedeclareSectionCommands[
	beforeskip=\parskip,
	indent=\ParIndent,
%	afterskip=1.5ex plus .5ex minus 0.2ex,
  ]{paragraph}
\addtokomafont{paragraph}{\raggedright} 
% Remove reference to chapter from section.
% \RedeclareSectionCommand[%
%   counterwithout=chapter,
%             ]{section}
%
%---------------------------------
% TABLE OF CONTENTS
%

%\usepackage[tocindentauto]{tocbasic}
%\usetocstyle{KOMAlike}
%---------------------------------
% Layout
% Page layouts
\setkomafont{pagehead}{\itshape}
\ohead{\upshape\thepage} % Outer side of pages
\rehead{\leftmark} % Right side of even pages
\lohead{\rightmark} % Left side of odd pages
\ofoot{\small\itshape G.E.~Leigh, version: \today}
\ifoot{}
\cfoot{}
\pagestyle{scrheadings}
%---------------------------------
% Control over lists:
\usepackage{enumitem}
% New defaults for enumerate & itemize:
\setlist{itemsep=0pt,topsep=\parsep}
%%   Label should be upright
\setlist[enumerate,itemize]{format=\normalfont}
%%   Label #1: roman (i), (ii), etc
%\setlist[enumerate,1]{label=(\roman*)}
%%   Label #2: alph (a), (b), etc
%\setlist[enumerate,2]{label=(\alph*)} 
%
% Special lists





%% 
%% Theorems et al
\usepackage[amsmath,thmmarks,hyperref]{ntheorem}
%% LTeX: enabled=false
%%%%%%%%%%%%%%%%%%%%%%%%%%%%%%%%%%%%%%%%%
%% BEGIN: §7. Environments
%%%%%%%%%%%%%%%%%%%%%%%%%%%%%%%%%%%
%%% §7.2. Default (plain) style
%% Our Theorem styles:
\makeatletter
%%%
%% plain:
%%   Theorem 6.13 · Löb's theorem  
\renewtheoremstyle{plain}
    {\item[\hskip\labelsep \theorem@headerfont ##1\ ##2\theorem@separator]}%
    {\item[\hskip\labelsep \theorem@headerfont ##1\ ##2\ \textperiodcentered\ ##3\theorem@separator]}
% plain w/o number:
\renewtheoremstyle{nonumberplain}
    {\item[\hskip\labelsep \theorem@headerfont ##1\theorem@separator]}%
    {\item[\hskip\labelsep \theorem@headerfont ##1\ \textperiodcentered\ ##3\theorem@separator]}
%%% 
% implicit: (uses optional argument for name)
%   9.20 Löb's Theorem
%
\newtheoremstyle{implicit}
    {\item[\hskip\labelsep \theorem@headerfont ##1\ ##2\theorem@separator]}%
    {\item[\hskip\labelsep \theorem@headerfont ##3\ ##2\theorem@separator]}
%
% no-number version  
\newtheoremstyle{nonumberimplicit}
    {\item[\hskip\labelsep \theorem@headerfont ##1\theorem@separator]}%
    {\item[\hskip\labelsep \theorem@headerfont ##3\theorem@separator]}
%%%
%%% remark:
%%%    Remark on A6
%%
%%\newtheoremstyle{remark}
%%  {\item[\hskip\labelsep \theorem@headerfont ##2\ ##1\theorem@separator]}%
%%  {\item[\hskip\labelsep \theorem@headerfont ##2\ ##1 on\ ##3\theorem@separator]}
%%% no-number version:
%%\newtheoremstyle{nonumberremark}
%%  {\item[\hskip\labelsep \theorem@headerfont ##1\theorem@separator]}%
%%  {\item[\hskip\labelsep \theorem@headerfont ##1\ on\ ##3\theorem@separator]}
%%%
%% PROOF
%%   Proof [of theorem 8.71]
\newtheoremstyle{proof}
  {\item[\hskip\labelsep \theorem@headerfont ##1\theorem@separator]}%
  {\item[\hskip\labelsep \theorem@headerfont ##1\ ##3\theorem@separator]}
\makeatother
%%%
\theoremstyle{plain}
%\theoremheaderfont{\upshape\scshape}
\newtheorem {theorem}           {Theorem} [chapter]
\newtheorem {lemma}[theorem]    {Lemma}
\newtheorem {proposition}[theorem] {Proposition}
\newtheorem {claim}[theorem]    {Claim}
\newtheorem {corollary}[theorem]{Corollary}
\newtheorem {example}[theorem]	{Example}
%%
\theoremstyle{implicit}
%\newtheorem*{thing}{thing}
%% \newtheorem*{covind}    {Course of values induction}
%% \newtheorem*{natind}    {Principle of induction for natural numbers}
%%
%\newtheorem{namedparadox}[theorem]{paradox}
\newtheorem{namedtheorem}[theorem]{Theorem}
\newtheorem{namedlemma}[theorem]{Lemma}
\newtheorem*{nameonly}{Thing}
%%%
%%%%%%%%%%%%%%%%%%%%%%%%%%%%%%%%%%%%
%%%% §7.3. Upright style
\theoremstyle{plain}
\theorembodyfont{\upshape}
\newtheorem{definition}[theorem] {Definition}
%%\newtheorem{assumption}[theorem]  {assumption}
%%%%%%%%%%%%%%%%%%%%%%%%%%%%%%%%%%%%
%%%% §7.4. Proof style (no number)
\theoremstyle{proof}
\theorembodyfont{\upshape}
%% \theoremprework{\arabicenum}
\theoremsymbol{\ensuremath{\dashv}}
\newtheorem {proof} {Proof}
%%%%%%%%%%%%%%%%%%%%%%%%%%%%%%%%%%%%
%%%% §7.5. Exercise (remove?)
\theoremstyle{plain}
\theorembodyfont{\upshape}
\newtheorem {exercise}[theorem]   {Exercise}
%%%%%%%%%%%%%%%%%%%%%%%%%%%%%%%%%%%%
%%%% §7.6. Remark
%%% Uses special remark formatting
\theoremstyle{remark}
\theoremsymbol{} % no end of remark symbol
\newtheorem*{remark}{remark}
%\newtheorem*{convention}{convention}
%
%% END

%
\input{input/unicodes}
%
\usepackage{xparse}
%---------------------------------
% 
%---------------------------------
% Proof-theoretic tools
\NewDocumentCommand{\pto}{m}{\lVert{#1}\rVert}
%
\usepackage[bookmarks]{hyperref}
\usepackage[nameinlink,noabbrev]{cleveref}
%%
\let\autoref\cref
%\crefname{assumption}{assumption}{assumptions}
%\crefname{claim}{claim}{claims}
%\crefname{proposition}{proposition}{propositions}
%\crefname{proposition2}{proposition}{propositions}
%\crefname{example}{example}{examples}
%\crefname{definition}{definition}{definitions}
%\crefname{corollary}{corollary}{corollaries}
%\crefname{corollary2}{corollary}{corollaries}
%\crefname{lemma}{lemma}{lemmas}
%\crefname{lemma2}{lemma}{lemmas}
%\crefname{namedparadox}{paradox}{paradoxes}
%\crefname{namedtheorem}{theorem}{theorems}
%\crefname{namedlemma}{lemma}{lemmas}
%\crefname{namedexample}{example}{examples}
%\crefname{namedassumption}{assumption}{assumptions}
%\crefname{principle}{principle}{principles}
%\newcommand{\creflastconjunction}{, and\nobreakspace}

% Meta
\title{Lecture Notes in Proof Theory}
\author{Graham E.\ Leigh}

\includeonly{
	module-0/intro,
	module-1/int-prop-logic,
	module-1/cla-prop-logic,
	module-2/FOL,
	module-2/conseq-of-ce,
	module-3/ord-anal,
	module-3/ordinal-interlude,
	module-3/ordinal-analysis,
	}

\begin{document}
\maketitle

% TOC
\tableofcontents

% Intro
%
\chapter{Lend me thy proof}
%

What does a proof tell about a theorem beyond its truth?
If the theorem states the existence of an object to what extent does the proof isolate the object in mind?
The reader will be familiar with the classical logic and the method of ‘proof by contradiction’ --- also known by the Latin phrase \emph{reductio ad absurdum} --- whereby an existential claim can be established by showing the negative \emph{universal} claim to be contradictory.
The mere statement of a theorem does not determine whether such method of proof is used or necessary.
One proof of a theorem may directly construct a witness.
Another may invoke only indirect reasoning but, perhaps, relies on fewer assumptions.
A third might be too complex to know because it appeals to lemmas whose proofs you do not have access to.
And only a characterisation of the mathematical theories in which the theorem holds can answer the question: Can the theorem be proved \emph{only} by indirect methods?

With logic in mind, other questions also stand out.
How \emph{complex} is logic? 
For that matter, what does it mean to say that one proof is more complex than another?
Neither question can be given a definite answer, but we do get a handle on issues like these by studying proofs, comparing and manipulating proofs.
I will show, for example, that every classically valid formula can be given a proof in which only subformulas of the conclusion are used.
Such a proof will not, in general, be the shortest such proof nor the most concise.
But it is the \emph{simplest} in the sense that it references only concepts no more complex than that being proved.
%Is there an algorithm that given a formula returns a proof of the formula if one exists?

The reader will also be shown situations of the opposite kind: an example of a mathematical theorem admitting an elementary proof but for which every proof necessarily refers to concepts \emph{more} complex than the conclusion.
No doubt you will have encountered such cases before, although you may not have realised at the time: the scenario is arithmetic and the theorem one of many examples of which all proofs (in the language of arithmetic) necessitate a stronger induction invariant than the theorem itself.

%Arithmetic, of course, 

Speaking of arithmetic, few readers would doubt the consistency of \emph{Peano} arithmetic, the first-order theory axiomatised by the defining equations for functions of successor, addition and multiplication, plus the axiom schema of induction.
One need only observe that each axiom is a true statement about the natural numbers, that is, that the structure of the natural numbers and its elementary functions forms a model of the Peano axioms.
But the standard model of arithmetic is overkill for the purpose of consistency of the Peano axioms.
Gödel's incompleteness theorem presents statements in the language of arithmetic that are true yet \emph{not} provable from the Peano axioms.
An obvious question arises:
What mathematical assumptions truly underpin the consistency of Peano arithmetic and, for that matter, other mathematical theories?
%How much \emph{more} complex is Peano arithmetic than, say, the subtheory with induction axioms removed?

This, in a nutshell, is
\emph{Proof Theory}: the mathematical theory of formal proofs and, by extension, the mathematical theory of axiomatic systems.
%
And through the course of this text you, dear reader, will see  for yourself the delights and delicacies that only a proof conceals.
Together we will taste the sweetness of the cherry on the top, break through the smooth crust and sample the richness beneath.
%Some will be sitting on top for all to see, others

But, as they say, the proof of the pudding is in the eating.
%
I hope you are hungry.


%
\part{Propositional Logic}
%
%
\chapter{Sequent calculus for intuitionistic propositional logic}
%
\chapter{Classical propositional logic}
%
And so it continues

\bigskip
Negation translation magic

Interpolation theorem -- Exercise.

%
\part{Predicate Logic}
%
%
\chapter{Predicate logic}
%
Once again

Here negation translation is exercise.
%
\chapter{Consequences of cut elimination}
\label{c-ce-conseq}
%
Now we are getting somewhere

\bigskip

Interpolation theorem -- Exercise.

Harrop's theorem
%
\part{An Introduction to Ordinal Analysis}
%
%
\chapter{Arithmetic and Proof Theory}\label{c-oa}
%
We now turn our attention to the theory of arithmetic.

%---------------------------------
\section{Peano and Heyting arithmetic}\label{s-oa-arithmetic}
%---------------------------------
%
\begin{definition}
	The \emph{language of arithmetic} is the first-order language \( \La \) with equality comprising of the nonlogical symbols:
	\begin{enumerate}
		\item function symbols: \( \0^0 \), \( \suc^1 \), \( +^2 \), \( ×^2 \).
		\item predicates: \( P^1 \).
	\end{enumerate}
\end{definition}
%
\begin{definition}
	The theory of Peano arithmetic is axiomatised by …
\end{definition}
%
As well as some basics of the theory of arithmetic, we recall the primitive recursive representation theorem.
%
\begin{definition}
	A formula is \( Σ_1 \) if it can be constructed from atomic formulas excluding \( P \) using the monotone propositional connectives, bounded universal quantifier and (unbounded) existential quantifier. That is, the \( Σ_1 \) formulas forms the smallest collection of \( \La \)-formulas satisfying:
	\begin{itemize}
		\item \( s = t , ⊥ ∈ Σ_1 \) for all (arithmetic) terms \( s \) and \( t \),
		\item if \( A \) and \( B \) are \( Σ_0 \), then so is \( A ∨ B \), \( A ∧ B \), and \( ∃ x\, A \),
		\item if \( A \) is \( Σ_1 \) and \( s \) is any (arithmetic) term not containing \( x \), then \( ∀x< s\, A \) is \( Σ_1 \), where this formula is shorthand for \( ∀x ( x< s → A ) \),
		\item if \( A \) and \( B \) are \( Σ_1 \) and, in addition, \( A \) does not contain the existential quantifier, then \( A → B \) is \( Σ_1 \).
	\end{itemize}
\end{definition}
%
\begin{theorem}
	Let \( R ⊆ \Nat^n \) be an \( n \)-ary relation on natural numbers. If \( R \) is primitive recursive then there exists a \( Σ_1 \) formula \( \Frml{R}(x_1, …, x_k ) \) of \( \La \) with at most the displayed variables occurring free such that for all \( n_1, …, n_k ∈ \Nat \),
	\[
		\HA ⊢ \Frml{R}(n_1 , …, n_k ) \quad\text{iff}\quad (n_1 , …, n_k ) ∈ R.
	\]
\end{theorem}
%

%---------------------------------
\section{Infinitary sequent calculi}\label{s-oa-omega-logic}
%---------------------------------

Onwards into infinitary proofs.
\begin{definition}\label{d-PAomega}
	\( \PAo \). Sequents/rules
	
	\( \HAo \). Likewise
\end{definition}
%
A look at induction.
%
\begin{namedlemma}[Embedding lemma]
	If \( \PA ⊢ Γ ⇒ Δ \) and \( Γ ⇒ Δ \) is a closed sequent.
	Then \( \PAo ⊢ Γ ⇒ Δ \).
\end{namedlemma}
%
To prove the lemma it is necessary to generalise its statement to include also
\( Γ^* ⇒ Δ^* \) is any closed instantiation of \( Γ ⇒ Δ \), then \( \PAo ⊢ Γ^* ⇒ Δ^* \).
%
%\begin{proof}
%	asd
%\end{proof}

Currently we have no means to measure the size of \( \PAo \)-proofs.
For this we will use ordinals.



%---------------------------------
\chapter{An ordinal interlude}\label{c-oa-ordinals}
%---------------------------------

To present the ordinals it is not necessary to have a set-theoretic definition of ordinals in mind (as, for example, arbitrary transitive sets).
Indeed, there is no need to consider the question of by what ordinals \emph{are} or from what they are \emph{formed}.
For a \emph{theory} of ordinals all that is relevant are the order-theoretic properties satisfied by the ordinals and a selection of operations that can be defined on them.
In short, ordinals are treated analogously to natural numbers: as a posited entity fulfilling specified criteria.
%
The material of this chapter draws from lecture notes by Michael Rathjen~\cite{RathjenLectures}.

%
\begin{definition}\label{d-ordinals}%[Ordinals]
	The \emph{ordinals} is a class \( \Ord \) equipped with a binary relation \( < \) satisfying three postulates, where \( ≤ \) is the reflexive closure of \( < \):
	\begin{axioms}[o]
		\item \( < \) is a strict linear order on \( \Ord \). That is, \( < \) is irreflexive, transitive and linear, where linear means that for all \( α , β ∈ \Ord \) either \( α ≤ β \) or \( β ≤ α \).\label{post-ord-lin}
		\item Every non-empty class of ordinals has a \( < \)-minimal element (necessarily unique by \ref{post-ord-lin}). That is, if \( O ⊆ \Ord \) is non-empty there exists \( ξ ∈ O \) such that \( ξ ≤ α \) for all \( α ∈ O \).\label{post-ord-wo}
		\item For every set \( X \) and function \( f \colon X → \Ord \) there exists \( ξ ∈ \Ord \) such that \( f(x) < ξ \) for every \( x ∈ X \).\label{post-ord-unbdd}
	\end{axioms}
\end{definition}

Set-theoretic concerns do matter in the language used to discuss ordinals.
As, for example, the Burali-Forte paradox shows, it is inconsistent  the Zermelo--Fraenkel (or Cantorian) conception of \emph{set} in mind to consider that the collection of (all) ordinals forms a set.
Hence use of term ‘class’ to refer to arbitrary collections of ordinals/objects and ‘set’ in specific case of \ref{post-ord-unbdd}.
Familiarity with set theory is not necessary for the elementary theory of ordinals presented here.
Indeed, it will suffice to replace every term ‘set’ in what follows by ‘countable set’ and ‘class’ by ‘countable or uncountable set’.
%We have side-stepped this concern by restricting attention to the \emph{countable} ordinals. Over Zermelo set theory, the collection of all countable ordinals, i.e., the collection \( \Ord \) above, forms a set; indeed, it is precisely the first uncountable ordinal.
%Such a restriction is not important for our later use of ordinals.
%Ultimately our attention will be constrained to a relatively small collection of (countable) ordinals.
%Our first lemma confirms that the relation \( <_\Ord \) is a well-order.

In the following, notation \( \setof{ t }[x ∈ X ] \) means the \emph{class} of objects \( t \) as \( x \) ranges over the (class) \( X \).
Usually a function \( f \colon U → V \) between classes has been specified along with a (sub)class \( X ⊆ U \) whence the notation \( \setof{ f(x) }[x ∈ X ] \) expresses the class of objects \( f(x) \) for \( x ∈ X \). This class will be written \( f[X] \).

\begin{convention}[Notating ordinals]
	Lowercase Greek letters \( α \), \( β \), etc.\ stand as metavariables for ordinals.
\end{convention}


\begin{lemma}\label{ord-well-order}
	Postulate \ref{post-ord-wo} is equivalent to the principle of transfinite induction.
	This is the statement that if \( O \) is progressive in the ordinals then \( \Ord ⊆ O \), where \( O \) is progressive means that for all ordinals \( α \), if \( β \in O \) for every \( β < α \) then \( α ∈ O \).
%	Let \( O ⊆ \Ord \) be non-empty.
%	Then \( O \) has a \( <_\Ord \)-minimal element and this is unique.
\end{lemma}
%
\begin{proof}
	Let \( O \) be progressive.
	Consider the class \( C = \Ord \setminus O \) of ordinals not in \( O \). If \( C \) is non-empty then, by \ref{post-ord-wo}, \( C \) contains a least ordinal, \( α \) say.
	As \( α \) is the least ordinal in \( C \), every \( ξ < α \) is element of \( O \). 
	Progressiveness implies that \( α ∈ O \) contradicting that \( α ∈ C \). 
	Hence, \( C \) is the empty class, so \( \Ord ⊆ O \).
	For the converse claim, assume postulate \ref{post-ord-lin} and the principle of transfinite induction (I could also assume \ref{post-ord-unbdd} but this is unnecessary).
	The aim is to establish \ref{post-ord-wo}. 
	Thus, let \( O \) be a non-empty class of ordinals and, for want of a contradiction, assume that \( O \) has no least element. 
	As in the other direction, I consider the complement of \( O \), the class \( C = \Ord \setminus O \).
	Suppose \( α \) be any ordinal such that \( ξ ∈ C \) for all \( ξ < α \). If \( α ∈ O \) then this is the least element of \( O \). As \( O \) has no least element therefore \( α ∈ C \).
	So \( C \) is progressive and \( C = \Ord \) by transfinite induction, contradicting the non-emptiness of \( O \).
\end{proof}

The next \namecref{ord-supremum} provides the primary means to infer the existence of ordinals.

\begin{lemma}
	\label{ord-supremum}\ 
	Let \( O \) be a class of ordinals.
	\begin{enumerate}
		\item There exists a least upper bound of \( O \). That is, an ordinal \( α \) such that \( ξ ≤ α \) for all \( ξ ∈ O \). This \( ξ \) is referred to as the \emph{supremum} of \( O \) and denoted \( \sup O \).
		\item There exists a strict least upper bound of \( O \), i.e., \( α \) such that \( ξ < α \) for all \( ξ ∈ O \).
	\end{enumerate}
	In each case the proclaimed ordinal is unique.
\end{lemma}
%
\begin{proof}
	Begin with 1. Let \( O \) be given.
	Consider the class \( O^≥ \) of all ordinals \( α \) such that \( ξ ≤ α \) for \emph{all} \( ξ ∈ O \).
	The \( < \)-least element of \( O^≥ \) (if such exists) is clearly the desired ordinal.
	But in order to apply postulate \ref{post-ord-wo} to this class it is necessary to establish that \( O^≥ \) is non-empty.
	For this I use the third postulate applied to identity function \( \textsf{id} \colon O → \Ord \colon ξ ↦ ξ \) (which is a function from \( O \) into \( \Ord \)).
	%
	For 2, the same argument works with the class \( O^> \) in place of \( O^≥ \) where this is the class of ordinals \emph{strictly} larger than all elements of \( O \).
	
	Uniqueness of each case is ensured by \ref{post-ord-lin}.
\end{proof}

Henceforth, I will not make explicit reference to the postulates.

The least ordinal is denoted \( 0 \). This happens to be the supremum of the empty set: \( 0 ≔ \sup ∅ \).
Given \( α ∈ \Ord \), the \emph{successor} of \( α \), in symbols \( α' \) or \( α + 1 \), is the least ordinal greater than \( α \), which exists (and is unique) by \cref{ord-supremum}(2) applied to the singleton set \( \setof{α} \).
That is, \( α' \) is such that \( ξ < α' \) iff \( ξ ≤ α  \).
The successor of \( 0 \) is denoted \( 1 ( =0') \), its successor \( 2 ( = 0'' ) \), etc.

A \emph{limit ordinal} is any non-zero ordinal \( λ \) such that \( η' < λ \), for all \( η < λ \).
Define a function \( f\colon ℕ → \Ord \) by \( f(0) = 0 \) and \( f(n+1) = f(n)' \).
That is, \( f(n) \) is the \emph{ordinal} representing the natural \( n \).
The supremum of \( \setof{ n }[n ∈ ℕ] \) is called \( ω \), which is a limit by construction and, therefore, the least limit ordinal.

\begin{lemma}
	\label{ord-suc-lim}
	Every non-zero ordinal is either a successor or a limit.
\end{lemma}

\begin{lemma}
	\label{ord-lim}
	An ordinal \( λ \) is a limit iff \( λ = \sup O \) for some non-empty set \( O \) closed under successor (meaning that \( ξ ∈ O \) implies \( ξ' ∈ O \)).
\end{lemma}

\begin{lemma}\label{ord-supremum-unique}
	Suppose \( O, O' \) are such that for every \( α \in O \) there exists \( β  \in O' \) such that \( α ≤ β \).
	Then \( \sup O ≤ \sup O' \).
\end{lemma}
\begin{exercise}
	Prove \cref{ord-suc-lim} to \ref{ord-supremum-unique}.
\end{exercise}


I will employ common set-theoretic abbreviations such as \( \sup_{i∈ I} α_i \) for \( \supof {α_i}[i ∈ I] \) and \( \sup_{i} α_i \) for \( \supof {α_i}[i < ω ] \).
I will also use \( λ \) as a metavariable for limit ordinals.

%--------------------------------------
\section{Elementary Ordinal Functions}
%--------------------------------------

A \emph{segment} of \( \Ord \) is any class \( O \) of ordinals which is closed downwards, i.e., if \( α < β ∈ O \) then \( α ∈ O \).
If \( X \) and \( Y \) are segments then either \( X ⊆ Y \) or \( Y ⊆ X \); in either case \( X ∩ Y \) is a segment.

Let \( O \) be a segment. A function \( f \colon O → \Ord \) is said to be:
\begin{itemize}
	\item \emph{order preserving} if \( α < β \) implies \( f(α) < f(β) \) for all \( α , β ∈ O \).
	\item \emph{continuous} if for all \( U ⊆ O \), if \( \sup U ∈ O \) then \( f( \sup U ) = \sup f[U] \).
%	\item \emph{normal} if \( O = \Ord \) and \( f \) is order preserving and continuous.
	\item an \emph{enumeration} (of \( X ⊆ \Ord \)) if \( f \) is order-preserving and \( f[O] = X \).
\end{itemize}

The identity function \( \mathsf{id} \colon \Ord → \Ord \) is all of the above. In particular, it is an enumeration of \( \Ord \).
Let \( f \colon ℕ → \Ord \) be given by \( f(0) = ω \) and \( f(n+1) = f(n)' \).
This function is order preserving and continuous (the latter is trivial). 
It is %not normal because the domain of \( f \) is not all ordinals, but it is
also an enumeration of the set \( \setof{ ω , ω' , … } \) because \( ℕ \) is a segment.
Notice that order preserving functions on ordinals are always injective.

\begin{lemma}\label{ord-o-p}
	If \( O \) is a segment and \( f \) is order preserving then \( α ≤ f(α) \) for all \( α ∈ O \).
\end{lemma}
%
\begin{exercise}
	Prove \cref{ord-o-p}.
\end{exercise}

The main property of ordinal functions I need is the summarised by
\begin{lemma}
	\label{ord-normal-exists}
	Every class of ordinals has a unique enumeration. The enumeration of \( Y ⊆ \Ord \) will be denoted \( E_Y \).
\end{lemma}
%
\begin{proof}
	\( E_Y \) is determined as the inverse of a particular function \( C_Y \colon Y → \Ord \), called the \emph{collapsing} function for \( Y \), defined by
	\[
		C_Y(α) = \sup \setof{ C_Y(ξ) + 1 }[ ξ ∈ Y \text{ and } ξ < α ].
	\]
	The collapsing function is clearly unique if it is well-defined. Moreover, \( C_Y \)
	This function is well-defined: Consider the class \( O \) of ordinals \( α \) for which the collapsing function on \( Y_α ≔ Y ∩ \setof{ ξ }[ξ ≤ α] \) exists.
	If \( C_{Y_ξ} \colon Y_ξ → \Ord \) is defined for each \( ξ < α \) I claim that \( C\colon Y_α → \Ord \) defined by 
	\[
		\begin{aligned}
			C(α) &= \sup\setof{ C_{Y_{ξ}}(ξ) + 1 }[ ξ < α \text{ and } ξ ∈ Y]
			\\
			C(ξ) &= C_{Y_ξ}(ξ) \text{ for \( ξ < α \)}
		\end{aligned}
	\]
	is the collapsing function for \( Y_α \).
	That this is the follows almost by definition. Indeed, all that is lacking is the observation that \( C_{Y_ξ}(β) = C_{Y_η}(β) \) whenever \( β ≤ ξ < η \).
	So \( O \) is progressive and transfinite induction implies that class \( Y_α \) has a collapsing function \( C_{Y_α} \). Now define \( C_Y \) as \( α ↦ C_{Y_α}(α) \).

	Clearly, \( C_Y \) is injective. Therefore the function admits a (right) inverse:
	\[
		E_Y ≔ C_Y^{-1} \colon C_Y[Y] → Y 
	\]
	As \( C_Y[Y ] \) is (clearly) a segment, \( E_Y \) is an enumeration of \( Y \).
	
	As to uniqueness of \( E_Y \), let \( O = C_Y[Y] \) and suppose \( f \colon O' → Y \) is any enumeration of \( Y \). In particular, \( O' \) is a segment. Transfinite induction implies that \( f(α) = E_Y(α) \) for all \( α ∈ O ∩ O' \).
	As both functions are injective and surjective into \( Y \) it follows that \( O = O' \).
\end{proof}

Two further properties of enumerations will be useful.

\begin{lemma}
	\label{ord-normal}
	Let \( f\colon \Ord → \Ord \) be continuous and order preserving (in particular, \( f \) is an enumeration of \( f[\Ord] \)).
	Then
	\begin{enumerate}
		\item For every \( α ≥ f(0) \) there is a unique \( β ≤ α \) such that \( f(β) ≤ α < f(β+1) \).\label{ord-normal-cover}
		\item For every \( α \) there is a unique \( β ≥ α \) such that \( β = f(β) \).\label{ord-normal-fix}
	\end{enumerate}
\end{lemma}
\begin{proof}
%	Continuity is implicit in the proof of \cref{ord-normal-exists}.
%	
	\ref{ord-normal-cover}. Consider the set \( O = \setof{ ξ }[ f(ξ) ≤ α ]\) and let \( β = \sup O \).
	Continuity yields
	\[
		f(β) = \sup f[O] = \sup \setof{ f(ξ) }[ f(ξ) ≤ α ] ≤ α
	\]
	whereas
	\(
		f(β+1) > α
	\)
	because \( β + 1 ∉ O \).
	
	\ref{ord-normal-fix}. Fix \( α \) and define \( O = \setof{ f(α) , f(f(α)) , …, f^n(α) , … } \) (arbitrary finite iterations of \( f \) on \( α \)). 
	Let \( β = \sup O \).
	Invoking continuity, \( f(β) = \sup f[O] = \sup O = β \). Moreover, \( α ≤ f(α) ≤ β \).
\end{proof}

%\begin{lemma}
%	\label{ord-normal-covers}
%	Let \( E\colon \Ord → \Ord \) be an enumeration (of \( E[\Ord] \)).
%	For every \( α ≥ E(0) \) there exists a unique \( β ≤ α \) such that \( E(β) ≤ α < E(β+1) \).
%\end{lemma}
%%
%\begin{proof}
%	Consider the set \( O = \setof{ ξ }[ E(ξ) ≤ α ]\) and let \( β = \sup O \).
%	Then 
%	\[
%		E(β) = \sup E[O] = \sup \setof{ E(ξ) }[ E(ξ) ≤ α ] ≤ α
%	\]
%	whereas
%	\(
%		E(β+1) > α
%	\)
%	because \( β + 1 ∉ O \).
%\end{proof}
%
%
%%As enumerations are order preserving it is always the case that \( α ≤ E(α) \). 
%
%\begin{lemma}
%	\label{ord-normal-fix}
%	Let \( E\colon \Ord → \Ord \) be an enumeration (of \( E[\Ord] \)). 
%	There exists \( ξ \) such that \( ξ = E(ξ) \).
%	Moreover, for every \( α \) there exists a unique \( ξ ≥ α \) such that \( ξ = E(ξ) \).
%\end{lemma}
%%
%\begin{proof}
%	Fix \( α \) and define \( O = \setof{ E(α) , E(E(α)) , …, E^n(α) , … } \) (arbitrary finite iterations of \( E \) on \( α \)). Then \( α ≤ E(α) ≤ \sup O \) and \( E(\sup O) \)
%\end{proof}


%--------------------------------------
\section{Elementary Ordinal Arithmetic}
%--------------------------------------
The basic operations of arithmetic can be extended to ordinals in a straightforward manner.
Often these are defined by transfinite recursion, but the two operations we desire, addition and exponentiation base \( ω \), can be expressed as enumeration functions.
I start with addition.
\begin{definition}
	Let \( α^≥ \) be the class of ordinals \( ≥ α \).
	\emph{Ordinal addition}, \( α + β \), is defined as \( α + β ≔ E_{α^≥}(β) \). That is, \( α + β \) is defined as the \( β \)-th ordinal in the enumeration of the ordinals \( ≥ α \).
\end{definition}
%
The following are direct consequences of this definition and left to the reader.

\begin{lemma}\label{ord-addition}
	For all \( α \), \( β \) and \( γ \).
	\begin{enumerate}
		\item \( α + 0 = α \).
		\item \( α + β' = ( α + β )' \).
		\item If \( β \) is a limit then \( α + β = \sup \setof{ α + ξ }[ξ < β ] \).
		\item \( α + ( β + γ ) = ( α + β ) + γ \).\label{ord-addition-assoc}
		\item \( α ≤ α + β \) and \( β ≤ α + β \).\label{ord-addition-inc}
	\end{enumerate}
\end{lemma}

\begin{example}\label{ex-ord-add}
	\( α + ω = \sup \setof{ α + n }[n∈ ℕ] = \sup \setof{ α , α' , α'' , …  } \). Thus \( α + ω \) is the least limit ordinal strictly above \( α \).
	
	In particular, \( n + ω = ω \) for every \( n < ω \).
	As \( 1 + ω = ω < ω + 1 \) ordinal addition is not commutative.
\end{example}

As addition is associative (item \ref{ord-addition-assoc} of the \namecref{ord-addition} above), I will omit brackets when stringing together applications of addition.
So \( α + β + γ \) can refer to either \( ( α + β ) + γ \) or \( α + ( β + γ ) \).

The next lemma is a consequence of \cref{ord-normal}.

\begin{lemma}
	For every \( α ≤ β \) there exists a unique \( ξ \) such that \( β = α + ξ \).
\end{lemma}
\begin{proof}
	\Cref{ord-normal} implies a unique \( ξ \) such that \( α + ξ ≤ β < α + ξ' \). Since \( α + ξ' = ( α + ξ ) + 1 \) it follows that \( α + ξ = β \).
\end{proof}

As \cref{ex-ord-add} demonstrates \( ω \) has the unusual property of being closed under addition: if \( ξ , η < ω \) then \( ξ + η < ω \).
Ordinals satisfying this condition are called \emph{additive principal} ordinals.
\begin{definition}
	\label{d-ord-AP}
	A ordinal \( α \) is additive principal iff \( α > 0 \) and \( ξ + η < α \) for all \( ξ , η < α \).
	The class of additive principal ordinals is denoted \( \AP \).
\end{definition}

The least additive principal ordinal is \( 1 \); the next is clearly \( ω \).
Most ordinals are \emph{not} additive principal. 
\( 1 \) is the only additive principal successor ordinal (because \( α + α ≥ α' \) provided \( α ≥ 1 \)).
Even most limit ordinals not additive principal: If \( α ≥ ω \) then \( α + ω ∉ \AP \) as \( α < α + ω \) but \( α + α ≮ α + ω \).

\begin{lemma}
	\label{ord-AP-normal}
	The enumeration function \( E_\AP \) for additive principal ordinals is continuous and has domain \( \Ord \).
%	The additive principal ordinals are
%	\begin{enumerate}
%		\item Closed: for every set \( O ⊆ \AP \), \( \sup O ∈ \AP \).
%		\item Unbounded in \( \Ord \). For every set \( α ∈ \AP \) there exists \( β > α \) such that \( β ∈ \AP \).
%	\end{enumerate}
\end{lemma}
\begin{proof}
%	Begin with the domain. By transfinite induction. Suppose \( ξ ∈ \dom E_\AP \) for every \( ξ < α \). I claim that \( α ∈ \dom E_\AP \). It suffices to show that there exists an additive principal ordinal \( β > E_\AP(ξ) \) for every \( ξ < α \).
%	Let \( O_0 = \setof{ E_\AP(ξ) }[ξ < α] \) and \( O_{n+1} = \setof{ ξ + η }[ξ , η ∈ O_n ] \). Set \(  \)
	Exercise.
\end{proof}

\Cref{ord-AP-normal} shows that the function enumerating the additive principal ordinals is defined on all ordinals, is order preserving and continuous.

\begin{lemma}
	\label{ord-AP}
	The following are equivalent for all \( α > 0 \):
	\begin{enumerate}
		\item \( α \) is additive principal.\label{ord-AP-1}
		\item \( α = 1 \) or \( α = \sup \setof{ ξ + ξ }[ξ < α ] \).\label{ord-AP-2}
		\item for all \( β < α \), \( β + α = α \).\label{ord-AP-3}
	\end{enumerate}
\end{lemma}
\begin{proof}
	\ref{ord-AP-1} $⇒$ \ref{ord-AP-2}. If \( α \) is additive principal then \( \sup \setof{ ξ + ξ }[ξ < α ] ≤ α \) by definition. Also, the additive principal ordinals except \( 1 \) are all limits, so if \( α ≠ 1 \) then \( α = \sup \setof{ ξ }[ξ < α ] ≤ \sup \setof{ ξ + ξ }[ξ < α ] \).
	
	\ref{ord-AP-2} $⇒$ \ref{ord-AP-3}. For \( α = 1 \) the claim is trivial. Otherwise, \( α  \) is a limit and \( β + α ≤ \sup \setof{ β + ξ }[ξ < α ] ≤ \sup \setof{ ξ + ξ }[ξ < α ] \). As \( α = \sup \setof{ ξ + ξ }[ξ < α ] \) the claim is established.
	
	\ref{ord-AP-3} $⇒$ \ref{ord-AP-1}. Straightforward.
\end{proof}
%

As a consequence of part \ref{ord-AP-3}, \( ω^α + ω^β = ω^β \) iff \( α < β \).
A corollary is the observation made earlier, that \( n + ω = ω \), which now follows from repeated applications of \cref{ord-AP}: \( α' + ω = α + ( ω^0 + ω^1 ) = α + ω^1 \).

Additive principal ordinals are central to the theory of ordinals.
As with addition, I will introduce more suggestive notation for the enumeration function for additive principal ordinals.

\begin{definition}
	\( ω^α ≔ E_\AP(α) \).
\end{definition}

By the definition \( ω^0 = 1 \) and \( ω^1 = ω \).
The reader can confirm that next additive principal ordinal above \( ω \) is the supremum of \( ω \), \( ω + ω \), \( ω + ω + ω \), …, \( ω + ⋯ + ω , … \) which is denoted \( ω^2 \).
%I will shortly show that the function \( ξ ↦ ω^ξ \) behaves as one would expect exponentiation (in particular, \( ω^(α + β) \)

\begin{lemma}
	\label{ord-AP-NF}
	For every \( α > 0 \) there exists unique \( β \) and \( ξ < α \) such that \( α = ω^β + ξ \).
\end{lemma}
\begin{proof}
	Let \( β \) be such that \( ω^β ≤ α < ω^{β'} \) and \( ξ \) such that \( α = ω^β + ξ \). Both ordinals are given by \cref{ord-normal}.
	What remains is to show uniqueness of this choice.
	Thus, suppose \( α = ω^γ + η \) for some \( γ \) and \( η < α \).
	The choice of \( β \) is clearly such that \( β ≥ γ \).
	As 
	\[ ω^β + ω^{γ+1} ≤ α + ω^{γ+1} ≤ ω^γ + η + ω^{γ + 1 } = ω^{γ+1} \]
	(the first inequality uses \cref{ord-addition}(\ref{ord-addition-inc}); the rest use \cref{ord-AP}(\ref{ord-AP-3})), also \( β ≤ γ \). Given that \( β = γ \), uniqueness of the rest is immediate.
\end{proof}
%\begin{lemma}
%	\label{ord-AP-add}
%	For all \( α > 0 \), \( α ∉ \AP \) iff \( α = β + γ \) for some \( β , γ < α \).
%\end{lemma}

%---------------------------------
\section{Normal forms and natural sum}
%---------------------------------

\Cref{ord-AP-NF} above provides the basis of a normal form representation of ordinals. This concept is introduced in the next definition.

\begin{definition}
	I write \( α =_\NF ω^β + γ \) to express that (i) \( α = ω^β + γ \) and (ii) \( γ < α \).
\end{definition}

Cantor, in 1897,\nocite{Cantor1897} established an expanded version of this normal form decomposition.
%
\begin{theorem}[Cantor normal form]\label{t-cantornf}
	For every ordinal \( α > 0 \) there exists \( n \) and ordinals \( α_n ≤ ⋯ ≤ α_0 \) such that 
	\[
		α = ω^{α_0} + ⋯ + ω^{α_n}.
	\]
	Moreover, this decomposition is unique. %, and \( α_0 < α \) if \( α < ε_0 \).
\end{theorem}
%
\begin{proof}
	The theorem is a simple generalisation of \cref{ord-AP-NF}.
%	The argument is by transfinite induction. Assume that each ordinal \( β < α \) admits a decomposition as described in the theorem. 
	Let \( α =_\NF ω^{α_0} + ξ_0 \) by \cref{ord-AP-NF}. If \( ξ_0 = 0 \) the decomposition is complete. 
	Otherwise, apply the \namecref{ord-AP-NF} again to express \( ξ_0 =_\NF ω^{α_1} + ξ_1 \), \( ξ_1 =_\NF ω^{α_2} + ξ_2 \), etc. 
	As \( α > ξ_0 > ξ_1 > ⋯ \) is a strictly decreasing sequence or ordinals, necessarily \( ξ_n = 0 \) for some \( n \). Thus, 
	\( α = ω^{α_0} + ⋯ + ω^{α_n} \). 
	Furthermore, \( α_0 ≥ α_1 ≥ ⋯ ≥ α_n \) because \( ω^{α_{i+1}} ≤ ξ_i < ω^{α_i+1} \) for each \( i \).
	Uniqueness is also a consequence of these normal forms.
\end{proof}


\begin{definition}
	The normal form notation is extended in the following way. Writing \( α =_\NF ω^{α_1} + ⋯ + ω^{α_n} \) expresses that (i) \( α = ω^{α_1} + ⋯ + ω^{α_n} \) and (ii) \( α ≥ α_1 ≥ ⋯ ≥ α_n \).
\end{definition}

\Cref{ord-normal} showed that every continuous order preserving function on the ordinals has fixed points. I.e., for each such function \( f \) there are ordinals \( β \) such that \( β = f(β) \).
As the function \( ξ ↦ ω^ξ \) (namely \( E_\AP \)) is an example of such a function, there must exist ordinals \( α \) such that \( α = ω^α \).
The proof of that lemma describes how to construct such an ordinal as the supremum of the sequence \( 0 \), \( 1 \), \( ω \), \( ω^ω \), …, \( α \), \( ω^α \), ….
This particular ordinal, conventionally denoted \( ε_0 \), will play a central role in the next chapter.

\begin{definition}\label{d-epsilon0}
	\( ε_0 ≔ \supseq ω_i \) where \( ω_0 = ω \) and \( ω_{k+1} = ω^{ω_k} \).
\end{definition}
%

%
\begin{lemma}
	\label{ord-e0}
	\( ε_0 \) is the least fixed point of the ordinal function \( α \mapsto ω^α \). 
	That is, \( ω^{ε_0} = ε_0 \) and \( α < ω^{α} \) for all \( α < ε_0 \).
\end{lemma}
%
\begin{exercise}
	Prove \cref{ord-e0}.
\end{exercise}

\begin{exercise}\label{ex-ord-mult-pre}
	Using the Cantor normal form theorem, define a multiplication operation where the first argument is restricted to additive principal ordinals: \( α , β ↦ ω^α.β \).
	The function should be continuous in \( β \) and satisfy the recursive clauses: \( ω^α.0 = 0 \) and \( ω^α.(β + 1) = ω^α.β + ω^α \).
\end{exercise}

\begin{exercise}\label{ex-ord-base-2}
	Define a function \( α ↦ 2^α \) satisfying
	\begin{align*}
		2^0 &= 1
		\\
		2^{α+1} &= 2^α + 2^α
		\\
		2^λ &= \supof{ 2^ξ }[ξ < λ]
	\end{align*}
	(You may find it useful to use the Cantor normal form theorem.)
	Show that this function is order preserving and continuous, and compute all fixed points of the function for ordinals \( α ≤ ε_0 \).
%	(It is not necessary to prove the existence of this function.)
\end{exercise}
%
\begin{exercise}
	\label{ex-epsilon-numbers}
	Let \( α ↦ ε_α \) be the enumerating function of the ordinals \( η \) such that \( η = ω^η \).
%	For each \( α \), define \( ε_α \) as the least ordinal such that
%	\begin{enumerate}
%		\item \( ε_β < ε_α \) for every \( β < α \),
%		\item \( ω^{ε_α} = ε_α \).
%	\end{enumerate}
	Express \( ε_α \) as a supremum of smaller ordinals as per \cref{d-epsilon0} and deduce that the enumerating function is defined for all ordinals.
\end{exercise}
%


\begin{exercise}
	\label{ex-ord-cnf-2}
	Prove the Cantor normal form theorem in base \( 2 \):
	\emph{For every ordinal \( α > 0 \) there exists unique ordinals \( α_n ≤ ⋯ ≤ α_0 ≤ α \) such that 
	\[
		α = 2^{α_0} + ⋯ + 2^{α_n}.
	\]}
\end{exercise}
%
\begin{exercise}
	\label{ex-ord-cnf-2o}
	What are the additive principal ordinals in base-$2$ normal form?
	Characterise the \( α \) such that \( 2^α = ω^α \).
\end{exercise}

%---------------------
%\section{Natural sum}
%---------------------

This brief foray into ordinals is concluded with another look at addition.
Recall that addition on ordinals is not commutative: \( 1 + ω ≠ ω + 1 \) for example.
It is possible to provide a natural notion of addition that \emph{is} commutative.
This is called the \emph{natural sum} (sometimes \emph{Hessenberg sum} after its originator Gerhard Hessenberg~\cite{Hess1906}). 
The Cantor normal theorem provides the means to achieve this.

\begin{definition}
	The natural sum of ordinals \( α \) and \( β \), denoted \( α \nsum β \) is defined by recursion on the two ordinals. \( 0 \nsum α = α \nsum 0 ≔ α \) for all \( α \).
	For non-zero \( α =_\NF ω^{α_0} + α_1 \) and \( β =_\NF ω^{β_0} + β_1 \)
	\[
		α \nsum β ≔ 
		\begin{cases}
			ω^{α_0} + ( α_1 \nsum β ), &\text{if \( α_0 ≥ β_0 \),}
			\\
			ω^{β_0} + ( α \nsum β_1 ), &\text{if \( α_0 ≤ β_0 \).}
		\end{cases}
	\]
	The operation of natural sum is well-defined as \( α_1 < α \) and \( β_1 < β \).
\end{definition}

As an operation on the Cantor normal form, the natural sum  has the following property.
\begin{lemma}
	For \( α =_\NF ω^{α_1} + ⋯ + ω^{α_m} \) and \( β =_\NF ω^{β_1} + ⋯ + ω^{β_n} \)
	\[
		α \nsum β ≔ ω^{γ_1} + ⋯ + ω^{γ_{m+n}}
	\]
	where \( γ_1 ≥ ⋯ ≥ γ_{m+n} \) enumerate the ordinals \( α_1, …, α_m , β_1 , …, β_n \) in descending order (with repetitions).
\end{lemma}

\begin{lemma}
	\label{ord-nsum}
	The natural sum is commutative and strongly increasing in both arguments: For all \( α \), \( β \), \( γ \),
	\begin{enumerate}
		\item \( α \nsum β = β \nsum α \);
		\item \( α < β \) implies \( α \nsum γ < β \nsum γ \).
	\end{enumerate}
\end{lemma}
%\begin{lemma}
%	\label{ord-nsum-2}
%	\begin{enumerate}
%		\item \( α \nsum β \)
%		\item \( α + β ≤ α \nsum β ≤ \max\setof{ α , β }.2 \).
%	\end{enumerate}
%\end{lemma}

\begin{exercise}
	Prove \cref{ord-nsum}.
\end{exercise}

\begin{exercise}
	\label{ex-ord-mult}
	Using the Cantor normal form theorem define a commutative multiplication \( α . β \) operation on ordinals. It should satisfy the distribution law:
	\(
		( α \nsum β ) . γ = (α .γ) \nsum (β .γ ).
	\)
	Hint, start from the function in exercise~\ref{ex-ord-mult-pre}.
\end{exercise}


%---------------------------------
\chapter{Ordinal analysis of arithmetic}\label{c-oa-PAo}
%---------------------------------
Ordinals will now be used to measure the \emph{height} of \( ω \)-proofs.
I begin by recalling the infinitary sequent calculi for arithmetic from the end of \cref{c-oa-arith}.
%
\begin{definition}%\label{d-PAomega}
	\( \PAo \) is the sequent calculus given by the following:
	\begin{itemize}
		\item sequents comprise formulas in the language of arithmetic (with equality).
		\item Initial sequents are \emph{closed} sequents of the form
		\begin{itemize}
			\item[(\botL)] \( ⊥, Γ ⇒ Δ \)
			\item[(\idRule)] \( Ps, Γ ⇒ Δ , Pt \) if \( ℕ ⊨ s = t \)
			\item[(\eqR)] \( Γ ⇒ Δ , s = t \) if \( ℕ ⊨ s = t \)
			\item[(\eqL)] \( s = t , Γ ⇒ Δ \) if \( ℕ ⊭ s = t \)
		\end{itemize}
		\item Inference rules are rules of \( \Gc \) but restricted to closed sequents and with \( \faR \) and \( \exL \) replaced by the two \( ω \)-rules:\note{\( \Gc \) or more general \( \Logic{G3} \)?}
		\begin{itemize}
			\item[(\omR)] \begin{prooftree} \hypo{ Γ ⇒ Δ , F(\nm n) \ \text{for every } n ∈ ℕ } \infer1{ Γ ⇒ Δ , ∀x F(x) }\end{prooftree}
			\item[(\omL)] \begin{prooftree} \hypo{ F(\nm n) , Γ ⇒ Δ \ \text{for every } n ∈ ℕ } \infer1{ ∃x F(x) , Γ ⇒ Δ }\end{prooftree}
		\end{itemize}
	\end{itemize}
%	Writing \( \PAo ⊢ Γ ⇒ Δ \) expresses that there is an \( ω \)-proof of \( Γ ⇒ Δ \) according to the above rules. In other words, there exists a well-founded tree labelled by sequents such that each leaf is an initial sequent and that each inner vertex together with its immediate successors in the tree forms a correct application of a rule of the calculus listed above.
%	
	\( \HAo \) is the same calculus but restricted to intuitionistic sequents.
\end{definition}
%
\begin{definition}\label{d-bound-omega-logic}
	Let \( \Theory{T} \) be \( \PAo \), \( \HAo \) or an extension of either calculus by rules that are at most \( ω \)-branching.
	The ternary relation \( \Theory{T} \prv{α}k Γ ⇒ Δ \), between a sequent \( Γ ⇒ Δ \), an ordinal \( α \) and \( k < ω \), is defined by transfinite recursion on the rules of \( \Theory{T} \):
	\begin{enumerate}
		\item If \( Γ ⇒ Δ \) is an initial sequent of \( \Theory{T} \), then \( \Theory{T} \prv{α}k Γ ⇒ Δ \) for all \( α \) and \( k \);
		\item For each inference \( (*) \) of \( \Theory{T} \) except cut of the form
		\[
			\Infer{\setof{ Γ_i ⇒ Δ_i }[i ∈ I]}[\( * \)]{ Γ ⇒ Δ }
		\]
		\( \Theory{T} \prv{α}k Γ ⇒ Δ \) holds if \( \Theory{T} \prv{α_i}k Γ_i ⇒ Δ_i \) and \( α_i < α \) for all \( i ∈ I \);
		\item If \( \Theory{T} \prv{α_0}k Γ ⇒ Δ , C \) and \( \Theory{T} \prv{α_1}k C , Γ ⇒ Σ \) for \( α_0,α_1 < α \) and \( \rk C < k \), then \( \Theory{T} \prv{α}k Γ ⇒ Δ, Σ \).
	\end{enumerate}
%	If \( \Theory{T} \) is a calculus of intuitionistic sequents, \( \Theory{T} \prv{α}k Γ ⇒ A \) is defined via the same conditions but with a modification of the final clause:
%	\begin{enumerate}[resume]
%		\item If \( \Theory{T} \prv{α_0}k Γ ⇒ C \) and \( \Theory{T} \prv{α_1}k C , Γ ⇒ A \) for \( α_0,α_1 < α \) and \( w(C) < k \), then \( \Theory{T} \prv{α}k Γ ⇒ A \).
%	\end{enumerate}
	Given \( \Theory{T} \prv{α}k Γ ⇒ Δ \) I will write that \( Γ ⇒ Δ \) is derivable (in \( \Theory T \)) with height \( ≤ α \) and cut rank \( ≤ k \).
\end{definition}
%

%Some of the terminology  to \( ω \)-proofs as before.

There is no requirement of minimality of \( α \) and \( k \) in the above definition. 
So the relation \( \prv{α}k \) is monotone in \( α \) and \( k \):
%
\begin{lemma}\label{oa-PAo-weak1}
	If \( α ≤ β \) and \( k ≤ l \) then \( \Theory{T} \prv{α}k Γ ⇒ Δ \) implies \( \Theory{T} \prv{β}l Γ ⇒ Δ \).
\end{lemma}
%
\begin{proof}
	By transfinite induction on \( α \). If \( Γ ⇒ Δ  \) is an initial sequent, the result is immediate.
	Otherwise, there is an inference rule of \( \Theory T \)
	\[
		\Infer{\setof{ Γ_i ⇒ Δ_i }[i ∈ I]}[\( * \)]{ Γ ⇒ Δ }
	\]
	and ordinals \( α_i < α \) such that \( \smash{\Theory T \prv{α_i}k Γ_i ⇒ Δ_i} \) for each \( i ∈ I \).
	The induction hypothesis implies that \( \Theory T \prv{α_i}l Γ_i ⇒ Δ_i \) for each \( i \), whereby \( \Theory T \prv{β}l Γ ⇒ Δ \) obtains.
\end{proof}

\Cref{oa-PAo-weak1} operates in the background of the majority of the results to follow. For that reason I will not make any explicit reference to the \namecref{oa-PAo-weak1}.

\begin{example}
	\tbw
\end{example}

\begin{lemma}\label{oa-HA-good}
	If \( \HAo \prv{α}k Γ ⇒ A \) then this fact can be observed by use of sequents of the form \( Σ ⇒ B \) (i.e., exactly one formula on the right).
\end{lemma}

\begin{exercise}
	Assign ordinal bounds on the \( ω \)-proofs of \( A , Γ ⇒ Δ , A \) constructed in exercise~\ref{ex:oa-id-simple}.
\end{exercise}

Revisiting the Embedding lemma (lemma \ref{oa-embed-weak}) it is possible provide ordinal bounds on the size of the resulting \( ω \)-proof.
Let \( α.k = \underbrace{α + ⋯ + α }_k \).

\begin{lemma}[Refined embedding lemma]\label{oa-embed-PAo-w-bounds}
	Suppose \( \PA ⊢ Γ ⇒ Δ \) and \( Γ ⇒ Δ \) is closed. Then there is \( n,k < ω \) such that \( \PAo \prv{ω.n}k Γ ⇒ Δ \) where \( ω .n = ω + ⋯ + ω \) (\( n \) times).
	Likewise, \( \HA \) into \( \HAo \).
\end{lemma}
%
\begin{exercise}
	Prove the refined embedding lemma following the schema of embedding lemma at the end of \cref{c-oa-arith}.
\end{exercise}
%
The next lemma hints at part of the usefulness of the \( ω \)-rule with the ability to isolate finitary reasoning from infinitary reasoning.
The result will be useful in \cref{s-oa-upper}.
%
\begin{proposition}\label{p-PAo-S1}
	Let \( A(a_1,…, a_k) \) be a \( Σ_1 \) formula. 
	There exists \( m < ω \) such that for all \( n_1, …, n_k ∈ \Nat \), 
	\[ \text{if }\thinspace  \Nat ⊨ A(\nm {n_1}, …, \nm {n_k} ) \text{ then } \HAo \prv m0 {⇒ A(\nm {n_1}, …, \nm {n_k})} .
	\]
\end{proposition}
%
\begin{proof}
	By induction on the rank of \( A \).
\end{proof}

Henceforth, I will omit explicit mention of \( \PAo \) and write \( \prv{α}k Γ ⇒ Δ \) to mean \( \PAo \prv{α}k Γ ⇒ Δ \).
The following results are stated only for \( \PAo \) but apply equally to \( \HAo \) in the expected way.
Admissibility of weakening becomes 
%
\begin{lemma}[Weakening Lemma]
	If \( \prv{α}k Γ ⇒ Δ \) and \( Γ' ⇒ Δ' \) is closed then \( \prv{α}k Γ' , Γ ⇒ Δ, Δ' \).
%	Likewise for \( \HAo \).
\end{lemma}
%
%\begin{proof}
%	By (transfinite) induction on \( α \).
%\end{proof}
\begin{exercise}
	Prove the weakening lemma.
\end{exercise}

The substitution lemma for \( \PAo \) takes a different formulation from previously. 
As sequents are closed, the correct formulation for \( ω \)-proofs is that provability depends on the \emph{value} of terms, not their \emph{form}.

\begin{lemma}[Substitution Lemma]
	Let \( Γ(a) ⇒ Δ(a) \) be a sequent and \( s \) and \( t \) be closed terms such that \( \Nat ⊨ s = t \). If \( \prv{α}k Γ(s) ⇒ Δ(s) \) implies \( \prv{α}k Γ(t) ⇒ Δ(t) \).
\end{lemma}
%
\begin{proof}
%	Analogous argument. I will treat the case of \( \exR \): 
	Suppose 
	\( \prv{α}k Γ(s) ⇒ Δ(s) \) and \( \Nat ⊨ s = t \).
	Let \( Γ(a) ⇒ Δ(a) \) be any sequent with at most \( a \) free.
	If \( Γ(s) ⇒ Δ(s) \) is initial then a case distinction on the different forms this sequent can take confirms that \( Γ(s) ⇒ Δ(s) \) is also initial provided \( \Nat ⊨ s = t \). The other case proceed by transfinite induction on \( α \).
\end{proof}

The final ingredient is the inversion lemma, the statement of which has the same form as before with two new cases treating equality.

\begin{lemma}[Inversion lemma]
	\label{oa-inversion}\ 
	\begin{enumerate}
		\item If \( \prv{α}k Γ ⇒ Δ , ⊥ \) then \( \prv{α}k Γ ⇒ Δ \).
		\item If \( \prv{α}k s = t , Γ ⇒ Δ \) and \( ℕ ⊨ s = t \) then \( \prv{α}k Γ ⇒ Δ \).\label{oa-inversion-eq1}
		\item If \( \prv{α}k Γ ⇒ Δ , s = t \) and \( ℕ ⊭ s = t \) then \( \prv{α}k Γ ⇒ Δ \).
		\item If \( \prv{α}k Γ ⇒ Δ , ∀x F(x) \) then \( \prv{α}k Γ ⇒ Δ , F(s) \) for every closed term \( s \).\label{oa-inversion-fa}
		\item If \( \prv{α}k ∃x F(x) , Γ ⇒ Δ \) then \( \prv{α}k F(s) , Γ ⇒ Δ \) for every closed term \( s \).
		\item Analogous inversion principles for the rules \( \disjL \), \( \conjR \), \( \impR \) and \( \impL \).
	\end{enumerate}
\end{lemma}
\begin{proof}
	I show cases \ref{oa-inversion-eq1} \& \ref{oa-inversion-fa}.
	
	\ref{oa-inversion-eq1}. By induction on \( α \). Suppose \( \prv{α}k s = t , Γ ⇒ Δ \) and \( ℕ ⊨ s = t \). If \( s = t , Γ ⇒ Δ \) is initial then so is \( Γ ⇒ Δ \). The other cases are straightforward because the equation \( s = t \) cannot be the principal formula of any rule.
	For if \( s = t , Γ ⇒ Δ \) is not initial, then there are sequents \( \setof{ Γ_i ⇒ Δ_i}[i<ω] \)
	and ordinals \( \setof{α_i}[i<ω] \) such that
	\begin{enumerate}[label=(\alph*)]
		\item \( \prv{α_i}k s=t , Γ_i ⇒ Δ_i \) for each \( i < ω \),
		\item \( α_i < α \) for all \( i \),
		\item \( \setof{ Γ_i ⇒ Δ_i}[i<ω] \) enumerate all premises of an inference of \( \PAo \) whose conclusion is \( Γ ⇒ Δ \).
	\end{enumerate}
	In the case of unary or binary rules, \( Γ_i = Γ_{i+1} \) and \( Δ_{i} = Δ_{i+1} \) for all \( i > 0 \) or \( 1 \). But in the case of either of the two \( ω \)-rules, the sequents enumerate the infinitely many premises.
	By (a)--(c) and the induction hypothesis, \( \prv{α}k Γ ⇒ Δ \) holds as desired.
	
	\ref{oa-inversion-fa}. The argument is a direct generalisation of the finitary inversion lemma. Suppose \( \prv{α}k Γ ⇒ Δ , ∀x F(x) \). If this sequent is initial, then so is \( Γ ⇒ Δ , F(s) \) for every closed term \( s \).
	The rest of the argument proceeds, essentially, as above by a case distinction on the inferences through which \( \prv{α}k Γ ⇒ Δ , ∀x F(x) \) can be derived.
	The case of \( \faR \) with \( ∀x F(x) \) principal bears treatment.
	The premises of this inference can be assumed to have the form \( Γ ⇒ Δ , ∀x F(x) , F(\nm n) \). An application of the induction hypothesis (to each premise) yields \( \prv{α}k Γ ⇒ Δ , F(\nm n) \) for every \( n \). If the desired closed term \( s \) is a numeral, this case is complete. Otherwise, let \( n \) be the value of \( s \), i.e., \( n ∈ ℕ \) is such that \( ℕ ⊨ \nm n = s \).
	The substitution lemma then yields \( \prv{α}k Γ ⇒ Δ , F(s) \).
\end{proof}

%---------------------------------
\section{Infinitary cut elimination}\label{s-oa-cutelim}
%---------------------------------

I begin with the transfinite version of the reduction lemma.
Recall, this is statement that borderline cuts can be simulated at the cost of increasing the depth of the proof by a controlled amount.
In the finitary case the depth increase was, in the case of classical logic, \( m + n \) where \( m \) and \( n \) bounded the depth of the two cut premises.

Lifting the statement of the reduction lemma to the transfinite realm is reasonably straightforward.
Given premises of a borderline cut of height \( α \) and \( β \) respectively, the cut can be simulated by a height of \( α \nsum β \).
The use of natural sum is crucial to the argument: the lifting of the finitary argument requires the resulting bound to be order-preserving in both arguments, a property we know fails for traditional ordinal sum \( α + β \).

%In fact, I will give two proofs of the reduction lemma.
%The first is the version just described: a direct lifting of the finitary argument to \( ω \)-proofs. 
%Some steps of the proof clearly require new insights, such as passing cuts through \( ω \)-rules and the new initial sequents with equations.
%But, by and large, the design of \( \PAo \) is such that the transfinite element is straightforward once one knows what to expect.
%
%Thus, the reduction lemma I will prove will be:

\begin{lemma}[Reduction lemma for \( \PAo \)]\label{oa-red-lem-PAo}
	Suppose \( \prv{α}k Γ ⇒ Δ , C \) and \( \prv{β}k C , Σ ⇒ Λ \).
	If \( \rk C ≤ k \) then \( \prv{α\nsum β }k Γ, Σ ⇒ Δ, Λ \).
\end{lemma}

The reader may surprised to know that there is a great deal of flexibility in proofs of the reduction lemma, which I will demonstrate by presenting a slightly different strategy than we used for in the analysis of classical predicate logic.
%The statement of the lemma already offers a minor, though insignificant, departure from previously by allowing \( \rk C < k \). 
%Provided that \( α , β \) are non-zero, \( \max\setof{ α , β } + 1 ≤ α \nsum β \), so whether.

%The second formulation will be presented after the proof of the above \namecref{oa-red-lem-PAo}.
%In short, it mitigates a shortcoming of the reduction lemma as described in its exaggeration of the complexity of cut elimination.
%The transfinite reduction lemma is optimal in the same way as the finitary form: Sequents can be readily chosen whose shortest cut-free proofs require a jump in ordinal height matching that expressed by the reduction lemma and its immediate corollaries.
%But by treating each logical connective as equally ‘expensive’, the reduction lemma ignores the true source of ‘large’ proofs: \emph{alternation} between \emph{positive} and \emph{negative} statements.
%
%This alternative version of the reduction lemma is closely related to the form proved in exercise~\ref{ex-red-lem-special}.

\begin{proof}%[of \cref{oa-red-lem-PAo}]
%	Suppose:
%	\begin{enumerate}
%		\item \( \prv{α}k Γ ⇒ Δ , C \).
%		\item \( \prv{β}k C , Σ ⇒ Λ \).
%	\end{enumerate}
%
	The proof branches into cases depending on the form of \( C \).
	In each case I will establish \( \prv{α \nsum β} k Γ, Σ ⇒ Δ, Λ \) but the induction will proceed over either \( α \) or \( β \) (depending on the case) rather than on the sum \( α \nsum β \).
	If the principal connective of \( C \) is among \( \setof{ ⊥ , ∀ ,  ∧ , → } \) I will refer to \( C \) as \emph{locally negative} (cf.~Canvas assignment no.~4).
	Otherwise, \( C \) will be \emph{locally positive}.
	
%	\paragraph{Case I: \( C \) is \( ⊥ \) or an equation.} This case can be dispensed with directly. If \( C = ⊥ \) or a false equation then \( \prv{α}k Γ ⇒ Δ \) is a consequence of the inversion lemma;  otherwise \( \prv{β}k Σ ⇒ Λ \).
%	In any case, weakening yields \( \prv{α\nsum β }k Γ, Σ ⇒ Δ , Λ \).
	
	\paragraph{Case I: \( C \) is atomic or locally negative.}
	Here I proceed by induction on \( β \) and show that \( \prv{ α \nsum β }k Γ, Σ ⇒ Δ , Λ \).
	I present two subcases:
	
	\( C = ∀x D(x) \). If \( C , Σ ⇒ Λ \) is initial then \( Σ ⇒ Λ \) is also initial and the claim holds by weakening.
	Otherwise, consider the rule that derives \( \prv{β}k C , Σ ⇒ Λ \).
	If the principal formula of the rule is \emph{not} \( C \) then the induction hypothesis can be applied directly to its premises and the rule re-applied to derive the desired sequent with correct bounds.
	If, however, the rule is \( \faL \) with \( C \) principal, the above argument does not work. But in this case there is \( γ < β \) and term \( t \) such that 
	\[
	  \prv{γ}k D(t) , C , Σ ⇒ Λ .
	\]
	The induction hypothesis yields
	\[
		\prv{α \nsum γ}k D(t) , Γ , Σ ⇒ Δ , Λ .
	\]
	From the inversion lemma (part \ref{oa-inversion-fa}) I know also that \( \prv{α}k Γ ⇒ Δ , D(t) \). Since \( \rk {D(t)} < \rk{ C} = k \), an application of cut yields 
	\( \prv{ α \nsum β }k Γ, Σ ⇒ Δ, Λ \).
	
	\( C = D → E \). I employ a similar argument as above but with a subtle difference in how the induction hypothesis is applied to account for the binary connectives. 
	By the previous argument I can jump directly to the case that \( C \) is principal in the derivation of \( \prv{β}k C , Σ ⇒ Λ \), for which there exist \( γ , δ < β \) and \( Λ = Λ_0 ∪ Λ_1 \) satisfying
	\begin{enumerate}
		\item \( \prv{γ}k C , Σ ⇒ Λ_0, D \).
		\item \( \prv{δ}k C , E , Σ ⇒ Λ_1 \).
	\end{enumerate}
	I start by applying the inversion lemma to my three hypotheses:
	\begin{enumerate}[resume]
		\item \( \prv{α}k D , Γ ⇒ Δ , E \).
		\item \( \prv{γ}k Σ ⇒ Λ_0, D \).
		\item \( \prv{δ}k E , Σ ⇒ Λ_1 \).
	\end{enumerate}
	Then I apply the induction hypothesis between the sequents in 3 and 5 (using ‘cut’ formula \( E \)):
	\begin{enumerate}[resume]
%		\item \( \prv{α+γ}k Γ , Σ ⇒ Δ ,Λ_0, D \).
		\item \( \prv{α \nsum δ}k D , Γ, Σ ⇒ Δ , Λ_1 \).
	\end{enumerate}
	I can now combine 6 and 3 with a (standard) cut:
	\[
		\prv{α \nsum β}k Γ , Σ ⇒ Δ ,Λ.
	\]
	The conjunction subcase is left to the reader.
	
	Case II: \( C \) is locally positive.
	This case is symmetric to the previous and left to the reader.
\end{proof}

%The observant reader will have noticed that the argument for the implication case could be simplified by applying the inversion lemma
% ------------------
\begin{figure}
	\centering
	\begin{prooftree}
%		\hypo{\prv{α}k Γ ⇒ Δ , C }
		\hypo{\prv{γ}k C , Σ ⇒ Λ_0, D}
		\infer[dashed]1[IL]{\prv{γ}k Σ ⇒ Λ_0, D}
%		\infer[dashed]2[IH]{\prv{α+γ}k Γ , Σ ⇒ Δ, Λ_0, D}
			
		\hypo{\prv{α}k Γ ⇒ Δ , C }
		\infer[dashed]1[IL]{\prv{α}k D , Γ ⇒ Δ , E }
			\hypo{\prv{δ}k C , E , Σ ⇒ Λ_1 }
			\infer[dashed]1[IL]{\prv{δ}k E , Σ ⇒ Λ_1 }
		\infer[dashed]2[IH]{\prv{α \nsum δ}k D , Γ , Σ ⇒ Δ , Λ_1 }
		\infer2[\Cut]{\prv{α \nsum β}k Γ , Σ ⇒ Δ , Λ }
	\end{prooftree}
	\caption{Illustration of the proof method in the reduction lemma for the case \( C = D → E \); IL = ‘inversion lemma’ and IH = ‘induction hypothesis’.}
	\label{f-oa-PAo}
\end{figure}
% ------------------

\begin{exercise}
	Complete the preceding proof.
\end{exercise}

\begin{exercise}
	Formulate and prove a reduction lemma for \( \HAo \) following the proof scheme above.
\end{exercise}

\begin{exercise}
	Give an alternative proof of \cref{oa-red-lem-PAo} using the proof strategy from the reduction lemma for \( \Gc \) (\cref{ce-red-lem-C}).
\end{exercise}

In the implication subcase of case II in the proof above, I used the induction hypothesis to simulate a cut on the formula \( E \)

\begin{theorem}[Reduction theorem for \( \PAo \)]\label{oa-red-thm-PAo}
	If \( \prv {α}{k+1} Γ ⇒ Δ  \) then \( \prv{ω^α}k Γ ⇒ Δ \).
\end{theorem}
\begin{proof}
	Induction on \( α \). If \( Γ ⇒ Δ \) is initial, the claim holds trivially.
	So suppose \( \prv {α}{k+1} Γ ⇒ Δ  \) is derived via a rule
	\[
	  \Infer{Γ_i ⇒ Δ_i \; \text{for } i ∈ I}[\( * \)]{ Γ ⇒ Δ }
	\]
	and for each \( i \) there is \( α_i < α  \) such that \( \prv{α_i}{k+1} Γ_i ⇒ Δ_i \).
	The induction hypothesis implies that \( \prv{ω^{α_i}}{k} Γ_i ⇒ Δ_i \) for each \( i \). So, if \( * \) is not cut, then 
	\[ \prv{ω^α}k Γ⇒ Δ \]
	obtains by re-applying the rule and observing that \( \supof{ ω^η }[η< α] ≤ ω^α \).
	Now suppose that the rule is cut, with cut formula \( C \). If \( \rk C < k \) the same argument as above applies.
	Otherwise \( \rk C = k \) and the reduction lemma is applicable, yielding
	\[
		\prv{ω^{α_0} \nsum ω^{α_1}}k Γ ⇒ Δ 
	\]
	Since \( ω^{α_0} \nsum ω^{α_1} < ω^α  \), the proof is complete.
\end{proof}

The bound in the reduction theorem can be improved fairly easily.
For the give proof strategy to work, it suffices to find an order-preserving function \( f \colon \Ord → \Ord \) such that \( f(α) ≥ \supof{ f(ξ) \nsum f(η) }[ξ,η<α] \).
An obvious candidate is \( f \colon α ↦ 2^α \) (see exercise~\ref{ex-ord-base-2}) and, indeed, \cref{oa-red-lem-PAo} can be strengthened by replacing \( ω^α \) with \( 2^α \).
Certainly, \( 2^α ≤ ω^α \) for all \( α \), so working with this bound seems a significant improvement. 
But given that for every additive principal ordinal \( α ≥ ω^ω \) in fact \( 2^α = ω^α \) (cf.\ exercise~\ref{ex-ord-cnf-2o}), the distinction between exponentiation in the two bases does little in reducing the complexity of cut elimination.

In the next section I will present a strict refinement of the cut elimination theorem in which ordinal exponentiation is directly tied to the \emph{quantifier} rank of the cut formula rather than the full rank.

Let \( ω_0^α ≔ α \) and \( ω_{k+1}^α ≔ ω^{ω_k^α} \).

\begin{theorem}[Cut elimination theorem]\label{oa-ce-PAo}
	If \( \prv{α}k Γ⇒ Δ \) then \( \prv{ω_k^α}0 Γ⇒ Δ \).
\end{theorem}
\begin{proof}
	Consequence of \cref{oa-red-thm-PAo}.
\end{proof}

\begin{exercise}
	Formulate and prove a corresponding reduction lemma and cut elimination theorem for \( \HAo \).
\end{exercise}

\begin{theorem}[Embedding theorem]
	\label{oa-embed-PA-ce}
	If \( \PA ⊢ Γ ⇒ Δ \) and this a closed sequent, then there exists \( α < ε_0 \) such that
	\[
		\PAo \prv{α}0 Γ ⇒ Δ.
	\]
	In addition, \( α \) is effectively computable from the given \( \PA \)-proof.
\end{theorem}
%
\begin{proof}
	Suppose \( \PA ⊢ Γ ⇒ Δ \).
	By the embedding lemma (\cref{oa-embed-PAo-w-bounds}) there is are \( n, k \) such that 
	\[
	  \PAo \prv{ω.n}k Γ ⇒ Δ .
	\]
	Let \( α = ω_k^{ω.n} \). Then \( α < ε_0 \) (by \cref{d-epsilon0}) and
	\[
	  \PAo \prv{α}0 Γ ⇒ Δ 
	\]
	by \cref{oa-ce-PAo}.
\end{proof}

On the basis of cut elimination, a few observations can be already made.

\begin{corollary}
	\label{PA-consis-weak}
	\( \PA \) and, hence, \( \HA \), are consistent.
\end{corollary}
%
\begin{proof}
	There can be no cut-free proof of the empty sequent.
\end{proof}

An inspection of the various proofs leading up to \cref{PA-consis-weak} can strengthen the result by clarifying what mathematical principles suffice to derive the consistency of arithmetic.
%
\begin{corollary}
	\label{PA-consis}
	Consistency of \( \PA \) can be deduced using only finitary reasoning plus the principle of transfinite induction for ordinals \( {≤} ε_0 \).
\end{corollary}
%
By ‘finitary reasoning’ I mean the ‘finite’ mathematics that can be carried out using only finite objects (such as natural numbers) and primitive recursive functions.
Examples include deciding whether one formula is a subformula of another, whether a given primitive recursive function enumerates the premises of an \( ω \)-rule (or Gödel codes of sequents) and what the concluding sequent is.
It is beyond the scope of these lecture notes to attempt to make the statement more precise, but the following proof ‘sketch’ hopefully elucidates how this could be achieved and proven.

\begin{proof}[sketch]
	Suppose there is a finite \( \PA \)-proof of the empty sequent. The embedding of \( \PA \) in \( \PAo \) (\cref{oa-embed-PAo-w-bounds}) provides an explicit number \( n < ω \) such that
	\[
		\PAo \prv{ω.n}n {} ⇒ {} .
	\]
	The existence of a cut-free proof of the empty sequent, along with the various results on which \cref{oa-ce-PAo} depends, can now be established by via finitary reasoning plus transfinite induction up to an ordinal strictly smaller than \( ε_0 \), for instance the ordinal \( ω_{n+2} \) suffices.
	
	As there can be no cut-free proof of the empty sequent, there is no derivation of the empty sequent in \( \PA \).
\end{proof}

\begin{corollary}
	If \( Γ \) is a set of \( Π^0_1 \) sentences and \( Δ \) a set of \( Σ^0_1 \) sentences, then \( \PAo ⊢ Γ ⇒ Δ \) iff there is a cut-free \( \PAo \) derivation of finite height.
\end{corollary}
%
\begin{proof}
	Exercise.
\end{proof}

% ----------------------
\section{Subtheories of Peano arithmetic}
\label{s-oa-ISigma}
% ----------------------

\tbw To cover
\begin{itemize}
	\item PRA
	\item \( \IS_n \)
\end{itemize}

The final part of this chapter will explore some important subtheories of \( \PA \) through the lens of \( ω \)-proofs.
As I will cover only classical theories, I take the opportunity to remove some (classically) definable connectives.

For the purpose of this section, the logical connectives are \( ⊥ \), \( ∧ \), \( → \) and \( ∀ \). Note, I am including implication rather than primitive negation as a matter of convenience.

The \emph{negation rank} (\emph{n-rank}) of a formula, \( \nrk F \), counts the nesting depth on the negative side of implications:
\begin{align*}
	\nrk{A} &= 0 \quad(\text{$A$ prime})
	&
	\nrk{F ∧ G} &= \max\setof{ \nrk F , \nrk G }
	\\
	\nrk{∀x F(x)} &= \nrk{F(a)}
	&
	\nrk{F → G} &= \max\setof{ \nrk F +1 , \nrk G }
\end{align*}

Compare with the quantifier hierarchy for this language fragment.

\begin{definition}
	The \( Π_{n}^P \) formulas (the \( ^P \) expresses that the predicate \( P \) is permitted, in contrast to our earlier definition of \( Π_1 \)) is the smallest set of formulas that contains
	\begin{enumerate}
		\item all prime formulas
		\item \( F ∧ G \) if \( F , G ∈ Π_{n}^P \),
		\item \( ∀x F(x) \) if \( F(a) ∈ Π_{n}^P \) and \( n > 0 \),
		\item \( F → G \) if \( F ∈ Π_{n-1}^P ∪ Π_0^P \) and \( G ∈ \Pi_{n}^P \).
	\end{enumerate}
\end{definition}

Since \( Π_0^P \) is closed under negation, there is no a priori bound on the negation depth of formulas in \( Π_n^P \).

\begin{lemma}
	Every \( Π_n^P \) formula is equivalent, over weak arithmetic, to a formula with n-rank \( n+1 \).
\end{lemma}
By \emph{weak arithmetic}, I have in mind the theory known as Robinson's \( Q \) (see~\cite[ch.~18]{LogThe}) but primitive recursive arithmetic or, equivalently, the theory known as ‘$\IS_1$’ suffices.

\begin{proof}
	Using the full logical language, express \( F ∈ Π_n^P \) in prefix normal form as
	\begin{gather}
		\label{oa-eqn-PNF}\tag{\dag}
		∀\vec x_1 ∃\vec x_2 ⋯ ∃ \vec x_{m} G
	\end{gather}
	where \( G \) is quantifier-free.
	We can choose \( m ∈ \setof{ n , n+1 } \) and \( G \) can be assumed to be \( Π_0^P \) because over weak arithmetic a disjunction \( E ∨ F \) is provably equivalent to 
	\[ 
		∃x ∃ y ( x + y = \suc \0 ∧ ( x = \0 → E ) ∧ ( x = \suc\0 → F )).
	\]
	and the leading existential quantifiers can be incorporated into the '$∃\vec x_m$' sequence of quantifiers.
	It is now straightforward to witness \eqref{oa-eqn-PNF} as an equivalent \( Π_{m}^P \)-formula whose negation rank is \( m \).
\end{proof}


I will jump the gun somewhat now.
%
A sequent is an expression \( Γ ⇒ Δ \) without free variables using the logical language isolated at the beginning of this section.
Let \( \prv{α}k Γ ⇒ Δ \) denote derivability in \( \PAo \) for such sequents in the usual way but with a more liberal cut rule bounded by negation rank:
\[
  \begin{prooftree}
	\hypo{\prv {α} k Γ ⇒ Δ , C }\hypo{ \prv{β} k C, Σ ⇒ Λ }
	\infer2[\Cut]{ \prv{γ}k Γ , Σ ⇒ Δ , Λ }
\end{prooftree}
\quad\text{for \( \nrk C < k \) and \( \max\setof{α,β} < γ \).}
\]

%The complication with adopting the above cut rule is that all transformations on proofs so far considered ‘reduce’
I assume this variation of \( \PAo \) satisfies weakening, substitution and inversion lemmas with the same bounds.
%
Given a finite sequence \( \vec A = (A_i)_{i≤k} \) of formulas, I will write \( \vec A , Γ ⇒ Δ \) for \( A_0 , …, A_n , Γ ⇒ Δ \).

%
\begin{lemma}[Refined reduction lemma]
	Suppose \( \prv{α}k Γ_i ⇒ Δ_i, C_i \) and \( \nrk {C_i} = k \) for each \( i ≤ n \). If \( \prv{β}k \vec C , Σ ⇒ Λ \), then
	\begin{gather}
		\label{oa-eqn-spec-red-lem}\tag{\dag}
		\prv{α + β} k Γ_0 , …, Γ_n , Σ ⇒ Δ_0 , …, Δ_n , Λ .
	\end{gather}
\end{lemma}
%
Applications of the refined reduction lemma, however, will also be to simulate an ordinary two-premise cut rule.
The significance of allowing multiple premises is hidden in the proof.
The multi-premise ‘cut’ can be visualised as either a generalisation of the binary cut rule:
\begin{prooftree*}
  \hypo{ Γ_0 ⇒ Δ_0, C_0 }
  \hypod 
  \hypo{ Γ_n ⇒ Δ_n, C_n }
  \hypo{ \vec C , Σ ⇒ Λ }
  \infer4[$n$-\Cut]{ Γ_0 , …, Γ_n , Σ ⇒ Δ_0 , …, Δ_n , Λ }
\end{prooftree*}
Or as sequence of binary cuts
\begin{prooftree*}
  \hypo{ Γ_n ⇒ Δ_n, C_n }
  \hypo{ Γ_1 ⇒ Δ_1, C_1 }
  \hypo{ Γ_0 ⇒ Δ_0, C_0 }
  \hypo{ \vec C , Σ ⇒ Λ }
  \infer2[\Cut]{ C_1 , …, C_{n} , Γ_0 , Σ ⇒ Δ_0 , Λ }
  \infer2[\Cut]{ }
  \ellipsis{}{ C_n, Γ_0 , …, Γ_{n-1} , Σ ⇒ Δ_0 , …, Δ_{n-1} , Λ }
  \infer2[\Cut]{ Γ_0 , …, Γ_n , Σ ⇒ Δ_0 , …, Δ_n , Λ }
\end{prooftree*}
The advantage of the former presentation is that the ‘size’ of the derivation does not depend on the order of the sequents.
This allows the application of an induction hypothesis in cases that the binary cut view grows too large.

%
\begin{proof}
	The overall structure of the proof will be recognisable as the strategy used in the proof of \cref{oa-red-lem-PAo}.
	I proceed by induction on \( β \).
	Suppose
	\begin{enumerate}
		\item \( \prv{α}k Γ_i ⇒ Δ_i, C_i \) and \( \nrk {C_i} = k \) for each \( i ≤ n \), and
		\item \( \prv{β}k \vec C , Σ ⇒ Λ \).
	\end{enumerate}
	I refer to \( \vec C \) as the \emph{cut} formulas.
	First, suppose no cut formula is principle in the final rule of assumption 2.
	If the sequent is initial, then \( Σ ⇒ Λ \) is initial and \eqref{oa-eqn-spec-red-lem} follows by weakening.
	Therefore, assume \( C_n \) is the principal formula in 2.
	There is a case distinction based on the form of \( C_n \).
	The focus will therefore be on assumption 2 above and
	\begin{gather}
		\label{oa-eqn-spec-red-lem-2}\tag{\ddag}
		\prv{α } k  Γ_n ⇒ Δ_n , C_n .
	\end{gather}

	If \( C_n = ⊥ \) or is a false equation then \eqref{oa-eqn-spec-red-lem} results from applying the inversion lemma to \eqref{oa-eqn-spec-red-lem-2}.
	If \( C_n = P s \), then \( P t ∈ Λ \) for some \( ℕ ⊨ s = t \) and \eqref{oa-eqn-spec-red-lem} also follows from \eqref{oa-eqn-spec-red-lem-2} via substitution.
	The final case is that \( C_n \) is a true equation. But it is not possible for such an atomic formula to be principal in \eqref{oa-eqn-spec-red-lem}.
	
	Moving on to the non-atomic case suppose, to begin, that \( C_n = D ∧ E \).
	From 2 I obtain \( γ < β \) and \( F ∈ \setof{ D, E} \) such that
	\begin{enumerate}[resume]
%		\label{oa-eqn-spec-red-lem}\tag{\dag}
		\item \( \prv{γ} k \vec C , F , Σ ⇒  Λ  \).
	\end{enumerate}
	Applying the inversion lemma to \eqref{oa-eqn-spec-red-lem-2} yields
	\begin{enumerate}[resume]
%		\label{oa-eqn-spec-red-lem}\tag{\dag}
		\item \( \prv{α } k  Γ_n ⇒ Δ_n , F  \).
	\end{enumerate}
	Adding this final sequent to the list of hypotheses in 1 above, and using 3 in place of 2, I can apply the induction hypothesis (as \( γ < β \)), which derives \eqref{oa-eqn-spec-red-lem}.
	
	The quantifier case, \( C_n = ∀x D(x) \) is essentially the same argument.
	From principality of \( C_n \) and the inversion lemma I know
	\begin{enumerate}[start=3,label=\arabic*'.]
		\item \( \prv{γ} k \vec C , D(s) , Σ ⇒  Λ \) for some \( γ < β \) and term \( s \).
		\item \( \prv{α } k  Γ_n ⇒ Δ_n , D(s) \).
	\end{enumerate}	
	I can then deduce \eqref{oa-eqn-spec-red-lem} from the induction hypothesis by adding 4' to the list in 1 and 3' in place of 2.
	
	The final case is involves a different in the argument.
	Suppose \( C_n = D → E \).
	Hypothesis 2 and the inversion lemma yields three derivations to work from:
	\begin{enumerate}[start=3,label=\arabic*''.]
		\item \( \prv{γ}k \vec C , E , Σ ⇒ Λ \),
		\item \( \prv{δ}k \vec C , Σ ⇒ Λ , D \),
		\item \( \prv{α}k D , Γ_n ⇒ Δ_n, E \),
	\end{enumerate}
	for \( γ, δ < β \).
	The first and third of these can be used with the induction hypothesis, obtaining as conclusion,
	\begin{enumerate}[resume,label=\arabic*''.]
		\item \( \prv{α + γ } k D , Γ_0 , …, Γ_n , Σ ⇒ Δ_0 , …, Δ_n , Λ \).
	\end{enumerate}
	To derive \eqref{oa-eqn-spec-red-lem}, I need to remove the formula \( D \) in 6'' I apply a cut against a second application of the induction hypothesis, this time using 4'' (and not expanding the list in 2):
	\begin{enumerate}[resume,label=\arabic*''.]
		\item \( \prv{ α + δ } k Γ_0 , …, Γ_n , Σ ⇒ Δ_0 , …, Δ_n , Λ , D \).
	\end{enumerate}
	As \( \nrk D < k \) a standard cut can be used between sequents 4'' and 6'', the conclusion being \eqref{oa-eqn-spec-red-lem}.
\end{proof}

As the focus is on better bounds on cut elimination, I will switch to base-$2$ exponentiation for the reduction theorem:

\begin{theorem}[Refined reduction theorem]
	Suppose \( \prv{α}{k+1} Γ ⇒ Δ \). Then \( \prv{2^α}{k} Γ ⇒ Δ \).
\end{theorem}
\begin{proof}
	This argument proceeds just as usual.
	Jumping to the main case, suppose \( \prv{α}{k+1} Γ ⇒ Δ \) is derived via cut:
	\[
		\prv{β}{k+1} Γ ⇒ Δ , C
		\qquad
		\prv{γ}{k+1} C, Σ ⇒ Λ 
	\]
	where \( β , γ < α \) and \( \nrk C ≤ k \).
	The induction hypothesis yields
	\[
		\prv{2^β}{k} Γ ⇒ Δ , C
		\qquad
		\prv{2^γ}{k} C, Σ ⇒ Λ 
	\]
	and the refined reduction lemma implies \( \prv{2^α}k Γ ⇒ Δ \).
\end{proof}

\begin{theorem}[Refined cut elimination]
	If \( \prv{α}k Γ ⇒ Δ \) then \( \prv{γ}0 Γ ⇒ Δ \) where \( γ = 2_k^α \).
\end{theorem}

\tbw

\begin{theorem}
	If \( \IS_n ⊢ A \) then \( \PAo \prv{α}0 Γ ⇒ Δ \) for some \( α < ω_{n+1} \).
\end{theorem}
%
\note{Check this.}
%
\begin{proof}[sketch]
	From \( \IS_n ⊢ {} ⇒ A \) we deduce that \( \PA \prv k {n+1} {} ⇒ A \).
	The reason is that the induction rule is only applied to formulas with n-rank \( ≤ n \) and finitary cut elimination is available in \( \PA \) to reduce the cut rank to formulas of the same n-rank as uses of induction.
	The embedding lemma of \( \PA \) into \( \PAo \) yields
	\( \PAo \prv{ω.k}{n} {}⇒ A \), so \( \PAo \prv{γ}0 {}⇒ A \) where
	\[
		γ = 2_{n}^{ω.k} .
	\]
	Recall that \( 2^{ω.k} = ω^k \), whence
	\[
		γ ≤ ω_{n}^{k} < ω_{n+1} .
	\]
\end{proof}



\end{document}
