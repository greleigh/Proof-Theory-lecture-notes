\documentclass[%
	paper=174mm:247mm,
%	11pt,
	DIV=10,
	leqno,
	titlepage,
	headsepline,
	%headings=twolinechapter,
	headsepline=false,
	toc=bib,
	toc=sectionentrywithoutdots,
	toc=chapterentrywithoutdots,
	% enabledeprecatedfontcommands,
	% final,
	unicode
	]%
	{scrbook}
%
%\usepackage{polyglossia}


% Meta
\title{Lecture Notes in Proof Theory}
\author{Graham E.\ Leigh}

\usepackage[utf8]{inputenc}
\usepackage[british]{babel}
\usepackage{xparse}
\usepackage{xcolor}
%
\usepackage[T1]{fontenc}
\usepackage{amsmath,amssymb}
%\usepackage[lining]{libertine}
%\usepackage{pifont}
\usepackage[osf]{newpxtext}
\usepackage[nosymbolsc]{newpxmath}
\usepackage[cal=rsfso]{mathalfa}
\useosf
%
\let\emptyset\emptysetAlt
\let\forall\forallAlt
\let\exists\existsAlt
%
% Packages
\usepackage{graphicx}
%
% Proofs
%\usepackage{ebproof} % ebprooffix for old versions of ebproof.
\usepackage{ebproof}
% Poor man's fix for ebproof
\makeatletter
\@ifpackagelater{ebproof}{2021/01/01}%
	{}%
	{\usepackage{ebprooffix}}
\makeatother
%\IfFileExists{ebprooffix.sty}{\usepackage{ebprooffix}}{} %
%\IfPackageAtLeast{ebproof}{2021/01/01}
%  {%
%    % Do something for the newer version
%  }
%  {%
%    % Do something different for the older version
%    \usepackage{ebprooffix}
%  }%

% Default settings for proof trees
\ebproofset{
	left label template={\normalfont\small (\inserttext)},
	right label template={\small \inserttext},
	rule margin=.5ex,% default .7ex
%	rule separation=1pt,% default 2pt % gap between double-rules
	separation=1.5em,% default 1.5em
	label separation=.5ex,% default .5ex
}
% Small size
\ebproofnewstyle{small}{
  template = \small$\inserttext$,
  left label template = \footnotesize{\inserttext},
  right label template = \footnotesize{\inserttext} }
% Footnote size
\ebproofnewstyle{smaller}{
  template = \footnotesize$\inserttext$,
  left label template = \scriptsize{\inserttext},
  right label template = \scriptsize{\inserttext} }
% Compact
\ebproofnewstyle{compact}{
  separation = 1.5ex, rule margin = .25ex,}
% tight
\ebproofnewstyle{tight}{
  separation = 1em, rule margin = .4ex,
  }
% Some additions to ebproof:
% for lists of premises 
\newcommand{\hypod}{\hypo{\dotsm}}
% axiom
\newcommand{\axiom}[2][]{\hypo{}\infer1[#1]{#2}}
% Sub-proof
\NewDocumentCommand{\subproof}{sO{}m}{%
	\IfBooleanTF{#1}{%
		\hypo{#2}\infer[no rule]1{#3}
	}{%
	\hypo{}\ellipsis{$#2$}{#3}
	}
	}
\providecommand{\subproof}[2][]{\hypo{#1}\infer[no rule]1{#2}}
% File styling.tex
% !TEX root = ../lecture-notes.tex
% LTeX: enabled=false
% Packages and default settings regarding typesetting and format
%---------------------------------
% Page layout assistance
\usepackage{scrlayer-scrpage}
\usepackage[marginparsep=2mm]{geometry}
%---------------------------------
% Theorem environments
\usepackage{mathtools}
%---------------------------------
% Styling for theorems
%
%---------------------------------
% BibLaTeX
%\usepackage[%
%  style=authoryear,% author-year style with extensions
%  backend=biber,%
%  articlein=false,% remove 'in' for journals
%  innamebeforetitle=true,% "in: Editor, Book"
%  dashed=false,% repeated authorship is not dashed
%  maxnames=1000,% don't cut listings short unless we want them too.
%  language=autobib,autolang=hyphen,% Can help with language
%  doi=false,isbn=false,eprint=false,% do not print doi, etc. 
%  ]{biblatex}
%
%---------------------------------
% ELEMENTS
\NewDocumentCommand{\Logic}{m}{\mathsf{#1}}
\NewDocumentCommand{\Theory}{m}{\mathsf{#1}}
\NewDocumentCommand{\Set}{m}{\mathrm{#1}}
\NewDocumentCommand{\Coll}{m}{\mathcal{#1}}
\NewDocumentCommand{\Frml}{m}{\mathsf{#1}} % Why not working?
\NewDocumentCommand{\Symbol}{m}{\mathsf{#1}}
\NewDocumentCommand{\Func}{m}{\mathit{#1}}
\NewDocumentCommand{\Lang}{m}{\mathscr{#1}}
% Rule names
\NewDocumentCommand{\Rule}{m}{\ensuremath{\mathsf{#1}}}
\NewDocumentCommand{\LeftRule}{m}{\Rule{L{#1}}}
\NewDocumentCommand{\RightRule}{m}{\Rule{R{#1}}}
\NewDocumentCommand{\IntroRule}{m}{\Rule{{#1}I}}
\NewDocumentCommand{\ElimRule}{m}{\Rule{{#1}E}}
%---------------------------------
% Disposition
% Part:
\renewcommand*{\partname}{Module} % Does not achieve anything!
\renewcommand*{\partformat}{Module~\thepart\autodot}
%\renewcommand*{\addparttocentry}[2]{%
%  \addtocentrydefault{part}{Module\nobreakspace #1}{#2}%
%	}%
\newcommand\partentrynumberformat[1]{Module~\ #1}
\RedeclareSectionCommand[
  tocentrynumberformat=\partentrynumberformat,
  tocnumwidth=6em,
]{part}

% Chapter:
\renewcommand{\raggedchapter}{\centering}
\setkomafont{chapter}{\sffamily\mdseries\LARGE}
\addtokomafont{chapterentry}{\rmfamily\mdseries}
\addtokomafont{chapterprefix}{\normalsize}

% (Sub)section
\renewcommand{\raggedsection}{\centering}
\setkomafont{section}{\rmfamily\mdseries\sffamily\large}
\setkomafont{subsection}{\centering\rmfamily\mdseries\itshape}

% Paragraph
% - Normal (par-)spacing with emphasis leading
\setkomafont{paragraph}{\rmfamily\mdseries\itshape}
\def\ParIndent{\the\parindent}
\RedeclareSectionCommands[
%	beforeskip=\parsep,
%	indent=\ParIndent,
%	afterskip=1.5ex plus .5ex minus 0.2ex,
  ]{paragraph}
\addtokomafont{paragraph}{\raggedright} 

% Remove reference to chapter from section.
% \RedeclareSectionCommand[%
%   counterwithout=chapter,
%             ]{section}
%
%---------------------------------
% TABLE OF CONTENTS
%

%\usepackage[tocindentauto]{tocbasic}
%\usetocstyle{KOMAlike}
%---------------------------------
% Layout
% Page layouts
\setkomafont{pagehead}{\itshape}
\ohead{\upshape\thepage} % Outer side of pages
\chead{}
\rehead{\leftmark} % Right side of even pages
\lohead{\rightmark} % Left side of odd pages
\ofoot{\small\itshape G.E.~Leigh, version: \today}
\ifoot{}
\cfoot{}
\pagestyle{scrheadings}
%---------------------------------



% Lists
%
\usepackage{enumitem}

% Alter spacing:
\setlist{itemsep=0pt,topsep=\parsep}

% Description format
\setlist[description]{format=\normalfont\scshape}

% Label should be upright
\setlist[enumerate,itemize]{format=\normalfont}
%%   Label #1: roman (i), (ii), etc
%\setlist[enumerate,1]{label=(\roman*)}
%%   Label #2: alph (a), (b), etc
%\setlist[enumerate,2]{label=(\alph*)} 
%
% Special lists





%% 
%% Theorems et al
\usepackage[amsmath,thmmarks,hyperref]{ntheorem}
%% LTeX: enabled=false

% Special lists

% Lists for axioms:
% \begin{axioms}[ax][opts]
% 	\item	
% \end{axioms}
%
\newlist{axiomlist}{enumerate}{1}
\setlist[axiomlist]{format=\scshape ax,label=\arabic*,ref={\scshape ax\arabic*}}
%
\NewDocumentEnvironment{axioms}{O{ax}O{}}%
	{\begin{axiomlist}[format=\scshape #1,ref={\scshape #1\arabic*},#2]}
	{\end{axiomlist}}


% Theorem Styles
% 	- default : [num] [opt] [type] 
%	- nameonly: [num] [opt]
%	- definition  :	[num] [type] · [opt] 
%	- proof   : Proof [opt] 
%	- plain   : [type] [num] · [opt]
%
\makeatletter

% default
\newtheoremstyle{default}
	{\item[\rlap{\vbox{\hbox{\hskip\labelsep \theorem@headerfont ##2\ ##1\theorem@separator}\hbox{\strut}}}]}%
	{\item[\rlap{\vbox{\hbox{\hskip\labelsep \theorem@headerfont ##2\ ##3\ \MakeLowercase{##1}\theorem@separator}\hbox{\strut}}}]}

% default w/o number
\newtheoremstyle{nonumberdefault}%
	{\item[\rlap{\vbox{\hbox{\hskip\labelsep \theorem@headerfont ##1\theorem@separator}\hbox{\strut}}}]}%
	{\item[\rlap{\vbox{\hbox{\hskip\labelsep \theorem@headerfont ##3\theorem@separator}\hbox{\strut}}}]}


% nameonly
\newtheoremstyle{nameonly}
	{\item[\rlap{\vbox{\hbox{\hskip\labelsep \theorem@headerfont ##2\ ##1\theorem@separator}\hbox{\strut}}}]}%
	{\item[\rlap{\vbox{\hbox{\hskip\labelsep \theorem@headerfont ##2\ ##3\theorem@separator}\hbox{\strut}}}]}

% nameonly w/o number
\newtheoremstyle{nonumbernameonly}%
	{\item[\rlap{\vbox{\hbox{\hskip\labelsep \theorem@headerfont ##1\theorem@separator}\hbox{\strut}}}]}%
	{\item[\rlap{\vbox{\hbox{\hskip\labelsep \theorem@headerfont ##3\theorem@separator}\hbox{\strut}}}]}

% convent(ion):
\newtheoremstyle{definition}
	{\item[\rlap{\vbox{\hbox{\hskip\labelsep \theorem@headerfont ##2\ ##1\theorem@separator}\hbox{\strut}}}]}%
	{\item[\rlap{\vbox{\hbox{\hskip\labelsep \theorem@headerfont ##2\ ##1\ \textperiodcentered\ ##3\theorem@separator}\hbox{\strut}}}]}

% convent(ion) w/o number:
\newtheoremstyle{nonumberdefinition}%
	{\item[\rlap{\vbox{\hbox{\hskip\labelsep \theorem@headerfont ##1\theorem@separator}\hbox{\strut}}}]}%
	{\item[\rlap{\vbox{\hbox{\hskip\labelsep \theorem@headerfont ##1\ \textperiodcentered\ ##3\theorem@separator}\hbox{\strut}}}]}

% proof :
\newtheoremstyle{proof}%
  {\item[\hskip\labelsep \theorem@headerfont ##1\theorem@separator]}%
  {\item[\hskip\labelsep \theorem@headerfont ##1\ ##3\theorem@separator]}

% plain 
\renewtheoremstyle{plain}%
    {\item[\hskip\labelsep \theorem@headerfont ##1\ ##2\theorem@separator]}%
    {\item[\hskip\labelsep \theorem@headerfont ##1\ ##2\ \textperiodcentered\ ##3\theorem@separator]}

% plain w/o number:
\renewtheoremstyle{nonumberplain}
    {\item[\hskip\labelsep \theorem@headerfont ##1\theorem@separator]}%
    {\item[\hskip\labelsep \theorem@headerfont ##1\ \textperiodcentered\ ##3\theorem@separator]}

%%%
\makeatother
%
% Global theorem settings:
%\theorempreskip{\parsep}
\theorempostskip{\parsep}

% 'default' presentation
\theoremstyle{default}

\newtheorem {theorem}           {Theorem} [chapter]
\newtheorem {lemma}[theorem]    {Lemma}
\newtheorem {proposition}[theorem] {Proposition}
\newtheorem {claim}[theorem]    {Claim}
\newtheorem {corollary}[theorem]{Corollary}
\newtheorem {example}[theorem]	{Example}

%% W/O numbers
\theoremstyle{definition}
\newtheorem* {remark}{Remark}

% nameonly
\theoremstyle{nameonly}
\newtheorem*{thing}{THIS HAS NO TITLE}

% typed
\theoremstyle{definition}
% upright mode
\theorembodyfont{\upshape}
\theoremsymbol{\ensuremath{\lrcorner}}
\newtheorem{definition}[theorem]{Definition}
\theoremsymbol{}
\newtheorem{convention}[theorem]{Convention}
\newtheorem{exercise}{Exercise}[chapter]


% Proof
\theoremstyle{proof}
\theorembodyfont{\upshape}
% \theoremprework{\arabicenum}
\theoremsymbol{\ensuremath{\dashv}}
\newtheorem {proof} {Proof}

\theoremsymbol{}


% Conventions
%\theoremstyle{definition}
%\theoremnumbering{Alph}
%\newtheorem{convention}{Convention}%[part]

% list (of conventions)
\makeatletter
\def\theoremlistoptname{% 
\let\thm@@thmlstart=\relax 
\let\thm@@thmlend=\relax 
\def\thm@@thmline##1##2##3##4##5{%
	\ifx\empty ##3% 
%	\@dottedtocline{-2}{0em}{0em}%
%		{\hyper@linkstart{link}{##5}{{##1}\ {##2} \protect\numberline{}}\hyper@linkend}% 
%		{{##4}}% 
	\else%
	\@dottedtocline{-2}{0em}{0em}%
		{\hyper@linkstart{link}{##5}{{##3}\protect\numberline{}}\hyper@linkend}% 
		{{##4}}% 
	\fi}}
\makeatother
%
%% END

%
\usepackage{input/unicode-char}
%
% COMMENT THESE TO USE NOTES:
\newcommand{\note}[1]{}
%\newcommand{\tbw}{}
%

%---------------------------------
% Note-making
\NewDocumentCommand{\marginnote}{m}{\marginpar{\scriptsize #1}}
\providecommand{\note}[1]{\marginnote{\textcolor{magenta}{#1}}}
\providecommand{\tbw}{\textcolor{magenta}{To be written.}}
%---------------------------------
% LANGUAGES
\newcommand{\La}{\Lang{L}_{\textsc{a}}}
\newcommand{\Lpra}{\Lang{L}_{\textsc{pra}}}
%---------------------------------
% LOGICS & THEORIES
\newcommand{\Nc}{\Logic{Nc}}
\newcommand{\Ni}{\Logic{N}}
\newcommand{\Nceq}{\Logic{Nc_=}}
\newcommand{\Nieq}{\Logic{N_=}}

\newcommand{\IL}{\Logic{I}}
\newcommand{\CL}{\Logic{C}}

\newcommand{\Gip}{\Logic{IPL}}
\newcommand{\Gcp}{\Logic{CPL}}
\newcommand{\Gmp}{\Logic{ML}}
\newcommand{\Gi}{\Logic{I}}
\newcommand{\Gc}{\Logic{C}}
\newcommand{\Gieq}{\Logic{IL}_=}
\newcommand{\Gceq}{\Logic{CL}_=}

\newcommand{\PA}{\Theory{PA}}
\newcommand{\HA}{\Theory{HA}}
%
\newcommand{\IS}{\Theory{I}\Sigma}
\newcommand{\IP}{\PA}
\newcommand{\PRA}{\Theory{PRA}}
%\newcommand{\HA}{\Theory{HA}}

\newcommand{\PAo}{\Theory{PAω}}
\newcommand{\HAo}{\Theory{HAω}}
\newcommand{\PAop}[1][\prec]{\Theory{PAω}+({#1})}
%---------------------------------
% SEQUENTS
\newcommand{\sa}{\Rightarrow}

\newcommand{\prv}[2]{\vdash^{\smash{#1}}_{\smash{#2}}}
\newcommand{\prvs}[2]{\mathrel{{\vdash}\mathllap{_*}}^{\smash{#1}}_{\smash{#2}}}
%---------------------------------
% INFERENCES
% \Rule, \LeftRule, \RightRule
% are defined in styling.tex
%
\newcommand{\Cut}{\Rule{cut}}
\newcommand{\mCut}{\Rule{mcut}}
%
\newcommand{\idRule}{\Rule{id}}
\newcommand{\botL}{\LeftRule{⊥}}
%
\newcommand{\faR}{\RightRule{∀}}
\newcommand{\faL}{\LeftRule{∀}}
\newcommand{\exR}{\RightRule{∃}}
\newcommand{\exL}{\LeftRule{∃}}
%
\newcommand{\impR}{\RightRule{→}}
\newcommand{\impL}{\LeftRule{→}}
%
\newcommand{\disjR}{\RightRule{∨}}
\newcommand{\disjL}{\LeftRule{∨}}
%
\newcommand{\conjR}{\RightRule{∧}}
\newcommand{\conjL}{\LeftRule{∧}}
%
% for equality:
\newcommand{\refRule}{\Rule{ref}}
\newcommand{\compRule}{\Rule{comp}}
%
% for arithmetic
\newcommand{\IRule}[1][]{\ensuremath{\Rule{ir}_{#1}}}
\newcommand{\SRule}[1][]{\ensuremath{\Rule{sub}_{#1}}}
%
% for ω-logic:
\newcommand{\omR}{\RightRule{ω}}
\newcommand{\omL}{\LeftRule{ω}}
\newcommand{\eqR}{\RightRule{=}}
\newcommand{\eqL}{\LeftRule{=}}

% Natural deduction

\newcommand{\assump}{\Rule{a}}

\newcommand{\botE}{\ElimRule{⊥}}
\newcommand{\RAA}{\Rule{RAA}}

\newcommand{\conjI}{\IntroRule{∧}}
\newcommand{\conjE}{\ElimRule{∧}}
%
\newcommand{\disjI}{\IntroRule{∨}}
\newcommand{\disjE}{\ElimRule{∨}}
%
\newcommand{\impI}{\IntroRule{→}}
\newcommand{\impE}{\ElimRule{→}}
%
\newcommand{\faI}{\IntroRule{∀}}
\newcommand{\faE}{\ElimRule{∀}}
%
\newcommand{\exI}{\IntroRule{∃}}
\newcommand{\exE}{\ElimRule{∃}}

%
%---------------------------------
% TERMS & FORMULAS
\newcommand{\arity}{\mathop{\mathit{ar}}}

\newcommand{\rk}[1]{\lvert{#1}\rvert}
\newcommand{\nrk}[1]{\lvert{#1}\rvert_*}
\newcommand{\qrk}[1]{\lvert{#1}\rvert_q}

\newcommand{\ID}{\Set{ID}}

\newcommand{\nm}[1]{\underline{#1}}
\newcommand{\cd}[1]{\overline{#1}}
%
\newcommand{\0}{\Symbol{0}}
\newcommand{\suc}{\Symbol{s}}
%\newcommand{\num}[1]{\underline{#1}}

% Under dot
\NewDocumentCommand{\subdot}{m}{\textup{\oalign{$#1$\cr\hfil.\hfil}}}%
\providecommand*{\subdot}[1]{\dot{#1}}
%  \ssubdot (for use in sub/supscript)
\newcommand*\ssubdot[1]{\oalign{${\scriptstyle #1}$\cr\hfil.\hfil}}%

% Shorthands
\newcommand{\Prog}[2][\prec]{\Frml{Prog}_{#1}{#2}}
\newcommand{\TI}[2][\prec]{\Frml{TI}_{#1}({#2})}
%---------------------------------

%---------------------------------
% SETS
\newcommand{\Nat}{\mathbb{N}}

\newcommand{\Child}{\Set{Child}}
\newcommand{\Pow}{\mathscr{P}}

\newcommand{\card}[1]{\lvert{#1}\rvert}

\NewDocumentCommand{\setof}{mo}{\{\,{#1}\IfValueT{#2}{\mid {#2}}\,\}}
\NewDocumentCommand{\Setof}{mo}{\bigl\{\,{#1}\IfValueT{#2}{\bigm| {#2}}\,\bigr\}}

\RenewDocumentCommand{\maxof}{mo}{\max\{\,{#1}\IfValueT{#2}{\mid {#2}}\,\}}

%---------------------------------
% Ordinals
\newcommand{\Ord}{\mathbb{O}}
\newcommand{\AP}{\Set{AP}}
\newcommand{\dom}{\mathop{\mathrm{dom}}}
\newcommand{\NF}{\textsc{nf}}

\newcommand{\nsum}{\mathbin{\#}}

\NewDocumentCommand{\supof}{mo}{\sup\{\,{#1}\IfValueT{#2}{\mid {#2}}\,\}}
\newcommand{\supseq}[1][i]{\sup_i}
%---------------------------------
% Proof-theoretic tools
\NewDocumentCommand{\pto}{m}{\lVert{#1}\rVert}
\NewDocumentCommand{\ot}{O{\prec}}{\lVert{#1}\rVert}
\NewDocumentCommand{\otin}{ O{\prec} m }{\lvert{#2}\rvert_{#1}}

%---------------------------------
% Proof-tree shortcuts
\NewDocumentCommand{\Infer}{omO{}m}{\begin{prooftree}\hypo{#2}\infer1[#3]{#4}\end{prooftree}}


%
\usepackage[bookmarks,unicode]{hyperref}
\makeatletter
\hypersetup{pdftitle={\@title},pdfauthor={\@author}}
\makeatother
\usepackage[nameinlink,noabbrev]{cleveref}
%%
%\let\autoref\cref
%\crefname{assumption}{assumption}{assumptions}
%\crefname{claim}{claim}{claims}
%\crefname{proposition}{proposition}{propositions}
%\crefname{proposition2}{proposition}{propositions}
%\crefname{example}{example}{examples}
%\crefname{exercise}{exercise}{exercises}
%\crefname{definition}{definition}{definitions}
%\crefname{corollary}{corollary}{corollaries}
%\crefname{corollary2}{corollary}{corollaries}
%\crefname{lemma}{lemma}{lemmas}
%\crefname{lemma2}{lemma}{lemmas}
%\crefname{namedparadox}{paradox}{paradoxes}
%\crefname{namedtheorem}{theorem}{theorems}
%\crefname{namedlemma}{lemma}{lemmas}
%\crefname{namedexample}{example}{examples}
%\crefname{namedassumption}{assumption}{assumptions}
%\crefname{principle}{principle}{principles}
%\newcommand{\creflastconjunction}{, and\nobreakspace}

\includeonly{
	module-0/about,
	module-0/intro,
	module-1/sequent-calculus,
	module-1/properties,
	module-2/ce,
	module-2/conseq-of-ce,
	module-2/equality,
	module-2/game,
	module-3/arithmetic,
	module-3/ordinal-interlude,
	module-3/ordinal-analysis,
	module-3/ti-and-pto,
	}

\begin{document}
\frontmatter
\maketitle

%%
\chapter{About this text}
\label{c-about}
%\addcontentsline{toc}{chapter}{\nameref{c-about}}
%
These lecture notes are written to accompany the course \emph{Proof Theory} given to second semester students of the \emph{Master in Logic} at the University of Gothenburg, Sweden.
This means that I assume reader is comfortable with elementary formal logic including propositional and predicate (i.e., first-order) logic and natural deduction.
The first half of the text \emph{Logical Theory} covers the assumed material and more.
The latter half of this course deals with Peano arithmetic and incompleteness.
Although designed to be self-contained, the reader will benefit from a having seen these two topics before (see, again, \emph{Logical Theory}).

\nocite{LogThe}

% TOC
\tableofcontents

\mainmatter

% Intro
%
\chapter{Lend me thy proof}
%

What does a proof tell about a theorem beyond its truth?
If the theorem states the existence of an object to what extent does the proof isolate the object in mind?
The reader will be familiar with the classical logic and the method of ‘proof by contradiction’ --- also known by the Latin phrase \emph{reductio ad absurdum} --- whereby an existential claim can be established by showing the negative \emph{universal} claim to be contradictory.
The mere statement of a theorem does not determine whether such method of proof is used or necessary.
One proof of a theorem may directly construct a witness.
Another may invoke only indirect reasoning but, perhaps, rely on fewer assumptions.
A third proof might be too complex to determine; it might, for instance, appeal to lemmas whose proofs you do not have access to.
And only a characterisation of the mathematical theories in which the theorem holds can answer the \emph{real} question: Can the theorem be proved \emph{only} by indirect methods?

With logic in mind, other questions also stand out.
How \emph{complex} is logic? 
For that matter, what does it mean to say that one logic --- or even one \emph{proof} --- is more complex than another?
Neither question can be given a definite answer, but we can get a handle on them by studying, comparing and manipulating proofs.
In these lecture notes
I will show, for example, that every classically valid formula can be given a proof in which only subformulas of the conclusion are used.
Such a proof will not, in general, be the shortest such proof nor the most concise.
But it is the \emph{simplest} in one concrete sense: it does not reference any concepts more complex than the one being proved.
%Is there an algorithm that given a formula returns a proof of the formula if one exists?

The reader will also be shown situations of the opposite kind: an example of a mathematical theorem admitting an elementary proof but for which every proof necessarily refers to concepts \emph{more} complex than the conclusion.
No doubt you will have encountered such cases before although you may not have realised at the time: the scenario is arithmetic and the theorem one of many examples whose proofs (in the language of arithmetic) necessitate a stronger induction invariant than the theorem itself.

%Arithmetic, of course, 

On the topic of arithmetic, I assume you won't deny me the consistency of \emph{Peano} arithmetic, the first-order theory axiomatised by the defining equations for functions of successor, addition and multiplication, plus the axiom schema of induction.
One need only observe that each axiom is a true statement about the natural numbers, that is, that the structure of the natural numbers and its elementary functions forms a model of the Peano axioms.
But the standard model of arithmetic is overkill for the purpose of consistency of the Peano axioms.
Gödel's incompleteness theorem presents statements in the language of arithmetic that are true yet \emph{not} provable from the Peano axioms.
So what mathematical assumptions truly underpin the consistency of Peano arithmetic and, for that matter, other mathematical theories?
And thinking of \emph{different} theories, 
can the deductive \emph{power} of a theory be measured, so that one theory can be directly compared to another?

This, in a nutshell, is
\emph{Proof Theory}: the mathematical theory of formal proofs and, by extension, the mathematical theory of mathematical proofs.
%
And through the course of this text you, dear reader, will see  for yourself the delights and delicacies that only a proof conceals.
Together we will taste the sweetness of the topping, break through its smooth crust and sample the richness beneath.
%Some will be sitting on top for all to see, others

But the proof of the pudding is in the eating.
%
I hope you are hungry.


%
\part{Two Calculi for Two Logics}
%
% --------------------------------------
\chapter{The sequent calculus}
% --------------------------------------


%
Some content

\begin{convention}[The rule \( \impL \) in sequent calculi]
	The rule \( \impL \) in classical or intuitionistic contexts.
\end{convention}

\bigskip
Negation translation magic (exercise)
%
\chapter{Properties of the sequent calculus}
%
Some content

%
\part{Cut elimination}
%
%---------------------------------
\chapter{Cut elimination}\label{c-cut-elim}
%---------------------------------
Here we present cut elimination for the calculi.

Cut rank,
Inversion lemma and the like

\begin{lemma}[First inversion lemma]\label{ce-inversion-lemma}
	Let \( ⊢ \) denoted provability in either \( \Gc \) or \( \Gi \).
	The following hold for all sequents and all \( n \), \( k \):
	\begin{enumerate}
		\item If \( \prv n k Γ ⇒ Δ , ⊥ \) then \( \prv n k Γ ⇒ Δ \).
		\item If \( \prv n k Γ ⇒ Δ , F ∧ G \) then \( \prv n k Γ ⇒ Δ , F \) and \( \prv n k Γ ⇒ Δ , G \).
		\item If \( \prv n k F ∧ G , Γ ⇒ Δ \) then \( \prv n k F , G , Γ ⇒ Δ \).
		\item If \( \prv n k F ∨ G , Γ ⇒ Δ \) then \( \prv n k F , Γ ⇒ Δ \) and \( \prv n k G , Γ ⇒ Δ \).
		\item If \( \prv n k Γ ⇒ Δ , F → G \) then \( \prv n k F , Γ ⇒ Δ , G \).
		\item If \( \prv n k Γ ⇒ Δ , ∀x F(x) \) then \( \prv n k Γ ⇒ Δ , F(s) \) for every term \( s \).
		\item If \( \prv n k ∃x F(x) , Γ ⇒ Δ \) then \( \prv n k F(s) , Γ ⇒ Δ \) for every term \( s \).
	\end{enumerate}
\end{lemma}

\begin{lemma}[Second inversion lemma]\ 
	\begin{enumerate}
		\item If \( \Gc \prv n k Γ ⇒ Δ , F ∨ G \) then \( \Gc \prv n k Γ ⇒ Δ , F , G \).
		\item If \( \Gc \prv n k F → G , Γ ⇒ Δ \) then \( \Gc \prv n k G ,Γ ⇒ Δ \) and \( \Gc \prv n k Γ ⇒ Δ , F \).
	\end{enumerate}
\end{lemma}

\begin{lemma}[Third inversion lemma]
	If \( \Gi \prv n k F → G , Γ ⇒ Δ \) then \( \Gi \prv n k G , Γ ⇒ Δ \).
\end{lemma}

% ---
\section{For classical logic}

\begin{lemma}[Reduction lemma for \( \Gc \)]
	\label{ce-red-lem-C}
	Suppose \( \prv m k Γ ⇒ Δ , C \) and \( \prv n k C , Σ ⇒ Λ  \).
	If \( \rk C = k \) then \( \prv {m+n}k Γ,Σ⇒ D,Λ  \).
\end{lemma}

\begin{exercise}
	\label{ex-red-lem-special}
	In this exercise you will prove a strengthening of the reduction lemma and, as a consequence, obtain more precise bounds on the cost of cut elimination in classical logic.
	
	See Canvas assign 4.
\end{exercise}


% ---
\section{For intuitionistic logic}
%
\chapter{Consequences of cut elimination}
\label{c-ce-conseq}
%
Now we are getting somewhere

\bigskip

Interpolation theorem -- Exercise.

Harrop's theorem
%
\chapter{Predicate logic with equality}
\label{c-equality}
%
We haven't treated equality (yet).

% ---
\section{Equality in natural deduction}

\( \Nceq \) \& \( \Nieq \).

% ---
\section{Equality in sequent calculus}

\( \Gieq \) \& \( \Gceq \).

% ---
\section{Cut elimination with equality}

Hmm!
% -------------------------------
\chapter{A game of cut and mouse}
% -------------------------------

Shall we? Or should this be for inf PA?
%
\part{An Introduction to Ordinal Analysis}
%
%
\chapter{Arithmetic and Sequent Calculi}\label{c-oa-arith}
%
As an application of proof theory beyond logic I will give an analysis of perhaps the most important formal theory in mathematics, the theory of Peano arithmetic.
Among the results we present is a syntactic characterisation of the theorems of the theory, and a proof of its consistency which does not invoke any semantic considerations.
A corollary of the analysis will be a characterisation of the non-finite mathematical assumptions required to establish the consistency of Peano arithmetic.

The proof I present has its origins in Gentzen's 1938 consistency proof\nocite{Gen-1938} but employs a simplification due to Kurt Schütte (1950)\nocite{Schu1950} whereby arithmetic is treated as a fragment of infinitary logic and a corresponding infinitary notion of sequent calculus proof.

Elementary results about this theory covered in the pre-requisite course \emph{Logical Theory} will be stated without proof; see corresponding chapters of \cite{LogThe} for details.

%---------------------------------
\section{Peano and Heyting arithmetic}\label{s-oa-arithmetic}
%---------------------------------
%
\begin{definition}
	The \emph{language of arithmetic} is the first-order language \( \La \) comprising the following nonlogical symbols with associated arities:
	\begin{enumerate}
		\item function symbols: \( \0^0 \), \( \suc^1 \), \( +^2 \), \( ×^2 \).
		\item predicates: \( P^1 \).
	\end{enumerate}
\end{definition}
%
The theory of arithmetic is formulated over predicate logic \emph{with equality}. 
I will first present theory with logic given by the classical natural deduction calculus with equality \( \NDceq \) before presenting an equivalent presentation based on the sequent calculus \( \Gceq \).
Both logics were introduced in \cref{c-equality}; see also~\cite[§6.5]{Negri_von_Plato}.

Henceforth, \emph{formula} will always refer to the language of arithmetic.
%
\begin{definition}
	The Peano axioms of arithmetic are the following sentences.
	\begin{itemize}
		\item \emph{Basic axioms:}
		\begin{axioms}[pa]
			\item \( ∀ x ¬ ( \0 = \suc x ) \)
			\item \( ∀ x ∀ y ( \suc x = \suc y → x = y ) \)
			\item \( ∀ x ( x + \0 = x ) \)
			\item \( ∀ x ∀ y ( x + \suc y = \suc ( x + y ) ) \)
			\item \( ∀ x ( x × \0 = \0 ) \)
			\item \( ∀ x ∀ y ( x × \suc y = ( x × y ) + x ) \)
		\end{axioms}
		\item \emph{Axiom scheme of induction:}
		\begin{axioms}[pa][start=7]
			\item The universal closure of \( A(\0) ∧ ∀x ( A(x) → A(\suc x) ) → ∀x A(x) \) for every formula \( A(a) \).
		\end{axioms}
	\end{itemize}
\end{definition}
%
\begin{definition}
	\emph{Peano arithmetic} (\( \PA \)) is theory over classical predicate logic axiomatised by the Peano axioms.
	I will write \( \PA \vdash A \) to express that \( A \) is a theorem of Peano arithmetic, that is, \( \Set{PA} ⊢_\NDceq A \) where \( \Set{PA} \) is the set of Peano axioms.
	\emph{Heyting arithmetic} (\( \HA \)) is the corresponding \emph{intuitionistic} theory, i.e., \( \HA ⊢ A \) expresses \( \Set{PA} ⊢_\NDieq A \).
\end{definition}
%
%\( \HA \vdash A \) means that \( A \) is a theorem of Heyting arithmetic.
%Observe that the theories above are specified over predicate logic \emph{with equality}.

The predicate \( P \) is auxiliary to the language of arithmetic in that it has no intended interpretation associated to it. 
It plays the role of a ‘free’ predicate as the next \namecref{oa-arith-P} demonstrates.

For a formulas \( A \) and \( B(a) \) in the language of arithmetic, let \( A[B/P] \) mark the result of replacing each occurrence of \( Ps \) in \( A \) by \( B(s) \) for every term \( s \).
That is, 
\begin{align*}
	(Ps)[B/P] &= B(s)
	\\
	A[B/P] &= A \quad\text{for \( A \) any other atomic formula}
	\\
	(A_0 → A_1)[B/P] &= ( A_0[B/P] → A_1[B/P] )
	\\
	(∀x A)[B/P] &= ∀ x( A[B/P] )
	\\
	&\text{etc}
\end{align*}

The following result is easy to prove.

\begin{proposition}
	\label{oa-arith-P}
	If \( \PA \vdash A \) then \( \PA \vdash A[B/P] \) for every formula \( B(a) \).
	Likewise for \( \HA \).
\end{proposition}
%
\begin{proof}
	Exercise.
\end{proof}

\begin{exercise}
	Show the following are theorems of Heyting arithmetic.
	\begin{enumerate}
		\item \( ∀ x( ¬ x = \0 → ∃ y( x = \suc y )) \).
		\item \( ∀ x ∀ y ( x + y = y + x ) \).
		\item \( ∀ x ∀ y ∀ z ( ( x + y ) + z = x + ( y + z ) ) \).
		\item \( ∀ x ∀ y ( x × y = y × x ) \).
	\end{enumerate}
\end{exercise}

As well as some basics of the theory of arithmetic, we recall the primitive recursive representation theorem. See \cite{LogThe}, for example, for details.
%
\begin{definition}
	A formula is \( Δ_0 \) if it can be constructed from atomic formulas excluding \( P \) by the propositional connectives and bounded quantifiers.
	That is, the \( Δ_0 \) formulas forms the smallest collection of \( \La \)-formulas satisfying:
	\begin{enumerate}
		\item all equations \( s = t \) are \( Δ_0 \) formulas,
		\item \( ⊥ \) is a \( Δ_0 \) formula,
		\item if \( F \) and \( G \) are \( Δ_0 \), then so is \( F → G \), \( F ∨ G \) and \( F ∧ G \),
		\item if \( F(a) \) is \( Δ_0 \) and \( s \) is a term, then \( ∀x < s\, F(x) \) and \( ∃x < s \, F(x) \) are \( Δ_0 \), where these formulas are shorthands for \( ∀x ( x< s → F(x) ) \) and \( ∃x ( x < s ∧ F(x) ) \) respectively.
%		\item if \( A \) and \( B \) are \( Σ_1 \) and, in addition, \( A \) does not contain the existential quantifier, then \( A → B \) is \( Σ_1 \).
	\end{enumerate}
	A formula is \( Σ_1 \) (\( Π_1 \)) if it has the form \( ∃x F(x) \) (respectively \( ∀x F(x) \)) where \( F(a) \) is \( Δ_0 \).
\end{definition}

Notice that the bound variable \( x \) does not occur in the ‘bounding’ term \( s \) in the construction \( ∀x < s\, F(x) \) above because terms do not contain bound variables.

Terms of the specific form \( \suc ⋯ \suc \0 \) are called \emph{numerals}.
The numeral evaluating to \( n ∈ ℕ \) is denoted \( \nm n \):
\[
	\nm n ≔ \underbrace{\suc ⋯ \suc}_n \0.
\]

I state the representation theorem for primitive recursive relations.
%
\begin{theorem}[Representation theorem]
	\label{representation-thm}
	Let \( R ⊆ \Nat^n \) be an \( k \)-ary relation on natural numbers. 
	If \( R \) is primitive recursive there exists a \( Δ_0 \) formula \( F_R(a_1, …, a_k ) \) of \( \La \) with at most the displayed variables free such that for all \( n_1, …, n_k ∈ \Nat \),
	\begin{align*}
		\PA ⊢ F_R(\nm n_1 , …, \nm n_k ) \quad &\text{iff}\quad (n_1 , …, n_k ) ∈ R
		\\
		\PA ⊢ ¬ F_R(\nm n_1 , …, \nm n_k ) \quad &\text{iff}\quad (n_1 , …, n_k ) ∉ R.
	\end{align*}
\end{theorem}
%

%---------------------------------
\section{Sequent calculi for arithmetic}\label{s-oa-omega-logic}
%---------------------------------

There are different ways to formulate arithmetic in sequent calculi.
One can incorporate all the axioms of arithmetic as initial sequents or treat each Peano axiom as contributing a rule of the calculus.
A convenient definition is the following.
\( \PA ⊢ Γ ⇒ Δ \) means that the sequent \( Γ ⇒ Δ \) has a derivation in the sequent calculus \( \Gceq \) expanded by:
\begin{itemize}
	\item Initial sequents \( Π ⇒ Σ , A \) for \( A \) a basic Peano axiom.
	\item The \emph{induction rule}:
	\[
	  \begin{prooftree}
	  	\hypo{ A(a) , Π ⇒ Σ , A( \suc a ) }
	  	\infer1[\IRule]{A(\0) , Π ⇒ Σ , ∀ x A(x) }
	  \end{prooftree}
	\]
	where \( a \) does not occur in the lower sequent.
\end{itemize}
The sequent calculus for Heyting arithmetic is the restriction to intuitionistic sequents. 
Recall, a sequent \( Γ ⇒ Δ \) is \emph{intuitionistic} if \( \card{Δ} = 1 \).
Define \( \HA ⊢ Γ ⇒ Δ \) as there exists a sequent calculus derivation witnessing \( \PA ⊢ Γ ⇒ Δ \) using only intuitionistic sequents.


\begin{proposition}
	\label{oa-PA-as-SC}\ 
	\begin{enumerate}
		\item \( \PA ⊢ A \) iff \( \PA ⊢ {} ⇒ A \).
		\item \( \HA ⊢ A \) iff \( \HA ⊢ {} ⇒ A \).
	\end{enumerate}
\end{proposition}
%
\begin{proof}
	I will show that every induction axiom admits a sequent calculus proof. The remainder of the proof is left as an exercise.
	
	Fix a formula \( F(a) \) and let \( Γ = \setof{ F(\0) , ∀x( F(x) → F( \suc x) ) } \).
	By logic, a derivation of \( F(a) , Γ ⇒ F( \suc a )  \) is readily obtained. An application of the induction rule, followed by more logic completes the derivation:
	\begin{prooftree*}
		\subproof{ F(a), Γ ⇒ F(a) }
		\subproof{ F( \suc a), Γ ⇒ F(\suc a)}
		\infer2[\impL]{ F(a) → F( \suc a ) , F(a), Γ ⇒ F(\suc a) }
		\infer1[\faL]{ F(a) , Γ ⇒ F(\suc a) }
		\infer1[\IRule]{ Γ ⇒ ∀ x F(x) }
%		\infer[double]1[\conjL, \impR]{ ⇒ F(\0) ∧ G → ∀x F(x)}
	\end{prooftree*}
\end{proof}

\begin{exercise}
	Show that \( \PA \) derives the same sequents as the calculus with induction axioms as initial sequents (along with the basic axioms) and no induction rule.
\end{exercise}

By \cref{oa-PA-as-SC}, \( \PA \) is consistent iff the empty sequent is not derivable.
As with the sequent calculi from previous chapters, it is clear that there can be no cut-free derivation of the empty sequent, neither in \( \PA \) nor \( \HA \).
Thus, consistency of either theory would follow directly from a cut-elimination theorem for the above sequent calculi.
%
There are sequents, however, that are provable but not \emph{cut-free} provable. 
We will not present the argument here, which appeals to Gödel's incompleteness theorems;\note{Sketch the argument} the finer details are beyond the scope of this book and can be found in, for example,~\cite{BBJ}.

Gentzen's observation was that every derivable \emph{equational} sequent can be shown to have a cut-free derivation, where an equational sequent is one of the form \( r_1 = s_1 , …, r_k = s_k ⇒ t_1 = u_1 , …, t_l = u_l \) wherein all terms are closed.
As the empty sequent is an example of an equational sequent, consistency is an immediate corollary of the (partial) cut-elimination result.

Gentzen's argument is highly intricate and was greatly streamlined by Kurt Schütte (1950)\nocite{Schu1950} who showed that full cut-elimination can be obtained by moving to a more relaxed notion of a sequent calculus derivation, termed ‘\( ω \)-proofs’, in which proofs are in general infinite objects.
%
The basic idea is to replace the logical rules \( \faR \) and \( \exL \) each by a rule with infinitely many premises:
\begin{gather*}
  \begin{prooftree}
	\hypo{ Γ ⇒ Δ , A(\0) }
	\hypo{ Γ ⇒ Δ , A(\nm 1) }
	\hypod
	\hypo{ Γ ⇒ Δ , A(\nm n) }
	\hypod
	\infer5[\omR]{ Γ ⇒ Δ , ∀x A(x) }
  \end{prooftree}
  \\[2ex]
  \begin{prooftree}
	\hypo{ A(\0) , Γ ⇒ Δ }
	\hypo{ A(\nm 1) , Γ ⇒ Δ }
	\hypod
	\hypo{ A(\nm n) , Γ ⇒ Δ }
	\hypod
	\infer5[\omL]{ ∃x A(x) , Γ ⇒ Δ }
  \end{prooftree}
\end{gather*}
The rules \( \omR \) and \( \omL \) above are collectively called the \( ω \)-rules.

‘Proof’ in the sense of the sequent calculi of previous chapters meant ‘finite tree labelled by sequents in agreement with the rules of the calculus’.
A ‘proof’ that uses an \( ω \)-rule can never be finite as these rules have infinitely many premises.
But the condition ‘finite or infinite tree labelled by sequents in agreement with the rules of the calculus’ is too liberal as it admits as 'proofs' trees with infinitely long branches, such as 
\begin{prooftree*}
	\subproof{ ⇒ ⊥ }
	\axiom[\botL]{ ⊥ ⇒ ⊥ }
	\infer2[\Cut]{ ⇒ ⊥ }
	\axiom[\botL]{ ⊥ ⇒ ⊥ }
	\infer2[\Cut]{ ⇒ ⊥ }
	\axiom[\botL]{ ⊥ ⇒ ⊥ }
	\infer2[\Cut]{ ⇒ ⊥ }
\end{prooftree*}
The answer is that, as in the finite case, there can be no infinite paths in an \( ω \)-proof but, unlike the finite case, the tree underlying an \( ω \)-proof may have infinitely wide branching.
Such trees are called \emph{well-founded}.

In sum, an \emph{\( ω \)-proof} is a well-founded tree that is labelled by sequents in a way consistent with the rules of the sequent calculus (the \( ω \)- and non \( ω \)-rules).
%A sequent derivation which (possibly) use \( ω \)-rules is called an \emph{\( ω \)-proof}.

\begin{proposition}\ 
	\label{oa-oproofs-simple}
	\begin{enumerate}
		\item There is an \( ω \)-proof of every sequent of the form \( F , Γ ⇒ Δ , F \)
		\item If there is an \( ω \)-proof of \( Γ(a) ⇒ Δ(a) \), then for every term \( s \) there is an \( ω \)-proof of \( Γ(s) ⇒ Δ(s) \).
	\end{enumerate}
\end{proposition}
%
\begin{proof}
	Exercise.
\end{proof}
%
\begin{proposition}
	\label{oa-ind-omega}
	The induction rule can be simulated via \( \omega \)-proofs.
\end{proposition}
%
%
\begin{proof}
	Fix a formula \( F(a) \) and as before let \( Γ = \setof{ F(\0), ∀x( F(x) → F( \suc x) ) } \).
	Suppose \( F(a) , Γ ⇒ Δ , F(\suc a) \) admits an \( ω \)-proof. 
	As this is a premise to an induction rule the variable \( a \) does not occur in \( Γ ∪ Δ \).
	By the previous \namecref{oa-oproofs-simple} there is an \( ω \)-proof of the sequent \( F(\nm n) , Γ ⇒ Δ , F(\nm {n+1}) \) for each \( n \).
	A sequence of cuts induces an \( ω \)-proof of \( F(\0) , Γ ⇒ Δ , F(\nm n) \): for \( n = 0,1 \) the claim is immediate. For \( n = m + 1 > 1 \) append the proof of the induction hypothesis by a single cut:
	\begin{prooftree*}
		\subproof{ F(\nm0) , Γ ⇒ Δ , F(\nm m) }
		\subproof{ F(\nm m) , Γ ⇒ Δ , F(\nm n) }
		\infer2[\Cut]{F(\0) , Γ ⇒ Δ , F(\nm n)}
	\end{prooftree*}
	As \( F(\0) , Γ ⇒ Δ , F(\nm n) \) is derivable for each \( n \), an application of \( \omR \) completes the (\( ω \)-)proof.
\end{proof}

\begin{exercise}
	Show that the induction axioms admit \emph{cut-free} \( ω \)-proofs.
\end{exercise}


With the \( ω \)-rules replacing the traditional quantifier rules \( \faR \) and \( \exL \) it turns out that free variables can be completely eliminated from the sequent calculus, meaning that only closed sequents are derived.
This convention serves to simplify much of the reasoning about \( ω \)-proofs.
It is also possible to dispense with the logical rules for equality by adopting more liberal initial sequents.

The next definition introduces both conventions and settles the notion of \( ω \)-proof used hereon.
Observe that it is decidable whether two closed terms \( s \) and \( t \) in the language of arithmetic evaluate to the same natural number. I will write \( ℕ ⊨ s = t \) if this is the case, and \( ℕ ⊭ s = t \) otherwise.
%
\begin{definition}[\( \PAo \) and \( \HAo \)]\label{d-PAomega}
	\( \PAo \) is the sequent calculus given by the following:
	\begin{itemize}
		\item sequents comprise formulas in the language of arithmetic (with equality).
		\item Initial sequents are \emph{closed} sequents of the form
		\begin{itemize}
			\item[(\botL)] \( ⊥, Γ ⇒ Δ \)
			\item[(\idRule)] \( Ps, Γ ⇒ Δ , Pt \) if \( ℕ ⊨ s = t \)
			\item[(\eqR)] \( Γ ⇒ Δ , s = t \) if \( ℕ ⊨ s = t \)
			\item[(\eqL)] \( s = t , Γ ⇒ Δ \) if \( ℕ ⊭ s = t \)
		\end{itemize}
		\item Inference rules are rules of \( \Gc \) but restricted to closed sequents and with \( \faR \) and \( \exL \) replaced by the two \( ω \)-rules:
		\begin{itemize}
			\item[(\omR)] \begin{prooftree} \hypo{ Γ ⇒ Δ , F(\nm n) \ \text{for every } n ∈ ℕ } \infer1{ Γ ⇒ Δ , ∀x F(x) }\end{prooftree}
			\item[(\omL)] \begin{prooftree} \hypo{ F(\nm n) , Γ ⇒ Δ \ \text{for every } n ∈ ℕ } \infer1{ ∃x F(x) , Γ ⇒ Δ }\end{prooftree}
		\end{itemize}
	\end{itemize}
	Writing \( \PAo ⊢ Γ ⇒ Δ \) expresses that there is an \( ω \)-proof of \( Γ ⇒ Δ \) according to the above rules. In other words, there exists a well-founded tree labelled by sequents such that each leaf is an initial sequent and that each inner vertex together with its immediate successors in the tree forms a correct application of a rule of the calculus listed above.
	
	\( \HAo \) is the calculus above restricted to intuitionistic sequents.
\end{definition}
%

With the sequent calculus formally defined, the realisation of finite \( \PA \)-proofs as \( ω \)-proofs can resume.
The first step is to give \( ω \)-proofs of the basic axioms of arithmetic.
%
\begin{proposition}
	\label{oa-embed-basic}
	Every closed initial sequent of \( \PA \) is derivable in \( \PAo \).
%	If \( F \) is a basic axiom of \( \PA \) and \( Γ ⇒ Δ \) any closed sequent, then \( \PAo ⊢ Γ ⇒ Δ , A \).
\end{proposition}
%
\begin{proof}
	Among the sequents to be shown derivable in \( \PA \) are all initial sequents of \( \Gc \) and the basic axioms of \( \PA \).
	I will treat the case of the basic axiom \textsc{pa}1, \( ∀x( ¬ \0 = \suc x ) \).
	Let \( n ∈ ℕ \) be arbitrary. As the equation \( \0 = \suc \nm n \) is false, \( \0 = \suc \nm n , Γ ⇒ Δ , ⊥ \) is an initial sequent of \( \PAo \) for all closed \( Γ , Δ  \).
	Therefore \( \PAo ⊢ Γ ⇒ Δ , ¬ \0 = \suc \nm n \) for every \( n ∈ ℕ \) and \( \PAo ⊢ Γ ⇒ Δ , ∀x( ¬ \0 = \suc x ) \) by \( \omR \).
\end{proof}
%
\begin{exercise}
	Complete the proof of \cref{oa-embed-basic}.
\end{exercise}
%
\begin{exercise}\label{ex:oa-id-simple}
	Show that all closed sequents of the form \( A, Γ ⇒ Δ , A \) are provable in \( \PAo \).
\end{exercise}

%
\begin{lemma}[Embedding lemma]\label{oa-embed-weak}
	Suppose \( \PA ⊢ Γ ⇒ Δ \) and let \( Γ^* ⇒ Δ^* \) be any closed substitution instance of \( Γ ⇒ Δ \) (obtained by substituting closed terms for free variables).
	Then \( \PAo ⊢ Γ ⇒ Δ \).
	Likewise, for Heyting arithmetic and \( \HAo \).
\end{lemma}
%
%
%\begin{proof}
%	Proceed by induction on the height of the (finite) \( \PA \)-proof.
%	\Cref{oa-embed-basic} covers the case of initial sequents as these are closed under substitution (if \( Γ(a) ⇒ Δ(a) \) is an initial sequent of \( \PA \) then so is \( Γ(s) ⇒ Δ(s) \) for every term \( s \)).
%\end{proof}

\begin{exercise}
	Prove the embedding lemma. Do not forget the equality rules implicit in \( \Gceq \).
\end{exercise}

The next task is to analyse \( ω \)-proofs and establish a cut elimination theorem.
Currently lacking, however, is some measure of the \emph{complexity} of an \( ω \)-proof analogous (or, perhaps, generalising) the height of finite sequent calculus proofs.
Although every path through an \( ω \)-proof is, by requirement, finite there are \( ω \)-proofs that admit paths of arbitrary (finite) length.
The \( ω \)-proof described by the proof of \cref{oa-ind-omega} is such an example.
It comprises a single application of an \( ω \)-rule at the root with the premise for the numeral \( n \) being derived by a (finite) sequent proof of height at least \( n \).

Thus the question comes down to how to associate a measure to \( ω \)-proofs such that strict subproofs (i.e., proofs of the premises of the root inference) can be recognised as being ‘smaller’ than the proof itself?
The answer to this conundrum is in the title of this part: \emph{ordinals}.

%Currently we have no means to measure the size of \( \PAo \)-proofs.
%For this we will use ordinals.



%---------------------------------
\chapter{An ordinal interlude}\label{c-oa-ordinals}
%---------------------------------

To present the ordinals it is not necessary to have a set-theoretic definition of ordinals in mind (as, for example, arbitrary transitive sets).
Indeed, there is no need to consider the question of by what ordinals \emph{are} or from what they are \emph{formed}.
For a \emph{theory} of ordinals all that is relevant are the order-theoretic properties satisfied by the ordinals and a selection of operations that can be defined on them.
In short, ordinals are treated analogously to natural numbers: as a posited entity fulfilling specified criteria.
%
The material of this chapter draws from lecture notes by Michael Rathjen~\cite{RathjenLectures}.

%
\begin{definition}\label{d-ordinals}%[Ordinals]
	The \emph{ordinals} is a class \( \Ord \) equipped with a binary relation \( < \) satisfying three postulates, where \( ≤ \) is the reflexive closure of \( < \):
	\begin{axioms}[o]
		\item \( < \) is a strict linear order on \( \Ord \). That is, \( < \) is irreflexive, transitive and linear, where linear means that for all \( α , β ∈ \Ord \) either \( α ≤ β \) or \( β ≤ α \).\label{post-ord-lin}
		\item Every non-empty class of ordinals has a \( < \)-minimal element (necessarily unique by \ref{post-ord-lin}). That is, if \( O ⊆ \Ord \) is non-empty there exists \( ξ ∈ O \) such that \( ξ ≤ α \) for all \( α ∈ O \).\label{post-ord-wo}
		\item For every set \( X \) and function \( f \colon X → \Ord \) there exists \( ξ ∈ \Ord \) such that \( f(x) < ξ \) for every \( x ∈ X \).\label{post-ord-unbdd}
	\end{axioms}
\end{definition}

Set-theoretic concerns do matter in the language used to discuss ordinals.
As, for example, the Burali-Forte paradox shows, it is inconsistent  the Zermelo--Fraenkel (or Cantorian) conception of \emph{set} in mind to consider that the collection of (all) ordinals forms a set.
Hence use of term ‘class’ to refer to arbitrary collections of ordinals/objects and ‘set’ in specific case of \ref{post-ord-unbdd}.
Familiarity with set theory is not necessary for the elementary theory of ordinals presented here.
Indeed, it will suffice to replace every term ‘set’ in what follows by ‘countable set’ and ‘class’ by ‘countable or uncountable set’.
%We have side-stepped this concern by restricting attention to the \emph{countable} ordinals. Over Zermelo set theory, the collection of all countable ordinals, i.e., the collection \( \Ord \) above, forms a set; indeed, it is precisely the first uncountable ordinal.
%Such a restriction is not important for our later use of ordinals.
%Ultimately our attention will be constrained to a relatively small collection of (countable) ordinals.
%Our first lemma confirms that the relation \( <_\Ord \) is a well-order.

In the following, notation \( \setof{ t }[x ∈ X ] \) means the \emph{class} of objects \( t \) as \( x \) ranges over the (class) \( X \).
Usually a function \( f \colon U → V \) between classes has been specified along with a (sub)class \( X ⊆ U \) whence the notation \( \setof{ f(x) }[x ∈ X ] \) expresses the class of objects \( f(x) \) for \( x ∈ X \). This class will be written \( f[X] \).

\begin{convention}
	Lowercase Greek letters \( α \), \( β \), etc.\ stand as metavariables for ordinals.
\end{convention}


\begin{lemma}\label{ord-well-order}
	Postulate \ref{post-ord-wo} is equivalent to the principle of transfinite induction.
	This is the statement that if \( O \) is progressive in the ordinals then \( \Ord ⊆ O \), where \( O \) is progressive means that for all ordinals \( α \), if \( β \in O \) for every \( β < α \) then \( α ∈ O \).
%	Let \( O ⊆ \Ord \) be non-empty.
%	Then \( O \) has a \( <_\Ord \)-minimal element and this is unique.
\end{lemma}
%
\begin{proof}
	Let \( O \) be progressive.
	Consider the class \( C = \Ord \setminus O \) of ordinals not in \( O \). If \( C \) is non-empty then, by \ref{post-ord-wo}, \( C \) contains a least ordinal, \( α \) say.
	As \( α \) is the least ordinal in \( C \), every \( ξ < α \) is element of \( O \). 
	Progressiveness implies that \( α ∈ O \) contradicting that \( α ∈ C \). 
	Hence, \( C \) is the empty class, so \( \Ord ⊆ O \).
	For the converse claim, assume postulate \ref{post-ord-lin} and the principle of transfinite induction (I could also assume \ref{post-ord-unbdd} but this is unnecessary).
	The aim is to establish \ref{post-ord-wo}. 
	Thus, let \( O \) be a non-empty class of ordinals and, for want of a contradiction, assume that \( O \) has no least element. 
	As in the other direction, I consider the complement of \( O \), the class \( C = \Ord \setminus O \).
	Suppose \( α \) be any ordinal such that \( ξ ∈ C \) for all \( ξ < α \). If \( α ∈ O \) then this is the least element of \( O \). As \( O \) has no least element therefore \( α ∈ C \).
	So \( C \) is progressive and \( C = \Ord \) by transfinite induction, contradicting the non-emptiness of \( O \).
\end{proof}

The next \namecref{ord-supremum} provides the primary means to infer the existence of ordinals.

\begin{lemma}
	\label{ord-supremum}\ 
	Let \( O \) be a class of ordinals.
	\begin{enumerate}
		\item There exists a least upper bound of \( O \). That is, an ordinal \( α \) such that \( ξ ≤ α \) for all \( ξ ∈ O \). This \( ξ \) is referred to as the \emph{supremum} of \( O \) and denoted \( \sup O \).
		\item There exists a strict least upper bound of \( O \), i.e., \( α \) such that \( ξ < α \) for all \( ξ ∈ O \).
	\end{enumerate}
	In each case the proclaimed ordinal is unique.
\end{lemma}
%
\begin{proof}
	Begin with 1. Let \( O \) be given.
	Consider the class \( O^≥ \) of all ordinals \( α \) such that \( ξ ≤ α \) for \emph{all} \( ξ ∈ O \).
	The \( < \)-least element of \( O^≥ \) (if such exists) is clearly the desired ordinal.
	But in order to apply postulate \ref{post-ord-wo} to this class it is necessary to establish that \( O^≥ \) is non-empty.
	For this I use the third postulate applied to identity function \( \textsf{id} \colon O → \Ord \colon ξ ↦ ξ \) (which is a function from \( O \) into \( \Ord \)).
	%
	For 2, the same argument works with the class \( O^> \) in place of \( O^≥ \) where this is the class of ordinals \emph{strictly} larger than all elements of \( O \).
	
	Uniqueness of each case is ensured by \ref{post-ord-lin}.
\end{proof}

Henceforth, I will not make explicit reference to the postulates.

The least ordinal is denoted \( 0 \). This happens to be the supremum of the empty set: \( 0 ≔ \sup ∅ \).
Given \( α ∈ \Ord \), the \emph{successor} of \( α \), in symbols \( α' \) or \( α + 1 \), is the least ordinal greater than \( α \), which exists (and is unique) by \cref{ord-supremum}(2) applied to the singleton set \( \setof{α} \).
That is, \( α' \) is such that \( ξ < α' \) iff \( ξ ≤ α  \).
The successor of \( 0 \) is denoted \( 1 ( =0') \), its successor \( 2 ( = 0'' ) \), etc.

A \emph{limit ordinal} is any non-zero ordinal \( λ \) such that \( η' < λ \), for all \( η < λ \).
Define a function \( f\colon ℕ → \Ord \) by \( f(0) = 0 \) and \( f(n+1) = f(n)' \).
That is, \( f(n) \) is the \emph{ordinal} representing the natural \( n \).
The supremum of \( \setof{ n }[n ∈ ℕ] \) is called \( ω \), which is a limit by construction and, therefore, the least limit ordinal.

\begin{lemma}
	\label{ord-suc-lim}
	Every non-zero ordinal is either a successor or a limit.
\end{lemma}

\begin{lemma}
	\label{ord-lim}
	An ordinal \( λ \) is a limit iff \( λ = \sup O \) for some non-empty set \( O \) closed under successor (meaning that \( ξ ∈ O \) implies \( ξ' ∈ O \)).
\end{lemma}

\begin{lemma}\label{ord-supremum-unique}
	Suppose \( O, O' \) are such that for every \( α \in O \) there exists \( β  \in O' \) such that \( α ≤ β \).
	Then \( \sup O ≤ \sup O' \).
\end{lemma}
\begin{exercise}
	Prove \cref{ord-suc-lim} to \ref{ord-supremum-unique}.
\end{exercise}


I will employ common set-theoretic abbreviations such as \( \sup_{i∈ I} α_i \) for \( \supof {α_i}[i ∈ I] \) and \( \sup_{i} α_i \) for \( \supof {α_i}[i < ω ] \).
I will also use \( λ \) as a metavariable for limit ordinals.

%--------------------------------------
\section{Elementary Ordinal Functions}
%--------------------------------------

A \emph{segment} of \( \Ord \) is any class \( O \) of ordinals which is closed downwards, i.e., if \( α < β ∈ O \) then \( α ∈ O \).
If \( X \) and \( Y \) are segments then either \( X ⊆ Y \) or \( Y ⊆ X \); in either case \( X ∩ Y \) is a segment.

Let \( O \) be a segment. A function \( f \colon O → \Ord \) is said to be:
\begin{itemize}
	\item \emph{order preserving} if \( α < β \) implies \( f(α) < f(β) \) for all \( α , β ∈ O \).
	\item \emph{continuous} if for all \( U ⊆ O \), if \( \sup U ∈ O \) then \( f( \sup U ) = \sup f[U] \).
%	\item \emph{normal} if \( O = \Ord \) and \( f \) is order preserving and continuous.
	\item an \emph{enumeration} (of \( X ⊆ \Ord \)) if \( f \) is order-preserving and \( f[O] = X \).
\end{itemize}

The identity function \( \mathsf{id} \colon \Ord → \Ord \) is all of the above. In particular, it is an enumeration of \( \Ord \).
Let \( f \colon ℕ → \Ord \) be given by \( f(0) = ω \) and \( f(n+1) = f(n)' \).
This function is order preserving and continuous (the latter is trivial). 
It is %not normal because the domain of \( f \) is not all ordinals, but it is
also an enumeration of the set \( \setof{ ω , ω' , … } \) because \( ℕ \) is a segment.
Notice that order preserving functions on ordinals are always injective.

\begin{lemma}\label{ord-o-p}
	If \( O \) is a segment and \( f \) is order preserving then \( α ≤ f(α) \) for all \( α ∈ O \).
\end{lemma}
%
\begin{exercise}
	Prove \cref{ord-o-p}.
\end{exercise}

The main property of ordinal functions I need is the summarised by
\begin{lemma}
	\label{ord-normal-exists}
	Every class of ordinals has a unique enumeration. The enumeration of \( Y ⊆ \Ord \) will be denoted \( E_Y \).
\end{lemma}
%
\begin{proof}
	\( E_Y \) is determined as the inverse of a particular function \( C_Y \colon Y → \Ord \), called the \emph{collapsing} function for \( Y \), defined by
	\[
		C_Y(α) = \sup \setof{ C_Y(ξ) + 1 }[ ξ ∈ Y \text{ and } ξ < α ].
	\]
	The collapsing function is clearly unique if it is well-defined. Moreover, \( C_Y \)
	This function is well-defined: Consider the class \( O \) of ordinals \( α \) for which the collapsing function on \( Y_α ≔ Y ∩ \setof{ ξ }[ξ ≤ α] \) exists.
	If \( C_{Y_ξ} \colon Y_ξ → \Ord \) is defined for each \( ξ < α \) I claim that \( C\colon Y_α → \Ord \) defined by 
	\[
		\begin{aligned}
			C(α) &= \sup\setof{ C_{Y_{ξ}}(ξ) + 1 }[ ξ < α \text{ and } ξ ∈ Y]
			\\
			C(ξ) &= C_{Y_ξ}(ξ) \text{ for \( ξ < α \)}
		\end{aligned}
	\]
	is the collapsing function for \( Y_α \).
	That this is the follows almost by definition. Indeed, all that is lacking is the observation that \( C_{Y_ξ}(β) = C_{Y_η}(β) \) whenever \( β ≤ ξ < η \).
	So \( O \) is progressive and transfinite induction implies that class \( Y_α \) has a collapsing function \( C_{Y_α} \). Now define \( C_Y \) as \( α ↦ C_{Y_α}(α) \).

	Clearly, \( C_Y \) is injective. Therefore the function admits a (right) inverse:
	\[
		E_Y ≔ C_Y^{-1} \colon C_Y[Y] → Y 
	\]
	As \( C_Y[Y ] \) is (clearly) a segment, \( E_Y \) is an enumeration of \( Y \).
	
	As to uniqueness of \( E_Y \), let \( O = C_Y[Y] \) and suppose \( f \colon O' → Y \) is any enumeration of \( Y \). In particular, \( O' \) is a segment. Transfinite induction implies that \( f(α) = E_Y(α) \) for all \( α ∈ O ∩ O' \).
	As both functions are injective and surjective into \( Y \) it follows that \( O = O' \).
\end{proof}

Two further properties of enumerations will be useful.

\begin{lemma}
	\label{ord-normal}
	Let \( f\colon \Ord → \Ord \) be continuous and order preserving (in particular, \( f \) is an enumeration of \( f[\Ord] \)).
	Then
	\begin{enumerate}
		\item For every \( α ≥ f(0) \) there is a unique \( β ≤ α \) such that \( f(β) ≤ α < f(β+1) \).\label{ord-normal-cover}
		\item For every \( α \) there is a unique \( β ≥ α \) such that \( β = f(β) \).\label{ord-normal-fix}
	\end{enumerate}
\end{lemma}
\begin{proof}
%	Continuity is implicit in the proof of \cref{ord-normal-exists}.
%	
	\ref{ord-normal-cover}. Consider the set \( O = \setof{ ξ }[ f(ξ) ≤ α ]\) and let \( β = \sup O \).
	Continuity yields
	\[
		f(β) = \sup f[O] = \sup \setof{ f(ξ) }[ f(ξ) ≤ α ] ≤ α
	\]
	whereas
	\(
		f(β+1) > α
	\)
	because \( β + 1 ∉ O \).
	
	\ref{ord-normal-fix}. Fix \( α \) and define \( O = \setof{ f(α) , f(f(α)) , …, f^n(α) , … } \) (arbitrary finite iterations of \( f \) on \( α \)). 
	Let \( β = \sup O \).
	Invoking continuity, \( f(β) = \sup f[O] = \sup O = β \). Moreover, \( α ≤ f(α) ≤ β \).
\end{proof}

%\begin{lemma}
%	\label{ord-normal-covers}
%	Let \( E\colon \Ord → \Ord \) be an enumeration (of \( E[\Ord] \)).
%	For every \( α ≥ E(0) \) there exists a unique \( β ≤ α \) such that \( E(β) ≤ α < E(β+1) \).
%\end{lemma}
%%
%\begin{proof}
%	Consider the set \( O = \setof{ ξ }[ E(ξ) ≤ α ]\) and let \( β = \sup O \).
%	Then 
%	\[
%		E(β) = \sup E[O] = \sup \setof{ E(ξ) }[ E(ξ) ≤ α ] ≤ α
%	\]
%	whereas
%	\(
%		E(β+1) > α
%	\)
%	because \( β + 1 ∉ O \).
%\end{proof}
%
%
%%As enumerations are order preserving it is always the case that \( α ≤ E(α) \). 
%
%\begin{lemma}
%	\label{ord-normal-fix}
%	Let \( E\colon \Ord → \Ord \) be an enumeration (of \( E[\Ord] \)). 
%	There exists \( ξ \) such that \( ξ = E(ξ) \).
%	Moreover, for every \( α \) there exists a unique \( ξ ≥ α \) such that \( ξ = E(ξ) \).
%\end{lemma}
%%
%\begin{proof}
%	Fix \( α \) and define \( O = \setof{ E(α) , E(E(α)) , …, E^n(α) , … } \) (arbitrary finite iterations of \( E \) on \( α \)). Then \( α ≤ E(α) ≤ \sup O \) and \( E(\sup O) \)
%\end{proof}


%--------------------------------------
\section{Elementary Ordinal Arithmetic}
%--------------------------------------
The basic operations of arithmetic can be extended to ordinals in a straightforward manner.
Often these are defined by transfinite recursion, but the two operations we desire, addition and exponentiation base \( ω \), can be expressed as enumeration functions.
I start with addition.
\begin{definition}
	Let \( α^≥ \) be the class of ordinals \( ≥ α \).
	\emph{Ordinal addition}, \( α + β \), is defined as \( α + β ≔ E_{α^≥}(β) \). That is, \( α + β \) is defined as the \( β \)-th ordinal in the enumeration of the ordinals \( ≥ α \).
\end{definition}
%
The following are direct consequences of this definition and left to the reader.

\begin{lemma}\label{ord-addition}
	For all \( α \), \( β \) and \( γ \).
	\begin{enumerate}
		\item \( α + 0 = α \).
		\item \( α + β' = ( α + β )' \).
		\item If \( β \) is a limit then \( α + β = \sup \setof{ α + ξ }[ξ < β ] \).
		\item \( α + ( β + γ ) = ( α + β ) + γ \).\label{ord-addition-assoc}
		\item \( α ≤ α + β \) and \( β ≤ α + β \).\label{ord-addition-inc}
	\end{enumerate}
\end{lemma}

\begin{example}\label{ex-ord-add}
	\( α + ω = \sup \setof{ α + n }[n∈ ℕ] = \sup \setof{ α , α' , α'' , …  } \). Thus \( α + ω \) is the least limit ordinal strictly above \( α \).
	
	In particular, \( n + ω = ω \) for every \( n < ω \).
	As \( 1 + ω = ω < ω + 1 \) ordinal addition is not commutative.
\end{example}

As addition is associative (item \ref{ord-addition-assoc} of the \namecref{ord-addition} above), I will omit brackets when stringing together applications of addition.
So \( α + β + γ \) can refer to either \( ( α + β ) + γ \) or \( α + ( β + γ ) \).

The next lemma is a consequence of \cref{ord-normal}.

\begin{lemma}
	For every \( α ≤ β \) there exists a unique \( ξ \) such that \( β = α + ξ \).
\end{lemma}
\begin{proof}
	\Cref{ord-normal} implies a unique \( ξ \) such that \( α + ξ ≤ β < α + ξ' \). Since \( α + ξ' = ( α + ξ ) + 1 \) it follows that \( α + ξ = β \).
\end{proof}

As \cref{ex-ord-add} demonstrates \( ω \) has the unusual property of being closed under addition: if \( ξ , η < ω \) then \( ξ + η < ω \).
Ordinals satisfying this condition are called \emph{additive principal} ordinals.
\begin{definition}
	\label{d-ord-AP}
	A ordinal \( α \) is additive principal iff \( α > 0 \) and \( ξ + η < α \) for all \( ξ , η < α \).
	The class of additive principal ordinals is denoted \( \AP \).
\end{definition}

The least additive principal ordinal is \( 1 \); the next is clearly \( ω \).
Most ordinals are \emph{not} additive principal. 
\( 1 \) is the only additive principal successor ordinal (because \( α + α ≥ α' \) provided \( α ≥ 1 \)).
Even most limit ordinals not additive principal: If \( α ≥ ω \) then \( α + ω ∉ \AP \) as \( α < α + ω \) but \( α + α ≮ α + ω \).

\begin{lemma}
	\label{ord-AP-normal}
	The enumeration function \( E_\AP \) for additive principal ordinals is continuous and has domain \( \Ord \).
%	The additive principal ordinals are
%	\begin{enumerate}
%		\item Closed: for every set \( O ⊆ \AP \), \( \sup O ∈ \AP \).
%		\item Unbounded in \( \Ord \). For every set \( α ∈ \AP \) there exists \( β > α \) such that \( β ∈ \AP \).
%	\end{enumerate}
\end{lemma}
\begin{proof}
%	Begin with the domain. By transfinite induction. Suppose \( ξ ∈ \dom E_\AP \) for every \( ξ < α \). I claim that \( α ∈ \dom E_\AP \). It suffices to show that there exists an additive principal ordinal \( β > E_\AP(ξ) \) for every \( ξ < α \).
%	Let \( O_0 = \setof{ E_\AP(ξ) }[ξ < α] \) and \( O_{n+1} = \setof{ ξ + η }[ξ , η ∈ O_n ] \). Set \(  \)
	Exercise.
\end{proof}

\Cref{ord-AP-normal} shows that the function enumerating the additive principal ordinals is defined on all ordinals, is order preserving and continuous.

\begin{lemma}
	\label{ord-AP}
	The following are equivalent for all \( α > 0 \):
	\begin{enumerate}
		\item \( α \) is additive principal.\label{ord-AP-1}
		\item \( α = 1 \) or \( α = \sup \setof{ ξ + ξ }[ξ < α ] \).\label{ord-AP-2}
		\item for all \( β < α \), \( β + α = α \).\label{ord-AP-3}
	\end{enumerate}
\end{lemma}
\begin{proof}
	\ref{ord-AP-1} $⇒$ \ref{ord-AP-2}. If \( α \) is additive principal then \( \sup \setof{ ξ + ξ }[ξ < α ] ≤ α \) by definition. Also, the additive principal ordinals except \( 1 \) are all limits, so if \( α ≠ 1 \) then \( α = \sup \setof{ ξ }[ξ < α ] ≤ \sup \setof{ ξ + ξ }[ξ < α ] \).
	
	\ref{ord-AP-2} $⇒$ \ref{ord-AP-3}. For \( α = 1 \) the claim is trivial. Otherwise, \( α  \) is a limit and \( β + α ≤ \sup \setof{ β + ξ }[ξ < α ] ≤ \sup \setof{ ξ + ξ }[ξ < α ] \). As \( α = \sup \setof{ ξ + ξ }[ξ < α ] \) the claim is established.
	
	\ref{ord-AP-3} $⇒$ \ref{ord-AP-1}. Straightforward.
\end{proof}
%

As a consequence of part \ref{ord-AP-3}, \( ω^α + ω^β = ω^β \) iff \( α < β \).
A corollary is the observation made earlier, that \( n + ω = ω \), which now follows from repeated applications of \cref{ord-AP}: \( α' + ω = α + ( ω^0 + ω^1 ) = α + ω^1 \).

Additive principal ordinals are central to the theory of ordinals.
As with addition, I will introduce more suggestive notation for the enumeration function for additive principal ordinals.

\begin{definition}
	\( ω^α ≔ E_\AP(α) \).
\end{definition}

By the definition \( ω^0 = 1 \) and \( ω^1 = ω \).
The reader can confirm that next additive principal ordinal above \( ω \) is the supremum of \( ω \), \( ω + ω \), \( ω + ω + ω \), …, \( ω + ⋯ + ω , … \) which is denoted \( ω^2 \).
%I will shortly show that the function \( ξ ↦ ω^ξ \) behaves as one would expect exponentiation (in particular, \( ω^(α + β) \)

\begin{lemma}
	\label{ord-AP-NF}
	For every \( α > 0 \) there exists unique \( β \) and \( ξ < α \) such that \( α = ω^β + ξ \).
\end{lemma}
\begin{proof}
	Let \( β \) be such that \( ω^β ≤ α < ω^{β'} \) and \( ξ \) such that \( α = ω^β + ξ \). Both ordinals are given by \cref{ord-normal}.
	What remains is to show uniqueness of this choice.
	Thus, suppose \( α = ω^γ + η \) for some \( γ \) and \( η < α \).
	The choice of \( β \) is clearly such that \( β ≥ γ \).
	As 
	\[ ω^β + ω^{γ+1} ≤ α + ω^{γ+1} ≤ ω^γ + η + ω^{γ + 1 } = ω^{γ+1} \]
	(the first inequality uses \cref{ord-addition}(\ref{ord-addition-inc}); the rest use \cref{ord-AP}(\ref{ord-AP-3})), also \( β ≤ γ \). Given that \( β = γ \), uniqueness of the rest is immediate.
\end{proof}
%\begin{lemma}
%	\label{ord-AP-add}
%	For all \( α > 0 \), \( α ∉ \AP \) iff \( α = β + γ \) for some \( β , γ < α \).
%\end{lemma}

%---------------------------------
\section{Normal forms and natural sum}
%---------------------------------

\Cref{ord-AP-NF} above provides the basis of a normal form representation of ordinals. This concept is introduced in the next definition.

\begin{definition}
	I write \( α =_\NF ω^β + γ \) to express that (i) \( α = ω^β + γ \) and (ii) \( γ < α \).
\end{definition}

Cantor, in 1897,\nocite{Cantor1897} established an expanded version of this normal form decomposition.
%
\begin{theorem}[Cantor normal form theorem]\label{t-cantornf}
	For every ordinal \( α > 0 \) there exists \( n \) and ordinals \( α_n ≤ ⋯ ≤ α_0 \) such that 
	\[
		α = ω^{α_0} + ⋯ + ω^{α_n}.
	\]
	Moreover, this decomposition is unique. %, and \( α_0 < α \) if \( α < ε_0 \).
\end{theorem}
%
\begin{proof}
	The theorem is a simple generalisation of \cref{ord-AP-NF}.
%	The argument is by transfinite induction. Assume that each ordinal \( β < α \) admits a decomposition as described in the theorem. 
	Let \( α =_\NF ω^{α_0} + ξ_0 \) by \cref{ord-AP-NF}. If \( ξ_0 = 0 \) the decomposition is complete. 
	Otherwise, apply the \namecref{ord-AP-NF} again to express \( ξ_0 =_\NF ω^{α_1} + ξ_1 \), \( ξ_1 =_\NF ω^{α_2} + ξ_2 \), etc. 
	As \( α > ξ_0 > ξ_1 > ⋯ \) is a strictly decreasing sequence or ordinals, necessarily \( ξ_n = 0 \) for some \( n \). Thus, 
	\( α = ω^{α_0} + ⋯ + ω^{α_n} \). 
	Furthermore, \( α_0 ≥ α_1 ≥ ⋯ ≥ α_n \) because \( ω^{α_{i+1}} ≤ ξ_i < ω^{α_i+1} \) for each \( i \).
	Uniqueness is also a consequence of these normal forms.
\end{proof}


\begin{definition}
	The normal form notation is extended in the following way. Writing \( α =_\NF ω^{α_1} + ⋯ + ω^{α_n} \) expresses that (i) \( α = ω^{α_1} + ⋯ + ω^{α_n} \) and (ii) \( α ≥ α_1 ≥ ⋯ ≥ α_n \).
\end{definition}

\Cref{ord-normal} showed that every continuous order preserving function on the ordinals has fixed points. I.e., for each such function \( f \) there are ordinals \( β \) such that \( β = f(β) \).
As the function \( ξ ↦ ω^ξ \) (namely \( E_\AP \)) is an example of such a function, there must exist ordinals \( α \) such that \( α = ω^α \).
The proof of that lemma describes how to construct such an ordinal as the supremum of the sequence \( 0 \), \( 1 \), \( ω \), \( ω^ω \), …, \( α \), \( ω^α \), ….
This particular ordinal, conventionally denoted \( ε_0 \), will play a central role in the next chapter.

\begin{definition}\label{d-epsilon0}
	\( ε_0 ≔ \supseq ω_i \) where \( ω_0 = ω \) and \( ω_{k+1} = ω^{ω_k} \).
\end{definition}
%

%
\begin{lemma}
	\label{ord-e0}
	\( ε_0 \) is the least fixed point of the ordinal function \( α \mapsto ω^α \). 
	That is, \( ω^{ε_0} = ε_0 \) and \( α < ω^{α} \) for all \( α < ε_0 \).
\end{lemma}
%
\begin{exercise}
	Prove \cref{ord-e0}.
\end{exercise}

\begin{exercise}\label{ex-ord-mult-pre}
	Using the Cantor normal form theorem, define a multiplication operation where the first argument is restricted to additive principal ordinals: \( α , β ↦ ω^α.β \).
	The function should be continuous in \( β \) and satisfy the recursive clauses: \( ω^α.0 = 0 \) and \( ω^α.(β + 1) = ω^α.β + ω^α \).
\end{exercise}

\begin{exercise}\label{ex-ord-base-2}
	Define a function \( α ↦ 2^α \) satisfying
	\begin{align*}
		2^0 &= 1
		\\
		2^{α+1} &= 2^α + 2^α
		\\
		2^λ &= \supof{ 2^ξ }[ξ < λ]
	\end{align*}
	(You may find it useful to use the Cantor normal form theorem.)
	Show that this function is order preserving and continuous, and compute all fixed points of the function for ordinals \( α ≤ ε_0 \).
%	(It is not necessary to prove the existence of this function.)
\end{exercise}
%
\begin{exercise}
	\label{ex-epsilon-numbers}
	Let \( α ↦ ε_α \) be the enumerating function of the ordinals \( η \) such that \( η = ω^η \).
%	For each \( α \), define \( ε_α \) as the least ordinal such that
%	\begin{enumerate}
%		\item \( ε_β < ε_α \) for every \( β < α \),
%		\item \( ω^{ε_α} = ε_α \).
%	\end{enumerate}
	Express \( ε_α \) as a supremum of smaller ordinals as per \cref{d-epsilon0} and deduce that the enumerating function is defined for all ordinals.
\end{exercise}
%


\begin{exercise}
	\label{ex-ord-cnf-2}
	Prove the Cantor normal form theorem in base \( 2 \):
	\emph{For every ordinal \( α > 0 \) there exists unique ordinals \( α_n ≤ ⋯ ≤ α_0 ≤ α \) such that 
	\[
		α = 2^{α_0} + ⋯ + 2^{α_n}.
	\]}
\end{exercise}
%
\begin{exercise}
	\label{ex-ord-cnf-2o}
	What are the additive principal ordinals in base-$2$ normal form?
	Characterise the \( α \) such that \( 2^α = ω^α \).
\end{exercise}

%---------------------
%\section{Natural sum}
%---------------------

This brief foray into ordinals is concluded with another look at addition.
Recall that addition on ordinals is not commutative: \( 1 + ω ≠ ω + 1 \) for example.
It is possible to provide a natural notion of addition that \emph{is} commutative.
This is called the \emph{natural sum} (sometimes \emph{Hessenberg sum} after its originator Gerhard Hessenberg~\cite{Hess1906}). 
The Cantor normal theorem provides the means to achieve this.

\begin{definition}
	The natural sum of ordinals \( α \) and \( β \), denoted \( α \nsum β \) is defined by recursion on the two ordinals. \( 0 \nsum α = α \nsum 0 ≔ α \) for all \( α \).
	For non-zero \( α =_\NF ω^{α_0} + α_1 \) and \( β =_\NF ω^{β_0} + β_1 \)
	\[
		α \nsum β ≔ 
		\begin{cases}
			ω^{α_0} + ( α_1 \nsum β ), &\text{if \( α_0 ≥ β_0 \),}
			\\
			ω^{β_0} + ( α \nsum β_1 ), &\text{if \( α_0 ≤ β_0 \).}
		\end{cases}
	\]
	The operation of natural sum is well-defined as \( α_1 < α \) and \( β_1 < β \).
\end{definition}

As an operation on the Cantor normal form, the natural sum  has the following property.
\begin{lemma}
	For \( α =_\NF ω^{α_1} + ⋯ + ω^{α_m} \) and \( β =_\NF ω^{β_1} + ⋯ + ω^{β_n} \)
	\[
		α \nsum β ≔ ω^{γ_1} + ⋯ + ω^{γ_{m+n}}
	\]
	where \( γ_1 ≥ ⋯ ≥ γ_{m+n} \) enumerate the ordinals \( α_1, …, α_m , β_1 , …, β_n \) in descending order (with repetitions).
\end{lemma}

\begin{lemma}
	\label{ord-nsum}
	The natural sum is commutative and strongly increasing in both arguments: For all \( α \), \( β \), \( γ \),
	\begin{enumerate}
		\item \( α \nsum β = β \nsum α \);
		\item \( α < β \) implies \( α \nsum γ < β \nsum γ \).
	\end{enumerate}
\end{lemma}
%\begin{lemma}
%	\label{ord-nsum-2}
%	\begin{enumerate}
%		\item \( α \nsum β \)
%		\item \( α + β ≤ α \nsum β ≤ \max\setof{ α , β }.2 \).
%	\end{enumerate}
%\end{lemma}

\begin{exercise}
	Prove \cref{ord-nsum}.
\end{exercise}

\begin{exercise}
	\label{ex-ord-mult}
	Using the Cantor normal form theorem define a commutative multiplication \( α . β \) operation on ordinals. It should satisfy the distribution law:
	\(
		( α \nsum β ) . γ = (α .γ) \nsum (β .γ ).
	\)
	Hint, start from the function in exercise~\ref{ex-ord-mult-pre}.
\end{exercise}


%---------------------------------
\chapter{Ordinal analysis of arithmetic}\label{c-oa-PAo}
%---------------------------------
Ordinals will now be used to measure the \emph{height} of \( ω \)-proofs.
I begin by formalising the infinitary sequent calculi for arithmetic described at the end of \cref{c-oa-arith}.
Building on that definition, it will prove convenient to view Peano arithmetic as extending primitive recursive arithmetic.
%
\begin{definition}%\label{d-PAomega}
	\( \PAo \) is the sequent calculus given by the following:
	Sequents comprise closed formulas in the language of primitive recursive arithmetic \( \Lpra \).
	
	Initial sequents are \emph{closed} sequents of the form
	\begin{itemize}
		\item[(\botL)] \( ⊥, Γ ⇒ Δ \).
		\item[(\idRule)] \( Ps, Γ ⇒ Δ , Pt \) if \( ℕ ⊨ s = t \).
		\item[(\eqR)] \( Γ ⇒ Δ , s = t \) if \( ℕ ⊨ s = t \).
		\item[(\eqL)] \( s = t , Γ ⇒ Δ \) if \( ℕ ⊭ s = t \).
	\end{itemize}
		
	Inference rules are rules of \( \Gc \) but restricted to closed sequents and with \( \faR \) and \( \exL \) replaced by the two \( ω \)-rules:\note{\( \Gc \) or more general \( \Logic{G3} \)?}
		\begin{itemize}
			\item[(\omR)] \begin{prooftree} \hypo{ Γ ⇒ Δ , F(\nm n) \ \text{for every } n ∈ ℕ } \infer1{ Γ ⇒ Δ , ∀x F(x) }\end{prooftree}
			\item[(\omL)] \begin{prooftree} \hypo{ F(\nm n) , Γ ⇒ Δ \ \text{for every } n ∈ ℕ } \infer1{ ∃x F(x) , Γ ⇒ Δ }\end{prooftree}
		\end{itemize}
	%	
	\( \HAo \) is the same calculus but restricted to intuitionistic sequents.
\end{definition}
%

It is important to note that the property of being an initial sequent of \( \PAo \) is decidable.
This is precisely because sequents do not contain free variables, whereby it is decidable whether an equation between primitive recursive terms is true (or not).
Likewise, for each of the rules of \( \PAo \): The property of being the \( n \)-th premise of a rule whose conclusion is \( Γ ⇒ Δ \) with specified principle formula is decidable.

%
\begin{definition}\label{d-bound-omega-logic}
	Let \( \Theory{T} \) be \( \PAo \), \( \HAo \) or an extension of either calculus by rules that are at most \( ω \)-branching.
	The ternary relation \( \Theory{T} \prv{α}k Γ ⇒ Δ \), between a sequent \( Γ ⇒ Δ \), an ordinal \( α \) and \( k < ω \), is defined by transfinite recursion on the rules of \( \Theory{T} \):
	\begin{enumerate}
		\item If \( Γ ⇒ Δ \) is an initial sequent of \( \Theory{T} \), then \( \Theory{T} \prv{α}k Γ ⇒ Δ \) for all \( α \) and \( k \);
		\item For each inference \( (*) \) of \( \Theory{T} \) except cut of the form
		\[
			\Infer{\setof{ Γ_i ⇒ Δ_i }[i ∈ I]}[\( * \)]{ Γ ⇒ Δ }
		\]
		\( \Theory{T} \prv{α}k Γ ⇒ Δ \) holds if \( \Theory{T} \prv{α_i}k Γ_i ⇒ Δ_i \) and \( α_i < α \) for all \( i ∈ I \);
		\item If \( \Theory{T} \prv{α_0}k Γ ⇒ Δ , C \) and \( \Theory{T} \prv{α_1}k C , Γ ⇒ Σ \) for \( α_0,α_1 < α \) and \( \rk C < k \), then \( \Theory{T} \prv{α}k Γ ⇒ Δ, Σ \).
	\end{enumerate}
%	If \( \Theory{T} \) is a calculus of intuitionistic sequents, \( \Theory{T} \prv{α}k Γ ⇒ A \) is defined via the same conditions but with a modification of the final clause:
%	\begin{enumerate}[resume]
%		\item If \( \Theory{T} \prv{α_0}k Γ ⇒ C \) and \( \Theory{T} \prv{α_1}k C , Γ ⇒ A \) for \( α_0,α_1 < α \) and \( w(C) < k \), then \( \Theory{T} \prv{α}k Γ ⇒ A \).
%	\end{enumerate}
	Given \( \Theory{T} \prv{α}k Γ ⇒ Δ \) I will write that \( Γ ⇒ Δ \) is derivable (in \( \Theory T \)) with height \( ≤ α \) and cut rank \( ≤ k \).
\end{definition}
%

%Some of the terminology  to \( ω \)-proofs as before.

There is no requirement of minimality of \( α \) and \( k \) in the above definition. 
So the relation \( \prv{α}k \) is monotone in \( α \) and \( k \):
%
\begin{lemma}\label{oa-PAo-weak1}
	If \( α ≤ β \) and \( k ≤ l \) then \( \Theory{T} \prv{α}k Γ ⇒ Δ \) implies \( \Theory{T} \prv{β}l Γ ⇒ Δ \).
\end{lemma}
%
\begin{proof}
	By transfinite induction on \( α \). If \( Γ ⇒ Δ  \) is an initial sequent, the result is immediate.
	Otherwise, there is an inference rule of \( \Theory T \)
	\[
		\Infer{\setof{ Γ_i ⇒ Δ_i }[i ∈ I]}[\( * \)]{ Γ ⇒ Δ }
	\]
	and ordinals \( α_i < α \) such that \( \smash{\Theory T \prv{α_i}k Γ_i ⇒ Δ_i} \) for each \( i ∈ I \).
	The induction hypothesis implies that \( \Theory T \prv{α_i}l Γ_i ⇒ Δ_i \) for each \( i \), whereby \( \Theory T \prv{β}l Γ ⇒ Δ \) obtains.
\end{proof}

\Cref{oa-PAo-weak1} operates in the background of the majority of the results to follow. For that reason I will not make any explicit reference to the \namecref{oa-PAo-weak1}.

\begin{example}
	\tbw
\end{example}

\begin{lemma}\label{oa-HA-good}
	If \( \HAo \prv{α}k Γ ⇒ A \) then this fact can be observed by use of sequents of the form \( Σ ⇒ B \) (i.e., exactly one formula on the right).
\end{lemma}

\begin{exercise}
	Assign ordinal bounds on the \( ω \)-proofs of \( A , Γ ⇒ Δ , A \) constructed in exercise~\ref{ex:oa-id-simple}.
\end{exercise}

Revisiting the Embedding lemma (lemma \ref{oa-embed-weak}) it is possible provide ordinal bounds on the size of the resulting \( ω \)-proof.
Let \( α.k = \underbrace{α + ⋯ + α }_k \).

\begin{lemma}[Refined embedding]\label{oa-embed-PAo-w-bounds}
	Suppose \( \PA ⊢ Γ ⇒ Δ \) and \( Γ ⇒ Δ \) is closed. Then there is \( n,k < ω \) such that \( \PAo \prv{ω.n}k Γ ⇒ Δ \) where \( ω .n = ω + ⋯ + ω \) (\( n \) times).
	Likewise, \( \HA \) into \( \HAo \).
\end{lemma}
%
\begin{exercise}
	Prove the refined embedding lemma following the schema of embedding lemma at the end of \cref{c-oa-arith}.
\end{exercise}
%
The next lemma hints at part of the usefulness of the \( ω \)-rule with the ability to isolate finitary reasoning from infinitary reasoning.
The result will be useful in \cref{s-oa-upper}.
%
\begin{proposition}\label{p-PAo-S1}
	Let \( A(a_1,…, a_k) \) be a \( Σ_1 \) formula. 
	There exists \( m < ω \) such that for all \( n_1, …, n_k ∈ \Nat \), 
	\[ \text{if }\thinspace  \Nat ⊨ A(\nm {n_1}, …, \nm {n_k} ) \text{ then } \HAo \prv m0 {⇒ A(\nm {n_1}, …, \nm {n_k})} .
	\]
\end{proposition}
%
\begin{proof}
	By induction on the rank of \( A \).
\end{proof}

Henceforth, I will omit explicit mention of \( \PAo \) and write \( \prv{α}k Γ ⇒ Δ \) to mean \( \PAo \prv{α}k Γ ⇒ Δ \).
The following results are stated only for \( \PAo \) but apply equally to \( \HAo \) in the expected way.
Admissibility of weakening becomes 
%
\begin{lemma}[Weakening]
	If \( \prv{α}k Γ ⇒ Δ \) and \( Γ' ⇒ Δ' \) is closed then \( \prv{α}k Γ' , Γ ⇒ Δ, Δ' \).
%	Likewise for \( \HAo \).
\end{lemma}
%
%\begin{proof}
%	By (transfinite) induction on \( α \).
%\end{proof}
\begin{exercise}
	Prove the weakening lemma.
\end{exercise}

The substitution lemma for \( \PAo \) takes a different formulation from previously. 
As sequents are closed, the correct formulation for \( ω \)-proofs is that provability depends on the \emph{value} of terms, not their \emph{form}.

\begin{lemma}[Substitution]
	Let \( Γ(a) ⇒ Δ(a) \) be a sequent and \( s \) and \( t \) be closed terms such that \( \Nat ⊨ s = t \). If \( \prv{α}k Γ(s) ⇒ Δ(s) \) implies \( \prv{α}k Γ(t) ⇒ Δ(t) \).
\end{lemma}
%
\begin{proof}
%	Analogous argument. I will treat the case of \( \exR \): 
	Suppose 
	\( \prv{α}k Γ(s) ⇒ Δ(s) \) and \( \Nat ⊨ s = t \).
	Let \( Γ(a) ⇒ Δ(a) \) be any sequent with at most \( a \) free.
	If \( Γ(s) ⇒ Δ(s) \) is initial then a case distinction on the different forms this sequent can take confirms that \( Γ(s) ⇒ Δ(s) \) is also initial provided \( \Nat ⊨ s = t \). The other case proceed by transfinite induction on \( α \).
\end{proof}

The final ingredient is the inversion lemma, the statement of which has the same form as before with two new cases treating equality.

\begin{lemma}[Inversion]
	\label{oa-inversion} The following hold for all parameters.
	\begin{enumerate}
		\item If \( \prv{α}k Γ ⇒ Δ , ⊥ \) then \( \prv{α}k Γ ⇒ Δ \).
		\item If \( \prv{α}k s = t , Γ ⇒ Δ \) and \( ℕ ⊨ s = t \) then \( \prv{α}k Γ ⇒ Δ \).\label{oa-inversion-eq1}
		\item If \( \prv{α}k Γ ⇒ Δ , s = t \) and \( ℕ ⊭ s = t \) then \( \prv{α}k Γ ⇒ Δ \).
		\item If \( \prv{α}k Γ ⇒ Δ , ∀x F(x) \) then \( \prv{α}k Γ ⇒ Δ , F(s) \) for every closed term \( s \).\label{oa-inversion-fa}
		\item If \( \prv{α}k ∃x F(x) , Γ ⇒ Δ \) then \( \prv{α}k F(s) , Γ ⇒ Δ \) for every closed term \( s \).
		\item Analogous inversion principles for the rules \( \disjL \), \( \conjR \), \( \impR \) and \( \impL \).
	\end{enumerate}
\end{lemma}
\begin{proof}
	I show cases \ref{oa-inversion-eq1} \& \ref{oa-inversion-fa}.
	
	\ref{oa-inversion-eq1}. By induction on \( α \). Suppose \( \prv{α}k s = t , Γ ⇒ Δ \) and \( ℕ ⊨ s = t \). If \( s = t , Γ ⇒ Δ \) is initial then so is \( Γ ⇒ Δ \). The other cases are straightforward because the equation \( s = t \) cannot be the principal formula of any rule.
	For if \( s = t , Γ ⇒ Δ \) is not initial, then there are sequents \( \setof{ Γ_i ⇒ Δ_i}[i<ω] \)
	and ordinals \( \setof{α_i}[i<ω] \) such that
	\begin{enumerate}[label=(\alph*)]
		\item \( \prv{α_i}k s=t , Γ_i ⇒ Δ_i \) for each \( i < ω \),
		\item \( α_i < α \) for all \( i \),
		\item \( \setof{ Γ_i ⇒ Δ_i}[i<ω] \) enumerate all premises of an inference of \( \PAo \) whose conclusion is \( Γ ⇒ Δ \).
	\end{enumerate}
	In the case of unary or binary rules, \( Γ_i = Γ_{i+1} \) and \( Δ_{i} = Δ_{i+1} \) for all \( i > 0 \) or \( 1 \). But in the case of either of the two \( ω \)-rules, the sequents enumerate the infinitely many premises.
	By (a)--(c) and the induction hypothesis, \( \prv{α}k Γ ⇒ Δ \) holds as desired.
	
	\ref{oa-inversion-fa}. The argument is a direct generalisation of the finitary inversion lemma. Suppose \( \prv{α}k Γ ⇒ Δ , ∀x F(x) \). If this sequent is initial, then so is \( Γ ⇒ Δ , F(s) \) for every closed term \( s \).
	The rest of the argument proceeds, essentially, as above by a case distinction on the inferences through which \( \prv{α}k Γ ⇒ Δ , ∀x F(x) \) can be derived.
	The case of \( \faR \) with \( ∀x F(x) \) principal bears treatment.
	The premises of this inference can be assumed to have the form \( Γ ⇒ Δ , ∀x F(x) , F(\nm n) \). An application of the induction hypothesis (to each premise) yields \( \prv{α}k Γ ⇒ Δ , F(\nm n) \) for every \( n \). If the desired closed term \( s \) is a numeral, this case is complete. Otherwise, let \( n \) be the value of \( s \), i.e., \( n ∈ ℕ \) is such that \( ℕ ⊨ \nm n = s \).
	The substitution lemma then yields \( \prv{α}k Γ ⇒ Δ , F(s) \).
\end{proof}

%---------------------------------
\section{Infinitary cut elimination}\label{s-oa-cutelim}
%---------------------------------

I begin with the transfinite version of the reduction lemma.
Recall, this is statement that borderline cuts can be simulated at the cost of increasing the depth of the proof by a controlled amount.
In the finitary case the depth increase was, in the case of classical logic, \( m + n \) where \( m \) and \( n \) bounded the depth of the two cut premises.

Lifting the statement of the reduction lemma to the transfinite realm is reasonably straightforward.
Given premises of a borderline cut of height \( α \) and \( β \) respectively, the cut can be simulated by a height of \( α \nsum β \).
The use of natural sum is crucial to the argument: the lifting of the finitary argument requires the resulting bound to be order-preserving in both arguments, a property we know fails for traditional ordinal sum \( α + β \).

%In fact, I will give two proofs of the reduction lemma.
%The first is the version just described: a direct lifting of the finitary argument to \( ω \)-proofs. 
%Some steps of the proof clearly require new insights, such as passing cuts through \( ω \)-rules and the new initial sequents with equations.
%But, by and large, the design of \( \PAo \) is such that the transfinite element is straightforward once one knows what to expect.
%
%Thus, the reduction lemma I will prove will be:

\begin{lemma}[Reduction lemma for \( \PAo \)]\label{oa-red-lem-PAo}
	Suppose \( \prv{α}k Γ ⇒ Δ , C \) and \( \prv{β}k C , Σ ⇒ Λ \).
	If \( \rk C ≤ k \) then \( \prv{α\nsum β }k Γ, Σ ⇒ Δ, Λ \).
\end{lemma}

The reader may surprised to know that there is a great deal of flexibility in proofs of the reduction lemma, which I will demonstrate by presenting a slightly different strategy than we used for in the analysis of classical predicate logic.
%The statement of the lemma already offers a minor, though insignificant, departure from previously by allowing \( \rk C < k \). 
%Provided that \( α , β \) are non-zero, \( \max\setof{ α , β } + 1 ≤ α \nsum β \), so whether.

%The second formulation will be presented after the proof of the above \namecref{oa-red-lem-PAo}.
%In short, it mitigates a shortcoming of the reduction lemma as described in its exaggeration of the complexity of cut elimination.
%The transfinite reduction lemma is optimal in the same way as the finitary form: Sequents can be readily chosen whose shortest cut-free proofs require a jump in ordinal height matching that expressed by the reduction lemma and its immediate corollaries.
%But by treating each logical connective as equally ‘expensive’, the reduction lemma ignores the true source of ‘large’ proofs: \emph{alternation} between \emph{positive} and \emph{negative} statements.
%
%This alternative version of the reduction lemma is closely related to the form proved in exercise~\ref{ex-red-lem-special}.

\begin{proof}%[of \cref{oa-red-lem-PAo}]
%	Suppose:
%	\begin{enumerate}
%		\item \( \prv{α}k Γ ⇒ Δ , C \).
%		\item \( \prv{β}k C , Σ ⇒ Λ \).
%	\end{enumerate}
%
	The proof branches into cases depending on the form of \( C \).
	In each case I will establish \( \prv{α \nsum β} k Γ, Σ ⇒ Δ, Λ \) but the induction will proceed over either \( α \) or \( β \) (depending on the case) rather than on the sum \( α \nsum β \).
	If the principal connective of \( C \) is among \( \setof{ ⊥ , ∀ ,  ∧ , → } \) I will refer to \( C \) as \emph{locally negative} (cf.~Canvas assignment no.~4).
	Otherwise, \( C \) will be \emph{locally positive}.
	
%	\paragraph{Case I: \( C \) is \( ⊥ \) or an equation.} This case can be dispensed with directly. If \( C = ⊥ \) or a false equation then \( \prv{α}k Γ ⇒ Δ \) is a consequence of the inversion lemma;  otherwise \( \prv{β}k Σ ⇒ Λ \).
%	In any case, weakening yields \( \prv{α\nsum β }k Γ, Σ ⇒ Δ , Λ \).
	
	\paragraph{Case I: \( C \) is atomic or locally negative.}
	Here I proceed by induction on \( β \) and show that \( \prv{ α \nsum β }k Γ, Σ ⇒ Δ , Λ \).
	I present two subcases:
	
	\( C = ∀x D(x) \). If \( C , Σ ⇒ Λ \) is initial then \( Σ ⇒ Λ \) is also initial and the claim holds by weakening.
	Otherwise, consider the rule that derives \( \prv{β}k C , Σ ⇒ Λ \).
	If the principal formula of the rule is \emph{not} \( C \) then the induction hypothesis can be applied directly to its premises and the rule re-applied to derive the desired sequent with correct bounds.
	If, however, the rule is \( \faL \) with \( C \) principal, the above argument does not work. But in this case there is \( γ < β \) and term \( t \) such that 
	\[
	  \prv{γ}k D(t) , C , Σ ⇒ Λ .
	\]
	The induction hypothesis yields
	\[
		\prv{α \nsum γ}k D(t) , Γ , Σ ⇒ Δ , Λ .
	\]
	From the inversion lemma (part \ref{oa-inversion-fa}) I know also that \( \prv{α}k Γ ⇒ Δ , D(t) \). Since \( \rk {D(t)} < \rk{ C} = k \), an application of cut yields 
	\( \prv{ α \nsum β }k Γ, Σ ⇒ Δ, Λ \).
	
	\( C = D → E \). I employ a similar argument as above but with a subtle difference in how the induction hypothesis is applied to account for the binary connectives. 
	By the previous argument I can jump directly to the case that \( C \) is principal in the derivation of \( \prv{β}k C , Σ ⇒ Λ \), for which there exist \( γ , δ < β \) and \( Λ = Λ_0 ∪ Λ_1 \) satisfying
	\begin{enumerate}
		\item \( \prv{γ}k C , Σ ⇒ Λ_0, D \).
		\item \( \prv{δ}k C , E , Σ ⇒ Λ_1 \).
	\end{enumerate}
	I start by applying the inversion lemma to my three hypotheses:
	\begin{enumerate}[resume]
		\item \( \prv{α}k D , Γ ⇒ Δ , E \).
		\item \( \prv{γ}k Σ ⇒ Λ_0, D \).
		\item \( \prv{δ}k E , Σ ⇒ Λ_1 \).
	\end{enumerate}
	Then I apply the induction hypothesis between the sequents in 3 and 5 (using ‘cut’ formula \( E \)):
	\begin{enumerate}[resume]
%		\item \( \prv{α+γ}k Γ , Σ ⇒ Δ ,Λ_0, D \).
		\item \( \prv{α \nsum δ}k D , Γ, Σ ⇒ Δ , Λ_1 \).
	\end{enumerate}
	I can now combine 6 and 3 with a (standard) cut:
	\[
		\prv{α \nsum β}k Γ , Σ ⇒ Δ ,Λ.
	\]
	The conjunction subcase is left to the reader.
	
	Case II: \( C \) is locally positive.
	This case is symmetric to the previous and left to the reader.
\end{proof}

%The observant reader will have noticed that the argument for the implication case could be simplified by applying the inversion lemma
% ------------------
\begin{figure}
	\centering
	\begin{prooftree}
%		\hypo{\prv{α}k Γ ⇒ Δ , C }
		\hypo{\prv{γ}k C , Σ ⇒ Λ_0, D}
		\infer[dashed]1[IL]{\prv{γ}k Σ ⇒ Λ_0, D}
%		\infer[dashed]2[IH]{\prv{α+γ}k Γ , Σ ⇒ Δ, Λ_0, D}
			
		\hypo{\prv{α}k Γ ⇒ Δ , C }
		\infer[dashed]1[IL]{\prv{α}k D , Γ ⇒ Δ , E }
			\hypo{\prv{δ}k C , E , Σ ⇒ Λ_1 }
			\infer[dashed]1[IL]{\prv{δ}k E , Σ ⇒ Λ_1 }
		\infer[dashed]2[IH]{\prv{α \nsum δ}k D , Γ , Σ ⇒ Δ , Λ_1 }
		\infer2[\Cut]{\prv{α \nsum β}k Γ , Σ ⇒ Δ , Λ }
	\end{prooftree}
	\caption{Illustration of the proof method in the reduction lemma for the case \( C = D → E \); IL = ‘inversion lemma’ and IH = ‘induction hypothesis’.}
	\label{f-oa-PAo}
\end{figure}
% ------------------

\begin{exercise}
	Complete the preceding proof.
\end{exercise}

\begin{exercise}
	Formulate and prove a reduction lemma for \( \HAo \) following the proof scheme above.
\end{exercise}

\begin{exercise}
	Give an alternative proof of \cref{oa-red-lem-PAo} using the proof strategy from the reduction lemma for \( \Gc \) (\cref{ce-red-lem-C}).
\end{exercise}

In the implication subcase of case II in the proof above, I used the induction hypothesis to simulate a cut on the formula \( E \)

\begin{theorem}[Reduction theorem for \( \PAo \)]\label{oa-red-thm-PAo}
	If \( \prv {α}{k+1} Γ ⇒ Δ  \) then \( \prv{ω^α}k Γ ⇒ Δ \).
\end{theorem}
\begin{proof}
	Induction on \( α \). If \( Γ ⇒ Δ \) is initial, the claim holds trivially.
	So suppose \( \prv {α}{k+1} Γ ⇒ Δ  \) is derived via a rule
	\[
	  \Infer{Γ_i ⇒ Δ_i \; \text{for } i ∈ I}[\( * \)]{ Γ ⇒ Δ }
	\]
	and for each \( i \) there is \( α_i < α  \) such that \( \prv{α_i}{k+1} Γ_i ⇒ Δ_i \).
	The induction hypothesis implies that \( \prv{ω^{α_i}}{k} Γ_i ⇒ Δ_i \) for each \( i \). So, if \( * \) is not cut, then 
	\[ \prv{ω^α}k Γ⇒ Δ \]
	obtains by re-applying the rule and observing that \( \supof{ ω^η }[η< α] ≤ ω^α \).
	Now suppose that the rule is cut, with cut formula \( C \). If \( \rk C < k \) the same argument as above applies.
	Otherwise \( \rk C = k \) and the reduction lemma is applicable, yielding
	\[
		\prv{ω^{α_0} \nsum ω^{α_1}}k Γ ⇒ Δ 
	\]
	Since \( ω^{α_0} \nsum ω^{α_1} < ω^α  \), the proof is complete.
\end{proof}

The bound in the reduction theorem can be improved fairly easily.
For the give proof strategy to work, it suffices to find an order-preserving function \( f \colon \Ord → \Ord \) such that \( f(α) ≥ \supof{ f(ξ) \nsum f(η) }[ξ,η<α] \).
An obvious candidate is \( f \colon α ↦ 2^α \) (see exercise~\ref{ex-ord-base-2}) and, indeed, \cref{oa-red-lem-PAo} can be strengthened by replacing \( ω^α \) with \( 2^α \).
Certainly, \( 2^α ≤ ω^α \) for all \( α \), so working with this bound seems a significant improvement. 
But given that for every additive principal ordinal \( α ≥ ω^ω \) in fact \( 2^α = ω^α \) (cf.\ exercise~\ref{ex-ord-cnf-2o}), the distinction between exponentiation in the two bases does little in reducing the complexity of cut elimination.

In the next section I will present a strict refinement of the cut elimination theorem in which ordinal exponentiation is directly tied to the \emph{quantifier} rank of the cut formula rather than the full rank.

Let \( ω_0^α ≔ α \) and \( ω_{k+1}^α ≔ ω^{ω_k^α} \).

\begin{theorem}[Cut elimination]\label{oa-ce-PAo}
	If \( \prv{α}k Γ⇒ Δ \) then \( \prv{ω_k^α}0 Γ⇒ Δ \).
\end{theorem}
\begin{proof}
	Consequence of \cref{oa-red-thm-PAo}.
\end{proof}

\begin{exercise}
	Formulate and prove a corresponding reduction lemma and cut elimination theorem for \( \HAo \).
\end{exercise}

\begin{theorem}[Embedding]
	\label{oa-embed-PA-ce}
	If \( \PA ⊢ Γ ⇒ Δ \) and this a closed sequent, then there exists \( α < ε_0 \) such that
	\[
		\PAo \prv{α}0 Γ ⇒ Δ.
	\]
	In addition, \( α \) is effectively computable from the given \( \PA \)-proof.
\end{theorem}
%
\begin{proof}
	Suppose \( \PA ⊢ Γ ⇒ Δ \).
	By the embedding lemma (\cref{oa-embed-PAo-w-bounds}) there is are \( n, k \) such that 
	\[
	  \PAo \prv{ω.n}k Γ ⇒ Δ .
	\]
	Let \( α = ω_k^{ω.n} \). Then \( α < ε_0 \) (by \cref{d-epsilon0}) and
	\[
	  \PAo \prv{α}0 Γ ⇒ Δ 
	\]
	by \cref{oa-ce-PAo}.
\end{proof}

On the basis of cut elimination, a few observations can be already made.

\begin{corollary}
	\label{PA-consis-weak}
	\( \PA \) and, hence, \( \HA \), are consistent.
\end{corollary}
%
\begin{proof}
	There can be no cut-free proof of the empty sequent.
\end{proof}

An inspection of the various proofs leading up to \cref{PA-consis-weak} can strengthen the result by clarifying what mathematical principles suffice to derive the consistency of arithmetic.
%
\begin{corollary}
	\label{PA-consis}
	Consistency of \( \PA \) can be deduced using only finitary reasoning plus the principle of transfinite induction for ordinals \( {≤} ε_0 \).
\end{corollary}
%
By ‘finitary reasoning’ I mean the ‘finite’ mathematics that can be carried out using only finite objects (such as natural numbers) and primitive recursive functions.
Examples include deciding whether one formula is a subformula of another, whether a given primitive recursive function enumerates the premises of an \( ω \)-rule (or Gödel codes of sequents) and what the concluding sequent is.
It is beyond the scope of these lecture notes to attempt to make the statement more precise, but the following proof ‘sketch’ hopefully elucidates how this could be achieved and proven.

\begin{proof}[sketch]
	Suppose there is a finite \( \PA \)-proof of the empty sequent. The embedding of \( \PA \) in \( \PAo \) (\cref{oa-embed-PAo-w-bounds}) provides an explicit number \( n < ω \) such that
	\[
		\PAo \prv{ω.n}n {} ⇒ {} .
	\]
	The existence of a cut-free proof of the empty sequent, along with the various results on which \cref{oa-ce-PAo} depends, can now be established by via finitary reasoning plus transfinite induction up to an ordinal strictly smaller than \( ε_0 \), for instance the ordinal \( ω_{n+2} \) suffices.
	
	As there can be no cut-free proof of the empty sequent, there is no derivation of the empty sequent in \( \PA \).
\end{proof}

\begin{corollary}
	If \( Γ \) is a set of \( Π^0_1 \) sentences and \( Δ \) a set of \( Σ^0_1 \) sentences, then \( \PAo ⊢ Γ ⇒ Δ \) iff there is a cut-free \( \PAo \) derivation of finite height.
\end{corollary}
%
\begin{proof}
	Exercise.
\end{proof}

% ----------------------
\section{On fragments of Peano arithmetic}
\label{s-oa-refined}
% ----------------------

It is worth considering the cut elimination theorem in the context of fragments of arithmetic, namely the theories \( \PA_n \) from \cref{s-oa-sub-PA}.
Recall the convention that formulas in these calculi are expressed without \( ∨ \) or \( ∃ \).

%
%A sequent is an expression \( Γ ⇒ Δ \) without free variables using the logical language isolated at the beginning of this section.
Let \( \PAo \prvs{α}q Γ ⇒ Δ \) denote derivability in \( \PAo \) for such sequents in the usual way but with the cut referring to implication depth:
\[
  \begin{prooftree}
	\hypo{\prvs {α} q Γ ⇒ Δ , C }\hypo{ \prvs{β} q C, Σ ⇒ Λ }
	\infer2[\Cut]{ \prvs{γ}q Γ , Σ ⇒ Δ , Λ }
\end{prooftree}
\quad\text{for \( \nrk C < k \) and \( \max\setof{α,β} < γ \).}
\]

%The complication with adopting the above cut rule is that all transformations on proofs so far considered ‘reduce’
%I assume this variation of \( \PAo \) satisfies weakening, substitution and inversion lemmas with the same bounds.
\begin{exercise}
	Verify that the above version of \( \PAo \) satisfies weakening, substitution and inversion lemmas with the same bounds
\end{exercise}
%
%Given a finite sequence \( \vec A = (A_i)_{i≤k} \) of formulas, I will write \( \vec A , Γ ⇒ Δ \) for \( A_0 , …, A_n , Γ ⇒ Δ \).

%
\begin{lemma}[Refined reduction]
	Suppose \( \prvs{α}k Γ_i ⇒ Δ_i, C_i \) and \( \nrk {C_i} ≤ k \) for each \( i ≤ n \). If \( \prvs{β}k C_0 , …, C_k , Σ ⇒ Λ \), then
	\begin{gather}
		\label{oa-eqn-spec-red-lem}\tag{\dag}
		\prvs{α + β} k Γ_0 , …, Γ_n , Σ ⇒ Δ_0 , …, Δ_n , Λ .
	\end{gather}
\end{lemma}
%

%
\begin{proof}
	The overall structure of the proof will be recognisable as the strategy used in the proof of \cref{oa-red-lem-PAo}.
	I proceed by induction on \( β \).
	Suppose
	\begin{enumerate}
		\item \( \prvs{α}k Γ_i ⇒ Δ_i, C_i \) and \( \nrk {C_i} ≤ k \) for each \( i ≤ n \), and
		\item \( \prvs{β}k C_0 , …, C_k , Σ ⇒ Λ \).
	\end{enumerate}
	I refer to the \( C_i \) as the \emph{cut} formulas and will henceforth write \( \vec C \) in place of \( \setof{ C_0 , …, C_k} \).
	First, suppose no cut formula is principle in the final rule of assumption 2.
	If the sequent is initial, then \( Σ ⇒ Λ \) is initial and \eqref{oa-eqn-spec-red-lem} follows by weakening.
	Therefore, assume \( C_n \) is the principal formula in 2.
	There is a case distinction based on the form of \( C_n \).
	The focus will therefore be on assumption 2 above and
	\begin{gather}
		\label{oa-eqn-spec-red-lem-2}\tag{\ddag}
		\prvs{α } k  Γ_n ⇒ Δ_n , C_n .
	\end{gather}

	If \( C_n = ⊥ \) or is a false equation then \eqref{oa-eqn-spec-red-lem} results from applying the inversion lemma to \eqref{oa-eqn-spec-red-lem-2}.
	If \( C_n = P s \), then \( P t ∈ Λ \) for some \( ℕ ⊨ s = t \) and \eqref{oa-eqn-spec-red-lem} also follows from \eqref{oa-eqn-spec-red-lem-2} via substitution.
	The final case is that \( C_n \) is a true equation. But it is not possible for such an atomic formula to be principal in \eqref{oa-eqn-spec-red-lem}.
	
	Moving on to the non-atomic case suppose, to begin, that \( C_n = D ∧ E \).
	From 2 I obtain \( γ < β \) and \( F ∈ \setof{ D, E} \) such that
	\begin{enumerate}[resume]
%		\label{oa-eqn-spec-red-lem}\tag{\dag}
		\item \( \prvs{γ} k \vec C , F , Σ ⇒  Λ  \).
	\end{enumerate}
	Applying the inversion lemma to \eqref{oa-eqn-spec-red-lem-2} yields
	\begin{enumerate}[resume]
%		\label{oa-eqn-spec-red-lem}\tag{\dag}
		\item \( \prvs{α } k  Γ_n ⇒ Δ_n , F  \).
	\end{enumerate}
	Adding this final sequent to the list of hypotheses in 1 above, and using 3 in place of 2, I can apply the induction hypothesis (as \( γ < β \)), which derives \eqref{oa-eqn-spec-red-lem}.
	
	The quantifier case, \( C_n = ∀x D(x) \) is essentially the same argument.
	From principality of \( C_n \) and the inversion lemma I know
	\begin{enumerate}[start=3,label=\arabic*'.]
		\item \( \prvs{γ} k \vec C , D(s) , Σ ⇒  Λ \) for some \( γ < β \) and term \( s \).
		\item \( \prvs{α } k  Γ_n ⇒ Δ_n , D(s) \).
	\end{enumerate}	
	I can then deduce \eqref{oa-eqn-spec-red-lem} from the induction hypothesis by adding 4' to the list in 1 and 3' in place of 2.
	
	The final case is involves a different in the argument.
	Suppose \( C_n = D → E \).
	Hypothesis 2 and the inversion lemma yields three derivations to work from:
	\begin{enumerate}[start=3,label=\arabic*''.]
		\item \( \prvs{γ}k \vec C , E , Σ ⇒ Λ \),
		\item \( \prvs{δ}k \vec C , Σ ⇒ Λ , D \),
		\item \( \prvs{α}k D , Γ_n ⇒ Δ_n, E \),
	\end{enumerate}
	for \( γ, δ < β \).
	The first and third of these can be used with the induction hypothesis, obtaining as conclusion,
	\begin{enumerate}[resume,label=\arabic*''.]
		\item \( \prvs{α + γ } k D , Γ_0 , …, Γ_n , Σ ⇒ Δ_0 , …, Δ_n , Λ \).
	\end{enumerate}
	To derive \eqref{oa-eqn-spec-red-lem}, I need to remove the formula \( D \) in 6'' I apply a cut against a second application of the induction hypothesis, this time using 4'' (and not expanding the list in 2):
	\begin{enumerate}[resume,label=\arabic*''.]
		\item \( \prvs{ α + δ } k Γ_0 , …, Γ_n , Σ ⇒ Δ_0 , …, Δ_n , Λ , D \).
	\end{enumerate}
	As \( \nrk D < k \) a standard cut can be used between sequents 4'' and 6'', the conclusion being \eqref{oa-eqn-spec-red-lem}.
\end{proof}

As the focus is on better bounds on cut elimination, I will switch to base-$2$ exponentiation for the reduction theorem:

\begin{theorem}[Refined reduction]
	Suppose \( \prvs{α}{k+1} Γ ⇒ Δ \). Then \( \prvs{2^α}{k} Γ ⇒ Δ \).
\end{theorem}
\begin{proof}
	This argument proceeds just as usual.
	Jumping to the main case, suppose \( \prvs{α}{k+1} Γ ⇒ Δ \) is derived via cut:
	\[
		\prvs{β}{k+1} Γ ⇒ Δ , C
		\qquad
		\prvs{γ}{k+1} C, Σ ⇒ Λ 
	\]
	where \( β , γ < α \) and \( \nrk C ≤ k \).
	The induction hypothesis yields
	\[
		\prv{2^β}{k} Γ ⇒ Δ , C
		\qquad
		\prv{2^γ}{k} C, Σ ⇒ Λ 
	\]
	and the refined reduction lemma implies \( \prvs{2^α}k Γ ⇒ Δ \).
\end{proof}

\begin{theorem}[Refined cut elimination]
	If \( \prvs{α}k Γ ⇒ Δ \) then \( \prvs{γ}0 Γ ⇒ Δ \) where \( γ = 2_k^α \).
\end{theorem}


\begin{theorem}
	\label{oa-embed-IS-ce}
	If \( \PA_n ⊢ Γ ⇒ Δ \) is closed, then \( \PAo \prv{α}0 Γ ⇒ Δ \) for some \( α < ω_{n+1} \).
\end{theorem}
%
%
\begin{proof}[sketch]
	From \( \PA_n ⊢ A \) we deduce that \( \PA ⊢ {} ⇒ A \) with a proof in which all cut formulas have implication rank \( < n \) (\cref{oa-partial-ce}).
%	The reason is that the induction rule is only applied to formulas with n-rank \( ≤ n \) and finitary cut elimination is available in \( \PA \) to reduce the cut rank to formulas of the same n-rank as uses of induction.
	The embedding lemma of \( \PA \) into \( \PAo \) yields
	\( \PAo \prvs{ω.k}{n} {}⇒ A \), so \( \PAo \prv{γ}0 {}⇒ A \) where
	\[
		γ = 2_{n}^{ω.k} .
	\]
	Recall that \( 2^{ω.k} = ω^k \), whence
	\[
		γ ≤ ω_{n}^{k} < ω_{n+1} .
	\]
\end{proof}


%---------------------------------
\chapter{Transfinite induction and proof-theoretic ordinals}
\label{c-TI-and-PTO}
%---------------------------------

The final chapter is devoted to proving the optimality of \cref{oa-embed-PA-ce}/\cref{PA-consis}.
I will show how the principle of transfinite induction can be rendered in arithmetic and show that it is precisely the ordinal \( ε_0 \) that marks the boundary between the provable and unprovable instance of transfinite induction.
%Such a characterisation of Peano arithmetic
It turns out that many interesting theories extending arithmetic (including set theories and theories of second-order arithmetic) can be characterised in such a way.
The ordinal corresponding to ‘provable instances of transfinite induction’ is one of a number of ways in which ordinals can be used to describe, delineate and compare mathematical theories.
\emph{Ordinal analysis}, in a nutshell, is the isolation and comparison of such ordinal measures.


%---------------------------------
\section{Provable transfinite induction}\label{s-oa-lower}
%---------------------------------
In the present section I will define precisely one way to assign an ordinal to a theory of arithmetic and show that under this measure the \emph{proof-theoretic ordinal} of Peano arithmetic is at least \( ε_0 \).
The following section will establish that this bound is optimal.

I begin by recalling some basic order-theory.
\begin{definition}
	Let \( ≺ \) be a relation on a non-empty set \( X \).
	\( ≺ \) is:
	\begin{itemize}
		\item \emph{well-founded} if there is no infinite \( ≺ \)-descending sequence, namely no sequence \( (x_i)_{i<ω} \) such that \( x_{i+1} ≺ x_i \) for every \( i \).
		\item a \emph{well-order} if \( ≺ \) is linear and well-founded.
	\end{itemize}
\end{definition}

\begin{example}\label{oa-ex-order-type}
	The following two orderings on natural numbers are well-orders.
	The third is well-founded but not a well-order.
	\begin{align*}
		m <_1 n \;&\text{iff}\; 0 < m < n \text{, or } n = 0 \text{ and } m ≠ 0 .
		\\
		m <_2 n \;&\text{iff}\; 
		\begin{cases}
			m < n, \text{and both are even or both odd, or}
			\\
			\text{$n$ even and \( m \) odd.}
		\end{cases}
		\\
		m <_2 n \;&\text{iff}\; m = 0 \text{ and } n ≠ 0.
	\end{align*}
\end{example}

The proof of the next lemma is left as an exercise.

\begin{lemma}
	A relation \( ≺  \) on a non-empty set \( X \) if a well-order iff every non-empty \( Y ⊆ X \) has a \( ≺ \)-least element.
\end{lemma}

Let \( ≺ \) be a well-founded ordering of \( \Nat \). I define
\begin{align*}
	\otin n &≔ \supof{ \otin m + 1 }[m ≺ n]
	\\
	\ot &≔ \supof{ \otin n + 1 }[ n ∈ \Nat ]
\end{align*}
Well-foundedness ensures the above notions are well-defined.
I call \( \otin[≺] n \) the order-type of \( n \) in \( ≺ \), and \( \ot \) the order-type of \( ≺ \).
The function \( \otin {·} \colon ℕ → \Ord \) is order-preserving: \( m ≺ n \) implies \( \otin m < \otin n \) and its range is a segment of \( \Ord \).
If \( ≺ \) is a well-order then the function is also injective, whence \( \otin {·}  \) is an order-preserving enumeration of \( ℕ \) in \( \Ord \).
%
\begin{example}
	I compute the order types of natural numbers in the three orderings from \cref{oa-ex-order-type}.
	Note, for the standard ordering on \( ℕ \),
	\begin{align*}
		\otin[<]{n} &= n \quad \text{for every \( n \)}
		\\
		\ot[<] &= \sup\setof{ n + 1 }[n ∈ \Nat] = ω .
	\end{align*}
	%
The ordering \( <_1 \) satisfies
\begin{gather*}
	\otin[<_1]{n+1} = n 
	\quad\text{and}\quad
	\otin[<_1]{0} =  ω
	\\
	\ot[<_1] = ω + 1 .
\end{gather*}
%
The ordering \( <_2 \) satisfies
\begin{align*}
	\otin[<_2]{2n} &= n 
	\\
	\otin[<_2]{2n+1} &= ω + n
	\\
	\ot[<_2] &= ω + ω .
\end{align*}
The ordering \( <_3 \) satisfies
\begin{align*}
	\otin[<_3]{0} &= 0
	\\
	\otin[<_3]{n} &= 1 \;\text{ for all $n>0$}
	\\
	\ot[<_3] &= 2.
\end{align*}
\end{example}
%
\begin{lemma}
	If \( ≺ \) is a well-founded relation on \( \Nat \) then for every \( α < \ot \) there exists \( n ∈ \Nat \) such that \( \otin n = α \).
	If \( ≺ \) is a well-ordering then \( n \) is unique.
\end{lemma}
%\begin{proof}
%	Let \( O = \setof{ \otin n }[n ∈ ℕ] \) and \( α ∈ O \).
%	If \(  \)
%%	Let \( ≺ \) be well-founded and assume, to the contrary, that \(  \)
%\end{proof}

For \( ≺ \) a primitive recursive relation on \( ℕ \) the representation theorem for arithmetic (\cref{representation-thm}) presents a \( Δ_0 \) formula \( {F_≺}(a,b) \) in the language of arithmetic (without the predicate \( P \)) such that for all \( n,m ∈ ℕ \),
\[
  \PA ⊢ {F_≺}(\nm m , \nm n) \text{ iff } m ≺ n.
\]
In what follows, I will write \( a ≺ b \) for the formula \( F_≺(a,b) \), and use \( ∀x ≺ a F(x) \) as an abbreviation for the formula \( ∀x ( x ≺ a ∧ F(x) ) \).

%
\begin{definition}
	For each primitive recursive ordering \( ≺ \) and formula \( A(x) \) define formulas:
	\begin{align*}
		\Prog {A} &≔ ∀x( ∀ y ≺ x \, A(y) → A(x) )
		\\
		\TI {A, a} &≔ \Prog {A} → ∀ y ≺ a\, A(y)
		\\
		\TI {A} &≔ ∀x\, \TI {A,x}
	\end{align*}
\end{definition}
%

If \( ≺ \) is a well-order, the formula \( \Prog A \) expresses progressiveness of the set of ordinals \( \otin n \) such that \( ℕ ⊨ A(\nm n) \).
In the case \( {≺} = {<} \) is the standard ordering on \( \Nat \), this is the same as \( A(x) \) being \emph{inductive}. 
As a result, \( \TI {A, a} \) states the principle of transfinite induction for this set restricted to the segment of ordinals \( \setof{ \otin n }[ n ≺ a ] \).


%
\begin{definition}
	Let \( \Theory{T} \) be a theory in the language \( \La \). The \emph{proof theoretic ordinal} of\, \( \Theory{T} \) is the ordinal \( \pto{\Theory T} \) defined by
	\[
	  \pto{\Theory T} = \supof{\ot }[\text{\( ≺ \) is a pr.~rec.,~well-founded  and \( \Theory{T} ⊢ \TI{P} \)}]
	\]
%	Equivalently, \( \pto{\Theory T} \) is the least primitive recursively representable ordinal for which the principal transfinite induction up to this ordinal for the predicate \( P \) is not derivable.
\end{definition}
%

%The restriction to primitive recursive orderings rather than of arbitrary complexity in the definition is because of its role in the formula \( \TI P \).
\noindent
The goal of this section is a lower bound on the proof-theoretic ordinal of Peano and Heyting arithmetic:
%
\begin{theorem}\label{pto-lower-bound}
	\( \pto \PA ≥ \pto \HA ≥ ε_0 \).
\end{theorem}
%
Unpacking \cref{pto-lower-bound}, it states that there exists a sequence of well-founded relations \( \setof{ ≺_i }_i \) such that \( \supseq \ot[≺_i] = ε_0 \) and \( \HA ⊢ \TI[≺_i] P \) for each \( i \).
A sequence of well-founded relations is not, strictly speaking, necessary as a single well-ordering can be defined of order-type \( ε_0 \) and for which transfinite induction can be proven for each proper initial segment.
I leave the proof of the next lemma as an exercise.
%
\begin{lemma}
	\label{oa-wo-e0}
	There exists a primitive recursive well-ordering of \( ℕ \) of order-type \( ε_0 \) and primitive recursive functions \( \oplus \) and \( \dot{ω} \) representing addition and exponentiation respectively in the sense that \( \oplus \colon ℕ × ℕ → ℕ \) and \( \dot{ω} \colon ℕ → ℕ \) satisfy
	\[
		\otin{ m \oplus n } = \otin m \nsum \otin n
		\qquad\text{and}\qquad
		\otin{ \dot{ω}(m) } = ω^{\otin m}
	\]
	for all \( m,n ∈ ℕ \).
\end{lemma}
%
\begin{exercise}
	\label{ex-wo-e0}
	Prove \cref{oa-wo-e0}.
	Hint: Utilise the Cantor normal form theorem and a (primitive recursive) bijection between \( ℕ \) and finite sequences of \( ℕ \).
\end{exercise}
%
\begin{exercise}
	Prove the following generalisation of \cref{oa-wo-e0}: Given a primitive recursive well-ordering of order-type \( α \) construct a primitive recursive ordering of \( ℕ \) of order-type \( ε_α \).
\end{exercise}

In the following \( ≺ \) denotes the primitive recursive well-ordering of order type \( ε_0 \) given by \cref{oa-wo-e0}.
%For \( α < ε_0 \) I will write \( ≺_α \) for the restriction of \( ≺ \) to numbers whose order-type in \( ≺ \) is \( < α \).
%That is,
%\[
%  {≺_α} ≔ \setof{ (m,n) }[ m ≺ n \text{ and } \otin[≺] n < α ]
%\]
%The relation \( ≺_α \) is not a well-order because every \( n \) for which \( \otin[≺] n ≥ α \) has the same order-type, namely \( 0 \).
%
%If \( ≺ \) is represented in arithmetic in the sense that \( \PA ⊢ \nm m ≺ \nm n \) iff \( m ≺ n \), the orderings \( ≺_α \) are represented through a formula of three variables:
%\[
%	{≺_*}(a,b,c) ≔ a ≺ b ∧ b ≺ c.
%\]
%This formula clearly has the property that
%\[
%  \PA ⊢ {≺_*}(\nm m , \nm n, \nm o) \text{ iff } \otin m ≺_{\otin o} \otin n .
%\]
%Henceforth, I will write \( a ≺_c b \) for the formula\( {≺_*}(a,b,c) \) above.
%
The proof of \cref{pto-lower-bound} relies on one lemma whose proof is rather time-consuming and will be omitted:
%
\begin{lemma}
	\label{pto-lower-bound-lem}
	For every formula \( A(a) \) in the language of arithmetic, there exists a formula \( A'(a) \) such that
	\[
		\PA ⊢ ∀x\bigl( \TI[≺] {A' , x } → \TI[≺] {A, \dot{ω}^x} \bigr).
%		\PA ⊢ \Prog[≺_{ω^α}] {A} → \Prog[≺_α] {A'}.
	\]
%	In particular, \( \PA ⊢ ∀x\bigl( \TI[≺] {A' , x } → \TI[≺] {A, \dot{ω}^x} \bigr) \).
\end{lemma}

%Viewing \( A \) as a set of ordinals,  
%\cref{pto-lower-bound-lem} expresses that there is a set \( A' ⊆ \ot \) such that for all \( α < \ot \), if transfinite induction holds for \( A' \) on the segment \( \ot \) up to \( α \) and \( A \) is progressive on the segment \( \ot \), then \( A \) contains all ordinals \( < ω^α \).
%
Although I won't present the proof, it will be useful to know how \( A' \) is constructed from \( A \).
First, I present the construction as an operation on sets of ordinals.
I write \( β ⊆ O \) as shorthand for \( (∀ξ< β)ξ ∈ O \).
Given \( O ⊆ \Ord \), define \( O' \) as the class
\[
  O' = \setof{ α }[ ∀ξ ( \, ξ ⊆ O \text{ implies } ξ + ω^α ⊆ O \, )].
\]
It is not difficult to see that \( O' \) is a segment and \( α ∈ O' \) implies \( ω^α ∈ O \), from which \( ω^{α+1} ⊆ O \) quickly follows.
%For the first observation, if \( β < α ∈ O' \) then \( ω^α ⊆ O \) (pick \( ξ = 0 \)), so \( β ∈ O \).
%And if \( α ∈ O' \) then \( ω^α ⊆ O \) (again picking \( ξ = 0 \)), so \( ω^α ∈ O \) (now picking \( ξ = ω^α \)).
%
%Looking closer at the definition, it is clear that \( O' \) is simply the largest segment of \( O \) such that \( \setof{ ω^α }[α ∈ O'] \) is a segment of \( O \).
%
Expressing the operation in the language of arithmetic provides the formula \( A' \) in \cref{pto-lower-bound-lem}:
\[
	A'(a) ≔ ∀x ( ∀ y ≺ x \, A(y) → ∀ y ≺ x \oplus \dot{ω}^a \, A(y) ).
\]

The above remarks concerning the properties of \( O \) and \( O' \) can be shown in \( \PA \) to hold for \( A \) and \( A' \). 
Thus, to prove the \namecref{pto-lower-bound-lem} it suffices to show that \( \PA ⊢ \Prog {A} → \Prog {A'} \), which involves similar argumentation.


%---------------------------------
\section{Bounding provable transfinite induction}\label{s-oa-upper}
%---------------------------------

The goal of this section is the converse to \cref{pto-lower-bound}:
%
\begin{theorem}\label{pto-upper-bound}
	\( \pto{\PA} ≤ ε_0 \).
\end{theorem}
%
The proof strategy is as follows. 
I fix an arbitrary primitive recursive well-ordering \( ≺ \) and suppose that \( \TI P \) is provable in \( \PA \). 
The embedding theorem for \( \PAo \) provides an ordinal \( α < ε_0 \) and a cut-free proof of \( \TI P \) bounded above by \( α \).
Applying the inversion lemma yields, for every \( n ∈ ℕ \),
\begin{equation}
	\PAo \prv{α}0 {\Prog P ⇒ ∀x ≺ \nm n \, Px} . \tag{\dag}\label{oa-eqn-TI}
\end{equation}
I want to infer from (\dag) that \( \otin n < ε_0 \) for every \( n \).
In fact, it will be the case that (\dag) holds only if \( \otin n ≤ α \).
%The goal of this section is to establish that from \eqref{oa-eqn-TI} we can infer that \( \ot  < ε_0 \).

To that aim I will utilise an extension of \( \PAo \), called \( \PAop \), such that \eqref{oa-eqn-TI} implies
\begin{equation}
	\PAop \prv{α}0 { ⇒ ∀x ≺ \nm n Px}. \tag{\ddag}\label{oa-eqn-TI2}
\end{equation}
%
The transfer from \eqref{oa-eqn-TI} to \eqref{oa-eqn-TI2} will depend on a cut elimination theorem for \( \PAop \).
An analysis of cut-free provability in \( \PAop \) will lead me from \eqref{oa-eqn-TI2} quite directly to \( \otin n ≤ α \) for all \( n \), i.e., \( \ot ≤ α < ε_0 \).
\medskip

I begin by introducing the extension of \( \PAo \) used in \eqref{oa-eqn-TI2}.
Henceforth, let \( ≺ \) be a fixed primitive recursive well-ordering on \( \Nat \). 
%Without loss of generality, I assume \( \ot \) is a limit ordinal.
For the sake of simplifying notation, I will write \( s^ℕ \) for the value of \( s \) in the standard model, i.e., the \( n \) such that \( ℕ ⊨ \nm n = s \).
This notation presupposes that \( s \) is closed.

%
\begin{definition}
%	Let \( ≺ \) be a well-ordering on \( \Nat \). 
%	We introduce the inference rule \( (≺) \):
	The rule \( (≺) \) comprises all instances of the inference
	\[
	  \Infer{Γ \sa Δ , P \nm n \quad\text{for every \( n ≺ s^\Nat \)}}[\( ≺ \)]{ Γ \sa Δ , Ps }
	\]
	The infinitary sequent calculus \( \PAop \) extends the axioms and rules of \( \PAo \) by the inference \( (≺) \) above. 
	The relation \( \PAop \prv{α}k Γ ⇒ Δ \) is given as in \cref{d-bound-omega-logic}.
\end{definition}

In general, the rule \( (≺) \) will have infinitely many premises like the \( ω \)-rules.
For instance, if there is an element \( m \) with order-type \( ω \), and \( M = \setof{ n ∈ ℕ }[n ≺ m] \) then one instance of the rule is
\[
  \Infer{ Γ ⇒ P \nm n \text{ for all } n ∈ M }[\( ≺ \)]{ Γ ⇒ P \nm m}
\]



The next three lemmas provide the motivation for this extension of \( \PAo \).
%
\begin{lemma}\label{oa-PAo-in-PAop}
	If \( \PAo \prv{α}k Γ ⇒ Δ \) then  \( \PAop \prv{α}k Γ ⇒ Δ \).
\end{lemma}
\begin{proof}
	Immediate.
\end{proof}
%
\begin{lemma}\label{oa-PAop-Prog}
	\( \PAop \prv{ω}0 {{}⇒ \Prog P} \).
\end{lemma}
%
\begin{proof}
	Recall that \( \Prog P = ∀ x ( ∀ y ( y ≺ x → Py ) → P x ) \). 
	Let \( k \) be the constant given by \cref{p-PAo-S1} such that \( \PAo \prv k 0 { ⇒ \nm m ≺ \nm n }\) for all \( m ≺ n \). 
	For every \( n ∈ \Nat \) I obtain the following derivation in \( \PAo \) (with implicit application of weakening) for all \( m , n ∈ ℕ \) satisfying \( m ≺ n \):
	\begin{prooftree*}
	  \axiom[\idRule]{ \prv00 P\nm m ⇒ P \nm m }
	  \subproof{\prv k0 {}⇒ \nm m \prec \nm n }
	  \infer[separation=3em]2[\impL]{\prv{k+1}0 \nm m ≺ \nm n → P \nm m  &⇒ P\nm m }
	  \infer1[\faL]{\prv{k+2}0 ∀y ≺ \nm n \, Py &⇒ P \nm m}
	\end{prooftree*}
	Continuing the derivation in \( \PAop \):
	\begin{prooftree*}
		\subproof{\prv{k+2}0 ∀y ≺ \nm n \, Py ⇒ P \nm m \text{ for all \( m ≺ n \)} }
		\infer1[\( ≺ \)]{\prv{k+3}0 ∀y ≺ \nm n \, Py ⇒ P\nm n }
		\infer1[\impR]{\prv{k+4}0 {} ⇒ ( ∀y ≺ \nm n \, Py ) → P\nm n }
		\hypo{\text{for every } n}
		\infer2[\faR]{\prv{k+5}0 {} ⇒ \Prog P }
	\end{prooftree*}
	An application of bound weakening completes the proof.
\end{proof}
%
\begin{lemma}[Refined embedding lemma]\label{oa-PAop-embedding}
	If\, \( \PA ⊢ \TI P \) then there exists \( k < ω \) such that for all \( n ∈ ℕ \), 
	\[ \PAop \prv{ω^2}k {{}⇒ ∀ y ≺ \nm n \, P y }. \]
\end{lemma}
%
\begin{proof}
	The embedding lemma for \( \PAo \) (\cref{oa-embed-PAo-w-bounds}) and inversions yields a \( k < ω \) such that for all \( n ∈ ℕ \):
	\[ 
		\PAo \prv{ω.k}k { \Prog P ⇒ ∀y ≺ \nm n \, P y } 
	\]
	\Cref{oa-PAop-Prog} and a pair of cuts completes the argument.
\end{proof}

Paired with cut elimination for \( \PAop \), treated in the next section, the \namecref{oa-PAop-embedding} above yields \eqref{oa-eqn-TI2}.
Under the assumption of cut elimination (with the same bounds as \( \PAo \)), just one lemma stands before an optimal upper bound on the proof-theoretic strength of \( \PA \).
This is the lemma below.

Since \( ≺ \) is a fixed well-ordering, for \( α < \ot \) I will write \( \cd{α} \) for the numeral \( \nm n \) such that \( α = \otin n \).
%
\begin{lemma}[Bounding lemma]
	\label{oa-bounding-lemma}
	Let \( α_1 , …, α_m , β_0, …, β_n < \ot \). If
	\[ \PAop \prv{γ}0 P\cd{α_1}, … , P\cd{α_m} ⇒ P\cd{β_0} , …, P\cd{β_n} \]
	then
	\(%begin{equation}\label{eqn-bounding-lemma}
		 \min\setof{β_0, …, β_n } ≤ \max \setof{α_1, …, α_m } + γ  . %\tag{\ast}
	\)%end{equation}
\end{lemma}
%
\begin{proof}
	Induction on \( γ \).
	If \( P\cd{α_1}, … , P\cd{α_m} ⇒ P\cd{β_0} , …, P\cd{β_n} \) is initial, then \( α_i = β_j \) for some \( i \) and \( j \) and the claim holds vacuously.
	If, however, the sequent is not initial then, as the derivation is cut-free, the final rule applied must be an instance of \( (≺) \). 
	I can assume, without loss of generality, that the principal formula is \( P \cd{β_n} \), i.e., that the inference applied is
	\begin{prooftree*}
		\hypo{ P\cd{α_1}, … , P\cd{α_m} ⇒ P\cd{β_0} , …, P\cd{β_n} , P\cd{δ} \quad\text{for all \( δ < β_n \)} }
		\infer1[\( ≺ \)]{ P\cd{α_1}, … , P\cd{α_m} ⇒ P\cd{β_0} , …, P\cd{β_n} }
	\end{prooftree*}
	For each \( δ <  β_n \) the corresponding premise has a cut-free derivation of height \( < γ \).
	That is, for every \( δ < β_n \) there exists \( γ_δ < γ \) such that 
	\[
	  \PAop \prv{γ_δ}0 P\cd{α_1}, … , P\cd{α_m} ⇒ P\cd{β_0} , …, P\cd{β_n} , P\cd{δ}.
	\]
	Let \( β = \min \setof{β_0, …, β_n } \) and \( α = \max \setof{α_1, …, α_m } \).
	The induction hypothesis implies that 
	\begin{equation}
		\label{eqn-bouding-lemma-sub}
		\text{for every \( δ < β_n \), } \min\setof{ β , δ } ≤ α + γ_δ.
	\end{equation}
	%
	Consider two cases. 
	First, suppose \( β < β_n \). 
	Choosing \( δ = β \) in \eqref{eqn-bouding-lemma-sub} yields
	\[
		β ≤ α + γ_β < α + γ,
	\]
	whereby the claim holds as desired.
	Otherwise, \( β = β_n \), and
	\begin{align*}
		β = \supof{ δ + 1 }[ δ < β ] &≤ \supof{ α + γ_δ + 1 }[ δ < β ] &&\text{by \eqref{eqn-bouding-lemma-sub}}
	  \\
	  &≤ α + \supof{ γ_δ + 1 }[ δ < β ] &&\text{continuity}
	  \\
	  &≤ α + γ.
	 \end{align*}
%	 as required.
\end{proof}

\begin{proof}[of \cref{pto-upper-bound} (assuming cut elmination)]
	Let \( ≺ \) be any primitive recursive well-order of \( ℕ \) and suppose \( \PA ⊢ \TI P \).
	Let \( α = \ot \).
	The refined embedding lemma~(\cref{oa-PAop-embedding}) provides a finite \( k \) such that for all \( n ∈ ℕ \)
	\[
		\PAop \prv{ω^2} k {}⇒ ∀ y ≺ \nm n\, Py .
	\]
	In particular, for every \( β < α \),
	\[
		\PAop \prv{ω^2} k {}⇒ P \cd {β} .
	\]
	Cut elimination for \( \PAop \) provides an ordinal \( γ < ε_0 \) such that (see next \namecref{s-oa-PAop-ce}) for every \( β < α \)
	\[
		\PAop \prv{γ} 0 {}⇒ P\cd{β} .
	\]
	The bounding lemma ensures that \( β ≤ γ \), meaning that \( \ot ≤ γ + 1 < ε_0 \).
\end{proof}

%---------------------------------
\section{Cut elimination, revisited}
\label{s-oa-PAop-ce}
%---------------------------------

What remains is to confirm cut elimination for the extended calculus \( \PAop \).
The reader can confirm that weakening and substitution remain admissible in this extension.
%
\begin{lemma}[Weakening Lemma]
	If \( \PAop \prv{α}k Γ ⇒ Δ \) and \( Γ'⇒Δ' \) is closed then \( \PAop \prv{α}k Γ' , Γ ⇒ Δ, Δ' \).
\end{lemma}
%
\begin{lemma}[Substitution Lemma]
	Let \( Γ(a) ⇒ Δ(a) \) be a sequent with \( a \) the only free variable, and let \( s \) and \( t \) be closed terms such that \( \Nat ⊨ s = t \). If \( \PAop \prv{α}k Γ(s) ⇒ Δ(s) \) then \( \PAop \prv{α}k Γ(t) ⇒ Δ(t) \).
\end{lemma}
%
%\begin{lemma}[Contraction Lemma]
%	If \( \PAop \prv{α}k A , A , Γ ⇒ Δ \) then \( \PAop \prv{α}k A , Γ ⇒ Δ \).
%	If \( \PAop \prv{α}k Γ ⇒ Δ , A , A \) then \( \PAop \prv{α}k Γ ⇒ Δ , A \).
%\end{lemma}
%
\begin{exercise}
	Prove the weakening and substitution lemmas for \( \PAop \).
\end{exercise}

Precisely the same formulation of the reduction lemma also holds, but here there are some notable changes to the proof.
%In fact, it is simple to see how 
%
I present only the ‘simple’ version of this result and leave the quantifier-relevant form for the reader.
%
\begin{lemma}[Reduction lemma]
	Suppose \( \PAop \prv{α}k Γ ⇒ Δ , C \) and \( \PAop \prv{β}k C, Γ ⇒ Λ \).
	If \( \rk C = k \) then \( \PAop \prv{α\nsum β}k Γ ⇒ Δ,Λ \).
\end{lemma}
%
\begin{proof}
	Like the inferences of \( \PAo \), the rule \( (≺) \) has the property of just one principal formula,
	namely
	\[
%		\text{If} \quad
		\Infer{Γ_i ⇒ Δ_i \;\text{for } i ∈ I}{Γ⇒Δ}
	\]
	is an instance iff there is \( F ∈ Δ \) such that
	\[
		\Infer{ Σ, Γ_i ⇒ Δ_i , Λ  \;\text{for } i ∈ I}{ Σ ⇒ Λ, F}
	\]
	is an instance for all \( Σ \) and \( Λ \).

	As such, the new rule does not affect the part of the argument where \( C \) is not principal in one of the assumptions.
	So it suffices to treat the case in which \( C = P s\) for some \( s \) and is principal in both assumptions.
	But if \( Ps \) is principal in the proof \( \prv{α}k Γ ⇒ Δ , C \) then inference deriving this sequent is either initial (whence \( P t ∈ Γ \) for \( ℕ ⊨ s = t \)) or the conclusion of \( (≺) \).
	In the latter case, however, it is not clear how to use the premises of the rule against the second assumption.
	Fortunately, though, it is not necessary because we are assuming that \( C \) is principal in the second hypothesis, \( \prv{β}k C,Γ ⇒ Λ \).
	For this to be the case, \( C , Γ ⇒ Λ \) must be an initial sequent, meaning that \( P t ∈ Λ \) such that \( ℕ ⊨ s = t \).
	The substitution lemma applied to the \emph{first} hypothesis, shows derivability of \( \prv{α}k Γ ⇒ Δ , Λ \).	
\end{proof}

\begin{exercise}
	Complete the proof of the reduction lemma.
\end{exercise}

\begin{exercise}
	Formulate and prove a quantifier-relevant formulation of the reduction lemma for \( \PAop \) following the schema of \cref{oa-red-lem-PAo-quant}.
\end{exercise}
%
\begin{theorem}[Cut elimination]
	If \( \PAop \prv{α}k Γ ⇒ Δ \) then \( \PAop \prv{ω_k^α}0 Γ ⇒ Δ  \).
\end{theorem}
%
\begin{proof}
	The proof proceeds precisely as before.
\end{proof}

%
%---------------------------------
\section{Characterisation of provable transfinite induction}
%---------------------------------

Combining the results in this chapter:
%
\begin{theorem}[Proof-theoretic characterisation theorem]
	The proof-the\-oretic ordinal of Peano and Heyting arithmetic is \( ε_0 \).
\end{theorem}
%
\begin{proof}
	As \( ε_0 ≤ \pto{\HA} ≤ \pto{\PA} \) by \cref{pto-lower-bound} and \( \pto{\PA} ≤ ε_0 \) by \cref{pto-upper-bound}.
\end{proof}
%
\begin{corollary}[Independence of transfinite induction]
	There is a primitive recursive well-ordering \( ≺ \) on \( \Nat \) and a formula \( A \) in the language of arithmetic such that \( \PA ⊬ \TI A \).
\end{corollary}
%

The following will be a consequence of \cref{oa-embed-IS-ce}, but it needs to be refined.
%
\begin{theorem}
	The proof-the\-oretic ordinal of \( \IS_n \) for \( n > 0 \) is \( ω_{n+1} \).
\end{theorem}
%

\backmatter

\bibliographystyle{plain}
\bibliography{bib/references.bib}

\end{document}
