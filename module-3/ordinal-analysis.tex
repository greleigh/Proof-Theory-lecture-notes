
%---------------------------------
\chapter{Ordinal analysis of arithmetic}\label{c-oa-PAo}
%---------------------------------
Ordinals will now be used to measure the \emph{height} of \( ω \)-proofs.
I begin by recalling the infinitary sequent calculi for arithmetic from the end of \cref{c-oa-arith}.
%
\begin{definition}%\label{d-PAomega}
	\( \PAo \) is the sequent calculus given by the following:
	\begin{itemize}
		\item sequents comprise formulas in the language of arithmetic (with equality).
		\item Initial sequents are \emph{closed} sequents of the form
		\begin{itemize}
			\item[(\botL)] \( ⊥, Γ ⇒ Δ \)
			\item[(\idRule)] \( Ps, Γ ⇒ Δ , Pt \) if \( ℕ ⊨ s = t \)
			\item[(\eqR)] \( Γ ⇒ Δ , s = t \) if \( ℕ ⊨ s = t \)
			\item[(\eqL)] \( s = t , Γ ⇒ Δ \) if \( ℕ ⊭ s = t \)
		\end{itemize}
		\item Inference rules are rules of \( \Gc \) but restricted to closed sequents and with \( \faR \) and \( \exL \) replaced by the two \( ω \)-rules:\note{\( \Gc \) or more general \( \Logic{G3} \)?}
		\begin{itemize}
			\item[(\omR)] \begin{prooftree} \hypo{ Γ ⇒ Δ , F(\nm n) \ \text{for every } n ∈ ℕ } \infer1{ Γ ⇒ Δ , ∀x F(x) }\end{prooftree}
			\item[(\omL)] \begin{prooftree} \hypo{ F(\nm n) , Γ ⇒ Δ \ \text{for every } n ∈ ℕ } \infer1{ ∃x F(x) , Γ ⇒ Δ }\end{prooftree}
		\end{itemize}
	\end{itemize}
%	Writing \( \PAo ⊢ Γ ⇒ Δ \) expresses that there is an \( ω \)-proof of \( Γ ⇒ Δ \) according to the above rules. In other words, there exists a well-founded tree labelled by sequents such that each leaf is an initial sequent and that each inner vertex together with its immediate successors in the tree forms a correct application of a rule of the calculus listed above.
%	
	\( \HAo \) is the same calculus but restricted to intuitionistic sequents.
\end{definition}
%
\begin{definition}\label{d-bound-omega-logic}
	Let \( \Theory{T} \) be \( \PAo \), \( \HAo \) or an extension of either calculus by rules that are at most \( ω \)-branching.
	The ternary relation \( \Theory{T} \prv{α}k Γ ⇒ Δ \), between a sequent \( Γ ⇒ Δ \), an ordinal \( α \) and \( k < ω \), is defined by transfinite recursion on the rules of \( \Theory{T} \):
	\begin{enumerate}
		\item If \( Γ ⇒ Δ \) is an initial sequent of \( \Theory{T} \), then \( \Theory{T} \prv{α}k Γ ⇒ Δ \) for all \( α \) and \( k \);
		\item For each inference \( (*) \) of \( \Theory{T} \) except cut of the form
		\[
			\Infer{\setof{ Γ_i ⇒ Δ_i }[i ∈ I]}[\( * \)]{ Γ ⇒ Δ }
		\]
		\( \Theory{T} \prv{α}k Γ ⇒ Δ \) holds if \( \Theory{T} \prv{α_i}k Γ_i ⇒ Δ_i \) and \( α_i < α \) for all \( i ∈ I \);
		\item If \( \Theory{T} \prv{α_0}k Γ ⇒ Δ , C \) and \( \Theory{T} \prv{α_1}k C , Γ ⇒ Σ \) for \( α_0,α_1 < α \) and \( \rk C < k \), then \( \Theory{T} \prv{α}k Γ ⇒ Δ, Σ \).
	\end{enumerate}
%	If \( \Theory{T} \) is a calculus of intuitionistic sequents, \( \Theory{T} \prv{α}k Γ ⇒ A \) is defined via the same conditions but with a modification of the final clause:
%	\begin{enumerate}[resume]
%		\item If \( \Theory{T} \prv{α_0}k Γ ⇒ C \) and \( \Theory{T} \prv{α_1}k C , Γ ⇒ A \) for \( α_0,α_1 < α \) and \( w(C) < k \), then \( \Theory{T} \prv{α}k Γ ⇒ A \).
%	\end{enumerate}
	Given \( \Theory{T} \prv{α}k Γ ⇒ Δ \) I will write that \( Γ ⇒ Δ \) is derivable (in \( \Theory T \)) with height \( ≤ α \) and cut rank \( ≤ k \).
\end{definition}
%

%Some of the terminology  to \( ω \)-proofs as before.

There is no requirement of minimality of \( α \) and \( k \) in the above definition. 
So the relation \( \prv{α}k \) is monotone in \( α \) and \( k \):
%
\begin{lemma}\label{oa-PAo-weak1}
	If \( α ≤ β \) and \( k ≤ l \) then \( \Theory{T} \prv{α}k Γ ⇒ Δ \) implies \( \Theory{T} \prv{β}l Γ ⇒ Δ \).
\end{lemma}
%
\begin{proof}
	By transfinite induction on \( α \). If \( Γ ⇒ Δ  \) is an initial sequent, the result is immediate.
	Otherwise, there is an inference rule of \( \Theory T \)
	\[
		\Infer{\setof{ Γ_i ⇒ Δ_i }[i ∈ I]}[\( * \)]{ Γ ⇒ Δ }
	\]
	and ordinals \( α_i < α \) such that \( \smash{\Theory T \prv{α_i}k Γ_i ⇒ Δ_i} \) for each \( i ∈ I \).
	The induction hypothesis implies that \( \Theory T \prv{α_i}l Γ_i ⇒ Δ_i \) for each \( i \), whereby \( \Theory T \prv{β}l Γ ⇒ Δ \) obtains.
\end{proof}

\Cref{oa-PAo-weak1} operates in the background of the majority of the results to follow. For that reason I will not make any explicit reference to the \namecref{oa-PAo-weak1}.

\begin{example}
	\tbw
\end{example}

\begin{lemma}\label{oa-HA-good}
	If \( \HAo \prv{α}k Γ ⇒ A \) then this fact can be observed by use of sequents of the form \( Σ ⇒ B \) (i.e., exactly one formula on the right).
\end{lemma}

\begin{exercise}
	Assign ordinal bounds on the \( ω \)-proofs of \( A , Γ ⇒ Δ , A \) constructed in exercise~\ref{ex:oa-id-simple}.
\end{exercise}

Revisiting the Embedding lemma (lemma \ref{oa-embed-weak}) it is possible provide ordinal bounds on the size of the resulting \( ω \)-proof.
Let \( α.k = \underbrace{α + ⋯ + α }_k \).

\begin{lemma}[Refined embedding lemma]\label{oa-embed-PAo-w-bounds}
	Suppose \( \PA ⊢ Γ ⇒ Δ \) and \( Γ ⇒ Δ \) is closed. Then there is \( n,k < ω \) such that \( \PAo \prv{ω.n}k Γ ⇒ Δ \) where \( ω .n = ω + ⋯ + ω \) (\( n \) times).
	Likewise, \( \HA \) into \( \HAo \).
\end{lemma}
%
\begin{exercise}
	Prove the refined embedding lemma following the schema of embedding lemma at the end of \cref{c-oa-arith}.
\end{exercise}
%
The next lemma hints at part of the usefulness of the \( ω \)-rule with the ability to isolate finitary reasoning from infinitary reasoning.
The result will be useful in \cref{s-oa-upper}.
%
\begin{proposition}\label{p-PAo-S1}
	Let \( A(a_1,…, a_k) \) be a \( Σ_1 \) formula. 
	There exists \( m < ω \) such that for all \( n_1, …, n_k ∈ \Nat \), 
	\[ \text{if }\thinspace  \Nat ⊨ A(\nm {n_1}, …, \nm {n_k} ) \text{ then } \HAo \prv m0 {⇒ A(\nm {n_1}, …, \nm {n_k})} .
	\]
\end{proposition}
%
\begin{proof}
	By induction on the rank of \( A \).
\end{proof}

Henceforth, I will omit explicit mention of \( \PAo \) and write \( \prv{α}k Γ ⇒ Δ \) to mean \( \PAo \prv{α}k Γ ⇒ Δ \).
The following results are stated only for \( \PAo \) but apply equally to \( \HAo \) in the expected way.
Admissibility of weakening becomes 
%
\begin{lemma}[Weakening Lemma]
	If \( \prv{α}k Γ ⇒ Δ \) and \( Γ' ⇒ Δ' \) is closed then \( \prv{α}k Γ' , Γ ⇒ Δ, Δ' \).
%	Likewise for \( \HAo \).
\end{lemma}
%
%\begin{proof}
%	By (transfinite) induction on \( α \).
%\end{proof}
\begin{exercise}
	Prove the weakening lemma.
\end{exercise}

The substitution lemma for \( \PAo \) takes a different formulation from previously. 
As sequents are closed, the correct formulation for \( ω \)-proofs is that provability depends on the \emph{value} of terms, not their \emph{form}.

\begin{lemma}[Substitution Lemma]
	Let \( Γ(a) ⇒ Δ(a) \) be a sequent and \( s \) and \( t \) be closed terms such that \( \Nat ⊨ s = t \). If \( \prv{α}k Γ(s) ⇒ Δ(s) \) implies \( \prv{α}k Γ(t) ⇒ Δ(t) \).
\end{lemma}
%
\begin{proof}
%	Analogous argument. I will treat the case of \( \exR \): 
	Suppose 
	\( \prv{α}k Γ(s) ⇒ Δ(s) \) and \( \Nat ⊨ s = t \).
	Let \( Γ(a) ⇒ Δ(a) \) be any sequent with at most \( a \) free.
	If \( Γ(s) ⇒ Δ(s) \) is initial then a case distinction on the different forms this sequent can take confirms that \( Γ(s) ⇒ Δ(s) \) is also initial provided \( \Nat ⊨ s = t \). The other case proceed by transfinite induction on \( α \).
\end{proof}

The final ingredient is the inversion lemma, the statement of which has the same form as before with two new cases treating equality.

\begin{lemma}[Inversion lemma]
	\label{oa-inversion}\ 
	\begin{enumerate}
		\item If \( \prv{α}k Γ ⇒ Δ , ⊥ \) then \( \prv{α}k Γ ⇒ Δ \).
		\item If \( \prv{α}k s = t , Γ ⇒ Δ \) and \( ℕ ⊨ s = t \) then \( \prv{α}k Γ ⇒ Δ \).\label{oa-inversion-eq1}
		\item If \( \prv{α}k Γ ⇒ Δ , s = t \) and \( ℕ ⊭ s = t \) then \( \prv{α}k Γ ⇒ Δ \).
		\item If \( \prv{α}k Γ ⇒ Δ , ∀x F(x) \) then \( \prv{α}k Γ ⇒ Δ , F(s) \) for every closed term \( s \).\label{oa-inversion-fa}
		\item If \( \prv{α}k ∃x F(x) , Γ ⇒ Δ \) then \( \prv{α}k F(s) , Γ ⇒ Δ \) for every closed term \( s \).
		\item Analogous inversion principles for the rules \( \disjL \), \( \conjR \), \( \impR \) and \( \impL \).
	\end{enumerate}
\end{lemma}
\begin{proof}
	I show cases \ref{oa-inversion-eq1} \& \ref{oa-inversion-fa}.
	
	\ref{oa-inversion-eq1}. By induction on \( α \). Suppose \( \prv{α}k s = t , Γ ⇒ Δ \) and \( ℕ ⊨ s = t \). If \( s = t , Γ ⇒ Δ \) is initial then so is \( Γ ⇒ Δ \). The other cases are straightforward because the equation \( s = t \) cannot be the principal formula of any rule.
	For if \( s = t , Γ ⇒ Δ \) is not initial, then there are sequents \( \setof{ Γ_i ⇒ Δ_i}[i<ω] \)
	and ordinals \( \setof{α_i}[i<ω] \) such that
	\begin{enumerate}[label=(\alph*)]
		\item \( \prv{α_i}k s=t , Γ_i ⇒ Δ_i \) for each \( i < ω \),
		\item \( α_i < α \) for all \( i \),
		\item \( \setof{ Γ_i ⇒ Δ_i}[i<ω] \) enumerate all premises of an inference of \( \PAo \) whose conclusion is \( Γ ⇒ Δ \).
	\end{enumerate}
	In the case of unary or binary rules, \( Γ_i = Γ_{i+1} \) and \( Δ_{i} = Δ_{i+1} \) for all \( i > 0 \) or \( 1 \). But in the case of either of the two \( ω \)-rules, the sequents enumerate the infinitely many premises.
	By (a)--(c) and the induction hypothesis, \( \prv{α}k Γ ⇒ Δ \) holds as desired.
	
	\ref{oa-inversion-fa}. The argument is a direct generalisation of the finitary inversion lemma. Suppose \( \prv{α}k Γ ⇒ Δ , ∀x F(x) \). If this sequent is initial, then so is \( Γ ⇒ Δ , F(s) \) for every closed term \( s \).
	The rest of the argument proceeds, essentially, as above by a case distinction on the inferences through which \( \prv{α}k Γ ⇒ Δ , ∀x F(x) \) can be derived.
	The case of \( \faR \) with \( ∀x F(x) \) principal bears treatment.
	The premises of this inference can be assumed to have the form \( Γ ⇒ Δ , ∀x F(x) , F(\nm n) \). An application of the induction hypothesis (to each premise) yields \( \prv{α}k Γ ⇒ Δ , F(\nm n) \) for every \( n \). If the desired closed term \( s \) is a numeral, this case is complete. Otherwise, let \( n \) be the value of \( s \), i.e., \( n ∈ ℕ \) is such that \( ℕ ⊨ \nm n = s \).
	The substitution lemma then yields \( \prv{α}k Γ ⇒ Δ , F(s) \).
\end{proof}

%---------------------------------
\section{Infinitary cut elimination}\label{s-oa-cutelim}
%---------------------------------

I begin with the transfinite version of the reduction lemma.
Recall, this is statement that borderline cuts can be simulated at the cost of increasing the depth of the proof by a controlled amount.
In the finitary case the depth increase was, in the case of classical logic, \( m + n \) where \( m \) and \( n \) bounded the depth of the two cut premises.

Lifting the statement of the reduction lemma to the transfinite realm is reasonably straightforward.
Given premises of a borderline cut of height \( α \) and \( β \) respectively, the cut can be simulated by a height of \( α \nsum β \).
The use of natural sum is crucial to the argument: the lifting of the finitary argument requires the resulting bound to be order-preserving in both arguments, a property we know fails for traditional ordinal sum \( α + β \).

%In fact, I will give two proofs of the reduction lemma.
%The first is the version just described: a direct lifting of the finitary argument to \( ω \)-proofs. 
%Some steps of the proof clearly require new insights, such as passing cuts through \( ω \)-rules and the new initial sequents with equations.
%But, by and large, the design of \( \PAo \) is such that the transfinite element is straightforward once one knows what to expect.
%
%Thus, the reduction lemma I will prove will be:

\begin{lemma}[Reduction lemma for \( \PAo \)]\label{oa-red-lem-PAo}
	Suppose \( \prv{α}k Γ ⇒ Δ , C \) and \( \prv{β}k C , Σ ⇒ Λ \).
	If \( \rk C ≤ k \) then \( \prv{α\nsum β }k Γ, Σ ⇒ Δ, Λ \).
\end{lemma}

The reader may surprised to know that there is a great deal of flexibility in proofs of the reduction lemma, which I will demonstrate by presenting a slightly different strategy than we used for in the analysis of classical predicate logic.
%The statement of the lemma already offers a minor, though insignificant, departure from previously by allowing \( \rk C < k \). 
%Provided that \( α , β \) are non-zero, \( \max\setof{ α , β } + 1 ≤ α \nsum β \), so whether.

%The second formulation will be presented after the proof of the above \namecref{oa-red-lem-PAo}.
%In short, it mitigates a shortcoming of the reduction lemma as described in its exaggeration of the complexity of cut elimination.
%The transfinite reduction lemma is optimal in the same way as the finitary form: Sequents can be readily chosen whose shortest cut-free proofs require a jump in ordinal height matching that expressed by the reduction lemma and its immediate corollaries.
%But by treating each logical connective as equally ‘expensive’, the reduction lemma ignores the true source of ‘large’ proofs: \emph{alternation} between \emph{positive} and \emph{negative} statements.
%
%This alternative version of the reduction lemma is closely related to the form proved in exercise~\ref{ex-red-lem-special}.

\begin{proof}%[of \cref{oa-red-lem-PAo}]
%	Suppose:
%	\begin{enumerate}
%		\item \( \prv{α}k Γ ⇒ Δ , C \).
%		\item \( \prv{β}k C , Σ ⇒ Λ \).
%	\end{enumerate}
%
	The proof branches into cases depending on the form of \( C \).
	In each case I will establish \( \prv{α \nsum β} k Γ, Σ ⇒ Δ, Λ \) but the induction will proceed over either \( α \) or \( β \) (depending on the case) rather than on the sum \( α \nsum β \).
	If the principal connective of \( C \) is among \( \setof{ ⊥ , ∀ ,  ∧ , → } \) I will refer to \( C \) as \emph{locally negative} (cf.~Canvas assignment no.~4).
	Otherwise, \( C \) will be \emph{locally positive}.
	
%	\paragraph{Case I: \( C \) is \( ⊥ \) or an equation.} This case can be dispensed with directly. If \( C = ⊥ \) or a false equation then \( \prv{α}k Γ ⇒ Δ \) is a consequence of the inversion lemma;  otherwise \( \prv{β}k Σ ⇒ Λ \).
%	In any case, weakening yields \( \prv{α\nsum β }k Γ, Σ ⇒ Δ , Λ \).
	
	\paragraph{Case I: \( C \) is atomic or locally negative.}
	Here I proceed by induction on \( β \) and show that \( \prv{ α \nsum β }k Γ, Σ ⇒ Δ , Λ \).
	I present two subcases:
	
	\( C = ∀x D(x) \). If \( C , Σ ⇒ Λ \) is initial then \( Σ ⇒ Λ \) is also initial and the claim holds by weakening.
	Otherwise, consider the rule that derives \( \prv{β}k C , Σ ⇒ Λ \).
	If the principal formula of the rule is \emph{not} \( C \) then the induction hypothesis can be applied directly to its premises and the rule re-applied to derive the desired sequent with correct bounds.
	If, however, the rule is \( \faL \) with \( C \) principal, the above argument does not work. But in this case there is \( γ < β \) and term \( t \) such that 
	\[
	  \prv{γ}k D(t) , C , Σ ⇒ Λ .
	\]
	The induction hypothesis yields
	\[
		\prv{α \nsum γ}k D(t) , Γ , Σ ⇒ Δ , Λ .
	\]
	From the inversion lemma (part \ref{oa-inversion-fa}) I know also that \( \prv{α}k Γ ⇒ Δ , D(t) \). Since \( \rk {D(t)} < \rk{ C} = k \), an application of cut yields 
	\( \prv{ α \nsum β }k Γ, Σ ⇒ Δ, Λ \).
	
	\( C = D → E \). I employ a similar argument as above but with a subtle difference in how the induction hypothesis is applied to account for the binary connectives. 
	By the previous argument I can jump directly to the case that \( C \) is principal in the derivation of \( \prv{β}k C , Σ ⇒ Λ \), for which there exist \( γ , δ < β \) and \( Λ = Λ_0 ∪ Λ_1 \) satisfying
	\begin{enumerate}
		\item \( \prv{γ}k C , Σ ⇒ Λ_0, D \).
		\item \( \prv{δ}k C , E , Σ ⇒ Λ_1 \).
	\end{enumerate}
	I start by applying the inversion lemma to my three hypotheses:
	\begin{enumerate}[resume]
		\item \( \prv{α}k D , Γ ⇒ Δ , E \).
		\item \( \prv{γ}k Σ ⇒ Λ_0, D \).
		\item \( \prv{δ}k E , Σ ⇒ Λ_1 \).
	\end{enumerate}
	Then I apply the induction hypothesis between the sequents in 3 and 5 (using ‘cut’ formula \( E \)):
	\begin{enumerate}[resume]
%		\item \( \prv{α+γ}k Γ , Σ ⇒ Δ ,Λ_0, D \).
		\item \( \prv{α \nsum δ}k D , Γ, Σ ⇒ Δ , Λ_1 \).
	\end{enumerate}
	I can now combine 6 and 3 with a (standard) cut:
	\[
		\prv{α \nsum β}k Γ , Σ ⇒ Δ ,Λ.
	\]
	The conjunction subcase is left to the reader.
	
	Case II: \( C \) is locally positive.
	This case is symmetric to the previous and left to the reader.
\end{proof}

%The observant reader will have noticed that the argument for the implication case could be simplified by applying the inversion lemma
% ------------------
\begin{figure}
	\centering
	\begin{prooftree}
%		\hypo{\prv{α}k Γ ⇒ Δ , C }
		\hypo{\prv{γ}k C , Σ ⇒ Λ_0, D}
		\infer[dashed]1[IL]{\prv{γ}k Σ ⇒ Λ_0, D}
%		\infer[dashed]2[IH]{\prv{α+γ}k Γ , Σ ⇒ Δ, Λ_0, D}
			
		\hypo{\prv{α}k Γ ⇒ Δ , C }
		\infer[dashed]1[IL]{\prv{α}k D , Γ ⇒ Δ , E }
			\hypo{\prv{δ}k C , E , Σ ⇒ Λ_1 }
			\infer[dashed]1[IL]{\prv{δ}k E , Σ ⇒ Λ_1 }
		\infer[dashed]2[IH]{\prv{α \nsum δ}k D , Γ , Σ ⇒ Δ , Λ_1 }
		\infer2[\Cut]{\prv{α \nsum β}k Γ , Σ ⇒ Δ , Λ }
	\end{prooftree}
	\caption{Illustration of the proof method in the reduction lemma for the case \( C = D → E \); IL = ‘inversion lemma’ and IH = ‘induction hypothesis’.}
	\label{f-oa-PAo}
\end{figure}
% ------------------

\begin{exercise}
	Complete the preceding proof.
\end{exercise}

\begin{exercise}
	Formulate and prove a reduction lemma for \( \HAo \) following the proof scheme above.
\end{exercise}

\begin{exercise}
	Give an alternative proof of \cref{oa-red-lem-PAo} using the proof strategy from the reduction lemma for \( \Gc \) (\cref{ce-red-lem-C}).
\end{exercise}

In the implication subcase of case II in the proof above, I used the induction hypothesis to simulate a cut on the formula \( E \)

\begin{theorem}[Reduction theorem for \( \PAo \)]\label{oa-red-thm-PAo}
	If \( \prv {α}{k+1} Γ ⇒ Δ  \) then \( \prv{ω^α}k Γ ⇒ Δ \).
\end{theorem}
\begin{proof}
	Induction on \( α \). If \( Γ ⇒ Δ \) is initial, the claim holds trivially.
	So suppose \( \prv {α}{k+1} Γ ⇒ Δ  \) is derived via a rule
	\[
	  \Infer{Γ_i ⇒ Δ_i \; \text{for } i ∈ I}[\( * \)]{ Γ ⇒ Δ }
	\]
	and for each \( i \) there is \( α_i < α  \) such that \( \prv{α_i}{k+1} Γ_i ⇒ Δ_i \).
	The induction hypothesis implies that \( \prv{ω^{α_i}}{k} Γ_i ⇒ Δ_i \) for each \( i \). So, if \( * \) is not cut, then 
	\[ \prv{ω^α}k Γ⇒ Δ \]
	obtains by re-applying the rule and observing that \( \supof{ ω^η }[η< α] ≤ ω^α \).
	Now suppose that the rule is cut, with cut formula \( C \). If \( \rk C < k \) the same argument as above applies.
	Otherwise \( \rk C = k \) and the reduction lemma is applicable, yielding
	\[
		\prv{ω^{α_0} \nsum ω^{α_1}}k Γ ⇒ Δ 
	\]
	Since \( ω^{α_0} \nsum ω^{α_1} < ω^α  \), the proof is complete.
\end{proof}

The bound in the reduction theorem can be improved fairly easily.
For the give proof strategy to work, it suffices to find an order-preserving function \( f \colon \Ord → \Ord \) such that \( f(α) ≥ \supof{ f(ξ) \nsum f(η) }[ξ,η<α] \).
An obvious candidate is \( f \colon α ↦ 2^α \) (see exercise~\ref{ex-ord-base-2}) and, indeed, \cref{oa-red-lem-PAo} can be strengthened by replacing \( ω^α \) with \( 2^α \).
Certainly, \( 2^α ≤ ω^α \) for all \( α \), so working with this bound seems a significant improvement. 
But given that for every additive principal ordinal \( α ≥ ω^ω \) in fact \( 2^α = ω^α \) (cf.\ exercise~\ref{ex-ord-cnf-2o}), the distinction between exponentiation in the two bases does little in reducing the complexity of cut elimination.

In the next section I will present a strict refinement of the cut elimination theorem in which ordinal exponentiation is directly tied to the \emph{quantifier} rank of the cut formula rather than the full rank.

Let \( ω_0^α ≔ α \) and \( ω_{k+1}^α ≔ ω^{ω_k^α} \).

\begin{theorem}[Cut elimination theorem]\label{oa-ce-PAo}
	If \( \prv{α}k Γ⇒ Δ \) then \( \prv{ω_k^α}0 Γ⇒ Δ \).
\end{theorem}
\begin{proof}
	Consequence of \cref{oa-red-thm-PAo}.
\end{proof}

\begin{exercise}
	Formulate and prove a corresponding reduction lemma and cut elimination theorem for \( \HAo \).
\end{exercise}

\begin{theorem}[Embedding theorem]
	\label{oa-embed-PA-ce}
	If \( \PA ⊢ Γ ⇒ Δ \) and this a closed sequent, then there exists \( α < ε_0 \) such that
	\[
		\PAo \prv{α}0 Γ ⇒ Δ.
	\]
	In addition, \( α \) is effectively computable from the given \( \PA \)-proof.
\end{theorem}
%
\begin{proof}
	Suppose \( \PA ⊢ Γ ⇒ Δ \).
	By the embedding lemma (\cref{oa-embed-PAo-w-bounds}) there is are \( n, k \) such that 
	\[
	  \PAo \prv{ω.n}k Γ ⇒ Δ .
	\]
	Let \( α = ω_k^{ω.n} \). Then \( α < ε_0 \) (by \cref{d-epsilon0}) and
	\[
	  \PAo \prv{α}0 Γ ⇒ Δ 
	\]
	by \cref{oa-ce-PAo}.
\end{proof}

On the basis of cut elimination, a few observations can be already made.

\begin{corollary}
	\label{PA-consis-weak}
	\( \PA \) and, hence, \( \HA \), are consistent.
\end{corollary}
%
\begin{proof}
	There can be no cut-free proof of the empty sequent.
\end{proof}

An inspection of the various proofs leading up to \cref{PA-consis-weak} can strengthen the result by clarifying what mathematical principles suffice to derive the consistency of arithmetic.
%
\begin{corollary}
	\label{PA-consis}
	Consistency of \( \PA \) can be deduced using only finitary reasoning plus the principle of transfinite induction for ordinals \( {≤} ε_0 \).
\end{corollary}
%
By ‘finitary reasoning’ I mean the ‘finite’ mathematics that can be carried out using only finite objects (such as natural numbers) and primitive recursive functions.
Examples include deciding whether one formula is a subformula of another, whether a given primitive recursive function enumerates the premises of an \( ω \)-rule (or Gödel codes of sequents) and what the concluding sequent is.
It is beyond the scope of these lecture notes to attempt to make the statement more precise, but the following proof ‘sketch’ hopefully elucidates how this could be achieved and proven.

\begin{proof}[sketch]
	Suppose there is a finite \( \PA \)-proof of the empty sequent. The embedding of \( \PA \) in \( \PAo \) (\cref{oa-embed-PAo-w-bounds}) provides an explicit number \( n < ω \) such that
	\[
		\PAo \prv{ω.n}n {} ⇒ {} .
	\]
	The existence of a cut-free proof of the empty sequent, along with the various results on which \cref{oa-ce-PAo} depends, can now be established by via finitary reasoning plus transfinite induction up to an ordinal strictly smaller than \( ε_0 \), for instance the ordinal \( ω_{n+2} \) suffices.
	
	As there can be no cut-free proof of the empty sequent, there is no derivation of the empty sequent in \( \PA \).
\end{proof}

\begin{corollary}
	If \( Γ \) is a set of \( Π^0_1 \) sentences and \( Δ \) a set of \( Σ^0_1 \) sentences, then \( \PAo ⊢ Γ ⇒ Δ \) iff there is a cut-free \( \PAo \) derivation of finite height.
\end{corollary}
%
\begin{proof}
	Exercise.
\end{proof}

% ----------------------
\section{Refining cut elimination arithmetic}
\label{s-oa-refined}
% ----------------------

\tbw
I will jump the gun somewhat.

The \emph{negation rank} (\emph{n-rank}) of a formula, \( \nrk F \), counts the nesting depth on the negative side of implications:
\begin{align*}
	\nrk{A} &= 0 \quad(\text{$A$ prime})
	&
	\nrk{F ∧ G} &= \max\setof{ \nrk F , \nrk G }
	\\
	\nrk{∀x F(x)} &= \nrk{F(a)}
	&
	\nrk{F → G} &= \max\setof{ \nrk F +1 , \nrk G }
\end{align*}


Compare with the quantifier hierarchy for this language fragment.
%
Since \( Π_0^P \) is closed under negation, there is no a priori bound on the negation depth of formulas in \( Π_n^P \).

\begin{lemma}
	Every \( Π_n^P \) formula is equivalent, over weak arithmetic, to a formula with n-rank \( n+1 \).
\end{lemma}
By \emph{weak arithmetic}, I have in mind the theory known as Robinson's \( Q \) (see~\cite[ch.~18]{LogThe}) but primitive recursive arithmetic or, equivalently, the theory known as ‘$\IS_1$’ suffices (cf.~\cref{s-oa-arithmetic}).

\begin{proof}
	Using the full logical language, express \( F ∈ Π_n^P \) in prefix normal form as
	\begin{gather}
		\label{oa-eqn-PNF}\tag{\dag}
		∀\vec x_1 ∃\vec x_2 ⋯ ∃ \vec x_{m} G
	\end{gather}
	where \( G \) is quantifier-free.
	We can choose \( m ∈ \setof{ n , n+1 } \) and \( G \) can be assumed to be \( Π_0^P \) because over weak arithmetic a disjunction \( E ∨ F \) is provably equivalent to 
	\[ 
		∃x ∃ y ( x + y = \suc \0 ∧ ( x = \0 → E ) ∧ ( x = \suc\0 → F )).
	\]
	and the leading existential quantifiers can be incorporated into the '$∃\vec x_m$' sequence of quantifiers.
	It is now straightforward to witness \eqref{oa-eqn-PNF} as an equivalent \( Π_{m}^P \)-formula whose negation rank is \( m \).
\end{proof}

%
A sequent is an expression \( Γ ⇒ Δ \) without free variables using the logical language isolated at the beginning of this section.
Let \( \prvs{α}k Γ ⇒ Δ \) denote derivability in \( \PAo \) for such sequents in the usual way but with a more liberal cut rule bounded by negation rank:
\[
  \begin{prooftree}
	\hypo{\prvs {α} k Γ ⇒ Δ , C }\hypo{ \prvs{β} k C, Σ ⇒ Λ }
	\infer2[\Cut]{ \prvs{γ}k Γ , Σ ⇒ Δ , Λ }
\end{prooftree}
\quad\text{for \( \nrk C < k \) and \( \max\setof{α,β} < γ \).}
\]

%The complication with adopting the above cut rule is that all transformations on proofs so far considered ‘reduce’
I assume this variation of \( \PAo \) satisfies weakening, substitution and inversion lemmas with the same bounds.
%
Given a finite sequence \( \vec A = (A_i)_{i≤k} \) of formulas, I will write \( \vec A , Γ ⇒ Δ \) for \( A_0 , …, A_n , Γ ⇒ Δ \).

%
\begin{lemma}[Refined reduction lemma]
	Suppose \( \prvs{α}k Γ_i ⇒ Δ_i, C_i \) and \( \nrk {C_i} ≤ k \) for each \( i ≤ n \). If \( \prvs{β}k \vec C , Σ ⇒ Λ \), then
	\begin{gather}
		\label{oa-eqn-spec-red-lem}\tag{\dag}
		\prvs{α + β} k Γ_0 , …, Γ_n , Σ ⇒ Δ_0 , …, Δ_n , Λ .
	\end{gather}
\end{lemma}
%
Applications of the refined reduction lemma, however, will also be to simulate an ordinary two-premise cut rule.
The significance of allowing multiple premises is hidden in the proof.
The multi-premise ‘cut’ can be visualised as either a generalisation of the binary cut rule:
\begin{prooftree*}
  \hypo{ Γ_0 ⇒ Δ_0, C_0 }
  \hypod 
  \hypo{ Γ_n ⇒ Δ_n, C_n }
  \hypo{ \vec C , Σ ⇒ Λ }
  \infer4[$n$-\Cut]{ Γ_0 , …, Γ_n , Σ ⇒ Δ_0 , …, Δ_n , Λ }
\end{prooftree*}
Or as sequence of binary cuts
\begin{prooftree*}
  \hypo{ Γ_n ⇒ Δ_n, C_n }
  \hypo{ Γ_1 ⇒ Δ_1, C_1 }
  \hypo{ Γ_0 ⇒ Δ_0, C_0 }
  \hypo{ \vec C , Σ ⇒ Λ }
  \infer2[\Cut]{ C_1 , …, C_{n} , Γ_0 , Σ ⇒ Δ_0 , Λ }
  \infer2[\Cut]{ }
  \ellipsis{}{ C_n, Γ_0 , …, Γ_{n-1} , Σ ⇒ Δ_0 , …, Δ_{n-1} , Λ }
  \infer2[\Cut]{ Γ_0 , …, Γ_n , Σ ⇒ Δ_0 , …, Δ_n , Λ }
\end{prooftree*}
The advantage of the former presentation is that the ‘size’ of the derivation does not depend on the order of the sequents.
This allows the application of an induction hypothesis in cases that the binary cut view grows too large.

%
\begin{proof}
	The overall structure of the proof will be recognisable as the strategy used in the proof of \cref{oa-red-lem-PAo}.
	I proceed by induction on \( β \).
	Suppose
	\begin{enumerate}
		\item \( \prvs{α}k Γ_i ⇒ Δ_i, C_i \) and \( \nrk {C_i} ≤ k \) for each \( i ≤ n \), and
		\item \( \prvs{β}k \vec C , Σ ⇒ Λ \).
	\end{enumerate}
	I refer to \( \vec C \) as the \emph{cut} formulas.
	First, suppose no cut formula is principle in the final rule of assumption 2.
	If the sequent is initial, then \( Σ ⇒ Λ \) is initial and \eqref{oa-eqn-spec-red-lem} follows by weakening.
	Therefore, assume \( C_n \) is the principal formula in 2.
	There is a case distinction based on the form of \( C_n \).
	The focus will therefore be on assumption 2 above and
	\begin{gather}
		\label{oa-eqn-spec-red-lem-2}\tag{\ddag}
		\prvs{α } k  Γ_n ⇒ Δ_n , C_n .
	\end{gather}

	If \( C_n = ⊥ \) or is a false equation then \eqref{oa-eqn-spec-red-lem} results from applying the inversion lemma to \eqref{oa-eqn-spec-red-lem-2}.
	If \( C_n = P s \), then \( P t ∈ Λ \) for some \( ℕ ⊨ s = t \) and \eqref{oa-eqn-spec-red-lem} also follows from \eqref{oa-eqn-spec-red-lem-2} via substitution.
	The final case is that \( C_n \) is a true equation. But it is not possible for such an atomic formula to be principal in \eqref{oa-eqn-spec-red-lem}.
	
	Moving on to the non-atomic case suppose, to begin, that \( C_n = D ∧ E \).
	From 2 I obtain \( γ < β \) and \( F ∈ \setof{ D, E} \) such that
	\begin{enumerate}[resume]
%		\label{oa-eqn-spec-red-lem}\tag{\dag}
		\item \( \prvs{γ} k \vec C , F , Σ ⇒  Λ  \).
	\end{enumerate}
	Applying the inversion lemma to \eqref{oa-eqn-spec-red-lem-2} yields
	\begin{enumerate}[resume]
%		\label{oa-eqn-spec-red-lem}\tag{\dag}
		\item \( \prvs{α } k  Γ_n ⇒ Δ_n , F  \).
	\end{enumerate}
	Adding this final sequent to the list of hypotheses in 1 above, and using 3 in place of 2, I can apply the induction hypothesis (as \( γ < β \)), which derives \eqref{oa-eqn-spec-red-lem}.
	
	The quantifier case, \( C_n = ∀x D(x) \) is essentially the same argument.
	From principality of \( C_n \) and the inversion lemma I know
	\begin{enumerate}[start=3,label=\arabic*'.]
		\item \( \prvs{γ} k \vec C , D(s) , Σ ⇒  Λ \) for some \( γ < β \) and term \( s \).
		\item \( \prvs{α } k  Γ_n ⇒ Δ_n , D(s) \).
	\end{enumerate}	
	I can then deduce \eqref{oa-eqn-spec-red-lem} from the induction hypothesis by adding 4' to the list in 1 and 3' in place of 2.
	
	The final case is involves a different in the argument.
	Suppose \( C_n = D → E \).
	Hypothesis 2 and the inversion lemma yields three derivations to work from:
	\begin{enumerate}[start=3,label=\arabic*''.]
		\item \( \prvs{γ}k \vec C , E , Σ ⇒ Λ \),
		\item \( \prvs{δ}k \vec C , Σ ⇒ Λ , D \),
		\item \( \prvs{α}k D , Γ_n ⇒ Δ_n, E \),
	\end{enumerate}
	for \( γ, δ < β \).
	The first and third of these can be used with the induction hypothesis, obtaining as conclusion,
	\begin{enumerate}[resume,label=\arabic*''.]
		\item \( \prvs{α + γ } k D , Γ_0 , …, Γ_n , Σ ⇒ Δ_0 , …, Δ_n , Λ \).
	\end{enumerate}
	To derive \eqref{oa-eqn-spec-red-lem}, I need to remove the formula \( D \) in 6'' I apply a cut against a second application of the induction hypothesis, this time using 4'' (and not expanding the list in 2):
	\begin{enumerate}[resume,label=\arabic*''.]
		\item \( \prvs{ α + δ } k Γ_0 , …, Γ_n , Σ ⇒ Δ_0 , …, Δ_n , Λ , D \).
	\end{enumerate}
	As \( \nrk D < k \) a standard cut can be used between sequents 4'' and 6'', the conclusion being \eqref{oa-eqn-spec-red-lem}.
\end{proof}

As the focus is on better bounds on cut elimination, I will switch to base-$2$ exponentiation for the reduction theorem:

\begin{theorem}[Refined reduction theorem]
	Suppose \( \prvs{α}{k+1} Γ ⇒ Δ \). Then \( \prvs{2^α}{k} Γ ⇒ Δ \).
\end{theorem}
\begin{proof}
	This argument proceeds just as usual.
	Jumping to the main case, suppose \( \prvs{α}{k+1} Γ ⇒ Δ \) is derived via cut:
	\[
		\prvs{β}{k+1} Γ ⇒ Δ , C
		\qquad
		\prvs{γ}{k+1} C, Σ ⇒ Λ 
	\]
	where \( β , γ < α \) and \( \nrk C ≤ k \).
	The induction hypothesis yields
	\[
		\prv{2^β}{k} Γ ⇒ Δ , C
		\qquad
		\prv{2^γ}{k} C, Σ ⇒ Λ 
	\]
	and the refined reduction lemma implies \( \prvs{2^α}k Γ ⇒ Δ \).
\end{proof}

\begin{theorem}[Refined cut elimination]
	If \( \prvs{α}k Γ ⇒ Δ \) then \( \prvs{γ}0 Γ ⇒ Δ \) where \( γ = 2_k^α \).
\end{theorem}

\tbw

\begin{theorem}
	\label{oa-embed-IS-ce}
	If \( \IS_n ⊢ A \) then \( \PAo \prv{α}0 Γ ⇒ Δ \) for some \( α < ω_{n+1} \).
\end{theorem}
%
\note{Check this.}
%
\begin{proof}[sketch]
	From \( \IS_n ⊢ A \) we deduce that \( \PA ⊢ {} ⇒ A \) with a proof in which all cuts have n-rank \( ≤ n \).
	The reason is that the induction rule is only applied to formulas with n-rank \( ≤ n \) and finitary cut elimination is available in \( \PA \) to reduce the cut rank to formulas of the same n-rank as uses of induction.
	The embedding lemma of \( \PA \) into \( \PAo \) yields
	\( \PAo \prvs{ω.k}{n} {}⇒ A \), so \( \PAo \prv{γ}0 {}⇒ A \) where
	\[
		γ = 2_{n}^{ω.k} .
	\]
	Recall that \( 2^{ω.k} = ω^k \), whence
	\[
		γ ≤ ω_{n}^{k} < ω_{n+1} .
	\]
\end{proof}

