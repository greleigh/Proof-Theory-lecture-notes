
%---------------------------------
\chapter{Ordinal Analsysis of Arithmetic}\label{s-oa-PAo}
%---------------------------------
We being by recalling the definition of the infinitary sequent calculi for arithmetic:
%
\begin{definition}
	\( \PAo \) and \( \HAo \).
\end{definition}
%
\begin{definition}\label{d-bound-omega-logic}
	Let \( \Theory{T} \) be a classical sequent calculus.
	The ternary relation \( \Theory{T} \prv{α}k Γ ⇒ Δ \) is defined by transfinite recursion on the rules of \( \Theory{T} \):
	\begin{enumerate}
		\item If \( Γ ⇒ Δ \) is an axiom of \( \Theory{T} \), then \( \Theory{T} \prv{α}k Γ ⇒ Δ \) for all \( α \) and \( k \);
		\item For each inference \( (*) \) of \( \Theory{T} \) except cut of the form
		\[
			\Infer{\setof{ Γ_i ⇒ Δ_i }[i ∈ I]}[*]{ Γ ⇒ Δ }
		\]
		\( \Theory{T} \prv{α}k Γ ⇒ Δ \) holds if \( \Theory{T} \prv{α_i}k Γ_i ⇒ Δ_i \) and \( α_i < α \) for all \( i ∈ I \);
		\item If \( \Theory{T} \prv{α_0}k Γ ⇒ Δ , C \) and \( \Theory{T} \prv{α_1}k C , Γ ⇒ Δ \) for \( α_0,α_1 < α \) and \( w(C) < k \), then \( \Theory{T} \prv{α}k Γ ⇒ Δ \).
	\end{enumerate}
	If \( \Theory{T} \) is a calculus of intuitionistic sequents, \( \Theory{T} \prv{α}k Γ ⇒ A \) is defined via the same conditions but with a modification of the final clause:
	\begin{enumerate}[resume]
		\item If \( \Theory{T} \prv{α_0}k Γ ⇒ C \) and \( \Theory{T} \prv{α_1}k C , Γ ⇒ A \) for \( α_0,α_1 < α \) and \( w(C) < k \), then \( \Theory{T} \prv{α}k Γ ⇒ A \).
	\end{enumerate}
	Given \( \Theory{T} \prv{α}k Γ ⇒ Δ \) we refer to \( α \) as the (ordinal) height and \( k \) the cut rank.
\end{definition}
%
There is no requirement of minimality of \( α \) and \( k \) in the above definition. In particular, the relation \( \prv{α}k \) is monotone in \( α \) and \( k \):
%
\begin{lemma}\label{l-PAo-index}
	If \( α ≤ β \) and \( k ≤ l \) then \( \Theory{T} \prv{α}k Γ ⇒ Δ \) implies \( \Theory{T} \prv{β}l Γ ⇒ Δ \).
\end{lemma}
%
Now we look at some examples … 
%
\begin{namedlemma}[Refined Embedding Lemma]\label{l-embedding-PAo-w-bounds}
	Suppose \( \PA ⊢ Γ ⇒ Δ \) and \( Γ ⇒ Δ \) is closed. Then there is \( k < ω \) such that \( \PAo \prv{ω.k}k Γ ⇒ Δ \).
\end{namedlemma}
%
\begin{proof}
	Exercise.
\end{proof}
%
The next lemma hints at the usefulness of the \( ω \)-rule. We will have need of this result in \cref{s-oa-upper}.
%
\begin{proposition}\label{p-PAo-S1}
	Let \( A(x_1,…, x_k) \) be a \( Σ_1 \) formula. There exists \( m < ω \) such that for all \( n_1, …, n_k ∈ \Nat \), if\, \( \Nat ⊨ A(\nm {n_1}, …, \nm {n_k} ) \) then \( \PAo \prv m0 {⇒ A(\nm {n_1}, …, \nm {n_k})} \).
\end{proposition}
%
\begin{proof}
	By induction on \( A \).
\end{proof}

Admissibility of weakening becomes 
%
\begin{namedlemma}[Weakening Lemma]
	If \( \PAo \prv{α}k Γ ⇒ Δ \) and \( A \) is closed, then \( \PAo \prv{α}k A , Γ ⇒ Δ \) and \( \PAo \prv{α}k Γ ⇒ Δ , A \).
\end{namedlemma}
%
\begin{proof}
	By (transfinite) induction on \( α \).
\end{proof}
%
\begin{namedlemma}[Substitution Lemma]
	Let \( Γ ⇒ Δ \) be a sequent with at most \( x \) free, and \( s \) and \( t \) be closed terms such that \( \Nat ⊨ s = t \). Then \( \PAo \prv{α}k Γ[s/x] ⇒ Δ[s/x] \) implies \( \PAo \prv{α}k Γ[t/x] ⇒ Δ[t/x] \).
\end{namedlemma}
%
\begin{proof}
	a
\end{proof}

\begin{namedlemma}[Contraction Lemma]
	If \( \PAo \prv{α}k A , A , Γ ⇒ Δ \) then \( \PAo \prv{α}k A , Γ ⇒ Δ \).
	If \( \PAo \prv{α}k Γ ⇒ Δ , A , A \) then \( \PAo \prv{α}k Γ ⇒ Δ , A \).
\end{namedlemma}
%
\begin{proof}
	b
\end{proof}

%---------------------------------
\section{Provable transfinite induction}\label{s-oa-lower}
%---------------------------------
Let \( ≺ \) be a well-founded ordering of \( \Nat \). We define
\begin{align*}
	\otin n &= \sup\setof{ \otin m + 1 }[m ≺ n]
	\\
	\ot &= \sup \setof{ \otin n + 1 }[ n ∈ \Nat ]
\end{align*}
%
\begin{example}
For the standard ordering \( 0 < 1 < 2 < ⋯ \) we have
\begin{align*}
	\otin[<]{n} &= n \quad \text{for every \( n ∈ \Nat \)}
	\\
	\ot[<] &= \sup\setof{ n + 1 }[n ∈ \Nat] = ω .
\end{align*}
%

The ordering \( <_1 \) was given by \( {<_1} = \setof{(m,n)}[m < n \text{ and } m≠0 ] ∪ \setof{(n,0)}[n ∈ \Nat] \), well-ordering \( \Nat \) as \( 1 <_1 2 <_1 3 <_1 ⋯ <_1 0 \).
For this well-ordering we have
\begin{align*}
	\otin[<_1]{n+1} &= n 
	\\
	\otin[<_1]{0} &=  ω
	\\
	\ot[<_1] &= ω + 1 .
\end{align*}
%

The well-ordering \( <_2 \) satisfies
\begin{align*}
	\otin[<_2]{2n} &= n 
	\\
	\otin[<_2]{2n+1} &= ω + n
	\\
	\ot[<_2] &= ω + ω = ω .2.
\end{align*}
\end{example}
%
\begin{lemma}
	If \( ≺ \) is a well-founded relation on \( \Nat \) then for every \( α < \ot \) there exists \( n ∈ \Nat \) such that \( \otin n = α \).
	If \( ≺ \) is a well-ordering then \( n \) is unique.
\end{lemma}
%
\begin{definition}
	For each primitive recursive ordering \( ≺ \) and formula \( A(x) \) we define formulas:
	\begin{align*}
		\Prog {A} &= ∀x( ∀ y ≺ x \, A(y) → A(x) )
		\\
		\TI {A} &= \Prog {A} → ∀ x\, A(x)
	\end{align*}
\end{definition}
%
In the case \( {≺} = {<} \) is the standard ordering on \( \Nat \), the formula \( \Prog A \) expresses that \( A(x) \) is \emph{inductive}.
%
\begin{definition}
	Let \( \Theory{T} \) be a theory in the language \( \La \). The \emph{proof theoretic ordinal} of\, \( \Theory{T} \) is the ordinal \( \pto{\Theory T} \) defined by
	\[
	  \pto{\Theory T} = \sup \setof{\ot }[\text{\( ≺ \) is a pr.~rec.~well-ordering and \( \Theory{T} ⊢ \TI{P} \)}]
	\]
	Equivalently, \( \pto{\Theory T} \) is the least primitive recursive ordinal for which the principal transfinite induction up to this ordinal for the predicate \( P \) is not derivable.
\end{definition}
%
\begin{theorem}\label{pto-lower-bound}
	\( \pto \PA ≥ \pto \HA ≥ ε_0 \).
\end{theorem}
%
To show \autoref{pto-lower-bound} we are required to present a sequence of well-orderings \( \{ ≺_i \}_i \) such that \( \supseq \ot[≺_i] = ε_0 \) and \( \HA ⊢ \TI[≺_i] P \) for each \( i \).


%---------------------------------
\section{Bounding provable transfinite induction}\label{s-oa-upper}
%---------------------------------

The goal of the remainder of the section is the converse to \cref{pto-lower-bound}:
%
\begin{theorem}\label{pto-upper-bound}
	\( \pto{\PA} ≤ ε_0 \).
\end{theorem}
%
The proof strategy is as follows. We fix a primitive recursive well-ordering \( ≺ \) and suppose that \( \TI P \) is provable in \( \PA \). The embedding theorem for \( \PAo \) implies that
\begin{equation}
	\PAo \prv{α}0 {\Prog P ⇒ ∀x Px} \text{ for some \( α < ε_0 \)}. \tag{\dag}\label{eqn-oa-TI}
\end{equation}
The goal of this section is to establish that from \eqref{eqn-oa-TI} we can infer that \( \ot  < ε_0 \).
We introduce an extension of \( \PAo \), called \( \PAop \), such that \eqref{eqn-oa-TI} implies
\begin{equation}
	\PAop \prv{α}0 { ⇒ ∀x Px} \text{ for some \( α < ε_0 \)}. \tag{\ddag}\label{eqn-oa-TI2}
\end{equation}
%
The transfer from \eqref{eqn-oa-TI} to \eqref{eqn-oa-TI2} will depend on a cut elimination theorem for \( \PAop \).
An analysis of cut-free provability in \( \PAop \) will lead us to deduce from \eqref{eqn-oa-TI2} that \( \ot ≤ α \), i.e., \( \ot < ε_0 \).
\smallskip

We start with introducing the extension of \( \PAo \) in \eqref{eqn-oa-TI2} for analysing proofs of \( \TI P \).
Henceforth, let \( ≺ \) be a fixed primitive recursive well-ordering on \( \Nat \).
%
\begin{definition}
	Let \( ≺ \) be a well-ordering on \( \Nat \). We introduce the inference rule \( (≺) \):
	\[
	  \Infer{Γ \sa Δ , Ps , P \nm n \quad\text{for every \( n ≺ s^\Nat \)}}[ \prec]{ Γ \sa Δ , Ps }
	\]
	The infinitary sequent calculus \( \PAop \) extends the axioms and rules of \( \PAo \) by the inference \( (≺) \) above. The relation \( \PAop \prv{α}k Γ ⇒ Δ \) is defined in \cref{d-bound-omega-logic}.
\end{definition}

In general, the rule \( (≺) \) will have infinitely many premises like the \( ω \)-rules.
For instance, if there is an element \( m \) with order-type \( ω \), then an instance of the rule
\[
  \Infer{X⇒Y \text{ for all Z}}[≺]{X ⇒ Y , P \nm m}
\]



The next three lemmas provide the motivation for this extension of \( \PAo \).
%
\begin{lemma}\label{l-PAo-in-PAop}
	If \( \PAo \prv{α}k Γ ⇒ Δ \) then  \( \PAop \prv{α}k Γ ⇒ Δ \).
\end{lemma}
\begin{proof}
	Immediate.
\end{proof}
%
\begin{lemma}\label{l-PAop-Prog}
	\( \PAop \prv{ω}0 {⇒ \Prog P} \).
\end{lemma}
%
\begin{proof}
	Recall that \( \Prog P = ∀ x ( ∀ y ( y ≺ x → Py ) → P x ) \). 
	Let \( K \) be the constant given by \cref{p-PAo-S1} such that \( \PAo \prv K 0 { ⇒ \nm m ≺ \nm n }\) for all \( m ≺ n \). 
	For every \( n ∈ \Nat \) we have the following derivation in \( \PAo \) (with implicit application of weakening):
	\begin{prooftree*}
	  \hypo{ \prv00 P\nm m ⇒ P \nm m }
	  \hypo{}\ellipsis{}{\prv K0 {}⇒ \nm m \prec \nm n \text{ for all \( m ≺ n \)} }
	  \infer[separation=3em]2[\impL]{\prv{K+1}0 \nm m ≺ \nm n → P \nm m  &⇒ P\nm m \text{ for all \( m ≺ n \)} }
	  \infer1[\faL]{\prv{K+2}0 ∀y ( y ≺ \nm n → Py ) &⇒ P \nm m \text{ for all \( m ≺ n \)} }
	\end{prooftree*}
	By weakening and \cref{l-PAo-in-PAop} we have, for every \( m ≺ n ∈ \Nat \), \[ \PAo \prv{K+2}0 ∀y ( y ≺ \nm n → Py ) &⇒ P \nm n , P \nm m .\]
	Continuing the derivation in \( \PAop \):
	\begin{prooftree*}
		\hypo{\prv{K+2}0 ∀y ( y ≺ \nm n → Py ) ⇒ P \nm m \text{ for all \( m ≺ n \)} }
		\infer1[\( ≺ \)]{\prv{K+3}0 ∀y ( y ≺ \nm n → Py ) ⇒ P\nm n }
		\infer1[\impR]{\prv{K+4}0 {} ⇒ ∀y ( y ≺ \nm n → Py ) → P\nm n }
	\end{prooftree*}
	An application of the \( ω \)-rule completes the proof.
\end{proof}
%
\begin{lemma}\label{l-PAop-embedding}
	If\, \( \PA ⊢ \TI P \) then\, \( \PAop \prv{ω^2}k {⇒ ∀ x \, P x} \) for some \( k < ω \).
\end{lemma}
%
\begin{proof}
	From the embedding lemma for \( \PAo \) (lemma~\ref{l-embedding-PAo-w-bounds}), from \( \PA ⊢ \TI P \) we obtain \( \PAo \prv{ω.k}k {⇒ \TI P} \) for some \( k < ω \).
	\Cref{l-PAop-Prog} and a pair of cuts yields the desired result.
\end{proof}

Once paired with cut elimination for \( \PAop \), which we tend to next, the final lemma above gives the claim in \eqref{eqn-oa-TI2}.

%---------------------------------
\section{Cut elimination, revisited}
%---------------------------------

The proof of the following admissibility lemmas is left as an exercise.
%
\begin{namedlemma}[Weakening Lemma]
	If \( \PAop \prv{α}k Γ ⇒ Δ \) and \( A \) is closed, then \( \PAop \prv{α}k A , Γ ⇒ Δ \) and \( \PAop \prv{α}k Γ ⇒ Δ , A \).
\end{namedlemma}
%
\begin{namedlemma}[Substitution Lemma]
	Let \( Γ ⇒ Δ \) be a sequent with at most \( x \) free, and \( s \) and \( t \) be closed terms such that \( \Nat ⊨ s = t \). Then \( \PAop \prv{α}k Γ[s/x] ⇒ Δ[s/x] \) implies \( \PAop \prv{α}k Γ[t/x] ⇒ Δ[t/x] \).
\end{namedlemma}
%
\begin{namedlemma}[Contraction Lemma]
	If \( \PAop \prv{α}k A , A , Γ ⇒ Δ \) then \( \PAop \prv{α}k A , Γ ⇒ Δ \).
	If \( \PAop \prv{α}k Γ ⇒ Δ , A , A \) then \( \PAop \prv{α}k Γ ⇒ Δ , A \).
\end{namedlemma}
%

Cut elimination for \( \PAop \) can be deduced from the proof we gave ….
%
\begin{namedlemma}[Reduction lemma]
	As expected
\end{namedlemma}
%
\begin{theorem}[Cut elimination for \( \PAop \)]
	\tbw
\end{theorem}
 
We require one further result on cut free proofs in this calculus before we can conclude our main result.
%
\begin{lemma}
	Suppose \( \PAop \prv{β}0 P\cd{α_1}, … , P\cd{α_k} ⇒ P\cd{γ_1} , …, P\cd{γ_l} \).
	Then \note{make sure \( \cd{α} \) is defined.}
	\begin{equation}\label{eqn-bounding-lemma}
		 \min\setof{γ_1, …, γ_l } ≤ \max \setof{α_1, …, α_k } + β  . 
	\end{equation}
\end{lemma}
%
\begin{proof}
	Induction on \( β \).
	If \( P\cd{α_1}, … , P\cd{α_k} ⇒ P\cd{γ_1} , …, P\cd{γ_l} \) is an axiom, then \( α_i = γ_j \) for some \( i \) and \( j \) and \eqref{eqn-bounding-lemma} is immediate.
	If, however, the sequent is not an axiom, then it is necessarily the conclusion of an instance of \( (≺) \) for which we may assume, without loss of generality, that the principal formula is \( P \cd{γ_l} \).
	Therefore, for every \( δ < γ_l \) there is \( β_δ < β \) such that
\end{proof}
%
%---------------------------------
\section{Characterisation of provable transfinite induction}
%---------------------------------

Combining \cref{pto-lower-bound,pto-upper-bound} is
%
\begin{theorem}[Proof-theoretic characterisation theorem]
	The proof-the\-oretic ordinal of Peano and Heyting arithmetic is \( ε_0 \).
\end{theorem}
%
%
\begin{corollary}[Independence of transfinite induction]
	There is a primitive recursive well-ordering \( ≺ \) on \( \Nat \) with ordertype \( ε_0 \) and a formula \( A \) in the language of arithmetic such that \( \PA ⊬ \TI A \).
\end{corollary}

