%---------------------------------
\chapter{Transfinite induction and proof-theoretic ordinals}
\label{c-TI-and-PTO}
%---------------------------------

The final chapter is devoted to proving the optimality of \cref{oa-embed-PA-ce}/\cref{PA-consis}.
I will show how the principle of transfinite induction can be rendered in arithmetic and show that it is precisely the ordinal \( ε_0 \) that marks the boundary between the provable and unprovable instance of transfinite induction.
%Such a characterisation of Peano arithmetic
It turns out that many interesting theories extending arithmetic (including set theories and theories of second-order arithmetic) can be characterised in such a way.
The ordinal corresponding to ‘provable instances of transfinite induction’ is one of a number of ways in which ordinals can be used to describe, delineate and compare mathematical theories.
\emph{Ordinal analysis}, in a nutshell, is the isolation and comparison of such ordinal measures.


%---------------------------------
\section{Provable transfinite induction}\label{s-oa-lower}
%---------------------------------
In the present section I will define precisely one way to assign an ordinal to a theory of arithmetic and show that under this measure the \emph{proof-theoretic ordinal} of Peano arithmetic is at least \( ε_0 \).
The following section will establish that this bound is optimal.

I begin by recalling some basic order-theory.
\begin{definition}
	Let \( ≺ \) be a relation on a non-empty set \( X \).
	\( ≺ \) is:
	\begin{itemize}
		\item \emph{well-founded} if there is no infinite \( ≺ \)-descending sequence, namely no sequence \( (x_i)_{i<ω} \) such that \( x_{i+1} ≺ x_i \) for every \( i \).
		\item a \emph{well-order} if \( ≺ \) is linear and well-founded.
	\end{itemize}
\end{definition}

\begin{example}\label{oa-ex-order-type}
	The following two orderings on natural numbers are well-orders.
	The third is well-founded but not a well-order.
	\begin{align*}
		m <_1 n \;&\text{iff}\; 0 < m < n \text{, or } n = 0 \text{ and } m ≠ 0 .
		\\
		m <_2 n \;&\text{iff}\; 
		\begin{cases}
			m < n, \text{and both are even or both odd, or}
			\\
			\text{$n$ even and \( m \) odd.}
		\end{cases}
		\\
		m <_2 n \;&\text{iff}\; m = 0 \text{ and } n ≠ 0.
	\end{align*}
\end{example}

The proof of the next lemma is left as an exercise.

\begin{lemma}
	A relation \( ≺  \) on a non-empty set \( X \) if a well-order iff every non-empty \( Y ⊆ X \) has a \( ≺ \)-least element.
\end{lemma}

Let \( ≺ \) be a well-founded ordering of \( \Nat \). I define
\begin{align*}
	\otin n &≔ \supof{ \otin m + 1 }[m ≺ n]
	\\
	\ot &≔ \supof{ \otin n + 1 }[ n ∈ \Nat ]
\end{align*}
Well-foundedness ensures the above notions are well-defined.
I call \( \otin[≺] n \) the order-type of \( n \) in \( ≺ \), and \( \ot \) the order-type of \( ≺ \).
The function \( \otin {·} \colon ℕ → \Ord \) is order-preserving: \( m ≺ n \) implies \( \otin m < \otin n \) and its range is a segment of \( \Ord \).
If \( ≺ \) is a well-order then the function is also injective, whence \( \otin {·}  \) is an order-preserving enumeration of \( ℕ \) in \( \Ord \).
%
\begin{example}
	I compute the order types of natural numbers in the three orderings from \cref{oa-ex-order-type}.
	Note, for the standard ordering on \( ℕ \),
	\begin{align*}
		\otin[<]{n} &= n \quad \text{for every \( n \)}
		\\
		\ot[<] &= \sup\setof{ n + 1 }[n ∈ \Nat] = ω .
	\end{align*}
	%
The ordering \( <_1 \) satisfies
\begin{gather*}
	\otin[<_1]{n+1} = n 
	\quad\text{and}\quad
	\otin[<_1]{0} =  ω
	\\
	\ot[<_1] = ω + 1 .
\end{gather*}
%
The ordering \( <_2 \) satisfies
\begin{align*}
	\otin[<_2]{2n} &= n 
	\\
	\otin[<_2]{2n+1} &= ω + n
	\\
	\ot[<_2] &= ω + ω .
\end{align*}
The ordering \( <_3 \) satisfies
\begin{align*}
	\otin[<_3]{0} &= 0
	\\
	\otin[<_3]{n} &= 1 \;\text{ for all $n>0$}
	\\
	\ot[<_3] &= 2.
\end{align*}
\end{example}
%
\begin{lemma}
	If \( ≺ \) is a well-founded relation on \( \Nat \) then for every \( α < \ot \) there exists \( n ∈ \Nat \) such that \( \otin n = α \).
	If \( ≺ \) is a well-ordering then \( n \) is unique.
\end{lemma}
%\begin{proof}
%	Let \( O = \setof{ \otin n }[n ∈ ℕ] \) and \( α ∈ O \).
%	If \(  \)
%%	Let \( ≺ \) be well-founded and assume, to the contrary, that \(  \)
%\end{proof}

For \( ≺ \) a primitive recursive relation on \( ℕ \) the representation theorem for arithmetic (\cref{representation-thm}) presents a \( Δ_0 \) formula \( {F_≺}(a,b) \) in the language of arithmetic (without the predicate \( P \)) such that for all \( n,m ∈ ℕ \),
\[
  \PA ⊢ {F_≺}(\nm m , \nm n) \text{ iff } m ≺ n.
\]
In what follows, I will write \( a ≺ b \) for the formula \( F_≺(a,b) \), and use \( ∀x ≺ a F(x) \) as an abbreviation for the formula \( ∀x ( x ≺ a ∧ F(x) ) \).

%
\begin{definition}
	For each primitive recursive ordering \( ≺ \) and formula \( A(x) \) define formulas:
	\begin{align*}
		\Prog {A} &≔ ∀x( ∀ y ≺ x \, A(y) → A(x) )
		\\
		\TI {A, a} &≔ \Prog {A} → ∀ y ≺ a\, A(y)
		\\
		\TI {A} &≔ ∀x\, \TI {A,x}
	\end{align*}
\end{definition}
%

If \( ≺ \) is a well-order, the formula \( \Prog A \) expresses progressiveness of the set of ordinals \( \otin n \) such that \( ℕ ⊨ A(\nm n) \).
In the case \( {≺} = {<} \) is the standard ordering on \( \Nat \), this is the same as \( A(x) \) being \emph{inductive}. 
As a result, \( \TI {A, a} \) states the principle of transfinite induction for this set restricted to the segment of ordinals \( \setof{ \otin n }[ n ≺ a ] \).


%
\begin{definition}
	Let \( \Theory{T} \) be a theory in the language \( \La \). The \emph{proof theoretic ordinal} of\, \( \Theory{T} \) is the ordinal \( \pto{\Theory T} \) defined by
	\[
	  \pto{\Theory T} = \supof{\ot }[\text{\( ≺ \) is a pr.~rec.,~well-founded  and \( \Theory{T} ⊢ \TI{P} \)}]
	\]
%	Equivalently, \( \pto{\Theory T} \) is the least primitive recursively representable ordinal for which the principal transfinite induction up to this ordinal for the predicate \( P \) is not derivable.
\end{definition}
%

%The restriction to primitive recursive orderings rather than of arbitrary complexity in the definition is because of its role in the formula \( \TI P \).
\noindent
The goal of this section is a lower bound on the proof-theoretic ordinal of Peano and Heyting arithmetic:
%
\begin{theorem}\label{pto-lower-bound}
	\( \pto \PA ≥ \pto \HA ≥ ε_0 \).
\end{theorem}
%
Unpacking \cref{pto-lower-bound}, it states that there exists a sequence of well-founded relations \( \setof{ ≺_i }_i \) such that \( \supseq \ot[≺_i] = ε_0 \) and \( \HA ⊢ \TI[≺_i] P \) for each \( i \).
A sequence of well-founded relations is not, strictly speaking, necessary as a single well-ordering can be defined of order-type \( ε_0 \) and for which transfinite induction can be proven for each proper initial segment.
I leave the proof of the next lemma as an exercise.
%
\begin{lemma}
	\label{oa-wo-e0}
	There exists a primitive recursive well-ordering of \( ℕ \) of order-type \( ε_0 \) and primitive recursive functions \( \oplus \) and \( \dot{ω} \) representing addition and exponentiation respectively in the sense that \( \oplus \colon ℕ × ℕ → ℕ \) and \( \dot{ω} \colon ℕ → ℕ \) satisfy
	\[
		\otin{ m \oplus n } = \otin m \nsum \otin n
		\qquad\text{and}\qquad
		\otin{ \dot{ω}(m) } = ω^{\otin m}
	\]
	for all \( m,n ∈ ℕ \).
\end{lemma}
%
\begin{exercise}
	\label{ex-wo-e0}
	Prove \cref{oa-wo-e0}.
	Hint: Utilise the Cantor normal form theorem and a (primitive recursive) bijection between \( ℕ \) and finite sequences of \( ℕ \).
\end{exercise}
%
\begin{exercise}
	Prove the following generalisation of \cref{oa-wo-e0}: Given a primitive recursive well-ordering of order-type \( α \) construct a primitive recursive ordering of \( ℕ \) of order-type \( ε_α \).
\end{exercise}

In the following \( ≺ \) denotes the primitive recursive well-ordering of order type \( ε_0 \) given by \cref{oa-wo-e0}.
%For \( α < ε_0 \) I will write \( ≺_α \) for the restriction of \( ≺ \) to numbers whose order-type in \( ≺ \) is \( < α \).
%That is,
%\[
%  {≺_α} ≔ \setof{ (m,n) }[ m ≺ n \text{ and } \otin[≺] n < α ]
%\]
%The relation \( ≺_α \) is not a well-order because every \( n \) for which \( \otin[≺] n ≥ α \) has the same order-type, namely \( 0 \).
%
%If \( ≺ \) is represented in arithmetic in the sense that \( \PA ⊢ \nm m ≺ \nm n \) iff \( m ≺ n \), the orderings \( ≺_α \) are represented through a formula of three variables:
%\[
%	{≺_*}(a,b,c) ≔ a ≺ b ∧ b ≺ c.
%\]
%This formula clearly has the property that
%\[
%  \PA ⊢ {≺_*}(\nm m , \nm n, \nm o) \text{ iff } \otin m ≺_{\otin o} \otin n .
%\]
%Henceforth, I will write \( a ≺_c b \) for the formula\( {≺_*}(a,b,c) \) above.
%
The proof of \cref{pto-lower-bound} relies on one lemma whose proof is rather time-consuming and will be omitted:
%
\begin{lemma}
	\label{pto-lower-bound-lem}
	For every formula \( A(a) \) in the language of arithmetic, there exists a formula \( A'(a) \) such that
	\[
		\PA ⊢ ∀x\bigl( \TI[≺] {A' , x } → \TI[≺] {A, \dot{ω}^x} \bigr).
%		\PA ⊢ \Prog[≺_{ω^α}] {A} → \Prog[≺_α] {A'}.
	\]
%	In particular, \( \PA ⊢ ∀x\bigl( \TI[≺] {A' , x } → \TI[≺] {A, \dot{ω}^x} \bigr) \).
\end{lemma}

%Viewing \( A \) as a set of ordinals,  
%\cref{pto-lower-bound-lem} expresses that there is a set \( A' ⊆ \ot \) such that for all \( α < \ot \), if transfinite induction holds for \( A' \) on the segment \( \ot \) up to \( α \) and \( A \) is progressive on the segment \( \ot \), then \( A \) contains all ordinals \( < ω^α \).
%
Although I won't present the proof, it will be useful to know how \( A' \) is constructed from \( A \).
First, I present the construction as an operation on sets of ordinals.
I write \( β ⊆ O \) as shorthand for \( (∀ξ< β)ξ ∈ O \).
Given \( O ⊆ \Ord \), define \( O' \) as the class
\[
  O' = \setof{ α }[ ∀ξ ( \, ξ ⊆ O \text{ implies } ξ + ω^α ⊆ O \, )].
\]
It is not difficult to see that \( O' \) is a segment and \( α ∈ O' \) implies \( ω^α ∈ O \), from which \( ω^{α+1} ⊆ O \) quickly follows.
%For the first observation, if \( β < α ∈ O' \) then \( ω^α ⊆ O \) (pick \( ξ = 0 \)), so \( β ∈ O \).
%And if \( α ∈ O' \) then \( ω^α ⊆ O \) (again picking \( ξ = 0 \)), so \( ω^α ∈ O \) (now picking \( ξ = ω^α \)).
%
%Looking closer at the definition, it is clear that \( O' \) is simply the largest segment of \( O \) such that \( \setof{ ω^α }[α ∈ O'] \) is a segment of \( O \).
%
Expressing the operation in the language of arithmetic provides the formula \( A' \) in \cref{pto-lower-bound-lem}:
\[
	A'(a) ≔ ∀x ( ∀ y ≺ x \, A(y) → ∀ y ≺ x \oplus \dot{ω}^a \, A(y) ).
\]

The above remarks concerning the properties of \( O \) and \( O' \) can be shown in \( \PA \) to hold for \( A \) and \( A' \). 
Thus, to prove the \namecref{pto-lower-bound-lem} it suffices to show that \( \PA ⊢ \Prog {A} → \Prog {A'} \), which involves similar argumentation.


%---------------------------------
\section{Bounding provable transfinite induction}\label{s-oa-upper}
%---------------------------------

The goal of this section is the converse to \cref{pto-lower-bound}:
%
\begin{theorem}\label{pto-upper-bound}
	\( \pto{\PA} ≤ ε_0 \).
\end{theorem}
%
The proof strategy is as follows. 
I fix an arbitrary primitive recursive well-ordering \( ≺ \) and suppose that \( \TI P \) is provable in \( \PA \). 
The embedding theorem for \( \PAo \) provides an ordinal \( α < ε_0 \) and a cut-free proof of \( \TI P \) bounded above by \( α \).
Applying the inversion lemma yields, for every \( n ∈ ℕ \),
\begin{equation}
	\PAo \prv{α}0 {\Prog P ⇒ ∀x ≺ \nm n \, Px} . \tag{\dag}\label{oa-eqn-TI}
\end{equation}
I want to infer from (\dag) that \( \otin n < ε_0 \) for every \( n \).
In fact, it will be the case that (\dag) holds only if \( \otin n ≤ α \).
%The goal of this section is to establish that from \eqref{oa-eqn-TI} we can infer that \( \ot  < ε_0 \).

To that aim I will utilise an extension of \( \PAo \), called \( \PAop \), such that \eqref{oa-eqn-TI} implies
\begin{equation}
	\PAop \prv{α}0 { ⇒ ∀x ≺ \nm n Px}. \tag{\ddag}\label{oa-eqn-TI2}
\end{equation}
%
The transfer from \eqref{oa-eqn-TI} to \eqref{oa-eqn-TI2} will depend on a cut elimination theorem for \( \PAop \).
An analysis of cut-free provability in \( \PAop \) will lead me from \eqref{oa-eqn-TI2} quite directly to \( \otin n ≤ α \) for all \( n \), i.e., \( \ot ≤ α < ε_0 \).
\medskip

I begin by introducing the extension of \( \PAo \) used in \eqref{oa-eqn-TI2}.
Henceforth, let \( ≺ \) be a fixed primitive recursive well-ordering on \( \Nat \). 
%Without loss of generality, I assume \( \ot \) is a limit ordinal.
For the sake of simplifying notation, I will write \( s^ℕ \) for the value of \( s \) in the standard model, i.e., the \( n \) such that \( ℕ ⊨ \nm n = s \).
This notation presupposes that \( s \) is closed.

%
\begin{definition}
%	Let \( ≺ \) be a well-ordering on \( \Nat \). 
%	We introduce the inference rule \( (≺) \):
	The rule \( (≺) \) comprises all instances of the inference
	\[
	  \Infer{Γ \sa Δ , P \nm n \quad\text{for every \( n ≺ s^\Nat \)}}[\( ≺ \)]{ Γ \sa Δ , Ps }
	\]
	The infinitary sequent calculus \( \PAop \) extends the axioms and rules of \( \PAo \) by the inference \( (≺) \) above. 
	The relation \( \PAop \prv{α}k Γ ⇒ Δ \) is given as in \cref{d-bound-omega-logic}.
\end{definition}

In general, the rule \( (≺) \) will have infinitely many premises like the \( ω \)-rules.
For instance, if there is an element \( m \) with order-type \( ω \), and \( M = \setof{ n ∈ ℕ }[n ≺ m] \) then one instance of the rule is
\[
  \Infer{ Γ ⇒ P \nm n \text{ for all } n ∈ M }[\( ≺ \)]{ Γ ⇒ P \nm m}
\]



The next three lemmas provide the motivation for this extension of \( \PAo \).
%
\begin{lemma}\label{oa-PAo-in-PAop}
	If \( \PAo \prv{α}k Γ ⇒ Δ \) then  \( \PAop \prv{α}k Γ ⇒ Δ \).
\end{lemma}
\begin{proof}
	Immediate.
\end{proof}
%
\begin{lemma}\label{oa-PAop-Prog}
	\( \PAop \prv{ω}0 {{}⇒ \Prog P} \).
\end{lemma}
%
\begin{proof}
	Recall that \( \Prog P = ∀ x ( ∀ y ( y ≺ x → Py ) → P x ) \). 
	Let \( k \) be the constant given by \cref{p-PAo-S1} such that \( \PAo \prv k 0 { ⇒ \nm m ≺ \nm n }\) for all \( m ≺ n \). 
	For every \( n ∈ \Nat \) I obtain the following derivation in \( \PAo \) (with implicit application of weakening) for all \( m , n ∈ ℕ \) satisfying \( m ≺ n \):
	\begin{prooftree*}
	  \axiom[\idRule]{ \prv00 P\nm m ⇒ P \nm m }
	  \subproof{\prv k0 {}⇒ \nm m \prec \nm n }
	  \infer[separation=3em]2[\impL]{\prv{k+1}0 \nm m ≺ \nm n → P \nm m  &⇒ P\nm m }
	  \infer1[\faL]{\prv{k+2}0 ∀y ≺ \nm n \, Py &⇒ P \nm m}
	\end{prooftree*}
	Continuing the derivation in \( \PAop \):
	\begin{prooftree*}
		\subproof{\prv{k+2}0 ∀y ≺ \nm n \, Py ⇒ P \nm m \text{ for all \( m ≺ n \)} }
		\infer1[\( ≺ \)]{\prv{k+3}0 ∀y ≺ \nm n \, Py ⇒ P\nm n }
		\infer1[\impR]{\prv{k+4}0 {} ⇒ ( ∀y ≺ \nm n \, Py ) → P\nm n }
		\hypo{\text{for every } n}
		\infer2[\faR]{\prv{k+5}0 {} ⇒ \Prog P }
	\end{prooftree*}
	An application of bound weakening completes the proof.
\end{proof}
%
\begin{lemma}[Refined embedding lemma]\label{oa-PAop-embedding}
	If\, \( \PA ⊢ \TI P \) then there exists \( k < ω \) such that for all \( n ∈ ℕ \), 
	\[ \PAop \prv{ω^2}k {{}⇒ ∀ y ≺ \nm n \, P y }. \]
\end{lemma}
%
\begin{proof}
	The embedding lemma for \( \PAo \) (\cref{oa-embed-PAo-w-bounds}) and inversions yields a \( k < ω \) such that for all \( n ∈ ℕ \):
	\[ 
		\PAo \prv{ω.k}k { \Prog P ⇒ ∀y ≺ \nm n \, P y } 
	\]
	\Cref{oa-PAop-Prog} and a pair of cuts completes the argument.
\end{proof}

Paired with cut elimination for \( \PAop \), treated in the next section, the \namecref{oa-PAop-embedding} above yields \eqref{oa-eqn-TI2}.
Under the assumption of cut elimination (with the same bounds as \( \PAo \)), just one lemma stands before an optimal upper bound on the proof-theoretic strength of \( \PA \).
This is the lemma below.

Since \( ≺ \) is a fixed well-ordering, for \( α < \ot \) I will write \( \cd{α} \) for the numeral \( \nm n \) such that \( α = \otin n \).
%
\begin{lemma}[Bounding lemma]
	\label{oa-bounding-lemma}
	Let \( α_1 , …, α_m , β_0, …, β_n < \ot \). If
	\[ \PAop \prv{γ}0 P\cd{α_1}, … , P\cd{α_m} ⇒ P\cd{β_0} , …, P\cd{β_n} \]
	then
	\(%begin{equation}\label{eqn-bounding-lemma}
		 \min\setof{β_0, …, β_n } ≤ \max \setof{α_1, …, α_m } + γ  . %\tag{\ast}
	\)%end{equation}
\end{lemma}
%
\begin{proof}
	Induction on \( γ \).
	If \( P\cd{α_1}, … , P\cd{α_m} ⇒ P\cd{β_0} , …, P\cd{β_n} \) is initial, then \( α_i = β_j \) for some \( i \) and \( j \) and the claim holds vacuously.
	If, however, the sequent is not initial then, as the derivation is cut-free, the final rule applied must be an instance of \( (≺) \). 
	I can assume, without loss of generality, that the principal formula is \( P \cd{β_n} \), i.e., that the inference applied is
	\begin{prooftree*}
		\hypo{ P\cd{α_1}, … , P\cd{α_m} ⇒ P\cd{β_0} , …, P\cd{β_n} , P\cd{δ} \quad\text{for all \( δ < β_n \)} }
		\infer1[\( ≺ \)]{ P\cd{α_1}, … , P\cd{α_m} ⇒ P\cd{β_0} , …, P\cd{β_n} }
	\end{prooftree*}
	For each \( δ <  β_n \) the corresponding premise has a cut-free derivation of height \( < γ \).
	That is, for every \( δ < β_n \) there exists \( γ_δ < γ \) such that 
	\[
	  \PAop \prv{γ_δ}0 P\cd{α_1}, … , P\cd{α_m} ⇒ P\cd{β_0} , …, P\cd{β_n} , P\cd{δ}.
	\]
	Let \( β = \min \setof{β_0, …, β_n } \) and \( α = \max \setof{α_1, …, α_m } \).
	The induction hypothesis implies that 
	\begin{equation}
		\label{eqn-bouding-lemma-sub}
		\text{for every \( δ < β_n \), } \min\setof{ β , δ } ≤ α + γ_δ.
	\end{equation}
	%
	Consider two cases. 
	First, suppose \( β < β_n \). 
	Choosing \( δ = β \) in \eqref{eqn-bouding-lemma-sub} yields
	\[
		β ≤ α + γ_β < α + γ,
	\]
	whereby the claim holds as desired.
	Otherwise, \( β = β_n \), and
	\begin{align*}
		β = \supof{ δ + 1 }[ δ < β ] &≤ \supof{ α + γ_δ + 1 }[ δ < β ] &&\text{by \eqref{eqn-bouding-lemma-sub}}
	  \\
	  &≤ α + \supof{ γ_δ + 1 }[ δ < β ] &&\text{continuity}
	  \\
	  &≤ α + γ.
	 \end{align*}
%	 as required.
\end{proof}

\begin{proof}[of \cref{pto-upper-bound} (assuming cut elmination)]
	Let \( ≺ \) be any primitive recursive well-order of \( ℕ \) and suppose \( \PA ⊢ \TI P \).
	Let \( α = \ot \).
	The refined embedding lemma~(\cref{oa-PAop-embedding}) provides a finite \( k \) such that for all \( n ∈ ℕ \)
	\[
		\PAop \prv{ω^2} k {}⇒ ∀ y ≺ \nm n\, Py .
	\]
	In particular, for every \( β < α \),
	\[
		\PAop \prv{ω^2} k {}⇒ P \cd {β} .
	\]
	Cut elimination for \( \PAop \) provides an ordinal \( γ < ε_0 \) such that (see next \namecref{s-oa-PAop-ce}) for every \( β < α \)
	\[
		\PAop \prv{γ} 0 {}⇒ P\cd{β} .
	\]
	The bounding lemma ensures that \( β ≤ γ \), meaning that \( \ot ≤ γ + 1 < ε_0 \).
\end{proof}

%---------------------------------
\section{Cut elimination, revisited}
\label{s-oa-PAop-ce}
%---------------------------------

What remains is to confirm cut elimination for the extended calculus \( \PAop \).
The reader can confirm that weakening and substitution remain admissible in this extension.
%
\begin{lemma}[Weakening Lemma]
	If \( \PAop \prv{α}k Γ ⇒ Δ \) and \( Γ'⇒Δ' \) is closed then \( \PAop \prv{α}k Γ' , Γ ⇒ Δ, Δ' \).
\end{lemma}
%
\begin{lemma}[Substitution Lemma]
	Let \( Γ(a) ⇒ Δ(a) \) be a sequent with \( a \) the only free variable, and let \( s \) and \( t \) be closed terms such that \( \Nat ⊨ s = t \). If \( \PAop \prv{α}k Γ(s) ⇒ Δ(s) \) then \( \PAop \prv{α}k Γ(t) ⇒ Δ(t) \).
\end{lemma}
%
%\begin{lemma}[Contraction Lemma]
%	If \( \PAop \prv{α}k A , A , Γ ⇒ Δ \) then \( \PAop \prv{α}k A , Γ ⇒ Δ \).
%	If \( \PAop \prv{α}k Γ ⇒ Δ , A , A \) then \( \PAop \prv{α}k Γ ⇒ Δ , A \).
%\end{lemma}
%
\begin{exercise}
	Prove the weakening and substitution lemmas for \( \PAop \).
\end{exercise}

Precisely the same formulation of the reduction lemma also holds, but here there are some notable changes to the proof.
%In fact, it is simple to see how 
%
I present only the ‘simple’ version of this result and leave the quantifier-relevant form for the reader.
%
\begin{lemma}[Reduction lemma]
	Suppose \( \PAop \prv{α}k Γ ⇒ Δ , C \) and \( \PAop \prv{β}k C, Γ ⇒ Λ \).
	If \( \rk C = k \) then \( \PAop \prv{α\nsum β}k Γ ⇒ Δ,Λ \).
\end{lemma}
%
\begin{proof}
	Like the inferences of \( \PAo \), the rule \( (≺) \) has the property of just one principal formula,
	namely
	\[
%		\text{If} \quad
		\Infer{Γ_i ⇒ Δ_i \;\text{for } i ∈ I}{Γ⇒Δ}
	\]
	is an instance iff there is \( F ∈ Δ \) such that
	\[
		\Infer{ Σ, Γ_i ⇒ Δ_i , Λ  \;\text{for } i ∈ I}{ Σ ⇒ Λ, F}
	\]
	is an instance for all \( Σ \) and \( Λ \).

	As such, the new rule does not affect the part of the argument where \( C \) is not principal in one of the assumptions.
	So it suffices to treat the case in which \( C = P s\) for some \( s \) and is principal in both assumptions.
	But if \( Ps \) is principal in the proof \( \prv{α}k Γ ⇒ Δ , C \) then inference deriving this sequent is either initial (whence \( P t ∈ Γ \) for \( ℕ ⊨ s = t \)) or the conclusion of \( (≺) \).
	In the latter case, however, it is not clear how to use the premises of the rule against the second assumption.
	Fortunately, though, it is not necessary because we are assuming that \( C \) is principal in the second hypothesis, \( \prv{β}k C,Γ ⇒ Λ \).
	For this to be the case, \( C , Γ ⇒ Λ \) must be an initial sequent, meaning that \( P t ∈ Λ \) such that \( ℕ ⊨ s = t \).
	The substitution lemma applied to the \emph{first} hypothesis, shows derivability of \( \prv{α}k Γ ⇒ Δ , Λ \).	
\end{proof}

\begin{exercise}
	Complete the proof of the reduction lemma.
\end{exercise}

\begin{exercise}
	Formulate and prove a quantifier-relevant formulation of the reduction lemma for \( \PAop \) following the schema of \cref{oa-red-lem-PAo-quant}.
\end{exercise}
%
\begin{theorem}[Cut elimination]
	If \( \PAop \prv{α}k Γ ⇒ Δ \) then \( \PAop \prv{ω_k^α}0 Γ ⇒ Δ  \).
\end{theorem}
%
\begin{proof}
	The proof proceeds precisely as before.
\end{proof}

%
%---------------------------------
\section{Characterisation of provable transfinite induction}
%---------------------------------

Combining the results in this chapter:
%
\begin{theorem}[Proof-theoretic characterisation theorem]
	The proof-the\-oretic ordinal of Peano and Heyting arithmetic is \( ε_0 \).
\end{theorem}
%
\begin{proof}
	As \( ε_0 ≤ \pto{\HA} ≤ \pto{\PA} \) by \cref{pto-lower-bound} and \( \pto{\PA} ≤ ε_0 \) by \cref{pto-upper-bound}.
\end{proof}
%
\begin{corollary}[Independence of transfinite induction]
	There is a primitive recursive well-ordering \( ≺ \) on \( \Nat \) and a formula \( A \) in the language of arithmetic such that \( \PA ⊬ \TI A \).
\end{corollary}
%

The following will be a consequence of \cref{oa-embed-IS-ce}, but it needs to be refined.
%
\begin{theorem}
	The proof-the\-oretic ordinal of \( \IS_n \) for \( n > 0 \) is \( ω_{n+1} \).
\end{theorem}