%---------------------------------
\chapter{An ordinal interlude}\label{c-oa-ordinals}
%---------------------------------

To present the ordinals it is not necessary to have a set-theoretic definition of ordinals in mind (as, for example, arbitrary transitive sets).
Indeed, there is no need to consider the question of by what ordinals \emph{are} or from what they are \emph{formed}.
For a \emph{theory} of ordinals all that is relevant are the order-theoretic properties satisfied by the ordinals and a selection of operations that can be defined on them.
In short, ordinals are treated analogously to natural numbers: as a posited entity fulfilling specified criteria.
%
The material of this chapter draws from lecture notes by Michael Rathjen~\cite{RathjenLectures}.

%
\begin{definition}\label{d-ordinals}%[Ordinals]
	The \emph{ordinals} is a class \( \Ord \) equipped with a binary relation \( < \) satisfying three postulates, where \( ≤ \) is the reflexive closure of \( < \):
	\begin{axioms}[o]
		\item \( < \) is a strict linear order on \( \Ord \). That is, \( < \) is irreflexive, transitive and linear, where linear means that for all \( α , β ∈ \Ord \) either \( α ≤ β \) or \( β ≤ α \).\label{post-ord-lin}
		\item Every non-empty class of ordinals has a \( < \)-minimal element (necessarily unique by \ref{post-ord-lin}). That is, if \( O ⊆ \Ord \) is non-empty there exists \( ξ ∈ O \) such that \( ξ ≤ α \) for all \( α ∈ O \).\label{post-ord-wo}
		\item For every set \( X \) and function \( f \colon X → \Ord \) there exists \( ξ ∈ \Ord \) such that \( f(x) < ξ \) for every \( x ∈ X \).\label{post-ord-unbdd}
	\end{axioms}
\end{definition}

Set-theoretic concerns do matter in the language used to discuss ordinals.
As, for example, the Burali-Forte paradox shows, it is inconsistent  the Zermelo--Fraenkel (or Cantorian) conception of \emph{set} in mind to consider that the collection of (all) ordinals forms a set.
Hence use of term ‘class’ to refer to arbitrary collections of ordinals/objects and ‘set’ in specific case of \ref{post-ord-unbdd}.
Familiarity with set theory is not necessary for the elementary theory of ordinals presented here.
Indeed, it will suffice to replace every term ‘set’ in what follows by ‘countable set’ and ‘class’ by ‘countable or uncountable set’.
%We have side-stepped this concern by restricting attention to the \emph{countable} ordinals. Over Zermelo set theory, the collection of all countable ordinals, i.e., the collection \( \Ord \) above, forms a set; indeed, it is precisely the first uncountable ordinal.
%Such a restriction is not important for our later use of ordinals.
%Ultimately our attention will be constrained to a relatively small collection of (countable) ordinals.
%Our first lemma confirms that the relation \( <_\Ord \) is a well-order.

In the following, notation \( \setof{ t }[x ∈ X ] \) means the \emph{class} of objects \( t \) as \( x \) ranges over the (class) \( X \).
Usually a function \( f \colon U → V \) between classes has been specified along with a (sub)class \( X ⊆ U \) whence the notation \( \setof{ f(x) }[x ∈ X ] \) expresses the class of objects \( f(x) \) for \( x ∈ X \). This class will be written \( f[X] \).

\begin{convention}[Notating ordinals]
	Lowercase Greek letters \( α \), \( β \), etc.\ stand as metavariables for ordinals.
\end{convention}


\begin{lemma}\label{ord-well-order}
	Postulate \ref{post-ord-wo} is equivalent to the principle of transfinite induction.
	This is the statement that if \( O \) is progressive in the ordinals then \( \Ord ⊆ O \), where \( O \) is progressive means that for all ordinals \( α \), if \( β \in O \) for every \( β < α \) then \( α ∈ O \).
%	Let \( O ⊆ \Ord \) be non-empty.
%	Then \( O \) has a \( <_\Ord \)-minimal element and this is unique.
\end{lemma}
%
\begin{proof}
	Let \( O \) be progressive.
	Consider the class \( C = \Ord \setminus O \) of ordinals not in \( O \). If \( C \) is non-empty then, by \ref{post-ord-wo}, \( C \) contains a least ordinal, \( α \) say.
	As \( α \) is the least ordinal in \( C \), every \( ξ < α \) is element of \( O \). 
	Progressiveness implies that \( α ∈ O \) contradicting that \( α ∈ C \). 
	Hence, \( C \) is the empty class, so \( \Ord ⊆ O \).
	For the converse claim, assume postulate \ref{post-ord-lin} and the principle of transfinite induction (I could also assume \ref{post-ord-unbdd} but this is unnecessary).
	The aim is to establish \ref{post-ord-wo}. 
	Thus, let \( O \) be a non-empty class of ordinals and, for want of a contradiction, assume that \( O \) has no least element. 
	As in the other direction, I consider the complement of \( O \), the class \( C = \Ord \setminus O \).
	Suppose \( α \) be any ordinal such that \( ξ ∈ C \) for all \( ξ < α \). If \( α ∈ O \) then this is the least element of \( O \). As \( O \) has no least element therefore \( α ∈ C \).
	So \( C \) is progressive and \( C = \Ord \) by transfinite induction, contradicting the non-emptiness of \( O \).
\end{proof}

The next \namecref{ord-supremum} provides the primary means to infer the existence of ordinals.

\begin{lemma}
	\label{ord-supremum}\ 
	Let \( O \) be a class of ordinals.
	\begin{enumerate}
		\item There exists a least upper bound of \( O \). That is, an ordinal \( α \) such that \( ξ ≤ α \) for all \( ξ ∈ O \). This \( ξ \) is referred to as the \emph{supremum} of \( O \) and denoted \( \sup O \).
		\item There exists a strict least upper bound of \( O \), i.e., \( α \) such that \( ξ < α \) for all \( ξ ∈ O \).
	\end{enumerate}
	In each case the proclaimed ordinal is unique.
\end{lemma}
%
\begin{proof}
	Begin with 1. Let \( O \) be given.
	Consider the class \( O^≥ \) of all ordinals \( α \) such that \( ξ ≤ α \) for \emph{all} \( ξ ∈ O \).
	The \( < \)-least element of \( O^≥ \) (if such exists) is clearly the desired ordinal.
	But in order to apply postulate \ref{post-ord-wo} to this class it is necessary to establish that \( O^≥ \) is non-empty.
	For this I use the third postulate applied to identity function \( \textsf{id} \colon O → \Ord \colon ξ ↦ ξ \) (which is a function from \( O \) into \( \Ord \)).
	%
	For 2, the same argument works with the class \( O^> \) in place of \( O^≥ \) where this is the class of ordinals \emph{strictly} larger than all elements of \( O \).
	
	Uniqueness of each case is ensured by \ref{post-ord-lin}.
\end{proof}

Henceforth, I will not make explicit reference to the postulates.

The least ordinal is denoted \( 0 \). This happens to be the supremum of the empty set: \( 0 ≔ \sup ∅ \).
Given \( α ∈ \Ord \), the \emph{successor} of \( α \), in symbols \( α' \) or \( α + 1 \), is the least ordinal greater than \( α \), which exists (and is unique) by \cref{ord-supremum}(2) applied to the singleton set \( \setof{α} \).
That is, \( α' \) is such that \( ξ < α' \) iff \( ξ ≤ α  \).
The successor of \( 0 \) is denoted \( 1 ( =0') \), its successor \( 2 ( = 0'' ) \), etc.

A \emph{limit ordinal} is any non-zero ordinal \( λ \) such that \( η' < λ \), for all \( η < λ \).
Define a function \( f\colon ℕ → \Ord \) by \( f(0) = 0 \) and \( f(n+1) = f(n)' \).
That is, \( f(n) \) is the \emph{ordinal} representing the natural \( n \).
The supremum of \( \setof{ n }[n ∈ ℕ] \) is called \( ω \), which is a limit by construction and, therefore, the least limit ordinal.

\begin{lemma}
	\label{ord-suc-lim}
	Every non-zero ordinal is either a successor or a limit.
\end{lemma}

\begin{lemma}
	\label{ord-lim}
	An ordinal \( λ \) is a limit iff \( λ = \sup O \) for some non-empty set \( O \) closed under successor (meaning that \( ξ ∈ O \) implies \( ξ' ∈ O \)).
\end{lemma}

\begin{lemma}\label{ord-supremum-unique}
	Suppose \( O, O' \) are such that for every \( α \in O \) there exists \( β  \in O' \) such that \( α ≤ β \).
	Then \( \sup O ≤ \sup O' \).
\end{lemma}
\begin{exercise}
	Prove \cref{ord-suc-lim} to \ref{ord-supremum-unique}.
\end{exercise}


I will employ common set-theoretic abbreviations such as \( \sup_{i∈ I} α_i \) for \( \supof {α_i}[i ∈ I] \) and \( \sup_{i} α_i \) for \( \supof {α_i}[i < ω ] \).
I will also use \( λ \) as a metavariable for limit ordinals.

%--------------------------------------
\section{Elementary Ordinal Functions}
%--------------------------------------

A \emph{segment} of \( \Ord \) is any class \( O \) of ordinals which is closed downwards, i.e., if \( α < β ∈ O \) then \( α ∈ O \).
If \( X \) and \( Y \) are segments then either \( X ⊆ Y \) or \( Y ⊆ X \); in either case \( X ∩ Y \) is a segment.

Let \( O \) be a segment. A function \( f \colon O → \Ord \) is said to be:
\begin{itemize}
	\item \emph{order preserving} if \( α < β \) implies \( f(α) < f(β) \) for all \( α , β ∈ O \).
	\item \emph{continuous} if for all \( U ⊆ O \), if \( \sup U ∈ O \) then \( f( \sup U ) = \sup f[U] \).
%	\item \emph{normal} if \( O = \Ord \) and \( f \) is order preserving and continuous.
	\item an \emph{enumeration} (of \( X ⊆ \Ord \)) if \( f \) is order-preserving and \( f[O] = X \).
\end{itemize}

The identity function \( \mathsf{id} \colon \Ord → \Ord \) is all of the above. In particular, it is an enumeration of \( \Ord \).
Let \( f \colon ℕ → \Ord \) be given by \( f(0) = ω \) and \( f(n+1) = f(n)' \).
This function is order preserving and continuous (the latter is trivial). 
It is %not normal because the domain of \( f \) is not all ordinals, but it is
also an enumeration of the set \( \setof{ ω , ω' , … } \) because \( ℕ \) is a segment.
Notice that order preserving functions on ordinals are always injective.

\begin{lemma}\label{ord-o-p}
	If \( O \) is a segment and \( f \) is order preserving then \( α ≤ f(α) \) for all \( α ∈ O \).
\end{lemma}
%
\begin{exercise}
	Prove \cref{ord-o-p}.
\end{exercise}

The main property of ordinal functions I need is the summarised by
\begin{lemma}
	\label{ord-normal-exists}
	Every class of ordinals has a unique enumeration. The enumeration of \( Y ⊆ \Ord \) will be denoted \( E_Y \).
\end{lemma}
%
\begin{proof}
	\( E_Y \) is determined as the inverse of a particular function \( C_Y \colon Y → \Ord \), called the \emph{collapsing} function for \( Y \), defined by
	\[
		C_Y(α) = \sup \setof{ C_Y(ξ) + 1 }[ ξ ∈ Y \text{ and } ξ < α ].
	\]
	The collapsing function is clearly unique if it is well-defined. Moreover, \( C_Y \)
	This function is well-defined: Consider the class \( O \) of ordinals \( α \) for which the collapsing function on \( Y_α ≔ Y ∩ \setof{ ξ }[ξ ≤ α] \) exists.
	If \( C_{Y_ξ} \colon Y_ξ → \Ord \) is defined for each \( ξ < α \) I claim that \( C\colon Y_α → \Ord \) defined by 
	\[
		\begin{aligned}
			C(α) &= \sup\setof{ C_{Y_{ξ}}(ξ) + 1 }[ ξ < α \text{ and } ξ ∈ Y]
			\\
			C(ξ) &= C_{Y_ξ}(ξ) \text{ for \( ξ < α \)}
		\end{aligned}
	\]
	is the collapsing function for \( Y_α \).
	That this is the follows almost by definition. Indeed, all that is lacking is the observation that \( C_{Y_ξ}(β) = C_{Y_η}(β) \) whenever \( β ≤ ξ < η \).
	So \( O \) is progressive and transfinite induction implies that class \( Y_α \) has a collapsing function \( C_{Y_α} \). Now define \( C_Y \) as \( α ↦ C_{Y_α}(α) \).

	Clearly, \( C_Y \) is injective. Therefore the function admits a (right) inverse:
	\[
		E_Y ≔ C_Y^{-1} \colon C_Y[Y] → Y 
	\]
	As \( C_Y[Y ] \) is (clearly) a segment, \( E_Y \) is an enumeration of \( Y \).
	
	As to uniqueness of \( E_Y \), let \( O = C_Y[Y] \) and suppose \( f \colon O' → Y \) is any enumeration of \( Y \). In particular, \( O' \) is a segment. Transfinite induction implies that \( f(α) = E_Y(α) \) for all \( α ∈ O ∩ O' \).
	As both functions are injective and surjective into \( Y \) it follows that \( O = O' \).
\end{proof}

Two further properties of enumerations will be useful.

\begin{lemma}
	\label{ord-normal}
	Let \( f\colon \Ord → \Ord \) be continuous and order preserving (in particular, \( f \) is an enumeration of \( f[\Ord] \)).
	Then
	\begin{enumerate}
		\item For every \( α ≥ f(0) \) there is a unique \( β ≤ α \) such that \( f(β) ≤ α < f(β+1) \).\label{ord-normal-cover}
		\item For every \( α \) there is a unique \( β ≥ α \) such that \( β = f(β) \).\label{ord-normal-fix}
	\end{enumerate}
\end{lemma}
\begin{proof}
%	Continuity is implicit in the proof of \cref{ord-normal-exists}.
%	
	\ref{ord-normal-cover}. Consider the set \( O = \setof{ ξ }[ f(ξ) ≤ α ]\) and let \( β = \sup O \).
	Continuity yields
	\[
		f(β) = \sup f[O] = \sup \setof{ f(ξ) }[ f(ξ) ≤ α ] ≤ α
	\]
	whereas
	\(
		f(β+1) > α
	\)
	because \( β + 1 ∉ O \).
	
	\ref{ord-normal-fix}. Fix \( α \) and define \( O = \setof{ f(α) , f(f(α)) , …, f^n(α) , … } \) (arbitrary finite iterations of \( f \) on \( α \)). 
	Let \( β = \sup O \).
	Invoking continuity, \( f(β) = \sup f[O] = \sup O = β \). Moreover, \( α ≤ f(α) ≤ β \).
\end{proof}

%\begin{lemma}
%	\label{ord-normal-covers}
%	Let \( E\colon \Ord → \Ord \) be an enumeration (of \( E[\Ord] \)).
%	For every \( α ≥ E(0) \) there exists a unique \( β ≤ α \) such that \( E(β) ≤ α < E(β+1) \).
%\end{lemma}
%%
%\begin{proof}
%	Consider the set \( O = \setof{ ξ }[ E(ξ) ≤ α ]\) and let \( β = \sup O \).
%	Then 
%	\[
%		E(β) = \sup E[O] = \sup \setof{ E(ξ) }[ E(ξ) ≤ α ] ≤ α
%	\]
%	whereas
%	\(
%		E(β+1) > α
%	\)
%	because \( β + 1 ∉ O \).
%\end{proof}
%
%
%%As enumerations are order preserving it is always the case that \( α ≤ E(α) \). 
%
%\begin{lemma}
%	\label{ord-normal-fix}
%	Let \( E\colon \Ord → \Ord \) be an enumeration (of \( E[\Ord] \)). 
%	There exists \( ξ \) such that \( ξ = E(ξ) \).
%	Moreover, for every \( α \) there exists a unique \( ξ ≥ α \) such that \( ξ = E(ξ) \).
%\end{lemma}
%%
%\begin{proof}
%	Fix \( α \) and define \( O = \setof{ E(α) , E(E(α)) , …, E^n(α) , … } \) (arbitrary finite iterations of \( E \) on \( α \)). Then \( α ≤ E(α) ≤ \sup O \) and \( E(\sup O) \)
%\end{proof}


%--------------------------------------
\section{Elementary Ordinal Arithmetic}
%--------------------------------------
The basic operations of arithmetic can be extended to ordinals in a straightforward manner.
Often these are defined by transfinite recursion, but the two operations we desire, addition and exponentiation base \( ω \), can be expressed as enumeration functions.
I start with addition.
\begin{definition}
	Let \( α^≥ \) be the class of ordinals \( ≥ α \).
	\emph{Ordinal addition}, \( α + β \), is defined as \( α + β ≔ E_{α^≥}(β) \). That is, \( α + β \) is defined as the \( β \)-th ordinal in the enumeration of the ordinals \( ≥ α \).
\end{definition}
%
The following are direct consequences of this definition and left to the reader.

\begin{lemma}\label{ord-addition}
	For all \( α \), \( β \) and \( γ \).
	\begin{enumerate}
		\item \( α + 0 = α \).
		\item \( α + β' = ( α + β )' \).
		\item If \( β \) is a limit then \( α + β = \sup \setof{ α + ξ }[ξ < β ] \).
		\item \( α + ( β + γ ) = ( α + β ) + γ \).\label{ord-addition-assoc}
		\item \( α ≤ α + β \) and \( β ≤ α + β \).\label{ord-addition-inc}
	\end{enumerate}
\end{lemma}

\begin{example}\label{ex-ord-add}
	\( α + ω = \sup \setof{ α + n }[n∈ ℕ] = \sup \setof{ α , α' , α'' , …  } \). Thus \( α + ω \) is the least limit ordinal strictly above \( α \).
	
	In particular, \( n + ω = ω \) for every \( n < ω \).
	As \( 1 + ω = ω < ω + 1 \) ordinal addition is not commutative.
\end{example}

As addition is associative (item \ref{ord-addition-assoc} of the \namecref{ord-addition} above), I will omit brackets when stringing together applications of addition.
So \( α + β + γ \) can refer to either \( ( α + β ) + γ \) or \( α + ( β + γ ) \).

The next lemma is a consequence of \cref{ord-normal}.

\begin{lemma}
	For every \( α ≤ β \) there exists a unique \( ξ \) such that \( β = α + ξ \).
\end{lemma}
\begin{proof}
	\Cref{ord-normal} implies a unique \( ξ \) such that \( α + ξ ≤ β < α + ξ' \). Since \( α + ξ' = ( α + ξ ) + 1 \) it follows that \( α + ξ = β \).
\end{proof}

As \cref{ex-ord-add} demonstrates \( ω \) has the unusual property of being closed under addition: if \( ξ , η < ω \) then \( ξ + η < ω \).
Ordinals satisfying this condition are called \emph{additive principal} ordinals.
\begin{definition}
	\label{d-ord-AP}
	A ordinal \( α \) is additive principal iff \( α > 0 \) and \( ξ + η < α \) for all \( ξ , η < α \).
	The class of additive principal ordinals is denoted \( \AP \).
\end{definition}

The least additive principal ordinal is \( 1 \); the next is clearly \( ω \).
Most ordinals are \emph{not} additive principal. 
\( 1 \) is the only additive principal successor ordinal (because \( α + α ≥ α' \) provided \( α ≥ 1 \)).
Even most limit ordinals not additive principal: If \( α ≥ ω \) then \( α + ω ∉ \AP \) as \( α < α + ω \) but \( α + α ≮ α + ω \).

\begin{lemma}
	\label{ord-AP-normal}
	The enumeration function \( E_\AP \) for additive principal ordinals is continuous and has domain \( \Ord \).
%	The additive principal ordinals are
%	\begin{enumerate}
%		\item Closed: for every set \( O ⊆ \AP \), \( \sup O ∈ \AP \).
%		\item Unbounded in \( \Ord \). For every set \( α ∈ \AP \) there exists \( β > α \) such that \( β ∈ \AP \).
%	\end{enumerate}
\end{lemma}
\begin{proof}
%	Begin with the domain. By transfinite induction. Suppose \( ξ ∈ \dom E_\AP \) for every \( ξ < α \). I claim that \( α ∈ \dom E_\AP \). It suffices to show that there exists an additive principal ordinal \( β > E_\AP(ξ) \) for every \( ξ < α \).
%	Let \( O_0 = \setof{ E_\AP(ξ) }[ξ < α] \) and \( O_{n+1} = \setof{ ξ + η }[ξ , η ∈ O_n ] \). Set \(  \)
	Exercise.
\end{proof}

\Cref{ord-AP-normal} shows that the function enumerating the additive principal ordinals is defined on all ordinals, is order preserving and continuous.

\begin{lemma}
	\label{ord-AP}
	The following are equivalent for all \( α > 0 \):
	\begin{enumerate}
		\item \( α \) is additive principal.\label{ord-AP-1}
		\item \( α = 1 \) or \( α = \sup \setof{ ξ + ξ }[ξ < α ] \).\label{ord-AP-2}
		\item for all \( β < α \), \( β + α = α \).\label{ord-AP-3}
	\end{enumerate}
\end{lemma}
\begin{proof}
	\ref{ord-AP-1} $⇒$ \ref{ord-AP-2}. If \( α \) is additive principal then \( \sup \setof{ ξ + ξ }[ξ < α ] ≤ α \) by definition. Also, the additive principal ordinals except \( 1 \) are all limits, so if \( α ≠ 1 \) then \( α = \sup \setof{ ξ }[ξ < α ] ≤ \sup \setof{ ξ + ξ }[ξ < α ] \).
	
	\ref{ord-AP-2} $⇒$ \ref{ord-AP-3}. For \( α = 1 \) the claim is trivial. Otherwise, \( α  \) is a limit and \( β + α ≤ \sup \setof{ β + ξ }[ξ < α ] ≤ \sup \setof{ ξ + ξ }[ξ < α ] \). As \( α = \sup \setof{ ξ + ξ }[ξ < α ] \) the claim is established.
	
	\ref{ord-AP-3} $⇒$ \ref{ord-AP-1}. Straightforward.
\end{proof}
%

As a consequence of part \ref{ord-AP-3}, \( ω^α + ω^β = ω^β \) iff \( α < β \).
A corollary is the observation made earlier, that \( n + ω = ω \), which now follows from repeated applications of \cref{ord-AP}: \( α' + ω = α + ( ω^0 + ω^1 ) = α + ω^1 \).

Additive principal ordinals are central to the theory of ordinals.
As with addition, I will introduce more suggestive notation for the enumeration function for additive principal ordinals.

\begin{definition}
	\( ω^α ≔ E_\AP(α) \).
\end{definition}

By the definition \( ω^0 = 1 \) and \( ω^1 = ω \).
The reader can confirm that next additive principal ordinal above \( ω \) is the supremum of \( ω \), \( ω + ω \), \( ω + ω + ω \), …, \( ω + ⋯ + ω , … \) which is denoted \( ω^2 \).
%I will shortly show that the function \( ξ ↦ ω^ξ \) behaves as one would expect exponentiation (in particular, \( ω^(α + β) \)

\begin{lemma}
	\label{ord-AP-NF}
	For every \( α > 0 \) there exists unique \( β \) and \( ξ < α \) such that \( α = ω^β + ξ \).
\end{lemma}
\begin{proof}
	Let \( β \) be such that \( ω^β ≤ α < ω^{β'} \) and \( ξ \) such that \( α = ω^β + ξ \). Both ordinals are given by \cref{ord-normal}.
	What remains is to show uniqueness of this choice.
	Thus, suppose \( α = ω^γ + η \) for some \( γ \) and \( η < α \).
	The choice of \( β \) is clearly such that \( β ≥ γ \).
	As 
	\[ ω^β + ω^{γ+1} ≤ α + ω^{γ+1} ≤ ω^γ + η + ω^{γ + 1 } = ω^{γ+1} \]
	(the first inequality uses \cref{ord-addition}(\ref{ord-addition-inc}); the rest use \cref{ord-AP}(\ref{ord-AP-3})), also \( β ≤ γ \). Given that \( β = γ \), uniqueness of the rest is immediate.
\end{proof}
%\begin{lemma}
%	\label{ord-AP-add}
%	For all \( α > 0 \), \( α ∉ \AP \) iff \( α = β + γ \) for some \( β , γ < α \).
%\end{lemma}

%---------------------------------
\section{Normal forms and natural sum}
%---------------------------------

\Cref{ord-AP-NF} above provides the basis of a normal form representation of ordinals. This concept is introduced in the next definition.

\begin{definition}
	I write \( α =_\NF ω^β + γ \) to express that (i) \( α = ω^β + γ \) and (ii) \( γ < α \).
\end{definition}

Cantor, in 1897,\nocite{Cantor1897} established an expanded version of this normal form decomposition.
%
\begin{theorem}[Cantor normal form]\label{t-cantornf}
	For every ordinal \( α > 0 \) there exists \( n \) and ordinals \( α_n ≤ ⋯ ≤ α_0 \) such that 
	\[
		α = ω^{α_0} + ⋯ + ω^{α_n}.
	\]
	Moreover, this decomposition is unique. %, and \( α_0 < α \) if \( α < ε_0 \).
\end{theorem}
%
\begin{proof}
	The theorem is a simple generalisation of \cref{ord-AP-NF}.
%	The argument is by transfinite induction. Assume that each ordinal \( β < α \) admits a decomposition as described in the theorem. 
	Let \( α =_\NF ω^{α_0} + ξ_0 \) by \cref{ord-AP-NF}. If \( ξ_0 = 0 \) the decomposition is complete. 
	Otherwise, apply the \namecref{ord-AP-NF} again to express \( ξ_0 =_\NF ω^{α_1} + ξ_1 \), \( ξ_1 =_\NF ω^{α_2} + ξ_2 \), etc. 
	As \( α > ξ_0 > ξ_1 > ⋯ \) is a strictly decreasing sequence or ordinals, necessarily \( ξ_n = 0 \) for some \( n \). Thus, 
	\( α = ω^{α_0} + ⋯ + ω^{α_n} \). 
	Furthermore, \( α_0 ≥ α_1 ≥ ⋯ ≥ α_n \) because \( ω^{α_{i+1}} ≤ ξ_i < ω^{α_i+1} \) for each \( i \).
	Uniqueness is also a consequence of these normal forms.
\end{proof}


\begin{definition}
	The normal form notation is extended in the following way. Writing \( α =_\NF ω^{α_1} + ⋯ + ω^{α_n} \) expresses that (i) \( α = ω^{α_1} + ⋯ + ω^{α_n} \) and (ii) \( α ≥ α_1 ≥ ⋯ ≥ α_n \).
\end{definition}

\Cref{ord-normal} showed that every continuous order preserving function on the ordinals has fixed points. I.e., for each such function \( f \) there are ordinals \( β \) such that \( β = f(β) \).
As the function \( ξ ↦ ω^ξ \) (namely \( E_\AP \)) is an example of such a function, there must exist ordinals \( α \) such that \( α = ω^α \).
The proof of that lemma describes how to construct such an ordinal as the supremum of the sequence \( 0 \), \( 1 \), \( ω \), \( ω^ω \), …, \( α \), \( ω^α \), ….
This particular ordinal, conventionally denoted \( ε_0 \), will play a central role in the next chapter.

\begin{definition}\label{d-epsilon0}
	\( ε_0 ≔ \supseq ω_i \) where \( ω_0 = ω \) and \( ω_{k+1} = ω^{ω_k} \).
\end{definition}
%

%
\begin{lemma}
	\label{ord-e0}
	\( ε_0 \) is the least fixed point of the ordinal function \( α \mapsto ω^α \). 
	That is, \( ω^{ε_0} = ε_0 \) and \( α < ω^{α} \) for all \( α < ε_0 \).
\end{lemma}
%
\begin{exercise}
	Prove \cref{ord-e0}.
\end{exercise}

\begin{exercise}\label{ex-ord-mult-pre}
	Using the Cantor normal form theorem, define a multiplication operation where the first argument is restricted to additive principal ordinals: \( α , β ↦ ω^α.β \).
	The function should be continuous in \( β \) and satisfy the recursive clauses: \( ω^α.0 = 0 \) and \( ω^α.(β + 1) = ω^α.β + ω^α \).
\end{exercise}

\begin{exercise}\label{ex-ord-base-2}
	Define a function \( α ↦ 2^α \) satisfying
	\begin{align*}
		2^0 &= 1
		\\
		2^{α+1} &= 2^α + 2^α
		\\
		2^λ &= \supof{ 2^ξ }[ξ < λ]
	\end{align*}
	(You may find it useful to use the Cantor normal form theorem.)
	Show that this function is order preserving and continuous, and compute all fixed points of the function for ordinals \( α ≤ ε_0 \).
%	(It is not necessary to prove the existence of this function.)
\end{exercise}
%
\begin{exercise}
	\label{ex-epsilon-numbers}
	Let \( α ↦ ε_α \) be the enumerating function of the ordinals \( η \) such that \( η = ω^η \).
%	For each \( α \), define \( ε_α \) as the least ordinal such that
%	\begin{enumerate}
%		\item \( ε_β < ε_α \) for every \( β < α \),
%		\item \( ω^{ε_α} = ε_α \).
%	\end{enumerate}
	Express \( ε_α \) as a supremum of smaller ordinals as per \cref{d-epsilon0} and deduce that the enumerating function is defined for all ordinals.
\end{exercise}
%


\begin{exercise}
	\label{ex-ord-cnf-2}
	Prove the Cantor normal form theorem in base \( 2 \):
	\emph{For every ordinal \( α > 0 \) there exists unique ordinals \( α_n ≤ ⋯ ≤ α_0 ≤ α \) such that 
	\[
		α = 2^{α_0} + ⋯ + 2^{α_n}.
	\]}
\end{exercise}
%
\begin{exercise}
	\label{ex-ord-cnf-2o}
	What are the additive principal ordinals in base-$2$ normal form?
	Characterise the \( α \) such that \( 2^α = ω^α \).
\end{exercise}

%---------------------
%\section{Natural sum}
%---------------------

This brief foray into ordinals is concluded with another look at addition.
Recall that addition on ordinals is not commutative: \( 1 + ω ≠ ω + 1 \) for example.
It is possible to provide a natural notion of addition that \emph{is} commutative.
This is called the \emph{natural sum} (sometimes \emph{Hessenberg sum} after its originator Gerhard Hessenberg~\cite{Hess1906}). 
The Cantor normal theorem provides the means to achieve this.

\begin{definition}
	The natural sum of ordinals \( α \) and \( β \), denoted \( α \nsum β \) is defined by recursion on the two ordinals. \( 0 \nsum α = α \nsum 0 ≔ α \) for all \( α \).
	For non-zero \( α =_\NF ω^{α_0} + α_1 \) and \( β =_\NF ω^{β_0} + β_1 \)
	\[
		α \nsum β ≔ 
		\begin{cases}
			ω^{α_0} + ( α_1 \nsum β ), &\text{if \( α_0 ≥ β_0 \),}
			\\
			ω^{β_0} + ( α \nsum β_1 ), &\text{if \( α_0 ≤ β_0 \).}
		\end{cases}
	\]
	The operation of natural sum is well-defined as \( α_1 < α \) and \( β_1 < β \).
\end{definition}

As an operation on the Cantor normal form, the natural sum  has the following property.
\begin{lemma}
	For \( α =_\NF ω^{α_1} + ⋯ + ω^{α_m} \) and \( β =_\NF ω^{β_1} + ⋯ + ω^{β_n} \)
	\[
		α \nsum β ≔ ω^{γ_1} + ⋯ + ω^{γ_{m+n}}
	\]
	where \( γ_1 ≥ ⋯ ≥ γ_{m+n} \) enumerate the ordinals \( α_1, …, α_m , β_1 , …, β_n \) in descending order (with repetitions).
\end{lemma}

\begin{lemma}
	\label{ord-nsum}
	The natural sum is commutative and strongly increasing in both arguments: For all \( α \), \( β \), \( γ \),
	\begin{enumerate}
		\item \( α \nsum β = β \nsum α \);
		\item \( α < β \) implies \( α \nsum γ < β \nsum γ \).
	\end{enumerate}
\end{lemma}
%\begin{lemma}
%	\label{ord-nsum-2}
%	\begin{enumerate}
%		\item \( α \nsum β \)
%		\item \( α + β ≤ α \nsum β ≤ \max\setof{ α , β }.2 \).
%	\end{enumerate}
%\end{lemma}

\begin{exercise}
	Prove \cref{ord-nsum}.
\end{exercise}

\begin{exercise}
	\label{ex-ord-mult}
	Using the Cantor normal form theorem define a commutative multiplication \( α . β \) operation on ordinals. It should satisfy the distribution law:
	\(
		( α \nsum β ) . γ = (α .γ) \nsum (β .γ ).
	\)
	Hint, start from the function in exercise~\ref{ex-ord-mult-pre}.
\end{exercise}
