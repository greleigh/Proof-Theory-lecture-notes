\documentclass[%
	paper=174mm:247mm,
%	11pt,
	DIV=11,
	leqno,
	titlepage,
	headsepline,
	%headings=twolinechapter,
	headsepline=false,
%	partprefix=true,
%	tocindentauto,
	dynindent,
	dynnumwidth,
	toc=bib,
	toc=sectionentrywithoutdots,
	toc=chapterentrywithoutdots,
	% enabledeprecatedfontcommands,
	% final,
	oneside,
	bookmark,
	unicode
	]%
	{scrbook}
%
%\usepackage{polyglossia}


% Meta
\title{Lecture Notes in Proof Theory}
\subtitle{Work in progress}
\author{Graham E.\ Leigh}

\usepackage[utf8]{inputenc}
\usepackage[british]{babel}
\usepackage{xparse}
\usepackage{xcolor}
%
\usepackage[T1]{fontenc}
\usepackage{amsmath,amssymb}
\usepackage[lining]{libertine}
%\usepackage{pifont}
\usepackage[libertine,nosymbolsc]{newtxmath}
\usepackage[cal=rsfso]{mathalfa}
\useosf
%
\let\emptyset\emptysetAlt
\let\forall\forallAlt
\let\exists\existsAlt
%
% Packages
\usepackage{graphicx}
%
% Proofs
\usepackage{ebproof}
\usepackage[round]{natbib}
\bibliographystyle{plainnat}
\defcitealias{LogThe}{\emph{Logical Theory}}

% File styling.tex
% !TEX root = ../lecture-notes.tex
% LTeX: enabled=false
% Packages and default settings regarding typesetting and format
%---------------------------------
% Page layout assistance
\usepackage{scrlayer-scrpage}
\usepackage[marginparsep=2mm]{geometry}
%---------------------------------
% Theorem environments
\usepackage{mathtools}
%---------------------------------
% Styling for theorems
%
%---------------------------------
% BibLaTeX
%\usepackage[%
%  style=authoryear,% author-year style with extensions
%  backend=biber,%
%  articlein=false,% remove 'in' for journals
%  innamebeforetitle=true,% "in: Editor, Book"
%  dashed=false,% repeated authorship is not dashed
%  maxnames=1000,% don't cut listings short unless we want them too.
%  language=autobib,autolang=hyphen,% Can help with language
%  doi=false,isbn=false,eprint=false,% do not print doi, etc. 
%  ]{biblatex}
%
%---------------------------------
% ELEMENTS
\NewDocumentCommand{\Logic}{m}{\mathsf{#1}}
\NewDocumentCommand{\Theory}{m}{\mathsf{#1}}
\NewDocumentCommand{\Set}{m}{\mathrm{#1}}
\NewDocumentCommand{\Frml}{m}{\mathsf{#1}} % Why not working?
\NewDocumentCommand{\Symbol}{m}{\mathsf{#1}}
\NewDocumentCommand{\Lang}{m}{\mathcal{#1}}
% Rule names
\NewDocumentCommand{\Rule}{m}{\ensuremath{\mathsf{#1}}}
\NewDocumentCommand{\LeftRule}{m}{\Rule{L{#1}}}
\NewDocumentCommand{\RightRule}{m}{\Rule{R{#1}}}
%---------------------------------
% Disposition
% Part:
\renewcommand*{\partname}{Module} % Does not achieve anything!
\renewcommand*{\partformat}{Module~\thepart\autodot}
%\renewcommand*{\addparttocentry}[2]{%
%  \addtocentrydefault{part}{Module\nobreakspace #1}{#2}%
%	}%
\newcommand\partentrynumberformat[1]{Module~\ #1}
\RedeclareSectionCommand[
  tocentrynumberformat=\partentrynumberformat,
  tocnumwidth=6em
]{part}
% Chapter:
\renewcommand{\raggedchapter}{\centering}
\setkomafont{chapter}{\sffamily\mdseries\LARGE}
\addtokomafont{chapterentry}{\rmfamily\mdseries}
\addtokomafont{chapterprefix}{\normalsize}
% (Sub)section
\renewcommand{\raggedsection}{\centering}
\setkomafont{section}{\rmfamily\mdseries\sffamily\large}
\setkomafont{subsection}{\centering\rmfamily\mdseries\itshape}
% Paragraph
% - Normal (par-)spacing with emphasis leading
\setkomafont{paragraph}{\rmfamily\mdseries\itshape}
\def\ParIndent{\the\parindent}
\RedeclareSectionCommands[
	beforeskip=\parskip,
	indent=\ParIndent,
%	afterskip=1.5ex plus .5ex minus 0.2ex,
  ]{paragraph}
\addtokomafont{paragraph}{\raggedright} 
% Remove reference to chapter from section.
% \RedeclareSectionCommand[%
%   counterwithout=chapter,
%             ]{section}
%
%---------------------------------
% TABLE OF CONTENTS
%

%\usepackage[tocindentauto]{tocbasic}
%\usetocstyle{KOMAlike}
%---------------------------------
% Layout
% Page layouts
\setkomafont{pagehead}{\itshape}
\ohead{\upshape\thepage} % Outer side of pages
\rehead{\leftmark} % Right side of even pages
\lohead{\rightmark} % Left side of odd pages
\ofoot{\small\itshape G.E.~Leigh, version: \today}
\ifoot{}
\cfoot{}
\pagestyle{scrheadings}
%---------------------------------
% Control over lists:
\usepackage{enumitem}
% New defaults for enumerate & itemize:
\setlist{itemsep=0pt,topsep=\parsep}
%%   Label should be upright
\setlist[enumerate,itemize]{format=\normalfont}
%%   Label #1: roman (i), (ii), etc
%\setlist[enumerate,1]{label=(\roman*)}
%%   Label #2: alph (a), (b), etc
%\setlist[enumerate,2]{label=(\alph*)} 
%
% Special lists





%% 
%% Theorems et al
\usepackage[amsmath,thmmarks,hyperref]{ntheorem}
%% LTeX: enabled=false
%%%%%%%%%%%%%%%%%%%%%%%%%%%%%%%%%%%%%%%%%
%% BEGIN: §7. Environments
%%%%%%%%%%%%%%%%%%%%%%%%%%%%%%%%%%%
%%% §7.2. Default (plain) style
%% Our Theorem styles:
\makeatletter
%%%
%% plain:
%%   Theorem 6.13 · Löb's theorem  
\renewtheoremstyle{plain}
    {\item[\hskip\labelsep \theorem@headerfont ##1\ ##2\theorem@separator]}%
    {\item[\hskip\labelsep \theorem@headerfont ##1\ ##2\ \textperiodcentered\ ##3\theorem@separator]}
% plain w/o number:
\renewtheoremstyle{nonumberplain}
    {\item[\hskip\labelsep \theorem@headerfont ##1\theorem@separator]}%
    {\item[\hskip\labelsep \theorem@headerfont ##1\ \textperiodcentered\ ##3\theorem@separator]}
%%% 
% implicit: (uses optional argument for name)
%   9.20 Löb's Theorem
%
\newtheoremstyle{implicit}
    {\item[\hskip\labelsep \theorem@headerfont ##1\ ##2\theorem@separator]}%
    {\item[\hskip\labelsep \theorem@headerfont ##3\ ##2\theorem@separator]}
%
% no-number version  
\newtheoremstyle{nonumberimplicit}
    {\item[\hskip\labelsep \theorem@headerfont ##1\theorem@separator]}%
    {\item[\hskip\labelsep \theorem@headerfont ##3\theorem@separator]}
%%%
%%% remark:
%%%    Remark on A6
%%
%%\newtheoremstyle{remark}
%%  {\item[\hskip\labelsep \theorem@headerfont ##2\ ##1\theorem@separator]}%
%%  {\item[\hskip\labelsep \theorem@headerfont ##2\ ##1 on\ ##3\theorem@separator]}
%%% no-number version:
%%\newtheoremstyle{nonumberremark}
%%  {\item[\hskip\labelsep \theorem@headerfont ##1\theorem@separator]}%
%%  {\item[\hskip\labelsep \theorem@headerfont ##1\ on\ ##3\theorem@separator]}
%%%
%% PROOF
%%   Proof [of theorem 8.71]
\newtheoremstyle{proof}
  {\item[\hskip\labelsep \theorem@headerfont ##1\theorem@separator]}%
  {\item[\hskip\labelsep \theorem@headerfont ##1\ ##3\theorem@separator]}
\makeatother
%%%
\theoremstyle{plain}
%\theoremheaderfont{\upshape\scshape}
\newtheorem {theorem}           {Theorem} [chapter]
\newtheorem {lemma}[theorem]    {Lemma}
\newtheorem {proposition}[theorem] {Proposition}
\newtheorem {claim}[theorem]    {Claim}
\newtheorem {corollary}[theorem]{Corollary}
\newtheorem {example}[theorem]	{Example}
%%
\theoremstyle{implicit}
%\newtheorem*{thing}{thing}
%% \newtheorem*{covind}    {Course of values induction}
%% \newtheorem*{natind}    {Principle of induction for natural numbers}
%%
%\newtheorem{namedparadox}[theorem]{paradox}
\newtheorem{namedtheorem}[theorem]{Theorem}
\newtheorem{namedlemma}[theorem]{Lemma}
\newtheorem*{nameonly}{Thing}
%%%
%%%%%%%%%%%%%%%%%%%%%%%%%%%%%%%%%%%%
%%%% §7.3. Upright style
\theoremstyle{plain}
\theorembodyfont{\upshape}
\newtheorem{definition}[theorem] {Definition}
%%\newtheorem{assumption}[theorem]  {assumption}
%%%%%%%%%%%%%%%%%%%%%%%%%%%%%%%%%%%%
%%%% §7.4. Proof style (no number)
\theoremstyle{proof}
\theorembodyfont{\upshape}
%% \theoremprework{\arabicenum}
\theoremsymbol{\ensuremath{\dashv}}
\newtheorem {proof} {Proof}
%%%%%%%%%%%%%%%%%%%%%%%%%%%%%%%%%%%%
%%%% §7.5. Exercise (remove?)
\theoremstyle{plain}
\theorembodyfont{\upshape}
\newtheorem {exercise}[theorem]   {Exercise}
%%%%%%%%%%%%%%%%%%%%%%%%%%%%%%%%%%%%
%%%% §7.6. Remark
%%% Uses special remark formatting
\theoremstyle{remark}
\theoremsymbol{} % no end of remark symbol
\newtheorem*{remark}{remark}
%\newtheorem*{convention}{convention}
%
%% END

%
\usepackage{input/unicode-char}
%
% COMMENT THESE TO USE NOTES:
\newcommand{\note}[1]{}
%\newcommand{\tbw}{}
%
\usepackage{xparse}
%---------------------------------
% 
%---------------------------------
% Proof-theoretic tools
\NewDocumentCommand{\pto}{m}{\lVert{#1}\rVert}
%
\usepackage[bookmarks,unicode]{hyperref}
\makeatletter
\hypersetup{pdftitle={\@title},pdfauthor={\@author}}
\makeatother
\usepackage[nameinlink,noabbrev]{cleveref}
%%
\let\autoref\cref
%\crefname{assumption}{assumption}{assumptions}
%\crefname{claim}{claim}{claims}
%\crefname{proposition}{proposition}{propositions}
%\crefname{proposition2}{proposition}{propositions}
%\crefname{example}{example}{examples}
%\crefname{definition}{definition}{definitions}
%\crefname{corollary}{corollary}{corollaries}
%\crefname{corollary2}{corollary}{corollaries}
%\crefname{lemma}{lemma}{lemmas}
%\crefname{lemma2}{lemma}{lemmas}
%\crefname{namedparadox}{paradox}{paradoxes}
%\crefname{namedtheorem}{theorem}{theorems}
%\crefname{namedlemma}{lemma}{lemmas}
%\crefname{namedexample}{example}{examples}
%\crefname{namedassumption}{assumption}{assumptions}
%\crefname{principle}{principle}{principles}
%\newcommand{\creflastconjunction}{, and\nobreakspace}

\includeonly{
	module-0/about,
	module-0/intro,
	module-1/natural-deduction,
	module-1/sequent-calculus,
	module-1/properties,
	module-2/ce,
	module-2/conseq-of-ce,
	module-2/equality,
%	module-2/game,
	module-3/arithmetic,
	module-3/ordinal-interlude,
	module-3/ordinal-analysis,
	module-3/ti-and-pto,
	%
%	solutions/part-1,
%	solutions/part-2,
%	solutions/part-3,
	}

\begin{document}
\frontmatter
\maketitle

%%
\chapter{About this text}
\label{c-about}
%\addcontentsline{toc}{chapter}{\nameref{c-about}}
%
These lecture notes are designed to accompany the course \emph{Proof Theory} given to second semester students of the \emph{Master in Logic} at the University of Gothenburg, Sweden.

% TOC
\tableofcontents

\mainmatter

%
\chapter{Lend me thy proof}
%

What does a proof tell about a theorem beyond its truth?
If the theorem states the existence of an object to what extent does the proof isolate the object in mind?
The reader will be familiar with the classical logic and the method of ‘proof by contradiction’ --- also known by the Latin phrase \emph{reductio ad absurdum} --- whereby an existential claim can be established by showing the negative \emph{universal} claim to be contradictory.
The mere statement of a theorem does not determine whether such method of proof is used or necessary.
One proof of a theorem may directly construct a witness.
Another may invoke only indirect reasoning but, perhaps, relies on fewer assumptions.
A third might be too complex to know because it appeals to lemmas whose proofs you do not have access to.
And only a characterisation of the mathematical theories in which the theorem holds can answer the question: Can the theorem be proved \emph{only} by indirect methods?

With logic in mind, other questions also stand out.
How \emph{complex} is logic? 
For that matter, what does it mean to say that one proof is more complex than another?
Neither question can be given a definite answer, but we do get a handle on issues like these by studying proofs, comparing and manipulating proofs.
I will show, for example, that every classically valid formula can be given a proof in which only subformulas of the conclusion are used.
Such a proof will not, in general, be the shortest such proof nor the most concise.
But it is the \emph{simplest} in the sense that it references only concepts no more complex than that being proved.
%Is there an algorithm that given a formula returns a proof of the formula if one exists?

The reader will also be shown situations of the opposite kind: an example of a mathematical theorem admitting an elementary proof but for which every proof necessarily refers to concepts \emph{more} complex than the conclusion.
No doubt you will have encountered such cases before, although you may not have realised at the time: the scenario is arithmetic and the theorem one of many examples of which all proofs (in the language of arithmetic) necessitate a stronger induction invariant than the theorem itself.

%Arithmetic, of course, 

Speaking of arithmetic, few readers would doubt the consistency of \emph{Peano} arithmetic, the first-order theory axiomatised by the defining equations for functions of successor, addition and multiplication, plus the axiom schema of induction.
One need only observe that each axiom is a true statement about the natural numbers, that is, that the structure of the natural numbers and its elementary functions forms a model of the Peano axioms.
But the standard model of arithmetic is overkill for the purpose of consistency of the Peano axioms.
Gödel's incompleteness theorem presents statements in the language of arithmetic that are true yet \emph{not} provable from the Peano axioms.
An obvious question arises:
What mathematical assumptions truly underpin the consistency of Peano arithmetic and, for that matter, other mathematical theories?
%How much \emph{more} complex is Peano arithmetic than, say, the subtheory with induction axioms removed?

This, in a nutshell, is
\emph{Proof Theory}: the mathematical theory of formal proofs and, by extension, the mathematical theory of axiomatic systems.
%
And through the course of this text you, dear reader, will see  for yourself the delights and delicacies that only a proof conceals.
Together we will taste the sweetness of the cherry on the top, break through the smooth crust and sample the richness beneath.
%Some will be sitting on top for all to see, others

But, as they say, the proof of the pudding is in the eating.
%
I hope you are hungry.



% Module 1
\part{Two Calculi for Two Logics}
\label{module-1}
% --------------------------------------
\chapter{Preliminaries}
% --------------------------------------

Set-theoretic notation. Let \( X \) be a set.
\begin{itemize}
	\item \( \Pow X \) is the power set of \( X \).
	\item \( X^{<ω} \) denotes the set of finite sequences in \( X \), namely the set of \( ( x_1 , …, x_n ) \) for \( n < ω \) and \( x_i ∈ X \).
	\item The cardinality of \( X \) is denoted \( \card X \); except where stated otherwise, this will invariably be either a natural number (\( X \) is finite) or \( ω \) (meaning \( X \) is countably infinite).
\end{itemize}


%I begin with defining the syntactic notions of formula, etc.

% ------------------------
\section{The language of logic}
% ------------------------

The \emph{logical symbols of propositional logic} are
\[ 
	∧ \quad  ∨ \quad  → \quad ⊥ 
\]
collectively called the \emph{(propositional) connectives}.
To these one adds a collection \( \Lang P \) of \emph{propositional variables} to obtain a \emph{propositional language}.

The \emph{logical symbols of (first-order) predicate logic} extends the propositional symbols by
\begin{itemize}
%	\item Connectives: \( ∧ \), \( ∨ \), \( → \) and \( ⊥ \).
	\item Quantifiers: \( ∀ \) and \( ∃ \).
	\item Bound variables: \( v_0 \), \( v_1 \), ….
	\item Free variables: \( a_0 \), \( a_1 \), ….
\end{itemize}
%
%
\begin{definition}
A \emph{first-order language} extends the logical symbols by two sets of \emph{non-logical} symbols:
\begin{itemize}
	\item a set \( \Lang F \) of \emph{function symbols};
	\item a set \( \Lang P \) of \emph{predicates};
	\item a function \( \arity \colon \Lang F ∪ \Lang P → ℕ \) assigning each symbol a natural number called the \emph{arity}.
\end{itemize}
The \emph{propositional} fragment of a first-order language arises by dropping all function symbols and all predicates of non-zero arity.
\end{definition}
%
A function symbol \( f ∈ \Lang F \) with arity \( \arity (f) = 0 \) is called a \emph{constant}; 
a predicate \( P ∈ \Lang P \) with arity \( 0 \) is a \emph{propositional variable}.
A symbol is \emph{nullary}, \emph{unary}, …, \emph{$n$-ary} if it has arity \( 0 \), \( 1 \), …, \( n \), etc..


Terms and formulas are defined as expected but should adhere to the free/bound variable distinction.
%\begin{itemize}
%%	\item Terms never contain bound variables.
%	\item Quantifiers only ever \emph{bind} bound variables.
%\end{itemize}
%
%
\begin{definition}[Term]
	The \emph{terms} are defined inductively as follows:
	\begin{enumerate}
		\item Every free variable is a term
		\item Given terms \( t_1 \), …, \( t_n \) and an \( n \)-ary function symbol \( f \), \( f t_1 ⋯ t_n \) is a term.
	\end{enumerate}
	Adding bound variables into the mix creates the class of syntactic objects I call \emph{pre-terms}:
	\begin{enumerate}
		\item Every variable (free or bound) is a pre-term.
		\item Given pre-terms \( t_1 \), …, \( t_n \) and an \( n \)-ary function symbol \( f \), \( f t_1 ⋯ t_n \) is a pre-term.
	\end{enumerate}
	The variables occurring in a pre-term \( t \) are called the \emph{active} (of the pre-term).
	In other words, a term is a pre-term in which only free variables are active.
	A (pre-)term with no active variables is \emph{closed}.
\end{definition}
%
With terms come formulas.

\begin{definition}[Atomic formula; prime formula]
	An expression \( P t_1 ⋯ t_n \) where \( P \) is an \( n \)-ary predicate and \( t_1 \), …, \( t_n \) are terms is called an \emph{atomic formula}.
	The atomic formulas together with the symbol \( ⊥ \) are, collectively, the \emph{prime} formulas.
\end{definition}
%
%
\begin{definition}[Pre-formula]
	The \emph{pre-formulas}, and their active variables, are generated by the following clauses.
	\begin{enumerate}
		\item Given pre-terms \( t_1 \), …, \( t_n \) and \( n \)-ary predicate \( P \), \( P t_1 ⋯ t_n \) is a pre-formula. A variable is \emph{active} in \( P t_1 ⋯ t_n \) iff it is active in (at least) one of the \( t_i \).
		\item \( ⊥ \) is a pre-formula with no active variables.
		\item If \( F \) and \( G \) are pre-formulas, then so is \( F → G \), \( F ∧ G \) and \( F ∨ G \). The active variables in each case is the union of active variables of each of \( F \) and \( G \).
		\item If \( F \) is a pre-formula and \( x \) a bound variable, then \( ∀x F \) and \( ∃ x F \) are pre-formulas. The active variables of \( ∀ x F \) and \( ∃ x F \) are the active variables of \( F \) minus   the variable \( x \).
	\end{enumerate}
\end{definition}
%
\begin{definition}[Formula]
	A \emph{formula} is a pre-formula for which only free variables are active.
	A formula with \emph{no} active variables is called \emph{closed} or a \emph{sentence}.
\end{definition}

It is helpful in many cases to associate a numerical measure of \emph{complexity} to (pre-)formulas.
There are many choices.
I take the following.

\begin{definition}[Formula complexity]
	The \emph{complexity} of a pre-formula \( F \) is denoted \( \rk F \) and determined by recursion on its generation:
	\begin{itemize}
		\item \( \rk F = 0 \) if \( F \) is prime.
		\item \( \rk{ F ∧ G } = \rk{ F ∨ G } = \rk{ F → G } = \maxof{ \rk F, \rk G} + 1 \).
		\item \( \rk{ ∀x F } = \rk{ ∃x F } = \rk{ F } + 1 \).
	\end{itemize}
\end{definition}


%
\begin{definition}[Substitution]
%	Let \( F \) be a pre-formula and \( \vec t = t_0 , …, t_n \) a finite sequence of pre-terms.
%
	Let \( F \) be a pre-formula, \( a \) a free variable and \( t \) a pre-term.
	I write \( F[t/a] \) for the pre-formula that results by substituting \( t \) for \( a \) in \( F \).
	This is defined recursively over terms and formulas:
	\begin{align*}
		a_j [ t/a_i ] &= 
		\begin{cases}
			t, &\text{if \( i = j \),}
			\\
			a_j, &\text{otherwise.}
		\end{cases}
		\\
		v_j [ t/a_i ] &= v_j.
		\\
		( S t_1 ⋯ t_n ) [ t/a_i ] &= S ( t_1 [ t/a_i ] ) ⋯ ( t_n [ t/a_i ] ) \; \text{for \( S ∈ \Lang F ∪ \Lang P \).}
%		\\
%		( P t_1 ⋯ t_n ) [ t/a_i ] &= P ( t_1 [ t/a_i ] ) ⋯ ( t_n [ t/a_i ] )
		\\
		⊥ [ t/a_i ] &= ⊥.
		\\
		( F * G ) [ t/a_i ] &= F [ t/a_i ] \,*\, G [ t/a_i ]  \; \text{for \( * ∈ \setof{∧ , ∨,→} \).}
		\\
		( Q x F ) [ t/a_i ] &= Qx \;  F [ t/a_i ] \; \text{for \( Q ∈ \setof{ ∀ , ∃ } \).}
	\end{align*}
	Simultaneously substituting a sequence of terms \( \vec s = s_0 , …, s_n \) for variables \( \vec c = c_{0} , …, c_{n} \) (where \( c_i ≠ c_{j} \) for all \( i<j ≤ n \)) is defined in the expected way and denoted \( F[ \vec s/\vec c] \) or \( F[ s_0/c_{0}, …, s_n/c_n ] \).
\end{definition}

\begin{lemma}
	If \( F \) is a formula and \( t \) a term,  \( F[t/a] \) is a formula.
\end{lemma}

\begin{convention}[Meta-variables]\label{conv-metavar}
	Henceforth, I adopt the following naming conventions except where explicitly stated otherwise.
	\begin{itemize}
		\item lower case roman letters:
		\begin{itemize}
		\item \( x \), \( y \), \( z \) (often with subscript, \( x_0 \), \( x_1 \), etc.\@) denote \emph{bound} variables,
		\item \( a \), \( b \), etc.\@ denote \emph{free} variables,
		\item \( r \), \( s \) and \( t \) range over terms (not pre-terms);
		\end{itemize}
		\item upper case Roman letters \( F \), \( G \), \( A \), \( B \), etc.\@ denote pre-formulas;
		\item upper case Greek letters \( Γ \), \( Δ \), etc.\@ denote finite sets of formulas.
	\end{itemize}
\end{convention}

\begin{convention}[Denoting substitution]\label{conv-subst}
	I will often introduce a formula along with a free variable, in the form of \( F(a) \). 
	This notation indicates that \( a \) is to be the focus of subsequent substitutions, whereby I write \( F(t) \) in place \( F[t/a] \).
	
	Sometimes the free variable \( a \) is not mentioned explicitly.
	I may introduce a (pre-)formula as \( F(x) \), subsequently writing \( F(t) \).
	In this context, \( F(x) \) formally represents \( G[x/a] \) for an appropriate choice of \( a \) (and \( G \)), whereby \( F(t) \) means \( G[t/a] \).
\end{convention}


Finally, I introduce three formulaic abbreviations:
\begin{itemize}
	\item \( ⊤ \) abbreviates the formula \( ⊥ → ⊥ \).
	\item \( ¬ F \) is shorthand for \( F → ⊥ \).
	\item \( F ↔ G \) abbreviates \( ( F → G ) ∧ ( G → F ) \).
\end{itemize}


% --------------------------------------
\section{Orders and trees}
% --------------------------------------

The various derivation calculi I present are all based on the mathematical notion of a tree.
%I will start with recapping the mathematical definition of a tree.
Recall that a relation \( ≤ \) on a set \( X \) is a \emph{partial order} iff it is:
\begin{enumerate}
	\item \emph{Reflexive:} \( x ≤ x \) for all \( x ∈ X \),
	\item \emph{Transitive:} if \( x ≤ y \) and \( y ≤ z \), then \( x ≤ z \),
	\item \emph{Anti-symmetric:} if \( x ≤ y \) and \( y ≤ x \) then \( x = y \),
\end{enumerate}
%
and is a \emph{linear order} if, in addition, it is
\begin{enumerate}[resume]
	\item \emph{Linear:} for all \( x,y ∈ X \), either \( x ≤ y \) or \( y ≤ x \).
\end{enumerate}
%
Given a partial order \( ≤ \), I use \( < \) to denote the derived \emph{strict} suborder, defined by \( x < y \) iff \( x ≤ y \) and \( x ≠ y \).

A \emph{tree} is a non-empty set \( T \), elements of which are called \emph{vertices}, equipped with a partial order \( ≤ \) satisfying:
\begin{enumerate}
	\item There exists a \( ≤ \)-minimal vertex \( * ∈ T \), called the \emph{root}. Minimality of \( * \) means that \( * ≤ v \) for all \( v ∈ T \).
	\item For every \( v ∈ T \), the set \( \setof{u ∈ T}[u ≤ v] \) of \emph{predecessors} of \( u \) is finite and linearly ordered by \( ≤ \).
\end{enumerate}

The following are straightforward consequences of the definition. Let \( ( T, ≤ ) \) be a tree and \( v \) a vertex.

\begin{lemma}
	Every non-root vertex has a \( ≤ \)-maximal predecessor.
	I.e., for every \( v ≠ * \) there is a unique \( v_* < v \) such that \( u < v \) iff \( u ≤ v_* \).
\end{lemma}

The vertex \( v_* \) in the above lemma is called the \emph{immediate predecessor} of \( v \), and I say that \( v \) is a \emph{child} of \( v_* \).
%A \emph{child} of \( v \) is any vertex \( u > v \) such that \( v = u_* \).
The set of \emph{children} of \( v \) is denoted
\[
	\Child_T(v) = \setof{u ∈ T}[v = u_*] .
\]
A \emph{leaf} is a vertex with no children.
If the tree is clear from context I will omit its mention, writing, for instance, \( \Child (v) \) for \( \Child_T(v) \).

For the most part, I will only treat finite trees.
In Module~\ref{module-3}, I present a sequent calculus whose derivations are, in general, infinitely branching trees, i.e., the set \( \Child(v) \) can be infinite for some \( v \).
Such trees will still maintain another notion of finiteness in the sense of not containing any infinite branches, called \emph{well-founded} trees.
%
\begin{definition}[Well-founded tree]
%	Let \( ≤ \) be a partial order on a set \( T \).
%	I call \( ≤ \) \emph{well-founded} if there is no infinite for every sequence
	Let \( (T , ≤ ) \) be a tree.
	A \emph{path} through \( T \) is a finite sequence \( (v_i)_{i ≤ k} \) such that \( v_{i+1} ∈ \Child (v_i) \) for all \( i < k \).
	An infinite sequence \( (v_i)_{i ∈ ℕ } \) for which the prefix \( ( v_i)_{i ≤ k} \) is a path for every \( k \) is called a \emph{branch}.
	
	A tree is \emph{well-founded} iff it has no branches.
\end{definition}

% --------------------------------------
\chapter{Natural deduction}
% --------------------------------------

%Although I recap the main
I assume the reader is familiar with some deduction calculus/proof system for classical predicate logic.
The precise calculus is not so important, so long as you understand the concept of a derivation system and the kind of object that constitutes a formal proof.

In these notes, I utilise two such calculi: \emph{natural deduction} and \emph{sequent calculi}.
The latter will be introduced and covered in much detail, beginning in the next chapter.
Natural deduction will also be given a formal definition.
But I will gloss over some aspects that would be considered important in a first introduction, assuming that you either have seen the material before (for instance, in~\citetalias{LogThe}) or can quickly relate the natural deduction calculus to the formal calculus you are most familiar with.

The various notions of \emph{(formal) deduction/proof}, including those addressed in this work, have a common underlying structure.
Namely, they are trees in which vertices are labelled by a certain kind of syntactic object --- formulas in the case of natural deduction --- and the relation between the label of a vertex and the label of its children corresponds to one of number of specified \emph{rules of inference}.

An \emph{inference over \( X \)} is a pair \( ( P , c ) ∈ X^{<ω} × X \), where \( P \) is a finite sequence of objects --- the \emph{premises} --- and \( c \) is a single deduction object, called the \emph{conclusion}.
For \( P = ( P_1 , …, P_n ) \), I visualise the inference \( ( P , C ) \) as a rule
\[
%\begin{prooftree}
%	\hypo{ \Setof{ p }[ p ∈ P] }
%	\infer1[$(P,c)$]{c}
%\end{prooftree}
%\qquad\text{or, if }P = \setof{p_1, …, p_n},\quad
\begin{prooftree}
	\hypo{ P_1 }
	\hypod \hypo{ P_n }
	\infer3{C}
\end{prooftree}
\]
which \emph{derives} the conclusion \( C \) from premises \( P_1 \), …, \( P_n \).

In natural deduction the set of formulas play the role of \( X \) and inferences relate finitely many premises (indeed, at most three) to a conclusion formula.
In short, an inference in natural deduction is a pair \( ( Γ , F ) \) where \( Γ \) is a finite sequence of formulas.
Examples are
\[
  \begin{prooftree}
  	\hypo{F}
  	\infer1{F ∨ G}
  \end{prooftree}
	\qquad
  \begin{prooftree}
  	\hypo{F}
  	\hypo{G}
  	\infer2{F ∧ G}
  \end{prooftree}
	\qquad
  \begin{prooftree}
  	\hypo{ F →  G }
  	\hypo{F}
  	\infer2{G}
  \end{prooftree}
\]
%The third inference above is the form of the elimination rule for disjunction.

Abstractly then, an derivation calculus comprises two pieces of information:
\begin{itemize}
	\item A set \( \Coll O \) of deduction \emph{objects};
	\item A set \( \Coll I ⊆ \Coll O^{<ω} × \Coll O \) of \emph{inferences} or \emph{rules}.
\end{itemize}

\begin{definition}[Derivation]
	A \emph{derivation} in the calculus \( (\Coll O , \Coll I ) \) is a tree \( (T, ≤) \) together with a map \( o \colon T → \Coll O \) assigning a deduction object to each vertex such that every vertex together with its children is an inference, i.e., for all \( v \) there exists an enumeration \( v_1 , …, v_n \) of \( v \)'s children such that
	\[
		\bigl( \, ( o(v_1) , …, o(v_n) ) , o(v) \, \bigr) ∈ \Coll I \!.
	\]
	The deduction object labelling the root of \( T \) is called the \emph{conclusion} of the derivation.
\end{definition}

Observe that every vertex of a derivation must be the conclusion of an inference. 
In particular, objects labelling \emph{leaves} must be derivable without premises.
In some calculi zero-premise inferences correspond to \emph{axioms} but in natural deduction they represent \emph{assumptions} which other rule can ‘mark’ as \emph{discharged} (or left \emph{undischarged}) by certain inferences.
Rules that ‘discharge’ assumptions are drawn as
\[
%  \begin{prooftree}
%	\hypo{[A]}
%	\ellipsis{}{B}
%	\infer1{C}
%\end{prooftree}
%\qquad \text{or}\qquad
  \begin{prooftree}
  	\hypo{b_0}
	\hypo{[a_1]}
	\ellipsis{}{b_1}
	\hypod
	\hypo{[a_n]}
	\ellipsis{}{b_n}
	\infer4{c}
\end{prooftree}
\]
illustrating that there is an inference \( \smash{\bigl( ( b_0,…,b_n) , c \bigr)} \) deriving \( c \) from premises \( b_0 \), …, \( b_n \) and that in the subtree above \( b_i \) (\( i>0 \)) occurrences of an \emph{assumption} \( a_i \) can be considered discharged.

Examples of assumption-discharging inferences in natural deduction are the introduction rule for implication and elimination rule for disjunction:
\[
\begin{prooftree}
	\hypo{[F]}
	\ellipsis{}{G}
	\infer1{ F → G }
\end{prooftree}
\qquad \qquad
\begin{prooftree}
  	\hypo{F ∨ G}
	\hypo{[F]}
	\ellipsis{}{H}
	\hypod
	\hypo{[G]}
	\ellipsis{}{H}
	\infer4{H}
\end{prooftree}
\]
%Labelling or recording assumptions as discharged does not need to be part of the deduction system, per se, but determines

A final quirk of natural deduction is that some inferences are only applicable if the subderivation above them fulfils some condition.
This is the case for two of the quantifier inferences:
\[
	\begin{prooftree}[center=false]
		\hypo{ F(a) }
		\infer1{ ∀x F(x) }
	\end{prooftree}
	\qquad
	\begin{prooftree}[center=false]
		\hypo{ ∃x F(x) }
		\hypo{ [ F(a) ] }
		\ellipsis{}{ H }
		\infer2{ H }
	\end{prooftree}
\]
which are associated a condition --- the \emph{eigenvariable} condition --- that restrict their application to contexts in which the distinguished free variable \( a \) does not occur in any \emph{undischarged} assumptions above the application of this rule.

It is these additional constraints on natural ‘deductions’ that explain the use of the term ‘derivation’ above rather than ‘deduction’ or ‘proof’.
Formally, some proof calculi are also equipped with a ‘correctness condition’ that identifies when a \emph{derivation} is \emph{well-formed} and can be considered a \emph{deduction} or a \emph{proof} in the calculus.

Before turning our full attention to natural deduction, I want the clarify the distinction between \emph{inference} and \emph{rule}.
Above, I used the two terms almost interchangeably, as a specification of the premises and conclusions that can be used in derivations of a particular calculus.
But when defining a system like natural deduction, we do not actually write down all the inferences. Rather, we present a set of schema or \emph{rules} that generate the \emph{inferences}.
%Usually, it is the rules that we give names to
The \emph{rule} called \( \conjI \), or \emph{\( ∧ \)-introduction}, is the set of all inferences
\[
  \begin{prooftree}
  	\hypo{F}
  	\hypo{G}
  	\infer2[$\conjI$]{F∧G}
  \end{prooftree}
\]
where $F$ and $G$ range over formulas.
%For derivations, I identify a


Thus, to fully specify a deduction calculus a third bit of information is sometimes (though not always) needed:

\begin{definition}[Calculus; deduction]
	A \emph{calculus} is a triple \( \Logic C = ( \Coll O , \Coll R , \Coll C ) \) where
	\begin{itemize}
		\item \( \Coll O \) is a set of \emph{deduction objects};
		\item \( \Coll R  \) is a set of \emph{rules}, where a rule is a set of inference over \( \Coll O \); % generating the inferences \( \Coll I = \bigcup _{\Rule r ∈ \Coll R} \Rule r \);
		\item \( \Coll C \) is a \emph{correctness condition} specifying which derivations in \( \Coll O \) and \( \Coll R \) are \emph{deductions}.
	\end{itemize}
\end{definition}
%
I could, of course, have incorporated the correctness condition into the definition of derivation.
Indeed, this is the approach that most texts in formal logic take.
I write ‘formal logic’ and not ‘proof theory’ because the majority of textbooks in proof theory including the standard references~\citep{Negri_von_Plato,Troelstra_Schwichtenberg_2000,Schu1950,Pohlers_1989}
are, understandably, more careful with the definition of proof.
%Nevertheless, the reader will have difficulty finding a text on proof theory that defines a calculus quite in the way I have.
%Either the text st

The purpose of drawing attention to the correctness condition is to emphasise that \emph{proofs} (in natural deduction) are not completely straightforward and, ultimately, isolating a deduction calculus with only trivial correctness conditions (finiteness say) will be extremely helpful.

Recall that only the deduction objects and rules are required for the definition of \emph{derivation}
I deliberately refrain from giving a formal definition of what constitutes a correctness condition as it is not particularly important.

% --------------------------------------
\section{Intuitionistic logic}
% --------------------------------------


%A \emph{deduction} is a finite tree of formulas such that the formula labelling a vertex is related to the formula(s) labelling the children
%The rules of intuitionistic natural deduction are
In natural deduction, each logical connective and quantifier is associated two kinds of rule: an \emph{introduction} rule, how to \emph{infer} a formula using the connective, and an \emph{elimination} rule, what can \emph{inferred} from the connective.
%In some cases, either category is further split into cases (such as \( \conjE \) below).

I present each rule in turn. Unless otherwise stated, in all cases symbols $F$, $G$, $a$, $t$, etc.\@ act as meta-variables ranging over all objects of the appropriate type, rather than denoting specific instances.
\begin{figure}
\centering \ebproofset{center=false}
\begin{gather*}
\begin{prooftree}
	\axiom[\assumpo]{F}
\end{prooftree}
\qquad
\begin{prooftree}
	\axiom[\assumpd]{F}
\end{prooftree}
\end{gather*}
\caption{Natural deduction assumption rules}
\label{f-ND-ass}
\end{figure}
%
\begin{figure}
\centering \ebproofset{center=false}
\begin{gather*}
\begin{prooftree}
	\hypo{ [F] }
	\ellipsis{}{G}
	\infer1[$\impI$]{ F → G }
\end{prooftree}
\qquad
\begin{prooftree}
	\hypo{F}
	\hypo{G}
	\infer2[$\conjI$]{ F ∧ G }
\end{prooftree}
\qquad
\begin{prooftree}
	\hypo{ F(a) }
	\infer1[$\faI_a$]{ ∀x F(x) }
\end{prooftree}
\\[8pt]
\begin{prooftree}
	\hypo{ F }
	\infer1[$\disjI_0$]{ F ∨ G }
\end{prooftree}
\qquad
\begin{prooftree}
	\hypo{ G }
	\infer1[$\disjI_1$]{ F ∨ G }
\end{prooftree}
\qquad
\begin{prooftree}
	\hypo{ F(t) }
	\infer1[$\exI$]{ ∃ x F(x) }
\end{prooftree}
\end{gather*}
\caption{Natural deduction introduction rules}
\label{f-ND-intro}
\end{figure}
%
\begin{figure}
\centering \ebproofset{center=false}
\begin{gather*}
\begin{prooftree}
	\hypo{ ⊥ }
	\infer1[$\botE$]{ F }
\end{prooftree}
\qquad
\begin{prooftree}
	\hypo{F ∧ G}
	\infer1[$\conjE_0$]{ F }
\end{prooftree}
\qquad
\begin{prooftree}
	\hypo{F ∧ G}
	\infer1[$\conjE_1$]{ G }
\end{prooftree}
\qquad
\begin{prooftree}%[center=false]
	\hypo{ ∀x F(x) }
	\infer1[$\faE$]{ F(t) }
\end{prooftree}
\\[8pt]
\begin{prooftree}
	\hypo{ F → G }
	\hypo{ F }
	\infer2[$\impE$]{ G }
\end{prooftree}
\qquad
\begin{prooftree}
	\hypo{ F ∨ G }
	\hypo{ [F] }
	\ellipsis{}{H}
	\hypo{ [G] }
	\ellipsis{}{H}
	\infer3[$\disjE$]{ H }
\end{prooftree}
\qquad
\begin{prooftree}[center=false]
	\hypo{ ∃x F(x) }
	\hypo{ [F(a)] }
	\ellipsis{}{H}
	\infer2[$\exE_a$]{ H }
\end{prooftree}
\end{gather*}
\caption{Natural deduction elimination rules}
\label{f-ND-elim}
\end{figure}

\paragraph{Assumption.} There are two zero-premise rules for initiating formulas as \emph{assumptions}:
\[
  \begin{prooftree}
  	\axiom[\assumpo]{F}
  \end{prooftree}
  \qquad
  \begin{prooftree}
  	\axiom[\assumpd]{F}
  \end{prooftree}
\]
The only difference between the rules is their name. The fact that we have two, identical, rules for assumptions is because of their double role, either as \emph{open} assumptions or as assumptions which have been \emph{discharged} by another rule.
The two rules \( (\assumpo) \) and \( (\assumpd) \) represent this dichotomy: the conclusion of the rule \( (\assumpo) \) will be called an \emph{open} assumption, and conclusions of \( (\assumpd) \) as \emph{discharged}, or \emph{closed}, assumptions.
%As the assumption rule is the only zero-premise rule in natural deduction, mention of it will be suppressed.

\paragraph{Falsum.} The connective \( ⊥ \) has no introduction rule, but an elimination rule that permits the derivation of any formula from it.
\[
  \begin{prooftree}[center=false]
  	\hypo{ ⊥ }
  	\infer1[$\botE$]{ F }
  \end{prooftree}
\]


\paragraph{Conjunction.} This connective is associated one introduction rule and two elimination rules:
\[
  \begin{prooftree}
  	\hypo{F}
  	\hypo{G}
  	\infer2[$\conjI$]{ F ∧ G }
  \end{prooftree}
  \qquad
  \begin{prooftree}
  	\hypo{F ∧ G}
  	\infer1[$\conjE_0$]{ F }
  \end{prooftree}
  \qquad
  \begin{prooftree}
  	\hypo{F ∧ G}
  	\infer1[$\conjE_1$]{ G }
  \end{prooftree}
\]

\paragraph{Implication.} An introduction rule (discharging an assumption) and one elimination rule
\[
  \begin{prooftree}[center=false]
  	\hypo{ [F] }
  	\ellipsis{}{G}
  	\infer1[$\impI$]{ F → G }
  \end{prooftree}
  \qquad
  \begin{prooftree}[center=false]
  	\hypo{ F → G }
  	\hypo{ F }
  	\infer2[$\impE$]{ G }
  \end{prooftree}
\]
As explained above, the introduction rule allows the assumption \( F \) to be considered as discharged in the subderivation, indicated by the notation
\begin{prooftree*}
	\hypo{[F]}
	\ellipsis{}{}
\end{prooftree*}%
This information is not strictly part of the inference but is relevant for the notion of entailment in natural deduction, the relation of ‘being derivable from assumptions’.

\paragraph{Disjunction.} Two introduction rules and one elimination rule, the latter with discharged assumptions.
\[
  \begin{prooftree}[center=false]
  	\hypo{ F }
  	\infer1[$\disjI_0$]{ F ∨ G }
  \end{prooftree}
  \qquad
  \begin{prooftree}[center=false]
  	\hypo{ G }
  	\infer1[$\disjI_1$]{ F ∨ G }
  \end{prooftree}
  \qquad
  \begin{prooftree}[center=false]
  	\hypo{ F ∨ G }
  	\hypo{ [F] }
  	\ellipsis{}{H}
  	\hypo{ [G] }
  	\ellipsis{}{H}
  	\infer3[$\disjE$]{ H }
  \end{prooftree}
\]

\paragraph{Universal quantifier.} An introduction and elimination rule. 
The introduction rules are separated according to the variable being ‘discharged’. Application of the introduction rules are subject to the \emph{eigenvariable condition} below.
\[
  \begin{prooftree}%[center=false]
  	\hypo{ F(a) }
  	\infer1[$\faI_a$]{ ∀x F(x) }
  \end{prooftree}
  \qquad
  \begin{prooftree}%[center=false]
  	\hypo{ ∀x F(x) }
  	\infer1[$\faE$]{ F(t) }
  \end{prooftree}
\]
Recall that per convention~\ref{conv-subst}, in the introduction rule \( F(x) \) means \( F[x/a] \). In particular, the variable \( a \) never occurs in the conclusion of \( \faI_a \). In the elimination rule \( \faE \), \( ∀x F(x) \) and \( F(t) \) mean \( ∀x F[x/b] \) and \( F[t/b] \) for some (unspecified) variable \( b \).


%\begin{thing}[Eigenvariable condition for \( \faI \)]
%	The rule \( \faI_a \) is applicable only if \( a \) does not occur in any assumptions above this inference which is \emph{undischarged} after applying this rule.
%\end{thing}


\paragraph{Existential quantifier.} To the universal quantifier as  disjunction is to conjunction:
\[
  \begin{prooftree}[center=false]
  	\hypo{ F(t) }
  	\infer1[$\exI$]{ ∃ x F(x) }
  \end{prooftree}
  \qquad
  \begin{prooftree}[center=false]
  	\hypo{ ∃x F(x) }
  	\hypo{ [F(a)] }
  	\ellipsis{}{H}
  	\infer2[$\exE_a$]{ H }
  \end{prooftree}
\]
Applications of \( \exE_a \) are also constrained by restricting where an eigenvariable condition:

\begin{thing}[Eigenvariable condition]
	The rules \( \faI_a \) and \( \exE_a \) are applicable only if \( a \) does not occur in the formula \( H \) nor in any assumption above this inference which is \emph{undischarged} after applying this rule.
\end{thing}

\begin{example}
	The following is derivation in the above rules of
	\( ∃x ¬F→¬∀x F \). Notice that all assumptions are discharged (and that discharged assumptions are annotated by the rule instance that discharged them):
	\begin{prooftree*}
		\hypo{ [∃x ¬F]^\dag }
		\hypo{ [ ¬ F(a) ]^{\dag\dag} }
		\hypo{ [∀x F]^\ddag }
		\infer1[$\faE$]{ F(a) }
		\infer2[$\impE$]{ ⊥ }
		\infer2[$\exE_a^{\dag\dag}$]{ ⊥ }
		\infer1[$\impI^\ddag$]{ ¬∀x F }
		\infer1[$\impI^\dag$]{∃x ¬F→¬∀x F }
	\end{prooftree*}
	The derivation above is well-formed: for the single inference to which the eigenvariable condition applies, \( \exE_a \), the variable \( a \) does not occur in an undischarged assumption (except for the assumption \( ¬ F(a) \) which is discharged by this inference).
\end{example}
%
\begin{definition}[Natural deduction, \( \Ni \)]
	\emph{Natural deduction} is the calculus \( \Ni \) over formulas comprising the aforementioned rules.
	A \emph{deduction} is a finite derivation in \( \Ni \) which fulfills the eigenvariable condition.
%	A well-formed derivation in \( \Ni \), i.e., one that satisfies the eigenvariable condition, is called a \emph{deduction}.
\end{definition}
	
\begin{definition}[Entailment]
	The \emph{entailment relation} is a relation \( ⊢ \) between a finite set of formulas and a formula defined as \( Γ ⊢ F \) iff that there exists a deduction in \( \Ni \) with conclusion \( F \) such that every undischarged assumption of this derivation is an element of \( Γ \).
	
	A \emph{validity} of \( \Ni \) is any formula \( F \) such that \( ∅ ⊢ F \) holds; also written as \( ⊢ F \).
\end{definition}


As a direct consequence of the definition, I deduce
\begin{lemma}\label{nd-i-transitive}
	If\/ \( Γ ⊢ F \) and \( Δ ∪ \setof F ⊢ G \), then \( Γ ∪ Δ ⊢ G \)
\end{lemma}
%
As \( Δ ∪ \setof F ⊢ F \) for all \( Δ \) and \( F \), a special case of \cref{nd-i-transitive} is
\begin{lemma}
	\( Γ ⊢ F \) implies \( Γ ∪ Δ ⊢ F \) for all \( Δ \).
\end{lemma}

\begin{theorem}[Deduction]
%	If \( F \) is a sentence, then
	For all \( F \) and \( G \):
	\( Γ ∪ \setof F ⊢ G \) iff \( Γ ⊢ F → G \).
\end{theorem}

\begin{lemma}\label{nd-i-fa}
	If\/ \( Γ \) is a set of sentences and \( F(a) \) any formula, then 
	\( Γ ⊢ F \) iff \( Γ ⊢ ∀x F(x) \).
\end{lemma}


\begin{exercise}
	Prove \ref{nd-i-transitive}--\ref{nd-i-fa}.
\end{exercise}

%
\begin{exercise}
	Express the correctness criterion for \( \Ni \) as a property of paths.
	That is, specify the correctness condition \( \Coll C \) as a set of sequences of formulas and rules such that a derivation \( D \) is a deduction iff every path through \( D \) is contained in \( \Coll C \).
\end{exercise}

The set of validities of \( \Ni \) determines a logic, known as intuitionistic logic.
%Not classical predicate logic, but intuitionistic logic.

\begin{definition}[Intuitionistic logic]
%	\emph{Intuitionistic logic} is identified with the natural deduction calculus \( \Ni \). 
	\( \IL \) is the set of validities of \( \Ni \).
	I write \( \IL ⊢ F \) iff \( F ∈ \IL \), iff \( ∅ ⊢ F \) holds (in \( \Ni \)).
\end{definition}

% --------------------------------------
\section{Classical logic}
% --------------------------------------

In contrast to intuitionistic logic, classical logic is defined via semantics, specifically, Tarskian semantics.
%Let \( \Lang L \) be a first-order language.
I will not recap the definition here;
the reader can consult, for example, \citetalias[ch.~4]{LogThe}.

Classical logic can also be captured syntactically.

\begin{definition}[Classical natural deduction, \( \Nc \)]
	The calculus \( \Nc \) is the extension of \( \Ni \) by the rule \( \RAA \):
	\[
		\begin{prooftree}
			\hypo{ [¬ F] }
			\ellipsis{}{ ⊥ }
			\infer1[$\RAA$]{ F }
		\end{prooftree}
	\]
	with the same eigenvariable conditions applying to deductions.
	
	The entailment relation for \( \Nc \) is denoted \( ⊢_\CL \).
	That is, \( Γ ⊢_\CL F \) express the existence of a deduction according to \( \Nc \) with conclusion \( F \) in which all undischarged assumptions are elements of \( Γ \).
\end{definition}

\( \RAA \) stands for \emph{reductio ad absurdum}.
To avoid potential confusion, I use \( ⊢_\IL \) for the entailment 

\begin{theorem}
	The following are equivalent.\label{nd-c-i}
	\begin{enumerate}
		\item \( Γ ⊢_\CL F \).
		\item \( Γ ∪ \setof{ ¬ F} ⊢_\CL ⊥ \).
		\item \( Γ ∪ \setof{ ¬ ¬ G → G }[G ∈ Δ] ⊢_\IL F \) for some set \( Δ \).
		\item \( Γ ∪ \setof{ G ∨ ¬ G }[G ∈ Δ] ⊢_\IL F \) for some set \( Δ \).
	\end{enumerate}
\end{theorem}

\begin{exercise}
	Prove \cref{nd-c-i}.
\end{exercise}

%In this course we will not need to assume



% --------------------------------------
\chapter{The sequent calculus}
% --------------------------------------


%
Some content

\begin{convention}[The rule \( \impL \) in sequent calculi]
	The rule \( \impL \) in classical or intuitionistic contexts.
\end{convention}

\bigskip
Negation translation magic (exercise)
%
\chapter{Properties of the sequent calculus}
%

Height \& cut rank.

\begin{exercise}
	Devise a function \( f \colon ℕ → ℕ \) such that \( \IL \prv {f(\rk A)} 0 A ⇒ A \) for every formula \( A \), and prove your claim.
\end{exercise}

% ------------------------------
\section{Height-preserving translations}
% ------------------------------

\begin{lemma}[Height preserving weakening]
	\label{weak-lem-hp}
	For all sequents \( Γ ⇒ Δ \) and all \( n ,k < ω \),
	\begin{enumerate}
		\item If \( \CL \prv n k Γ ⇒ Δ \) then \( \CL \prv n k Γ , Σ ⇒ Δ , Π \) for every \( Σ \) and \( Π \).
		\item If \( \IL \prv n k Γ ⇒ Δ \) and \( \card{Δ ∪ Π } ≤ 1 \) then \( \IL \prv n k Γ , Σ ⇒ Δ , Π \).
	\end{enumerate}
\end{lemma}


% ------------------------------
\section{Negative translations}
% ------------------------------



% Module 2
\part{Cut elimination}
\label{module-2}
%---------------------------------
\chapter{Cut elimination}\label{c-cut-elim}
%---------------------------------
Here we present cut elimination for the calculi.

Cut rank,
Inversion lemma and the like

\begin{lemma}[First inversion]\label{ce-inversion-lemma}
	Let \( ⊢ \) denoted provability in either \( \Gc \) or \( \Gi \).
	The following hold for all sequents and all \( n \), \( k \):
	\begin{enumerate}
		\item If \( \prv n k Γ ⇒ Δ , ⊥ \) then \( \prv n k Γ ⇒ Δ \).
		\item If \( \prv n k Γ ⇒ Δ , F ∧ G \) then \( \prv n k Γ ⇒ Δ , F \) and \( \prv n k Γ ⇒ Δ , G \).
		\item If \( \prv n k F ∧ G , Γ ⇒ Δ \) then \( \prv n k F , G , Γ ⇒ Δ \).
		\item If \( \prv n k F ∨ G , Γ ⇒ Δ \) then \( \prv n k F , Γ ⇒ Δ \) and \( \prv n k G , Γ ⇒ Δ \).
		\item If \( \prv n k Γ ⇒ Δ , F → G \) then \( \prv n k F , Γ ⇒ Δ , G \).
		\item If \( \prv n k Γ ⇒ Δ , ∀x F(x) \) then \( \prv n k Γ ⇒ Δ , F(s) \) for every term \( s \).
		\item If \( \prv n k ∃x F(x) , Γ ⇒ Δ \) then \( \prv n k F(s) , Γ ⇒ Δ \) for every term \( s \).
	\end{enumerate}
\end{lemma}

\begin{exercise}
	Prove that the rules \( \exR \) and \( \faL \) are not invertible in the above sense.
	That is, show the following two statements are \emph{false}:
	\begin{itemize}
		\item If \( \prv n k ∀x F(x) , Γ ⇒ Δ , ∀x F(x) \) then \( \prv n k F(s) , Γ ⇒ Δ \) for some term \( s \).
		\item If \( \prv n k Γ ⇒ Δ , ∃x F(x) \) then \( \prv n k Γ ⇒ Δ  , F(s) \) for some term \( s \).
	\end{itemize}
\end{exercise}

For the classical sequent calculus two additional ‘inversion’ principles hold.

\begin{lemma}[Second inversion] 
	In addition, classical predicate logic admits the inversions:
	\begin{enumerate}[start=8]
		\item If \( \Gc \prv n k Γ ⇒ Δ , F ∨ G \) then \( \Gc \prv n k Γ ⇒ Δ , F , G \).
		\item If \( \Gc \prv n k F → G , Γ ⇒ Δ \) then \( \Gc \prv n k G ,Γ ⇒ Δ \) and \( \Gc \prv n k Γ ⇒ Δ , F \).
	\end{enumerate}
\end{lemma}

The intuitionistic sequent calculus supports just one part of the classical inversions.

\begin{lemma}[Third inversion]
	In addition to \cref{ce-inversion-lemma}, intuitionistic predicate logic admits the inversion
	\begin{enumerate}[start=8]
		\item If \( \Gi \prv n k F → G , Γ ⇒ Δ \) then \( \Gi \prv n k G , Γ ⇒ Δ \).
	\end{enumerate}
\end{lemma}

\begin{exercise}
	Show that the ‘missing’ inversions for intuitionistic logic are false:
	\begin{enumerate}[start=8,label=\arabic*'.]
		\item If \( \Gi \prv n k Γ ⇒ F ∨ G \) then either \( \Gi \prv n k Γ ⇒ F \) or \( \Gi \prv n k Γ ⇒ G \).
		\item If \( \Gi \prv n k F → G , Γ ⇒ Δ \) then \( \Gi \prv n k Γ ⇒ Δ , F \).
	\end{enumerate}
\end{exercise}

% ----------------------------------
\section{Classical logic}
% ----------------------------------

For this section, I treat the classical sequent calculus \( \Gc \).
Thus, \( \prv n k \) means \( \Gc \prv n k \) throughout.

\begin{lemma}[Reduction]
	\label{ce-red-lem-C}
	Suppose \( \prv m k Γ ⇒ Δ , C \) and \( \prv n k C , Σ ⇒ Λ  \).
	If \( \rk C = k \) then \( \prv {m+n}k Γ,Σ ⇒ Δ,Λ  \).
\end{lemma}
%
\begin{proof}
	See lectures.

	One case for reference. (Induction is on \( m + n \).)
	Suppose \( C = D ∨ E \) and
	\begin{enumerate}
		\item \( \prv m k Γ ⇒ Δ , C \) \label{itm-ce-reduct-a}
		\item \( \prv n k C , Σ ⇒ Λ  \) \label{itm-ce-reduct-b}
	\end{enumerate}
	arise from
	\begin{enumerate}[resume]
		\item \( \prv {m'} k Γ ⇒ Δ , C , D \) \label{itm-ce-reduct-1}
		\item \( \prv {n'} k D , C, Σ ⇒ Λ  \), and\label{itm-ce-reduct-2}
		\item \( \prv {n''} k E , C , Σ ⇒ Λ  \) \label{itm-ce-reduct-3}
	\end{enumerate}
	by the rules \( \disjR \) and \( \disjL \) respectively, where \( m' < m \) and \( n' , n'' < n \).
	\Cref{f-ce-CL} provides an illustration of the argument in this case.
	As \( m' + n < m + n \), the induction hypothesis can be applied to the pair \eqref{itm-ce-reduct-1} and \eqref{itm-ce-reduct-b}, yielding
	\begin{enumerate}[resume]
		\item \( \prv {m'+n} k Γ , Σ ⇒ Δ , Λ , D \).
	\end{enumerate}

	I also apply the inversion lemma to \eqref{itm-ce-reduct-2}, deducing
	\begin{enumerate}[resume]
		\item \( \prv {n'} k D , Σ ⇒ Λ  \).
	\end{enumerate}
	As \( \rk D < \rk C \), it is possible to cut the final pair of sequents, resulting in
	\begin{enumerate}[resume]
		\item \( \prv {h} k Γ ,  Σ ⇒ Δ , Λ  \) for \( h = \max \setof{ m' + n , n' } + 1 \).
	\end{enumerate}
	As \( h ≤ m + n \), this case is complete.
\end{proof}

% ------------------
\begin{figure}
	\begin{mdframed}
	\centering
	\begin{prooftree}
		\subproof*[\eqref{itm-ce-reduct-1}]{\prv{m'}k Γ ⇒ Δ , C , D }
			\subproof*[\eqref{itm-ce-reduct-b}]{\prv{n}k C , Σ ⇒ Λ }
		\infer[dashed]2[IH]{\prv{ m' + n } k Γ , Σ ⇒ Δ , Λ , D }
		
		\subproof*[\eqref{itm-ce-reduct-2}]{\prv{n'}k D , C , Σ ⇒ Λ }
		\infer[dashed]1[IL]{\prv{n'}k D , Σ ⇒ Λ }
			
		\infer2[\Cut]{\prv{m + n}k Γ , Σ ⇒ Δ , Λ }
	\end{prooftree}
	\caption{Illustration of the proof method in the reduction lemma for the case \( C = D ∨ E \). Numbering refers to the proof; IL = ‘inversion lemma’ and IH = ‘induction hypothesis’.}
	\label{f-ce-CL}
	\end{mdframed}
\end{figure}
% ------------------

\begin{theorem}[Reduction]
	If \( \prv m {k+1} Γ ⇒ Δ  \) then \( \prv {2^m} k Γ ⇒ Δ \).
\end{theorem}
%
\begin{proof}
	In lectures.
\end{proof}

%\begin{exercise}
%	\label{ex-red-lem-special}
%	In this exercise you will prove a strengthening of the reduction lemma and, as a consequence, obtain more precise bounds on the cost of cut elimination in classical logic.
%	
%	See Canvas assign 4.
%\end{exercise}

Iterating the reduction lemma induces a cut-free proof.
Define \( 2_k^n \) as \( 2^n_0 = n \) and \( 2^n_{k+1} = 2^{2^n_k} \).
The following result is due to Gerhard Gentzen and often referred to as \emph{Genzten’s Hauptsatz}.

\begin{theorem}[Cut elimination]
	Every sequent provable in classical predicate logic is provable without cut.
	In particular, if \( \Gc \prv n k Γ ⇒ Δ  \) then \( \Gc \prv m 0 Γ ⇒ Δ \) for some \( m ≤ 2^n_k \).
\end{theorem}
%
\begin{proof}
	Induction on the cut rank. Recall, \( \prv {} 0 \) is synonymous with ‘cut-free provable’.
\end{proof}


% ----------------------------------
\section{Intuitionistic logic}
% ----------------------------------

The rule of cut is also admissible in intuitionistic sequent calculus with essentially the same bounds.
The failure of inversion creates some complications while the restriction to intuitionistic logic simplifies other cases.

\begin{lemma}[Reduction]
	\label{ce-red-lem-C}
	Suppose \( \Gi \prv m k Γ ⇒ C \) and \( \Gi \prv n k C , Σ ⇒ Λ  \).
	If \( \rk C = k \) then \( \Gi \prv {(m+n).2}k Γ,Σ⇒ Λ  \).
\end{lemma}
%
\begin{proof}
	See lectures.
\end{proof}

\begin{theorem}[Reduction]
	If \( \Gi \prv m {k+1} Γ ⇒ Δ  \) then \( \Gi \prv {4^m} k Γ ⇒ Δ \).
\end{theorem}
%
\begin{proof}
	In lectures.
\end{proof}

%\begin{exercise}
%	\label{ex-red-lem-special}
%	In this exercise you will prove a strengthening of the reduction lemma and, as a consequence, obtain more precise bounds on the cost of cut elimination in classical logic.
%	
%	See Canvas assign 4.
%\end{exercise}

Iterating the reduction lemma induces a cut-free proof.
Define \( 4_k^n \) as \( 4^n_0 = n \) and \( 4^n_{k+1} = 4^{4^n_k} \).

\begin{theorem}[Cut elimination]
	Every sequent provable in intuitionistic predicate logic is provable in the same calculus without cut.
	In particular, if \( \Gi \prv n k Γ ⇒ Δ  \) then \( \Gi \prv m 0 Γ ⇒ Δ \) for some \( m ≤ 4^n_k \).
\end{theorem}

As a comparison, the reader can verify that \( 4^n_k ≤ 2^n_{2k} \) for all \( n \) and \( k \).

The next section looks more closely at the structure of cut-free proofs to derive a variety of structural results about classical and intuitionistic logic.
Before that, I present a strengthening of the cut elimination bounds that can be achieved by a more careful proof of the reduction lemma.
The proof of the following results are exercises.

% -----------------------------
\section{Refining cut elimination}
% -----------------------------

As formulated, the cut elimination theorem suggests that every logical connective contributes an exponential in the size of a proof.
A careful examination of some cases show that many connectives can be dispensed ‘cheaply’, that is to say without inducing an exponential blow-up in proof height.
As an example, consider the reduction lemma for a conjunction (classical or intuitionistic sequent calculus --- they behave equally in this particular example):
\begin{enumerate}
	\item \( \prv m k Γ ⇒ Δ , D ∧ E \),
	\item \( \prv n k D ∧ E , Σ ⇒ Λ \).
%	\item \( k = \rk { D ∧ E } \).
\end{enumerate}
The reduction lemma states that if \( k = \rk { D ∧ E } \) then \( \prv { m+n} k Γ , Σ ⇒ Δ , Λ \).
However, the inversion lemma presents a different strategy to derive the same sequent. I ‘invert’ the sequents above to obtain
\begin{enumerate}
	\item \( \prv m k Γ ⇒ Δ , D \),
	\item \( \prv m k Γ ⇒ Δ , E \),
	\item \( \prv n k D , E , Σ ⇒ Λ \),
\end{enumerate}
and recombine as a sequence of two cuts of rank \(  < k \):
\begin{prooftree*}
	\hypo{\prv m k Γ ⇒ Δ , D }
	\hypo{\prv m k Γ ⇒ Δ , E }
	\hypo{\prv n k D , E , Σ ⇒ Λ }
	\infer2[\Cut]{\prv {\max\setof{ m, n }+1} k D , Γ, Σ ⇒ Δ ,  Λ }
	\infer2[\Cut]{\prv {\max\setof{ m, n } +2} k Γ , Σ ⇒ Δ , Λ }
\end{prooftree*}
%
If \( m, n > 1 \) then \( \max\setof{ m, n } + 2 < m + n \).

The same trick can be applied to the other propositional connectives (\( ∨ \) and \( → \)) for the classical sequent calculus:
\begin{exercise}
	\label{ex-ce-refined-cpl}
	Using the method outlined above, state and prove an improved reduction lemma for \emph{classical propositional} logic. Use your results to show that \( \Gcp \prv n k Γ ⇒ Δ \) implies \( \Gcp \prv h 0 Γ ⇒ Δ \) for \( h = 2^{n. 2^k} \).
\end{exercise}

\begin{exercise}
	\label{ex-ce-refined-cl}
	Generalise the previous exercise to deduce more efficient bounds on cut elimination for classical predicate logic.
\end{exercise}

The inversion method does not apply to intuitionistic logic in the same way because that logic lacks the requisite inversions.
%
Still, there is another way to structure the reduction lemma that works smoothly for intuitionistic logic.
The idea is to rely on the asymmetry of the intuitionistic sequent calculus and the observation that in this calculus the main complexity of cut elimination arises when cut formulas ‘switch’ sides of the sequent arrow.
To show this, I introduce a new complexity measure on formulas, called the implication depth, that counts the ‘nesting’ of implications. 
Let \( \nrk A \) be the implication rank of \( A \), defined by:
\begin{itemize}
	\item \( \nrk A = 0 \) for \( A \) prime.
	\item \( \nrk {QxA} = \nrk A \) for \( Q ∈ \{ ∀ , ∃ \} \).
	\item \( \nrk {A ∘ B} = \max\setof{ \nrk A , \nrk B }\) for \( ∘ ∈ \{ ∧ , ∨ \} \).
	\item \( \nrk {A → B} = \max \setof{ \nrk A + 1 , \nrk B } \).
\end{itemize}

It will be necessary to restrict attention to the fragment without \( ∃ \) and \( ∨ \) as these add a level of complexity to the process that is  beyond the scope of this course.

\begin{definition}
	Let \( \prvs n q \) express derivability in the intuitionistic sequent calculus by a proof with height \( ≤ n \) in which all cut formulas have \emph{implication rank} \( < q \) and do not contain \( ∨ \) or \( ∃ \).
\end{definition}
%
So the version of the cut rule used in \( \prvs{}{} \) is
\[ 
\begin{prooftree}
	\hypo{ \prvs n q Γ ⇒ C }
	\hypo{ \prvs m q C , Λ ⇒ B }
	\infer2[\Cut]{\prvs h q Γ, Λ ⇒ B}
\end{prooftree}
\quad\text{provided } \nrk C < q  \text{ and } h > \max\setof{ m, n }.
\]

You will prove:

\begin{theorem}[Refined cut elimination]\label{ce-refined}
	If \( \prvs n q Γ ⇒ A \) then \( \prvs m 0 Γ ⇒ A \) for some \( m ≤ 2^{n}_{q} \).
\end{theorem}


The reduction lemma that leads to the above result shifts the focus from reducing a single cut to reducing a batch of cuts as a single operation.
Such a collection of cuts will have a specific form that can be expressed as a lifting of the standard two-premise cut rule to a multi-premise version, called the \emph{multi-cut}: 
%(and also originating from Gentzen), compresses a sequence of \( k \) cuts into a single \( k+1 \)-premise rule:
\[ 
\begin{prooftree}
	\hypo{ Γ_0 ⇒ C_0 }
	\hypod
	\hypo{ Γ_k ⇒ C_k }
	\hypo{ C_0 , …, C_k , Λ ⇒ B }
	\infer4[\mCut]{ Γ_0 , …, Γ_k , Λ ⇒ B }
\end{prooftree}
%\quad\text{provided } \nrk {C_i} < q \text{for } .
\]
%
The multi-cut has is origins in Gentzen's original proof of the \emph{Haupsatz} and can be viewed as abbreviating the sequence of binary cuts:
\[ 
\begin{prooftree}
	\hypo{ Γ_0 ⇒ C_0 }
	\hypo{ Γ_{k-1} ⇒ C_{k-1} }
	\hypo{ Γ_k ⇒ C_k }
	\hypo{ C_0 , …, C_k , Λ ⇒ B }
	\infer2[\Cut]{ C_0 , …, C_{k-1} , Γ_k , Λ ⇒ B }
	\infer2[\Cut]{}
	\ellipsis{}{ C_0 , Γ_1 , …, Γ_k , Λ ⇒ B }
	\infer2[\Cut]{ Γ_0 , …, Γ_k , Λ ⇒ B }
\end{prooftree}
\]
The purpose of the multi-cut, however, is to hold together (in a single inference) the many individual cuts that are to be ‘eliminated’ at the once.
A typical example of when the multi-cut rule is useful is in the reduction process attached to a binary cut with a conjunctive cut formula (see \cref{f-ce-mcut-IL}).
If using the implication rank of formulas, cuts on the formulas \( A \) and \( B \) are considered as complex as on the conjunction \( A ∧ B \) itself.
% ------------------
\begin{figure}
	\begin{mdframed}
	\raggedright
	\begin{prooftree}
		\subproof{ Γ ⇒ A }
			\subproof{ Γ ⇒ B }
		\infer2[\conjR]{ Γ ⇒ A ∧ B }
		
		\subproof{ A , B , Σ ⇒ D }
		\infer1[\conjL]{ A ∧ B , Σ ⇒ D }
			
		\infer2[\Cut]{ Γ , Σ ⇒ D }
	\end{prooftree}
	\\[1ex]\raggedleft
	reduces to\quad
	\begin{prooftree}
		\subproof{ Γ ⇒ A }
			\subproof{ Γ ⇒ B }
		
		\subproof{ A , B , Σ ⇒ D }
			
		\infer3[\mCut]{ Γ , Σ ⇒ D }
	\end{prooftree}
	\caption{Illustration of the \emph{multi-cut} reduction for a conjunctive cut formula. According to the implication rank, cut formulas in the second inference are as complex as in the first.}
	\label{f-ce-mcut-IL}
	\end{mdframed}
\end{figure}
% ------------------

The reduction lemma for \( \prvs{}{} \) is the following:
%
\begin{lemma}[Multi-cut reduction]\label{ce-reduct-mcut}
	The following applies to \( \Gi \).
	Suppose \( m_0 , …, m_k \), \( n \) and \( q \) are such that 
	\begin{itemize}
		\item \( \prvs {m} q Γ_i ⇒ C_i \) for each \( i ≤ k \),
		\item \( \prvs n {q} C_0,…, C_k , Σ ⇒ D \).
%		\item \( \nrk {C_i} = q \) for all \( i \).
	\end{itemize}
	If \( \nrk{C_i} = q \) for all \( i ≤ k \), then 
	\( \prvs {m + n} q Γ_0 , …, Γ_k , Σ ⇒ D \).
%	where \( m = \textstyle\sum_i m_i \) and \( f(n) = 2^{2^n} \).
\end{lemma}
%
In the special case of the standard ‘two-premise’ cut (where \( k = 0 \) above), the multi-cut reduction lemma states
\begin{center}
	If \( \prvs {m} q Γ ⇒ C \)
	and \( \prvs n {q} C , Σ ⇒ D \) where \( \nrk C = q \),
	then \( \prvs { m + n} q Γ , Σ ⇒ D \).
\end{center}
%
The bound provided by the multi-cut reduction lemma is almost identical to the usual reduction lemma (the reduction lemma for intuitionistic logic employs the bound \( (m+n).2 \) rather than \( m+n \)).
But, crucially, the lemma claims that this bound is sufficient for reducing the \emph{implication rank} of the cut, not only the traditional rank.

\begin{exercise}
	Prove the multi-cut reduction lemma.
	You will find the inversion lemma is needed to obtain the desired bound which you can assume holds for \( \prvs{}{} \) with the same bounds.
\end{exercise}

\begin{exercise}
	Using \cref{ce-reduct-mcut} and any other observations to prove the refined cut elimination theorem (\cref{ce-refined}).
\end{exercise}

\begin{exercise}
	Using an appropriate negation translation of \( \Gc \) into \( \Gi \), deduce a refined cut elimination theorem for classical logic.
%	What is the derived complexity measure in place of implication depth, and what is the appropriate ‘multi-cut’ reduction lemma? 
\end{exercise}

\begin{exercise}
	Propose a refined cut elimination theorem for classical predicate logic. What is the appropriate measure in place of implication depth, and what is the appropriate ‘multi-cut’ reduction lemma? 
\end{exercise}


%
\chapter{Consequences of cut elimination}
\label{c-ce-conseq}
%
Now we are getting somewhere

\bigskip

Interpolation theorem -- Exercise.

Harrop's theorem
%
\chapter{Predicate logic with equality}
\label{c-equality}
%
We haven't treated equality (yet).

% ---
\section{Equality in natural deduction}

\( \NDceq \) \& \( \NDieq \).

% ---
\section{Equality in sequent calculus}

\( \Gieq \) \& \( \Gceq \).

% ---
\section{Cut elimination with equality}

Hmm!
% -------------------------------
\chapter{A game of cut and mouse}
% -------------------------------

Shall we? Or should this be for inf PA?

% Module 3
\part{An Introduction to Ordinal Analysis}
\label{module-3}
%
\chapter{Arithmetic and Sequent Calculi}\label{c-oa-arith}
%
As an application of proof theory beyond logic I will give an analysis of perhaps the most important formal theory in mathematics, the theory of Peano arithmetic.
Among the results we present is a syntactic characterisation of the theorems of the theory, and a proof of its consistency which does not invoke any semantic considerations.
A corollary of the analysis will be a characterisation of the non-finite mathematical assumptions required to establish the consistency of Peano arithmetic.

The proof I present has its origins in Gentzen's 1938 consistency proof\nocite{Gen-1938} but employs a simplification due to Kurt Schütte (1950)\nocite{Schu1950} whereby arithmetic is treated as a fragment of infinitary logic and a corresponding infinitary notion of sequent calculus proof.

Elementary results about this theory covered in the pre-requisite course \emph{Logical Theory} will be stated without proof; see corresponding chapters of \cite{LogThe} for details.

%---------------------------------
\section{Peano and Heyting arithmetic}\label{s-oa-arithmetic}
%---------------------------------
%
\begin{definition}
	The \emph{language of arithmetic} is the first-order language \( \La \) comprising the following nonlogical symbols with associated arities:
	\begin{enumerate}
		\item function symbols: \( \0^0 \), \( \suc^1 \), \( +^2 \), \( ×^2 \).
		\item predicates: \( P^1 \).
	\end{enumerate}
\end{definition}
%
The theory of arithmetic is formulated over predicate logic \emph{with equality}. 
I will first present theory with logic given by the classical natural deduction calculus with equality \( \NDceq \) before presenting an equivalent presentation based on the sequent calculus \( \Gceq \).
Both logics were introduced in \cref{c-equality}; see also~\cite[§6.5]{Negri_von_Plato}.

Henceforth, \emph{formula} will always refer to the language of arithmetic.
%
\begin{definition}
	The Peano axioms of arithmetic are the following sentences.
	\begin{itemize}
		\item \emph{Basic axioms:}
		\begin{axioms}[pa]
			\item \( ∀ x ¬ ( \0 = \suc x ) \)
			\item \( ∀ x ∀ y ( \suc x = \suc y → x = y ) \)
			\item \( ∀ x ( x + \0 = x ) \)
			\item \( ∀ x ∀ y ( x + \suc y = \suc ( x + y ) ) \)
			\item \( ∀ x ( x × \0 = \0 ) \)
			\item \( ∀ x ∀ y ( x × \suc y = ( x × y ) + x ) \)
		\end{axioms}
		\item \emph{Axiom scheme of induction:}
		\begin{axioms}[pa][start=7]
			\item The universal closure of \( A(\0) ∧ ∀x ( A(x) → A(\suc x) ) → ∀x A(x) \) for every formula \( A(a) \).
		\end{axioms}
	\end{itemize}
\end{definition}
%
\begin{definition}
	\emph{Peano arithmetic} (\( \PA \)) is theory over classical predicate logic axiomatised by the Peano axioms.
	I will write \( \PA \vdash A \) to express that \( A \) is a theorem of Peano arithmetic, that is, \( \Set{PA} ⊢_\NDceq A \) where \( \Set{PA} \) is the set of Peano axioms.
	\emph{Heyting arithmetic} (\( \HA \)) is the corresponding \emph{intuitionistic} theory, i.e., \( \HA ⊢ A \) expresses \( \Set{PA} ⊢_\NDieq A \).
\end{definition}
%
%\( \HA \vdash A \) means that \( A \) is a theorem of Heyting arithmetic.
%Observe that the theories above are specified over predicate logic \emph{with equality}.

The predicate \( P \) is auxiliary to the language of arithmetic in that it has no intended interpretation associated to it. 
It plays the role of a ‘free’ predicate as the next \namecref{oa-arith-P} demonstrates.

For a formulas \( A \) and \( B(a) \) in the language of arithmetic, let \( A[B/P] \) mark the result of replacing each occurrence of \( Ps \) in \( A \) by \( B(s) \) for every term \( s \).
That is, 
\begin{align*}
	(Ps)[B/P] &= B(s)
	\\
	A[B/P] &= A \quad\text{for \( A \) any other atomic formula}
	\\
	(A_0 → A_1)[B/P] &= ( A_0[B/P] → A_1[B/P] )
	\\
	(∀x A)[B/P] &= ∀ x( A[B/P] )
	\\
	&\text{etc}
\end{align*}

The following result is easy to prove.

\begin{proposition}
	\label{oa-arith-P}
	If \( \PA \vdash A \) then \( \PA \vdash A[B/P] \) for every formula \( B(a) \).
	Likewise for \( \HA \).
\end{proposition}
%
\begin{proof}
	Exercise.
\end{proof}

\begin{exercise}
	Show the following are theorems of Heyting arithmetic.
	\begin{enumerate}
		\item \( ∀ x( ¬ x = \0 → ∃ y( x = \suc y )) \).
		\item \( ∀ x ∀ y ( x + y = y + x ) \).
		\item \( ∀ x ∀ y ∀ z ( ( x + y ) + z = x + ( y + z ) ) \).
		\item \( ∀ x ∀ y ( x × y = y × x ) \).
	\end{enumerate}
\end{exercise}

As well as some basics of the theory of arithmetic, we recall the primitive recursive representation theorem. See \cite{LogThe}, for example, for details.
%
\begin{definition}
	A formula is \( Δ_0 \) if it can be constructed from atomic formulas excluding \( P \) by the propositional connectives and bounded quantifiers.
	That is, the \( Δ_0 \) formulas forms the smallest collection of \( \La \)-formulas satisfying:
	\begin{enumerate}
		\item all equations \( s = t \) are \( Δ_0 \) formulas,
		\item \( ⊥ \) is a \( Δ_0 \) formula,
		\item if \( F \) and \( G \) are \( Δ_0 \), then so is \( F → G \), \( F ∨ G \) and \( F ∧ G \),
		\item if \( F(a) \) is \( Δ_0 \) and \( s \) is a term, then \( ∀x < s\, F(x) \) and \( ∃x < s \, F(x) \) are \( Δ_0 \), where these formulas are shorthands for \( ∀x ( x< s → F(x) ) \) and \( ∃x ( x < s ∧ F(x) ) \) respectively.
%		\item if \( A \) and \( B \) are \( Σ_1 \) and, in addition, \( A \) does not contain the existential quantifier, then \( A → B \) is \( Σ_1 \).
	\end{enumerate}
	A formula is \( Σ_1 \) (\( Π_1 \)) if it has the form \( ∃x F(x) \) (respectively \( ∀x F(x) \)) where \( F(a) \) is \( Δ_0 \).
\end{definition}

Notice that the bound variable \( x \) does not occur in the ‘bounding’ term \( s \) in the construction \( ∀x < s\, F(x) \) above because terms do not contain bound variables.

Terms of the specific form \( \suc ⋯ \suc \0 \) are called \emph{numerals}.
The numeral evaluating to \( n ∈ ℕ \) is denoted \( \nm n \):
\[
	\nm n ≔ \underbrace{\suc ⋯ \suc}_n \0.
\]

I state the representation theorem for primitive recursive relations.
%
\begin{theorem}[Representation]
	\label{representation-thm}
	Let \( R ⊆ \Nat^n \) be an \( k \)-ary relation on natural numbers. 
	If \( R \) is primitive recursive there exists a \( Δ_0 \) formula \( F_R(a_1, …, a_k ) \) of \( \La \) with at most the displayed variables free such that for all \( n_1, …, n_k ∈ \Nat \),
	\begin{align*}
		\PA ⊢ F_R(\nm n_1 , …, \nm n_k ) \quad &\text{iff}\quad (n_1 , …, n_k ) ∈ R
		\\
		\PA ⊢ ¬ F_R(\nm n_1 , …, \nm n_k ) \quad &\text{iff}\quad (n_1 , …, n_k ) ∉ R.
	\end{align*}
\end{theorem}
%

%---------------------------------
\section{Fragments of arithmetic}\label{s-oa-sub-PA}
%---------------------------------

There are important subtheories of Peano arithmetic that are worth introducing.
I will begin with the theory of primitive recursion, called \emph{primitive recursive arithmetic}, \( \PRA \). 
Primitive recursive arithmetic is, in essence, the equational theory of primitive recursive functions. For a recap on primitive recursive functions see, e.g.~\cite[ch~15]{LogThe}.
The theory is formulated in the extension of \( \La \) by function symbols for all primitive recursive functions.

As the theories introduced this section will all be formulated over classical logic, I take the opportunity to the logical connectives are \( ⊥ \), \( ∧ \), \( → \) and \( ∀ \). Note, I am including implication rather than primitive negation as a matter of convenience.


%\begin{definition}
	The language of primitive recursive arithmetic,
	\( \Lpra \), contains a function symbol \( \subdot h \) (of arity \( n \)) for each primitive recursive function \( h \colon ℕ^n → ℕ \).
	\emph{Primitive recursive arithmetic}, \( \PRA \), is the theory in classical logic whose non-logical axioms are the defining equations of each primitive recursive function.
%\end{definition}

%To be more formal, I take the function symbols of \( \Lpra \) to be generated by:
\begin{definition}[Language of primitive recursive arithmetic]
%	\( \Lpra \) is the language containing a unary predicate sy
	The symbols of \( \Lpra \) are generated as follows.
\begin{enumerate}
	\item \( P ∈ \Lpra \) is a unary predicate symbol.
	\item \( \suc ∈ \Lpra \) (unary) and \( \0_n ∈ \Lpra \) ($n$-ary) for every \( n \).
	\item \( \Symbol p_{n,k} ∈ \Lpra \) ($n$-ary) for every \( n \) and \( k < n \).
	\item For each \( m \)-ary function symbol \( g \) and \( n \)-ary function symbols \( \vec h = h_1, …, h_m \), an \( n \)-ary function symbol \( \Symbol c_{ g , \vec h} ∈ \Lpra \).
	\item For each \( n \)-ary function symbol \( g \) and \( (n+2) \)-ary function symbol \( h \), an \( (n + 1 ) \)-ary function symbol \( \Symbol r_{f,g} ∈ \Lpra \).
\end{enumerate}
\end{definition}

\begin{definition}[Primitive recursive arithmetic]
\( \PRA \) is the theory in classical logic axiomatised by the following sentences where \( g, h, h_1 , …, h_n \) range over primitive recursive function symbols of appropriate arity.
\begin{axioms}[pr]
	\item \( ∀x  ¬ \, \suc x = \0_0 \).\label{axiom1}
	\item \( ∀x_1 ⋯ x_n \, \Symbol p_{n,k} \vec x = x_{k+1} \).
	\item \( ∀ x_1 ⋯ x_n \, \Symbol c_{g, (h_1, …, h_n)} \vec x = g ( h_1 \vec x ) ⋯ ( h_n \vec x ) \).

	\item For the function symbol \( \Symbol r_{g,h} \), the axioms of primitive recursion:
	\begin{align*}
		∀ x_1 ⋯ x_n \, \Symbol r_{g,h} \0 \vec x &= g \vec x 
		\\
		∀ y x_1 ⋯ x_n \, \Symbol r_{g,h} (\suc y) \vec x &= h y ( \Symbol r_{g,h} y \vec x ) \vec x . 
	\end{align*}
	\item The universal closure of \( A(\0) ∧ ∀x( A(x) → A(\suc x) ) → ∀x A(x) \) for every quantifier-free formula \( A \).
\end{axioms}
\end{definition}
The reader can confirm that the axioms are well-formed (that each function symbol is associated the correct arity according to the definition).


I assume that \( \Lpra \) extends \( \La \) in the sense that the symbols \( \0 \), \( \suc \), \( + \) and \( × \)  name the corresponding function symbol in \( \Lpra \) and that \( \Lpra \) contains the unary predicate \( P \).
In the case of \( + \), for example, 
\[
	{+} ≔ \Symbol c_{\hat+,\Symbol p^2_1,\Symbol p^2_0}
	\text{ where }
	{\hat+} ≔ \Symbol r_{\Symbol p^1_0, \suc_{3,1}}
	\text{ and }
	\suc_{3,1} ≔ \Symbol c_{\suc,\Symbol p^2_1}
\]
Checking the definition, the axioms express \( \suc_{3,1} \) as the ternary function that returns the successor of its middle argument (and ignores the other), \( \hat+ \) as addition defined by recursion on its \emph{first} argument, and \( + \) as \( \hat+ \) with the order of arguments exchanged.
Although \( \hat+ \) clearly also \emph{defines} addition on \( ℕ \), it is only for \( + \) that the basic Peano axioms are provable in from the \( \PRA \)-axioms.

\begin{exercise}
	Find a function symbol \( × \) in \( \Lpra \) such that the basic Peano axioms for the symbol are provable in \( \PRA \).
\end{exercise}

\begin{lemma}
	The basic Peano axioms are provable in \( \PRA \).
\end{lemma}
\begin{proof}
	Only the second basic axiom, \( ∀x∀y( \suc x = \suc y → x = y ) \), is not obvious.
	For this, we consider the \( \PRA \)-axioms for a particular instance of primitive recursion, the unary function \( f = \Symbol r_{\0,\Symbol p_{2,0}} \) that has associated axioms
	\[
		f \0 = \0
		\qquad
		f ( \suc a ) = \Symbol p_{2,0} a ( f a )
		\quad\text{and}\quad
		\Symbol p_{2,0} a b = a.
	\]
	Reasoning informally in \( \PRA \):
	Assume \( \suc a = \suc b \). Substitution of equal terms yields \( a = \Symbol p_{2,0} a ( f a ) = f (\suc a ) = f( \suc b ) = \Symbol p_{2,0} b ( f b ) = b  \).
\end{proof}

\begin{convention}[Representating primitive recursive functions]
	For each primitive recursive function \( h \colon ℕ^n → ℕ  \) is associated a canonical function symbol \( \subdot h \) in \( \Lpra \) that expresses the construction \( h \) by the rules of primitive recursion.
\end{convention}

As an example of the above convention, consider the exponentiation function \( \mathit{ex} \colon n ↦ 2^n \).
There is a formula \( E(a,b) \) in the language of arithmetic such that
\begin{enumerate}
	\item \( \PA ⊢ ∀x ∃!y \, E(x,y) \). %(where \( ∃! \) means ‘exists a unique’, defined in the obvious way).
	\item \( \PA ⊢ E(\nm n, \nm {2^n}) \) for every \( n ∈ ℕ \).
\end{enumerate}
Exponentiation base \( 2 \) is, however, clearly primitive recursive, so there exists a function symbol \( \Symbol e \) in \( \Lpra \) such that
\(
  \PRA ⊢ \Symbol e \nm n = \nm{2^n} 
\)
for each \( n \). 
By the above convention, the function symbol \( \Symbol e \) is denoted by writing \( \subdot{\mathit{exp}} \) or, more suggestively, as \( \subdot 2 ^s \) in place of \( \Symbol e s \).


\begin{lemma}
%	Let \( s \) and \( t \) be closed terms in \( \Lpra \).
	It is decidable whether \( ℕ ⊨ s = t \) holds for any two closed terms in \( \Lpra \).
%	There exists a primitive recursive function \( f \) for which the question of whether \( f n = \0 \) is undecidable.
\end{lemma}

%The level in the quantifier hierarchy at a formula first appears is, unsurprisingly, closely related to
Recall the implication rank introduced at the end of \cref{c-cut-elim}, defined by counting the nesting depth on the negative side of implications:
\begin{align*}
	\nrk{A} &= 0 \quad(\text{$A$ prime})
	&
	\nrk{F ∧ G} &= \max\setof{ \nrk F , \nrk G }
	\\
	\nrk{∀x F(x)} &= \nrk{F(a)}
	&
	\nrk{F → G} &= \max\setof{ \nrk F +1 , \nrk G }
\end{align*}

\begin{definition}[The quantifier hierarchy in arithmetic]
	The \( Π_{n}^P \) formulas (the \( ^P \) expresses that the predicate \( P \) is permitted, in contrast to our earlier definition of \( Π_1 \)) is the set of formulas of implication depth \( < n \).
%	smallest set of formulas that contains
%	\begin{enumerate}
%		\item all prime formulas 
%		\item \( F ∧ G \) if \( F , G ∈ Π_{n}^P \),
%		\item \( ∀x F(x) \) if \( F(a) ∈ Π_{n}^P \) and \( n > 0 \),
%		\item \( F → G \) if \( F ∈ Π_{n-1}^P ∪ Π_0^P \) and \( G ∈ \Pi_{n}^P \).
%	\end{enumerate}
\end{definition}

To understand the definition, notice that the formula \( ∀x ∃y ∀ z F \) will have implication depth \( 2 \) if \( \nrk F = 0 \). 
Thus, \( Π^0_2 \) formulas in the usual sense correspond to implication depth \( 1 \).

\begin{comment}

As \( Π_0^P \) happens to be closed under arbitrary implications, there is no a priori bound on the negation depth of formulas in \( Π_n^P \).
Modulo equivalence, however, \( Π_n^P \) corresponds to implication rank \( n+1 \)

\begin{lemma}
	Every formula is equivalent to \( Π_n^P \) formula for some \( n \), and every \( Π_n^P \) formula is equivalent, over \( \PRA \), to a formula with implication rank \( n+1 \).
\end{lemma}

\begin{proof}
	It is routine to show that every \( Π_0^P \) formula is equivalent to a quantifier-free formula of implication rank \( 1 \).
	The case of \( n > 0 \) is now straightforward.
%
%	Using the full logical language, express \( F ∈ Π_n^P \) in prefix normal form as
%	\begin{gather}
%		\label{oa-eqn-PNF}\tag{\dag}
%		∀\vec x_1 ∃\vec x_2 ⋯ ∃ \vec x_{m} G
%	\end{gather}
%	where \( G \) is quantifier-free.
%	We can choose \( m ∈ \setof{ n , n+1 } \) and \( G \) can be assumed to be \( Π_0^P \) because over weak arithmetic a disjunction \( E ∨ F \) is provably equivalent to 
%	\[ 
%		∃x ∃ y ( x + y = \suc \0 ∧ ( x = \0 → E ) ∧ ( x = \suc\0 → F )).
%	\]
%	and the leading existential quantifiers can be incorporated into the '$∃\vec x_m$' sequence of quantifiers.
%	It is now straightforward to witness \eqref{oa-eqn-PNF} as an equivalent \( Π_{m}^P \)-formula whose negation rank is \( m \).
\end{proof}

\end{comment}

\begin{definition}[Theories with restricted induction]
	For each \( n \), \( \PA_n \) denotes the theory in the language \( \Lpra \) extending \( \PRA \) by the axiom of induction for \( \Pi^P_n \) formulas.
	That is, \( \PA_n \) extends \( \PRA \) by the universal closure of 
	\[ A(\0) ∧ ∀x ( A(x) → A(\suc x) ) → ∀x A(x) \] 
	for every formula \( A(a) \) satisfying \( \nrk{A(a)} < n \).
\end{definition}


%---------------------------------
\section{Sequent calculi for arithmetic}\label{s-oa-omega-logic}
%---------------------------------

There are different ways to formulate arithmetic in sequent calculi.
One can incorporate all the axioms of arithmetic as initial sequents or treat each Peano axiom as contributing a rule of the calculus.
A convenient definition is the following.
\( \PA ⊢ Γ ⇒ Δ \) means that the sequent \( Γ ⇒ Δ \) has a derivation in the sequent calculus \( \Gceq \) expanded by:
\begin{itemize}
	\item Initial sequents \( Π ⇒ Σ , A \) for \( A \) a basic Peano axiom.
	\item The \emph{induction rule}:
	\[
	  \begin{prooftree}
	  	\hypo{ A(a) , Π ⇒ Σ , A( \suc a ) }
	  	\infer1[\IRule]{A(\0) , Π ⇒ Σ , ∀ x A(x) }
	  \end{prooftree}
	\]
	where \( a \) does not occur in the lower sequent.
\end{itemize}
The sequent calculus for Heyting arithmetic is the restriction to intuitionistic sequents. 
Recall, a sequent \( Γ ⇒ Δ \) is \emph{intuitionistic} if \( \card{Δ} = 1 \).
Define \( \HA ⊢ Γ ⇒ Δ \) as there exists a sequent calculus derivation witnessing \( \PA ⊢ Γ ⇒ Δ \) using only intuitionistic sequents.


\begin{proposition}\relax
	\label{oa-PA-as-SC}
	For every sentence \( A \),
	\begin{enumerate}
		\item \( \PA ⊢ A \) iff \( \PA ⊢ {} ⇒ A \).
		\item \( \HA ⊢ A \) iff \( \HA ⊢ {} ⇒ A \).
	\end{enumerate}
\end{proposition}
%
\begin{proof}
	I will show that every induction axiom admits a sequent calculus proof. The remainder of the proof is left as an exercise.
	
	Fix a formula \( F(a) \) and let \( Γ = \setof{ F(\0) , ∀x( F(x) → F( \suc x) ) } \).
	By logic, a derivation of \( F(a) , Γ ⇒ F( \suc a )  \) is readily obtained. An application of the induction rule, followed by more logic completes the derivation:
	\begin{prooftree*}
		\subproof{ F(a), Γ ⇒ F(a) }
		\subproof{ F( \suc a), Γ ⇒ F(\suc a)}
		\infer2[\impL]{ F(a) → F( \suc a ) , F(a), Γ ⇒ F(\suc a) }
		\infer1[\faL]{ F(a) , Γ ⇒ F(\suc a) }
		\infer1[\IRule]{ Γ ⇒ ∀ x F(x) }
%		\infer[double]1[\conjL, \impR]{ ⇒ F(\0) ∧ G → ∀x F(x)}
	\end{prooftree*}
\end{proof}

\begin{exercise}
	Show that \( \PA \) derives the same sequents as the calculus with induction axioms as initial sequents (along with the basic axioms) and no induction rule.
\end{exercise}


Sequent calculi for fragments of arithmetic can be obtained by simply restricting the complexity of formulas in the induction rule.
I write \( \PA_n ⊢ Γ ⇒ Δ \) if in all uses of the induction rule \( \IRule \) above, \( \nrk{A} ≤ n \).
When induction is restricted, so can the cut rank.

\begin{theorem}[Partial cut elimination]\label{oa-partial-ce}
	Suppose \( \PA_n ⊢ Γ ⇒ Δ \) and \( n ≥ 1 \).
	Then \( \PA ⊢ Γ ⇒ Δ \) with a derivation in which all cut formulas have implication depth \( < n \).
\end{theorem}
%
\begin{proof}
	By induction on the number of induction rules in the derivation.
	I leave the base case, when there are no induction rules, to the reader.
%	Let \( U \) be the (finite) set of Peano axioms.
%	Consider first a derivation \( ⊢ Γ ⇒ Δ \) with no use of the induction rule.
%	Then clearly \( \Gc ⊢ U, Γ ⇒ Δ \), whence there is a cut-free proof of the sequent, and a \( \PA \)-proof with only cuts on basic axioms. 
%	The basic axioms are all implication rank \( 0 \).
	
	Now consider a sequent proof containing an instance of \( \IRule \):
	\[
	  \begin{prooftree}
	  	\subproof{ A(a) , Π ⇒ Σ , A( \suc a ) }
	  	\infer1[\IRule]{A(\0) , Π ⇒ Σ , ∀ x A(x) }
	  	\ellipsis{}{ Γ ⇒ Δ }
	  \end{prooftree}
	\]
	I split this into two \( \PA \)-proofs:
	\[
	  \begin{prooftree} 
	  	\subproof{ A(a) , Π ⇒ Σ , A( \suc a ) }
	  	\infer1[\IRule]{A(\0) , Π ⇒ Σ , ∀ x A(x) }
	  \end{prooftree}
	  \quad\text{and}\quad
	  \begin{prooftree}
	  	\axiom[\idRule]{∀x A(x) , A(\0) , Π ⇒ Σ , ∀ x A(x) }
	  	\ellipsis{}{∀x A(x) , Γ ⇒ Δ }
	  \end{prooftree}
	\]
	By the induction hypothesis, each of the two sequents can be derived with cut formulas of rank \( < n \) and a cut on \( ∀x A(x) \) combines the two proofs.
\end{proof}

\begin{exercise}
	Show the base case of previous induction:
	That if \( \IS_0 ⊢ Γ ⇒ Δ \) then there is derivation in which all cuts have implication depth \( 0 \).
\end{exercise}


%---------------------------------
\section{Small proofs and big proofs}
\label{s-oa-o-proofs}
%---------------------------------

By \cref{oa-PA-as-SC}, \( \PA \) is consistent iff the empty sequent is not derivable.
As with the sequent calculi from previous chapters, it is clear that there can be no cut-free derivation of the empty sequent, neither in \( \PA \) nor \( \HA \).
Thus, consistency of either theory would follow directly from a cut-elimination theorem for the above sequent calculi.
%
There are sequents, however, that are provable but not \emph{cut-free} provable. 
We will not present the argument here, which appeals to Gödel's incompleteness theorems;\note{Sketch the argument} the finer details are beyond the scope of this book and can be found in, for example,~\cite{BBJ}.

Gentzen's observation was that every derivable \emph{equational} sequent can be shown to have a cut-free derivation, where an equational sequent is one of the form \( r_1 = s_1 , …, r_k = s_k ⇒ t_1 = u_1 , …, t_l = u_l \) wherein all terms are closed.
As the empty sequent is an example of an equational sequent, consistency is an immediate corollary of the (partial) cut-elimination result.

Gentzen's argument is highly intricate and was greatly streamlined by Kurt Schütte (1950)\nocite{Schu1950} who showed that full cut-elimination can be obtained by moving to a more relaxed notion of a sequent calculus derivation, termed ‘\( ω \)-proofs’, in which proofs are in general infinite objects.
%
The basic idea is to replace the logical rules \( \faR \) and \( \exL \) each by a rule with infinitely many premises:
\begin{gather*}
  \begin{prooftree}
	\hypo{ Γ ⇒ Δ , A(\0) }
	\hypo{ Γ ⇒ Δ , A(\nm 1) }
	\hypod
	\hypo{ Γ ⇒ Δ , A(\nm n) }
	\hypod
	\infer5[\omR]{ Γ ⇒ Δ , ∀x A(x) }
  \end{prooftree}
  \\[2ex]
  \begin{prooftree}
	\hypo{ A(\0) , Γ ⇒ Δ }
	\hypo{ A(\nm 1) , Γ ⇒ Δ }
	\hypod
	\hypo{ A(\nm n) , Γ ⇒ Δ }
	\hypod
	\infer5[\omL]{ ∃x A(x) , Γ ⇒ Δ }
  \end{prooftree}
\end{gather*}
The rules \( \omR \) and \( \omL \) above are collectively called the \( ω \)-rules.

‘Proof’ in the sense of the sequent calculi of previous chapters meant ‘finite tree labelled by sequents in agreement with the rules of the calculus’.
A ‘proof’ that uses an \( ω \)-rule can never be finite as these rules have infinitely many premises.
But the condition ‘finite or infinite tree labelled by sequents in agreement with the rules of the calculus’ is too liberal as it admits as 'proofs' trees with infinitely long branches, such as 
\begin{prooftree*}
	\subproof{ ⇒ ⊥ }
	\axiom[\botL]{ ⊥ ⇒ ⊥ }
	\infer2[\Cut]{ ⇒ ⊥ }
	\axiom[\botL]{ ⊥ ⇒ ⊥ }
	\infer2[\Cut]{ ⇒ ⊥ }
	\axiom[\botL]{ ⊥ ⇒ ⊥ }
	\infer2[\Cut]{ ⇒ ⊥ }
\end{prooftree*}
The answer is that, as in the finite case, there can be no infinite paths in an \( ω \)-proof but, unlike the finite case, the tree underlying an \( ω \)-proof may have infinitely wide branching.
Such trees are called \emph{well-founded}.

In sum, an \emph{\( ω \)-proof} is a well-founded tree that is labelled by sequents in a way consistent with the rules of the sequent calculus (the \( ω \)- and non \( ω \)-rules).
%A sequent derivation which (possibly) use \( ω \)-rules is called an \emph{\( ω \)-proof}.

\begin{proposition}
	\label{oa-oproofs-simple}
	\begin{enumerate}
		\item There is an \( ω \)-proof of every sequent of the form \( F , Γ ⇒ Δ , F \)
		\item If there is an \( ω \)-proof of \( Γ(a) ⇒ Δ(a) \), then for every term \( s \) there is an \( ω \)-proof of \( Γ(s) ⇒ Δ(s) \).
	\end{enumerate}
\end{proposition}
%
\begin{proof}
	Exercise.
\end{proof}
%
\begin{proposition}
	\label{oa-ind-omega}
	The induction rule can be simulated via \( \omega \)-proofs.
\end{proposition}
%
%
\begin{proof}
	Fix a formula \( F(a) \) and as before let \( Γ = \setof{ F(\0), ∀x( F(x) → F( \suc x) ) } \).
	Suppose \( F(a) , Γ ⇒ Δ , F(\suc a) \) admits an \( ω \)-proof. 
	As this is a premise to an induction rule the variable \( a \) does not occur in \( Γ ∪ Δ \).
	By the previous \namecref{oa-oproofs-simple} there is an \( ω \)-proof of the sequent \( F(\nm n) , Γ ⇒ Δ , F(\nm {n+1}) \) for each \( n \).
	A sequence of cuts induces an \( ω \)-proof of \( F(\0) , Γ ⇒ Δ , F(\nm n) \): for \( n = 0,1 \) the claim is immediate. For \( n = m + 1 > 1 \) append the proof of the induction hypothesis by a single cut:
	\begin{prooftree*}
		\subproof{ F(\nm0) , Γ ⇒ Δ , F(\nm m) }
		\subproof{ F(\nm m) , Γ ⇒ Δ , F(\nm n) }
		\infer2[\Cut]{F(\0) , Γ ⇒ Δ , F(\nm n)}
	\end{prooftree*}
	As \( F(\0) , Γ ⇒ Δ , F(\nm n) \) is derivable for each \( n \), an application of \( \omR \) completes the (\( ω \)-)proof.
\end{proof}

\begin{exercise}
	Show that the induction axioms admit \emph{cut-free} \( ω \)-proofs.
\end{exercise}


With the \( ω \)-rules replacing the traditional quantifier rules \( \faR \) and \( \exL \) it turns out that free variables can be completely eliminated from the sequent calculus, meaning that only closed sequents are derived.
This convention serves to simplify much of the reasoning about \( ω \)-proofs.
It is also possible to dispense with the logical rules for equality by adopting more liberal initial sequents.

The next definition introduces both conventions and settles the notion of \( ω \)-proof used hereon.
Observe that it is decidable whether two closed terms \( s \) and \( t \) in the language of arithmetic evaluate to the same natural number. I will write \( ℕ ⊨ s = t \) if this is the case, and \( ℕ ⊭ s = t \) otherwise.
%
\begin{definition}[\( \PAo \) and \( \HAo \)]\label{d-PAomega}
	\( \PAo \) is the sequent calculus given by the following:
	\begin{itemize}
		\item sequents comprise formulas in the language of arithmetic (with equality).
		\item Initial sequents are \emph{closed} sequents of the form
		\begin{itemize}
			\item[(\botL)] \( ⊥, Γ ⇒ Δ \)
			\item[(\idRule)] \( Ps, Γ ⇒ Δ , Pt \) if \( ℕ ⊨ s = t \)
			\item[(\eqR)] \( Γ ⇒ Δ , s = t \) if \( ℕ ⊨ s = t \)
			\item[(\eqL)] \( s = t , Γ ⇒ Δ \) if \( ℕ ⊭ s = t \)
		\end{itemize}
		\item Inference rules are rules of \( \Gc \) but restricted to closed sequents and with \( \faR \) and \( \exL \) replaced by the two \( ω \)-rules:
		\begin{itemize}
			\item[(\omR)] \begin{prooftree} \hypo{ Γ ⇒ Δ , F(\nm n) \ \text{for every } n ∈ ℕ } \infer1{ Γ ⇒ Δ , ∀x F(x) }\end{prooftree}
			\item[(\omL)] \begin{prooftree} \hypo{ F(\nm n) , Γ ⇒ Δ \ \text{for every } n ∈ ℕ } \infer1{ ∃x F(x) , Γ ⇒ Δ }\end{prooftree}
		\end{itemize}
	\end{itemize}
	Writing \( \PAo ⊢ Γ ⇒ Δ \) expresses that there is an \( ω \)-proof of \( Γ ⇒ Δ \) according to the above rules. In other words, there exists a well-founded tree labelled by sequents such that each leaf is an initial sequent and that each inner vertex together with its immediate successors in the tree forms a correct application of a rule of the calculus listed above.
	
	\( \HAo \) is the calculus above restricted to intuitionistic sequents.
\end{definition}
%

With the sequent calculus formally defined, the realisation of finite \( \PA \)-proofs as \( ω \)-proofs can resume.
The first step is to give \( ω \)-proofs of the basic axioms of arithmetic.
%
\begin{proposition}
	\label{oa-embed-basic}
	Every closed initial sequent of \( \PA \) is derivable in \( \PAo \).
%	If \( F \) is a basic axiom of \( \PA \) and \( Γ ⇒ Δ \) any closed sequent, then \( \PAo ⊢ Γ ⇒ Δ , A \).
\end{proposition}
%
\begin{proof}
	Among the sequents to be shown derivable in \( \PA \) are all initial sequents of \( \Gc \) and the basic axioms of \( \PA \).
	I will treat the case of the basic axiom \textsc{pa}1, \( ∀x( ¬ \0 = \suc x ) \).
	Let \( n ∈ ℕ \) be arbitrary. As the equation \( \0 = \suc \nm n \) is false, \( \0 = \suc \nm n , Γ ⇒ Δ , ⊥ \) is an initial sequent of \( \PAo \) for all closed \( Γ , Δ  \).
	Therefore \( \PAo ⊢ Γ ⇒ Δ , ¬ \0 = \suc \nm n \) for every \( n ∈ ℕ \) and \( \PAo ⊢ Γ ⇒ Δ , ∀x( ¬ \0 = \suc x ) \) by \( \omR \).
\end{proof}
%
\begin{exercise}
	Complete the proof of \cref{oa-embed-basic}.
\end{exercise}
%
\begin{exercise}\label{ex:oa-id-simple}
	Show that all closed sequents of the form \( A, Γ ⇒ Δ , A \) are provable in \( \PAo \).
\end{exercise}

%
\begin{lemma}[Embedding]\label{oa-embed-weak}
	Suppose \( \PA ⊢ Γ ⇒ Δ \) and let \( Γ^* ⇒ Δ^* \) be any closed substitution instance of \( Γ ⇒ Δ \) (obtained by substituting closed terms for free variables).
	Then \( \PAo ⊢ Γ ⇒ Δ \).
	Likewise, for Heyting arithmetic and \( \HAo \).
\end{lemma}
%
%
%\begin{proof}
%	Proceed by induction on the height of the (finite) \( \PA \)-proof.
%	\Cref{oa-embed-basic} covers the case of initial sequents as these are closed under substitution (if \( Γ(a) ⇒ Δ(a) \) is an initial sequent of \( \PA \) then so is \( Γ(s) ⇒ Δ(s) \) for every term \( s \)).
%\end{proof}

\begin{exercise}
	Prove the embedding lemma. Do not forget the equality rules implicit in \( \Gceq \).
\end{exercise}

The next task is to analyse \( ω \)-proofs and establish a cut elimination theorem.
Currently lacking, however, is some measure of the \emph{complexity} of an \( ω \)-proof analogous (or, perhaps, generalising) the height of finite sequent calculus proofs.
Although every path through an \( ω \)-proof is, by requirement, finite there are \( ω \)-proofs that admit paths of arbitrary (finite) length.
The \( ω \)-proof described by the proof of \cref{oa-ind-omega} is such an example.
It comprises a single application of an \( ω \)-rule at the root with the premise for the numeral \( n \) being derived by a (finite) sequent proof of height at least \( n \).

Thus the question comes down to how to associate a measure to \( ω \)-proofs such that strict subproofs (i.e., proofs of the premises of the root inference) can be recognised as being ‘smaller’ than the proof itself?
The answer to this conundrum is in the title of this module: \emph{ordinals}.

%Currently we have no means to measure the size of \( \PAo \)-proofs.
%For this we will use ordinals.


%---------------------------------
\chapter{An ordinal interlude}\label{c-oa-ordinals}
%---------------------------------

To present the ordinals it is not necessary to have a set-theoretic definition of ordinals in mind (as, for example, arbitrary transitive sets).
Indeed, there is no need to consider the question of by what ordinals \emph{are} or from what they are \emph{formed}.
For a \emph{theory} of ordinals all that is relevant are the order-theoretic properties satisfied by the ordinals and a selection of operations that can be defined on them.
In short, ordinals are treated analogously to natural numbers: as a posited entity fulfilling specified criteria.
%
The material of this chapter draws from lecture notes by Michael Rathjen~\cite{RathjenLectures}.

%
\begin{definition}\label{d-ordinals}%[Ordinals]
	The \emph{ordinals} is a class \( \Ord \) equipped with a binary relation \( < \) satisfying three postulates, where \( ≤ \) is the reflexive closure of \( < \):
	\begin{axioms}[o]
		\item \( < \) is a strict linear order on \( \Ord \). That is, \( < \) is irreflexive, transitive and linear, where linear means that for all \( α , β ∈ \Ord \) either \( α ≤ β \) or \( β ≤ α \).\label{post-ord-lin}
		\item Every non-empty class of ordinals has a \( < \)-minimal element (necessarily unique by \ref{post-ord-lin}). That is, if \( O ⊆ \Ord \) is non-empty there exists \( ξ ∈ O \) such that \( ξ ≤ α \) for all \( α ∈ O \).\label{post-ord-wo}
		\item For every set \( X \) and function \( f \colon X → \Ord \) there exists \( ξ ∈ \Ord \) such that \( f(x) < ξ \) for every \( x ∈ X \).\label{post-ord-unbdd}
	\end{axioms}
\end{definition}

Set-theoretic concerns do matter in the language used to discuss ordinals.
As, for example, the Burali-Forte paradox shows, it is inconsistent  the Zermelo--Fraenkel (or Cantorian) conception of \emph{set} in mind to consider that the collection of (all) ordinals forms a set.
Hence use of term ‘class’ to refer to arbitrary collections of ordinals/objects and ‘set’ in specific case of \ref{post-ord-unbdd}.
Familiarity with set theory is not necessary for the elementary theory of ordinals presented here.
Indeed, it will suffice to replace every term ‘set’ in what follows by ‘countable set’ and ‘class’ by ‘countable or uncountable set’.
%We have side-stepped this concern by restricting attention to the \emph{countable} ordinals. Over Zermelo set theory, the collection of all countable ordinals, i.e., the collection \( \Ord \) above, forms a set; indeed, it is precisely the first uncountable ordinal.
%Such a restriction is not important for our later use of ordinals.
%Ultimately our attention will be constrained to a relatively small collection of (countable) ordinals.
%Our first lemma confirms that the relation \( <_\Ord \) is a well-order.

In the following, notation \( \setof{ t }[x ∈ X ] \) means the \emph{class} of objects \( t \) as \( x \) ranges over the (class) \( X \).
Usually a function \( f \colon U → V \) between classes has been specified along with a (sub)class \( X ⊆ U \) whence the notation \( \setof{ f(x) }[x ∈ X ] \) expresses the class of objects \( f(x) \) for \( x ∈ X \). This class will be written \( f[X] \).

\begin{convention}[Notating ordinals]
	Lowercase Greek letters \( α \), \( β \), etc.\ stand as metavariables for ordinals.
\end{convention}


\begin{lemma}\label{ord-well-order}
	Postulate \ref{post-ord-wo} is equivalent to the principle of transfinite induction.
	This is the statement that if \( O \) is progressive in the ordinals then \( \Ord ⊆ O \), where \( O \) is progressive means that for all ordinals \( α \), if \( β \in O \) for every \( β < α \) then \( α ∈ O \).
%	Let \( O ⊆ \Ord \) be non-empty.
%	Then \( O \) has a \( <_\Ord \)-minimal element and this is unique.
\end{lemma}
%
\begin{proof}
	Let \( O \) be progressive.
	Consider the class \( C = \Ord \setminus O \) of ordinals not in \( O \). If \( C \) is non-empty then, by \ref{post-ord-wo}, \( C \) contains a least ordinal, \( α \) say.
	As \( α \) is the least ordinal in \( C \), every \( ξ < α \) is element of \( O \). 
	Progressiveness implies that \( α ∈ O \) contradicting that \( α ∈ C \). 
	Hence, \( C \) is the empty class, so \( \Ord ⊆ O \).
	For the converse claim, assume postulate \ref{post-ord-lin} and the principle of transfinite induction (I could also assume \ref{post-ord-unbdd} but this is unnecessary).
	The aim is to establish \ref{post-ord-wo}. 
	Thus, let \( O \) be a non-empty class of ordinals and, for want of a contradiction, assume that \( O \) has no least element. 
	As in the other direction, I consider the complement of \( O \), the class \( C = \Ord \setminus O \).
	Suppose \( α \) be any ordinal such that \( ξ ∈ C \) for all \( ξ < α \). If \( α ∈ O \) then this is the least element of \( O \). As \( O \) has no least element therefore \( α ∈ C \).
	So \( C \) is progressive and \( C = \Ord \) by transfinite induction, contradicting the non-emptiness of \( O \).
\end{proof}

The next \namecref{ord-supremum} provides the primary means to infer the existence of ordinals.

\begin{lemma}
	\label{ord-supremum}\ 
	Let \( O \) be a class of ordinals.
	\begin{enumerate}
		\item There exists a least upper bound of \( O \). That is, an ordinal \( α \) such that \( ξ ≤ α \) for all \( ξ ∈ O \). This \( ξ \) is referred to as the \emph{supremum} of \( O \) and denoted \( \sup O \).
		\item There exists a strict least upper bound of \( O \), i.e., \( α \) such that \( ξ < α \) for all \( ξ ∈ O \).
	\end{enumerate}
	In each case the proclaimed ordinal is unique.
\end{lemma}
%
\begin{proof}
	Begin with 1. Let \( O \) be given.
	Consider the class \( O^≥ \) of all ordinals \( α \) such that \( ξ ≤ α \) for \emph{all} \( ξ ∈ O \).
	The \( < \)-least element of \( O^≥ \) (if such exists) is clearly the desired ordinal.
	But in order to apply postulate \ref{post-ord-wo} to this class it is necessary to establish that \( O^≥ \) is non-empty.
	For this I use the third postulate applied to identity function \( \textsf{id} \colon O → \Ord \colon ξ ↦ ξ \) (which is a function from \( O \) into \( \Ord \)).
	%
	For 2, the same argument works with the class \( O^> \) in place of \( O^≥ \) where this is the class of ordinals \emph{strictly} larger than all elements of \( O \).
	
	Uniqueness of each case is ensured by \ref{post-ord-lin}.
\end{proof}

Henceforth, I will not make explicit reference to the postulates.

The least ordinal is denoted \( 0 \). This happens to be the supremum of the empty set: \( 0 ≔ \sup ∅ \).
Given \( α ∈ \Ord \), the \emph{successor} of \( α \), in symbols \( α' \) or \( α + 1 \), is the least ordinal greater than \( α \), which exists (and is unique) by \cref{ord-supremum}(2) applied to the singleton set \( \setof{α} \).
That is, \( α' \) is such that \( ξ < α' \) iff \( ξ ≤ α  \).
The successor of \( 0 \) is denoted \( 1 ( =0') \), its successor \( 2 ( = 0'' ) \), etc.

A \emph{limit ordinal} is any non-zero ordinal \( λ \) such that \( η' < λ \), for all \( η < λ \).
Define a function \( f\colon ℕ → \Ord \) by \( f(0) = 0 \) and \( f(n+1) = f(n)' \).
That is, \( f(n) \) is the \emph{ordinal} representing the natural \( n \).
The supremum of \( \setof{ n }[n ∈ ℕ] \) is called \( ω \), which is a limit by construction and, therefore, the least limit ordinal.

\begin{lemma}
	\label{ord-suc-lim}
	Every non-zero ordinal is either a successor or a limit.
\end{lemma}

\begin{lemma}
	\label{ord-lim}
	An ordinal \( λ \) is a limit iff \( λ = \sup O \) for some non-empty set \( O \) closed under successor (meaning that \( ξ ∈ O \) implies \( ξ' ∈ O \)).
\end{lemma}

\begin{lemma}\label{ord-supremum-unique}
	Suppose \( O, O' \) are such that for every \( α \in O \) there exists \( β  \in O' \) such that \( α ≤ β \).
	Then \( \sup O ≤ \sup O' \).
\end{lemma}
\begin{exercise}
	Prove \cref{ord-suc-lim} to \ref{ord-supremum-unique}.
\end{exercise}


I will employ common set-theoretic abbreviations such as \( \sup_{i∈ I} α_i \) for \( \supof {α_i}[i ∈ I] \) and \( \sup_{i} α_i \) for \( \supof {α_i}[i < ω ] \).
I will also use \( λ \) as a metavariable for limit ordinals.

%--------------------------------------
\section{Elementary Ordinal Functions}
%--------------------------------------

A \emph{segment} of \( \Ord \) is any class \( O \) of ordinals which is closed downwards, i.e., if \( α < β ∈ O \) then \( α ∈ O \).
If \( X \) and \( Y \) are segments then either \( X ⊆ Y \) or \( Y ⊆ X \); in either case \( X ∩ Y \) is a segment.

Let \( O \) be a segment. A function \( f \colon O → \Ord \) is said to be:
\begin{itemize}
	\item \emph{order preserving} if \( α < β \) implies \( f(α) < f(β) \) for all \( α , β ∈ O \).
	\item \emph{continuous} if for all \( U ⊆ O \), if \( \sup U ∈ O \) then \( f( \sup U ) = \sup f[U] \).
%	\item \emph{normal} if \( O = \Ord \) and \( f \) is order preserving and continuous.
	\item an \emph{enumeration} (of \( X ⊆ \Ord \)) if \( f \) is order-preserving and \( f[O] = X \).
\end{itemize}

The identity function \( \mathsf{id} \colon \Ord → \Ord \) is all of the above. In particular, it is an enumeration of \( \Ord \).
Let \( f \colon ℕ → \Ord \) be given by \( f(0) = ω \) and \( f(n+1) = f(n)' \).
This function is order preserving and continuous (the latter is trivial). 
It is %not normal because the domain of \( f \) is not all ordinals, but it is
also an enumeration of the set \( \setof{ ω , ω' , … } \) because \( ℕ \) is a segment.
Notice that order preserving functions on ordinals are always injective.

\begin{lemma}\label{ord-o-p}
	If \( O \) is a segment and \( f \) is order preserving then \( α ≤ f(α) \) for all \( α ∈ O \).
\end{lemma}
%
\begin{exercise}
	Prove \cref{ord-o-p}.
\end{exercise}

The main property of ordinal functions I need is the summarised by
\begin{lemma}
	\label{ord-normal-exists}
	Every class of ordinals has a unique enumeration. The enumeration of \( Y ⊆ \Ord \) will be denoted \( E_Y \).
\end{lemma}
%
\begin{proof}
	\( E_Y \) is determined as the inverse of a particular function \( C_Y \colon Y → \Ord \), called the \emph{collapsing} function for \( Y \), defined by
	\[
		C_Y(α) = \sup \setof{ C_Y(ξ) + 1 }[ ξ ∈ Y \text{ and } ξ < α ].
	\]
	The collapsing function is clearly unique if it is well-defined. Moreover, \( C_Y \)
	This function is well-defined: Consider the class \( O \) of ordinals \( α \) for which the collapsing function on \( Y_α ≔ Y ∩ \setof{ ξ }[ξ ≤ α] \) exists.
	If \( C_{Y_ξ} \colon Y_ξ → \Ord \) is defined for each \( ξ < α \) I claim that \( C\colon Y_α → \Ord \) defined by 
	\[
		\begin{aligned}
			C(α) &= \sup\setof{ C_{Y_{ξ}}(ξ) + 1 }[ ξ < α \text{ and } ξ ∈ Y]
			\\
			C(ξ) &= C_{Y_ξ}(ξ) \text{ for \( ξ < α \)}
		\end{aligned}
	\]
	is the collapsing function for \( Y_α \).
	That this is the follows almost by definition. Indeed, all that is lacking is the observation that \( C_{Y_ξ}(β) = C_{Y_η}(β) \) whenever \( β ≤ ξ < η \).
	So \( O \) is progressive and transfinite induction implies that class \( Y_α \) has a collapsing function \( C_{Y_α} \). Now define \( C_Y \) as \( α ↦ C_{Y_α}(α) \).

	Clearly, \( C_Y \) is injective. Therefore the function admits a (right) inverse:
	\[
		E_Y ≔ C_Y^{-1} \colon C_Y[Y] → Y 
	\]
	As \( C_Y[Y ] \) is (clearly) a segment, \( E_Y \) is an enumeration of \( Y \).
	
	As to uniqueness of \( E_Y \), let \( O = C_Y[Y] \) and suppose \( f \colon O' → Y \) is any enumeration of \( Y \). In particular, \( O' \) is a segment. Transfinite induction implies that \( f(α) = E_Y(α) \) for all \( α ∈ O ∩ O' \).
	As both functions are injective and surjective into \( Y \) it follows that \( O = O' \).
\end{proof}

Two further properties of enumerations will be useful.

\begin{lemma}
	\label{ord-normal}
	Let \( f\colon \Ord → \Ord \) be continuous and order preserving (in particular, \( f \) is an enumeration of \( f[\Ord] \)).
	Then
	\begin{enumerate}
		\item For every \( α ≥ f(0) \) there is a unique \( β ≤ α \) such that \( f(β) ≤ α < f(β+1) \).\label{ord-normal-cover}
		\item For every \( α \) there is a unique \( β ≥ α \) such that \( β = f(β) \).\label{ord-normal-fix}
	\end{enumerate}
\end{lemma}
\begin{proof}
%	Continuity is implicit in the proof of \cref{ord-normal-exists}.
%	
	\ref{ord-normal-cover}. Consider the set \( O = \setof{ ξ }[ f(ξ) ≤ α ]\) and let \( β = \sup O \).
	Continuity yields
	\[
		f(β) = \sup f[O] = \sup \setof{ f(ξ) }[ f(ξ) ≤ α ] ≤ α
	\]
	whereas
	\(
		f(β+1) > α
	\)
	because \( β + 1 ∉ O \).
	
	\ref{ord-normal-fix}. Fix \( α \) and define \( O = \setof{ f(α) , f(f(α)) , …, f^n(α) , … } \) (arbitrary finite iterations of \( f \) on \( α \)). 
	Let \( β = \sup O \).
	Invoking continuity, \( f(β) = \sup f[O] = \sup O = β \). Moreover, \( α ≤ f(α) ≤ β \).
\end{proof}

%\begin{lemma}
%	\label{ord-normal-covers}
%	Let \( E\colon \Ord → \Ord \) be an enumeration (of \( E[\Ord] \)).
%	For every \( α ≥ E(0) \) there exists a unique \( β ≤ α \) such that \( E(β) ≤ α < E(β+1) \).
%\end{lemma}
%%
%\begin{proof}
%	Consider the set \( O = \setof{ ξ }[ E(ξ) ≤ α ]\) and let \( β = \sup O \).
%	Then 
%	\[
%		E(β) = \sup E[O] = \sup \setof{ E(ξ) }[ E(ξ) ≤ α ] ≤ α
%	\]
%	whereas
%	\(
%		E(β+1) > α
%	\)
%	because \( β + 1 ∉ O \).
%\end{proof}
%
%
%%As enumerations are order preserving it is always the case that \( α ≤ E(α) \). 
%
%\begin{lemma}
%	\label{ord-normal-fix}
%	Let \( E\colon \Ord → \Ord \) be an enumeration (of \( E[\Ord] \)). 
%	There exists \( ξ \) such that \( ξ = E(ξ) \).
%	Moreover, for every \( α \) there exists a unique \( ξ ≥ α \) such that \( ξ = E(ξ) \).
%\end{lemma}
%%
%\begin{proof}
%	Fix \( α \) and define \( O = \setof{ E(α) , E(E(α)) , …, E^n(α) , … } \) (arbitrary finite iterations of \( E \) on \( α \)). Then \( α ≤ E(α) ≤ \sup O \) and \( E(\sup O) \)
%\end{proof}


%--------------------------------------
\section{Elementary Ordinal Arithmetic}
%--------------------------------------
The basic operations of arithmetic can be extended to ordinals in a straightforward manner.
Often these are defined by transfinite recursion, but the two operations we desire, addition and exponentiation base \( ω \), can be expressed as enumeration functions.
I start with addition.
\begin{definition}
	Let \( α^≥ \) be the class of ordinals \( ≥ α \).
	\emph{Ordinal addition}, \( α + β \), is defined as \( α + β ≔ E_{α^≥}(β) \). That is, \( α + β \) is defined as the \( β \)-th ordinal in the enumeration of the ordinals \( ≥ α \).
\end{definition}
%
The following are direct consequences of this definition and left to the reader.

\begin{lemma}\label{ord-addition}
	For all \( α \), \( β \) and \( γ \).
	\begin{enumerate}
		\item \( α + 0 = α \).
		\item \( α + β' = ( α + β )' \).
		\item If \( β \) is a limit then \( α + β = \sup \setof{ α + ξ }[ξ < β ] \).
		\item \( α + ( β + γ ) = ( α + β ) + γ \).\label{ord-addition-assoc}
		\item \( α ≤ α + β \) and \( β ≤ α + β \).\label{ord-addition-inc}
	\end{enumerate}
\end{lemma}

\begin{example}\label{ex-ord-add}
	\( α + ω = \sup \setof{ α + n }[n∈ ℕ] = \sup \setof{ α , α' , α'' , …  } \). Thus \( α + ω \) is the least limit ordinal strictly above \( α \).
	
	In particular, \( n + ω = ω \) for every \( n < ω \).
	As \( 1 + ω = ω < ω + 1 \) ordinal addition is not commutative.
\end{example}

As addition is associative (item \ref{ord-addition-assoc} of the \namecref{ord-addition} above), I will omit brackets when stringing together applications of addition.
So \( α + β + γ \) can refer to either \( ( α + β ) + γ \) or \( α + ( β + γ ) \).

The next lemma is a consequence of \cref{ord-normal}.

\begin{lemma}
	For every \( α ≤ β \) there exists a unique \( ξ \) such that \( β = α + ξ \).
\end{lemma}
\begin{proof}
	\Cref{ord-normal} implies a unique \( ξ \) such that \( α + ξ ≤ β < α + ξ' \). Since \( α + ξ' = ( α + ξ ) + 1 \) it follows that \( α + ξ = β \).
\end{proof}

As \cref{ex-ord-add} demonstrates \( ω \) has the unusual property of being closed under addition: if \( ξ , η < ω \) then \( ξ + η < ω \).
Ordinals satisfying this condition are called \emph{additive principal} ordinals.
\begin{definition}
	\label{d-ord-AP}
	A ordinal \( α \) is additive principal iff \( α > 0 \) and \( ξ + η < α \) for all \( ξ , η < α \).
	The class of additive principal ordinals is denoted \( \AP \).
\end{definition}

The least additive principal ordinal is \( 1 \); the next is clearly \( ω \).
Most ordinals are \emph{not} additive principal. 
\( 1 \) is the only additive principal successor ordinal (because \( α + α ≥ α' \) provided \( α ≥ 1 \)).
Even most limit ordinals not additive principal: If \( α ≥ ω \) then \( α + ω ∉ \AP \) as \( α < α + ω \) but \( α + α ≮ α + ω \).

\begin{lemma}
	\label{ord-AP-normal}
	The enumeration function \( E_\AP \) for additive principal ordinals is continuous and has domain \( \Ord \).
%	The additive principal ordinals are
%	\begin{enumerate}
%		\item Closed: for every set \( O ⊆ \AP \), \( \sup O ∈ \AP \).
%		\item Unbounded in \( \Ord \). For every set \( α ∈ \AP \) there exists \( β > α \) such that \( β ∈ \AP \).
%	\end{enumerate}
\end{lemma}
\begin{proof}
%	Begin with the domain. By transfinite induction. Suppose \( ξ ∈ \dom E_\AP \) for every \( ξ < α \). I claim that \( α ∈ \dom E_\AP \). It suffices to show that there exists an additive principal ordinal \( β > E_\AP(ξ) \) for every \( ξ < α \).
%	Let \( O_0 = \setof{ E_\AP(ξ) }[ξ < α] \) and \( O_{n+1} = \setof{ ξ + η }[ξ , η ∈ O_n ] \). Set \(  \)
	Exercise.
\end{proof}

\Cref{ord-AP-normal} shows that the function enumerating the additive principal ordinals is defined on all ordinals, is order preserving and continuous.

\begin{lemma}
	\label{ord-AP}
	The following are equivalent for all \( α > 0 \):
	\begin{enumerate}
		\item \( α \) is additive principal.\label{ord-AP-1}
		\item \( α = 1 \) or \( α = \sup \setof{ ξ + ξ }[ξ < α ] \).\label{ord-AP-2}
		\item for all \( β < α \), \( β + α = α \).\label{ord-AP-3}
	\end{enumerate}
\end{lemma}
\begin{proof}
	\ref{ord-AP-1} $⇒$ \ref{ord-AP-2}. If \( α \) is additive principal then \( \sup \setof{ ξ + ξ }[ξ < α ] ≤ α \) by definition. Also, the additive principal ordinals except \( 1 \) are all limits, so if \( α ≠ 1 \) then \( α = \sup \setof{ ξ }[ξ < α ] ≤ \sup \setof{ ξ + ξ }[ξ < α ] \).
	
	\ref{ord-AP-2} $⇒$ \ref{ord-AP-3}. For \( α = 1 \) the claim is trivial. Otherwise, \( α  \) is a limit and \( β + α ≤ \sup \setof{ β + ξ }[ξ < α ] ≤ \sup \setof{ ξ + ξ }[ξ < α ] \). As \( α = \sup \setof{ ξ + ξ }[ξ < α ] \) the claim is established.
	
	\ref{ord-AP-3} $⇒$ \ref{ord-AP-1}. Straightforward.
\end{proof}
%

As a consequence of part \ref{ord-AP-3}, \( ω^α + ω^β = ω^β \) iff \( α < β \).
A corollary is the observation made earlier, that \( n + ω = ω \), which now follows from repeated applications of \cref{ord-AP}: \( α' + ω = α + ( ω^0 + ω^1 ) = α + ω^1 \).

Additive principal ordinals are central to the theory of ordinals.
As with addition, I will introduce more suggestive notation for the enumeration function for additive principal ordinals.

\begin{definition}
	\( ω^α ≔ E_\AP(α) \).
\end{definition}

By the definition \( ω^0 = 1 \) and \( ω^1 = ω \).
The reader can confirm that next additive principal ordinal above \( ω \) is the supremum of \( ω \), \( ω + ω \), \( ω + ω + ω \), …, \( ω + ⋯ + ω , … \) which is denoted \( ω^2 \).
%I will shortly show that the function \( ξ ↦ ω^ξ \) behaves as one would expect exponentiation (in particular, \( ω^(α + β) \)

\begin{lemma}
	\label{ord-AP-NF}
	For every \( α > 0 \) there exists unique \( β \) and \( ξ < α \) such that \( α = ω^β + ξ \).
\end{lemma}
\begin{proof}
	Let \( β \) be such that \( ω^β ≤ α < ω^{β'} \) and \( ξ \) such that \( α = ω^β + ξ \). Both ordinals are given by \cref{ord-normal}.
	What remains is to show uniqueness of this choice.
	Thus, suppose \( α = ω^γ + η \) for some \( γ \) and \( η < α \).
	The choice of \( β \) is clearly such that \( β ≥ γ \).
	As 
	\[ ω^β + ω^{γ+1} ≤ α + ω^{γ+1} ≤ ω^γ + η + ω^{γ + 1 } = ω^{γ+1} \]
	(the first inequality uses \cref{ord-addition}(\ref{ord-addition-inc}); the rest use \cref{ord-AP}(\ref{ord-AP-3})), also \( β ≤ γ \). Given that \( β = γ \), uniqueness of the rest is immediate.
\end{proof}
%\begin{lemma}
%	\label{ord-AP-add}
%	For all \( α > 0 \), \( α ∉ \AP \) iff \( α = β + γ \) for some \( β , γ < α \).
%\end{lemma}

%---------------------------------
\section{Normal forms and natural sum}
%---------------------------------

\Cref{ord-AP-NF} above provides the basis of a normal form representation of ordinals. This concept is introduced in the next definition.

\begin{definition}
	I write \( α =_\NF ω^β + γ \) to express that (i) \( α = ω^β + γ \) and (ii) \( γ < α \).
\end{definition}

Cantor, in 1897,\nocite{Cantor1897} established an expanded version of this normal form decomposition.
%
\begin{theorem}[Cantor normal form]\label{t-cantornf}
	For every ordinal \( α > 0 \) there exists \( n \) and ordinals \( α_n ≤ ⋯ ≤ α_0 \) such that 
	\[
		α = ω^{α_0} + ⋯ + ω^{α_n}.
	\]
	Moreover, this decomposition is unique. %, and \( α_0 < α \) if \( α < ε_0 \).
\end{theorem}
%
\begin{proof}
	The theorem is a simple generalisation of \cref{ord-AP-NF}.
%	The argument is by transfinite induction. Assume that each ordinal \( β < α \) admits a decomposition as described in the theorem. 
	Let \( α =_\NF ω^{α_0} + ξ_0 \) by \cref{ord-AP-NF}. If \( ξ_0 = 0 \) the decomposition is complete. 
	Otherwise, apply the \namecref{ord-AP-NF} again to express \( ξ_0 =_\NF ω^{α_1} + ξ_1 \), \( ξ_1 =_\NF ω^{α_2} + ξ_2 \), etc. 
	As \( α > ξ_0 > ξ_1 > ⋯ \) is a strictly decreasing sequence or ordinals, necessarily \( ξ_n = 0 \) for some \( n \). Thus, 
	\( α = ω^{α_0} + ⋯ + ω^{α_n} \). 
	Furthermore, \( α_0 ≥ α_1 ≥ ⋯ ≥ α_n \) because \( ω^{α_{i+1}} ≤ ξ_i < ω^{α_i+1} \) for each \( i \).
	Uniqueness is also a consequence of these normal forms.
\end{proof}


\begin{definition}
	The normal form notation is extended in the following way. Writing \( α =_\NF ω^{α_1} + ⋯ + ω^{α_n} \) expresses that (i) \( α = ω^{α_1} + ⋯ + ω^{α_n} \) and (ii) \( α ≥ α_1 ≥ ⋯ ≥ α_n \).
\end{definition}

\Cref{ord-normal} showed that every continuous order preserving function on the ordinals has fixed points. I.e., for each such function \( f \) there are ordinals \( β \) such that \( β = f(β) \).
As the function \( ξ ↦ ω^ξ \) (namely \( E_\AP \)) is an example of such a function, there must exist ordinals \( α \) such that \( α = ω^α \).
The proof of that lemma describes how to construct such an ordinal as the supremum of the sequence \( 0 \), \( 1 \), \( ω \), \( ω^ω \), …, \( α \), \( ω^α \), ….
This particular ordinal, conventionally denoted \( ε_0 \), will play a central role in the next chapter.

\begin{definition}\label{d-epsilon0}
	\( ε_0 ≔ \supseq ω_i \) where \( ω_0 = ω \) and \( ω_{k+1} = ω^{ω_k} \).
\end{definition}
%

%
\begin{lemma}
	\label{ord-e0}
	\( ε_0 \) is the least fixed point of the ordinal function \( α \mapsto ω^α \). 
	That is, \( ω^{ε_0} = ε_0 \) and \( α < ω^{α} \) for all \( α < ε_0 \).
\end{lemma}
%
\begin{exercise}
	Prove \cref{ord-e0}.
\end{exercise}

\begin{exercise}\label{ex-ord-mult-pre}
	Using the Cantor normal form theorem, define a multiplication operation where the first argument is restricted to additive principal ordinals: \( α , β ↦ ω^α.β \).
	The function should be continuous in \( β \) and satisfy the recursive clauses: \( ω^α.0 = 0 \) and \( ω^α.(β + 1) = ω^α.β + ω^α \).
\end{exercise}

\begin{exercise}\label{ex-ord-base-2}
	Define a function \( α ↦ 2^α \) satisfying
	\begin{align*}
		2^0 &= 1
		\\
		2^{α+1} &= 2^α + 2^α
		\\
		2^λ &= \supof{ 2^ξ }[ξ < λ]
	\end{align*}
	(You may find it useful to use the Cantor normal form theorem.)
	Show that this function is order preserving and continuous, and compute all fixed points of the function for ordinals \( α ≤ ε_0 \).
%	(It is not necessary to prove the existence of this function.)
\end{exercise}
%
\begin{exercise}
	\label{ex-epsilon-numbers}
	Let \( α ↦ ε_α \) be the enumerating function of the ordinals \( η \) such that \( η = ω^η \).
%	For each \( α \), define \( ε_α \) as the least ordinal such that
%	\begin{enumerate}
%		\item \( ε_β < ε_α \) for every \( β < α \),
%		\item \( ω^{ε_α} = ε_α \).
%	\end{enumerate}
	Express \( ε_α \) as a supremum of smaller ordinals as per \cref{d-epsilon0} and deduce that the enumerating function is defined for all ordinals.
\end{exercise}
%


\begin{exercise}
	\label{ex-ord-cnf-2}
	Prove the Cantor normal form theorem in base \( 2 \):
	\emph{For every ordinal \( α > 0 \) there exists unique ordinals \( α_n ≤ ⋯ ≤ α_0 ≤ α \) such that 
	\[
		α = 2^{α_0} + ⋯ + 2^{α_n}.
	\]}
\end{exercise}
%
\begin{exercise}
	\label{ex-ord-cnf-2o}
	What are the additive principal ordinals in base-$2$ normal form?
	Characterise the \( α \) such that \( 2^α = ω^α \).
\end{exercise}

%---------------------
%\section{Natural sum}
%---------------------

This brief foray into ordinals is concluded with another look at addition.
Recall that addition on ordinals is not commutative: \( 1 + ω ≠ ω + 1 \) for example.
It is possible to provide a natural notion of addition that \emph{is} commutative.
This is called the \emph{natural sum} (sometimes \emph{Hessenberg sum} after its originator Gerhard Hessenberg~\cite{Hess1906}). 
The Cantor normal theorem provides the means to achieve this.

\begin{definition}
	The natural sum of ordinals \( α \) and \( β \), denoted \( α \nsum β \) is defined by recursion on the two ordinals. \( 0 \nsum α = α \nsum 0 ≔ α \) for all \( α \).
	For non-zero \( α =_\NF ω^{α_0} + α_1 \) and \( β =_\NF ω^{β_0} + β_1 \)
	\[
		α \nsum β ≔ 
		\begin{cases}
			ω^{α_0} + ( α_1 \nsum β ), &\text{if \( α_0 ≥ β_0 \),}
			\\
			ω^{β_0} + ( α \nsum β_1 ), &\text{if \( α_0 ≤ β_0 \).}
		\end{cases}
	\]
	The operation of natural sum is well-defined as \( α_1 < α \) and \( β_1 < β \).
\end{definition}

As an operation on the Cantor normal form, the natural sum  has the following property.
\begin{lemma}
	For \( α =_\NF ω^{α_1} + ⋯ + ω^{α_m} \) and \( β =_\NF ω^{β_1} + ⋯ + ω^{β_n} \)
	\[
		α \nsum β ≔ ω^{γ_1} + ⋯ + ω^{γ_{m+n}}
	\]
	where \( γ_1 ≥ ⋯ ≥ γ_{m+n} \) enumerate the ordinals \( α_1, …, α_m , β_1 , …, β_n \) in descending order (with repetitions).
\end{lemma}

\begin{lemma}
	\label{ord-nsum}
	The natural sum is commutative and strongly increasing in both arguments: For all \( α \), \( β \), \( γ \),
	\begin{enumerate}
		\item \( α \nsum β = β \nsum α \);
		\item \( α < β \) implies \( α \nsum γ < β \nsum γ \).
	\end{enumerate}
\end{lemma}
%\begin{lemma}
%	\label{ord-nsum-2}
%	\begin{enumerate}
%		\item \( α \nsum β \)
%		\item \( α + β ≤ α \nsum β ≤ \max\setof{ α , β }.2 \).
%	\end{enumerate}
%\end{lemma}

\begin{exercise}
	Prove \cref{ord-nsum}.
\end{exercise}

\begin{exercise}
	\label{ex-ord-mult}
	Using the Cantor normal form theorem define a commutative multiplication \( α . β \) operation on ordinals. It should satisfy the distribution law:
	\(
		( α \nsum β ) . γ = (α .γ) \nsum (β .γ ).
	\)
	Hint, start from the function in exercise~\ref{ex-ord-mult-pre}.
\end{exercise}


%---------------------------------
\chapter{Ordinal analysis of arithmetic}\label{c-oa-PAo}
%---------------------------------
Ordinals will now be used to measure the \emph{height} of \( ω \)-proofs.
I begin by recalling the infinitary sequent calculi for arithmetic from the end of \cref{c-oa-arith}.
%
\begin{definition}%\label{d-PAomega}
	\( \PAo \) is the sequent calculus given by the following:
	\begin{itemize}
		\item sequents comprise formulas in the language of arithmetic (with equality).
		\item Initial sequents are \emph{closed} sequents of the form
		\begin{itemize}
			\item[(\botL)] \( ⊥, Γ ⇒ Δ \)
			\item[(\idRule)] \( Ps, Γ ⇒ Δ , Pt \) if \( ℕ ⊨ s = t \)
			\item[(\eqR)] \( Γ ⇒ Δ , s = t \) if \( ℕ ⊨ s = t \)
			\item[(\eqL)] \( s = t , Γ ⇒ Δ \) if \( ℕ ⊭ s = t \)
		\end{itemize}
		\item Inference rules are rules of \( \Gc \) but restricted to closed sequents and with \( \faR \) and \( \exL \) replaced by the two \( ω \)-rules:\note{\( \Gc \) or more general \( \Logic{G3} \)?}
		\begin{itemize}
			\item[(\omR)] \begin{prooftree} \hypo{ Γ ⇒ Δ , F(\nm n) \ \text{for every } n ∈ ℕ } \infer1{ Γ ⇒ Δ , ∀x F(x) }\end{prooftree}
			\item[(\omL)] \begin{prooftree} \hypo{ F(\nm n) , Γ ⇒ Δ \ \text{for every } n ∈ ℕ } \infer1{ ∃x F(x) , Γ ⇒ Δ }\end{prooftree}
		\end{itemize}
	\end{itemize}
%	Writing \( \PAo ⊢ Γ ⇒ Δ \) expresses that there is an \( ω \)-proof of \( Γ ⇒ Δ \) according to the above rules. In other words, there exists a well-founded tree labelled by sequents such that each leaf is an initial sequent and that each inner vertex together with its immediate successors in the tree forms a correct application of a rule of the calculus listed above.
%	
	\( \HAo \) is the same calculus but restricted to intuitionistic sequents.
\end{definition}
%
\begin{definition}\label{d-bound-omega-logic}
	Let \( \Theory{T} \) be \( \PAo \), \( \HAo \) or an extension of either calculus by rules that are at most \( ω \)-branching.
	The ternary relation \( \Theory{T} \prv{α}k Γ ⇒ Δ \), between a sequent \( Γ ⇒ Δ \), an ordinal \( α \) and \( k < ω \), is defined by transfinite recursion on the rules of \( \Theory{T} \):
	\begin{enumerate}
		\item If \( Γ ⇒ Δ \) is an initial sequent of \( \Theory{T} \), then \( \Theory{T} \prv{α}k Γ ⇒ Δ \) for all \( α \) and \( k \);
		\item For each inference \( (*) \) of \( \Theory{T} \) except cut of the form
		\[
			\Infer{\setof{ Γ_i ⇒ Δ_i }[i ∈ I]}[\( * \)]{ Γ ⇒ Δ }
		\]
		\( \Theory{T} \prv{α}k Γ ⇒ Δ \) holds if \( \Theory{T} \prv{α_i}k Γ_i ⇒ Δ_i \) and \( α_i < α \) for all \( i ∈ I \);
		\item If \( \Theory{T} \prv{α_0}k Γ ⇒ Δ , C \) and \( \Theory{T} \prv{α_1}k C , Γ ⇒ Σ \) for \( α_0,α_1 < α \) and \( \rk C < k \), then \( \Theory{T} \prv{α}k Γ ⇒ Δ, Σ \).
	\end{enumerate}
%	If \( \Theory{T} \) is a calculus of intuitionistic sequents, \( \Theory{T} \prv{α}k Γ ⇒ A \) is defined via the same conditions but with a modification of the final clause:
%	\begin{enumerate}[resume]
%		\item If \( \Theory{T} \prv{α_0}k Γ ⇒ C \) and \( \Theory{T} \prv{α_1}k C , Γ ⇒ A \) for \( α_0,α_1 < α \) and \( w(C) < k \), then \( \Theory{T} \prv{α}k Γ ⇒ A \).
%	\end{enumerate}
	Given \( \Theory{T} \prv{α}k Γ ⇒ Δ \) I will write that \( Γ ⇒ Δ \) is derivable (in \( \Theory T \)) with height \( ≤ α \) and cut rank \( ≤ k \).
\end{definition}
%

%Some of the terminology  to \( ω \)-proofs as before.

There is no requirement of minimality of \( α \) and \( k \) in the above definition. 
So the relation \( \prv{α}k \) is monotone in \( α \) and \( k \):
%
\begin{lemma}\label{oa-PAo-weak1}
	If \( α ≤ β \) and \( k ≤ l \) then \( \Theory{T} \prv{α}k Γ ⇒ Δ \) implies \( \Theory{T} \prv{β}l Γ ⇒ Δ \).
\end{lemma}
%
\begin{proof}
	By transfinite induction on \( α \). If \( Γ ⇒ Δ  \) is an initial sequent, the result is immediate.
	Otherwise, there is an inference rule of \( \Theory T \)
	\[
		\Infer{\setof{ Γ_i ⇒ Δ_i }[i ∈ I]}[\( * \)]{ Γ ⇒ Δ }
	\]
	and ordinals \( α_i < α \) such that \( \smash{\Theory T \prv{α_i}k Γ_i ⇒ Δ_i} \) for each \( i ∈ I \).
	The induction hypothesis implies that \( \Theory T \prv{α_i}l Γ_i ⇒ Δ_i \) for each \( i \), whereby \( \Theory T \prv{β}l Γ ⇒ Δ \) obtains.
\end{proof}

\Cref{oa-PAo-weak1} operates in the background of the majority of the results to follow. For that reason I will not make any explicit reference to the \namecref{oa-PAo-weak1}.

\begin{example}
	\tbw
\end{example}

\begin{lemma}\label{oa-HA-good}
	If \( \HAo \prv{α}k Γ ⇒ A \) then this fact can be observed by use of sequents of the form \( Σ ⇒ B \) (i.e., exactly one formula on the right).
\end{lemma}

\begin{exercise}
	Assign ordinal bounds on the \( ω \)-proofs of \( A , Γ ⇒ Δ , A \) constructed in exercise~\ref{ex:oa-id-simple}.
\end{exercise}

Revisiting the Embedding lemma (lemma \ref{oa-embed-weak}) it is possible provide ordinal bounds on the size of the resulting \( ω \)-proof.
Let \( α.k = \underbrace{α + ⋯ + α }_k \).

\begin{lemma}[Refined embedding lemma]\label{oa-embed-PAo-w-bounds}
	Suppose \( \PA ⊢ Γ ⇒ Δ \) and \( Γ ⇒ Δ \) is closed. Then there is \( n,k < ω \) such that \( \PAo \prv{ω.n}k Γ ⇒ Δ \) where \( ω .n = ω + ⋯ + ω \) (\( n \) times).
	Likewise, \( \HA \) into \( \HAo \).
\end{lemma}
%
\begin{exercise}
	Prove the refined embedding lemma following the schema of embedding lemma at the end of \cref{c-oa-arith}.
\end{exercise}
%
The next lemma hints at part of the usefulness of the \( ω \)-rule with the ability to isolate finitary reasoning from infinitary reasoning.
The result will be useful in \cref{s-oa-upper}.
%
\begin{proposition}\label{p-PAo-S1}
	Let \( A(a_1,…, a_k) \) be a \( Σ_1 \) formula. 
	There exists \( m < ω \) such that for all \( n_1, …, n_k ∈ \Nat \), 
	\[ \text{if }\thinspace  \Nat ⊨ A(\nm {n_1}, …, \nm {n_k} ) \text{ then } \HAo \prv m0 {⇒ A(\nm {n_1}, …, \nm {n_k})} .
	\]
\end{proposition}
%
\begin{proof}
	By induction on the rank of \( A \).
\end{proof}

Henceforth, I will omit explicit mention of \( \PAo \) and write \( \prv{α}k Γ ⇒ Δ \) to mean \( \PAo \prv{α}k Γ ⇒ Δ \).
The following results are stated only for \( \PAo \) but apply equally to \( \HAo \) in the expected way.
Admissibility of weakening becomes 
%
\begin{lemma}[Weakening Lemma]
	If \( \prv{α}k Γ ⇒ Δ \) and \( Γ' ⇒ Δ' \) is closed then \( \prv{α}k Γ' , Γ ⇒ Δ, Δ' \).
%	Likewise for \( \HAo \).
\end{lemma}
%
%\begin{proof}
%	By (transfinite) induction on \( α \).
%\end{proof}
\begin{exercise}
	Prove the weakening lemma.
\end{exercise}

The substitution lemma for \( \PAo \) takes a different formulation from previously. 
As sequents are closed, the correct formulation for \( ω \)-proofs is that provability depends on the \emph{value} of terms, not their \emph{form}.

\begin{lemma}[Substitution Lemma]
	Let \( Γ(a) ⇒ Δ(a) \) be a sequent and \( s \) and \( t \) be closed terms such that \( \Nat ⊨ s = t \). If \( \prv{α}k Γ(s) ⇒ Δ(s) \) implies \( \prv{α}k Γ(t) ⇒ Δ(t) \).
\end{lemma}
%
\begin{proof}
%	Analogous argument. I will treat the case of \( \exR \): 
	Suppose 
	\( \prv{α}k Γ(s) ⇒ Δ(s) \) and \( \Nat ⊨ s = t \).
	Let \( Γ(a) ⇒ Δ(a) \) be any sequent with at most \( a \) free.
	If \( Γ(s) ⇒ Δ(s) \) is initial then a case distinction on the different forms this sequent can take confirms that \( Γ(s) ⇒ Δ(s) \) is also initial provided \( \Nat ⊨ s = t \). The other case proceed by transfinite induction on \( α \).
\end{proof}

The final ingredient is the inversion lemma, the statement of which has the same form as before with two new cases treating equality.

\begin{lemma}[Inversion lemma]
	\label{oa-inversion}\ 
	\begin{enumerate}
		\item If \( \prv{α}k Γ ⇒ Δ , ⊥ \) then \( \prv{α}k Γ ⇒ Δ \).
		\item If \( \prv{α}k s = t , Γ ⇒ Δ \) and \( ℕ ⊨ s = t \) then \( \prv{α}k Γ ⇒ Δ \).\label{oa-inversion-eq1}
		\item If \( \prv{α}k Γ ⇒ Δ , s = t \) and \( ℕ ⊭ s = t \) then \( \prv{α}k Γ ⇒ Δ \).
		\item If \( \prv{α}k Γ ⇒ Δ , ∀x F(x) \) then \( \prv{α}k Γ ⇒ Δ , F(s) \) for every closed term \( s \).\label{oa-inversion-fa}
		\item If \( \prv{α}k ∃x F(x) , Γ ⇒ Δ \) then \( \prv{α}k F(s) , Γ ⇒ Δ \) for every closed term \( s \).
		\item Analogous inversion principles for the rules \( \disjL \), \( \conjR \), \( \impR \) and \( \impL \).
	\end{enumerate}
\end{lemma}
\begin{proof}
	I show cases \ref{oa-inversion-eq1} \& \ref{oa-inversion-fa}.
	
	\ref{oa-inversion-eq1}. By induction on \( α \). Suppose \( \prv{α}k s = t , Γ ⇒ Δ \) and \( ℕ ⊨ s = t \). If \( s = t , Γ ⇒ Δ \) is initial then so is \( Γ ⇒ Δ \). The other cases are straightforward because the equation \( s = t \) cannot be the principal formula of any rule.
	For if \( s = t , Γ ⇒ Δ \) is not initial, then there are sequents \( \setof{ Γ_i ⇒ Δ_i}[i<ω] \)
	and ordinals \( \setof{α_i}[i<ω] \) such that
	\begin{enumerate}[label=(\alph*)]
		\item \( \prv{α_i}k s=t , Γ_i ⇒ Δ_i \) for each \( i < ω \),
		\item \( α_i < α \) for all \( i \),
		\item \( \setof{ Γ_i ⇒ Δ_i}[i<ω] \) enumerate all premises of an inference of \( \PAo \) whose conclusion is \( Γ ⇒ Δ \).
	\end{enumerate}
	In the case of unary or binary rules, \( Γ_i = Γ_{i+1} \) and \( Δ_{i} = Δ_{i+1} \) for all \( i > 0 \) or \( 1 \). But in the case of either of the two \( ω \)-rules, the sequents enumerate the infinitely many premises.
	By (a)--(c) and the induction hypothesis, \( \prv{α}k Γ ⇒ Δ \) holds as desired.
	
	\ref{oa-inversion-fa}. The argument is a direct generalisation of the finitary inversion lemma. Suppose \( \prv{α}k Γ ⇒ Δ , ∀x F(x) \). If this sequent is initial, then so is \( Γ ⇒ Δ , F(s) \) for every closed term \( s \).
	The rest of the argument proceeds, essentially, as above by a case distinction on the inferences through which \( \prv{α}k Γ ⇒ Δ , ∀x F(x) \) can be derived.
	The case of \( \faR \) with \( ∀x F(x) \) principal bears treatment.
	The premises of this inference can be assumed to have the form \( Γ ⇒ Δ , ∀x F(x) , F(\nm n) \). An application of the induction hypothesis (to each premise) yields \( \prv{α}k Γ ⇒ Δ , F(\nm n) \) for every \( n \). If the desired closed term \( s \) is a numeral, this case is complete. Otherwise, let \( n \) be the value of \( s \), i.e., \( n ∈ ℕ \) is such that \( ℕ ⊨ \nm n = s \).
	The substitution lemma then yields \( \prv{α}k Γ ⇒ Δ , F(s) \).
\end{proof}

%---------------------------------
\section{Infinitary cut elimination}\label{s-oa-cutelim}
%---------------------------------

I begin with the transfinite version of the reduction lemma.
Recall, this is statement that borderline cuts can be simulated at the cost of increasing the depth of the proof by a controlled amount.
In the finitary case the depth increase was, in the case of classical logic, \( m + n \) where \( m \) and \( n \) bounded the depth of the two cut premises.

Lifting the statement of the reduction lemma to the transfinite realm is reasonably straightforward.
Given premises of a borderline cut of height \( α \) and \( β \) respectively, the cut can be simulated by a height of \( α \nsum β \).
The use of natural sum is crucial to the argument: the lifting of the finitary argument requires the resulting bound to be order-preserving in both arguments, a property we know fails for traditional ordinal sum \( α + β \).

%In fact, I will give two proofs of the reduction lemma.
%The first is the version just described: a direct lifting of the finitary argument to \( ω \)-proofs. 
%Some steps of the proof clearly require new insights, such as passing cuts through \( ω \)-rules and the new initial sequents with equations.
%But, by and large, the design of \( \PAo \) is such that the transfinite element is straightforward once one knows what to expect.
%
%Thus, the reduction lemma I will prove will be:

\begin{lemma}[Reduction lemma for \( \PAo \)]\label{oa-red-lem-PAo}
	Suppose \( \prv{α}k Γ ⇒ Δ , C \) and \( \prv{β}k C , Σ ⇒ Λ \).
	If \( \rk C ≤ k \) then \( \prv{α\nsum β }k Γ, Σ ⇒ Δ, Λ \).
\end{lemma}

The reader may surprised to know that there is a great deal of flexibility in proofs of the reduction lemma, which I will demonstrate by presenting a slightly different strategy than we used for in the analysis of classical predicate logic.
%The statement of the lemma already offers a minor, though insignificant, departure from previously by allowing \( \rk C < k \). 
%Provided that \( α , β \) are non-zero, \( \max\setof{ α , β } + 1 ≤ α \nsum β \), so whether.

%The second formulation will be presented after the proof of the above \namecref{oa-red-lem-PAo}.
%In short, it mitigates a shortcoming of the reduction lemma as described in its exaggeration of the complexity of cut elimination.
%The transfinite reduction lemma is optimal in the same way as the finitary form: Sequents can be readily chosen whose shortest cut-free proofs require a jump in ordinal height matching that expressed by the reduction lemma and its immediate corollaries.
%But by treating each logical connective as equally ‘expensive’, the reduction lemma ignores the true source of ‘large’ proofs: \emph{alternation} between \emph{positive} and \emph{negative} statements.
%
%This alternative version of the reduction lemma is closely related to the form proved in exercise~\ref{ex-red-lem-special}.

\begin{proof}%[of \cref{oa-red-lem-PAo}]
%	Suppose:
%	\begin{enumerate}
%		\item \( \prv{α}k Γ ⇒ Δ , C \).
%		\item \( \prv{β}k C , Σ ⇒ Λ \).
%	\end{enumerate}
%
	The proof branches into cases depending on the form of \( C \).
	In each case I will establish \( \prv{α \nsum β} k Γ, Σ ⇒ Δ, Λ \) but the induction will proceed over either \( α \) or \( β \) (depending on the case) rather than on the sum \( α \nsum β \).
	If the principal connective of \( C \) is among \( \setof{ ⊥ , ∀ ,  ∧ , → } \) I will refer to \( C \) as \emph{locally negative} (cf.~Canvas assignment no.~4).
	Otherwise, \( C \) will be \emph{locally positive}.
	
%	\paragraph{Case I: \( C \) is \( ⊥ \) or an equation.} This case can be dispensed with directly. If \( C = ⊥ \) or a false equation then \( \prv{α}k Γ ⇒ Δ \) is a consequence of the inversion lemma;  otherwise \( \prv{β}k Σ ⇒ Λ \).
%	In any case, weakening yields \( \prv{α\nsum β }k Γ, Σ ⇒ Δ , Λ \).
	
	\paragraph{Case I: \( C \) is atomic or locally negative.}
	Here I proceed by induction on \( β \) and show that \( \prv{ α \nsum β }k Γ, Σ ⇒ Δ , Λ \).
	I present two subcases:
	
	\( C = ∀x D(x) \). If \( C , Σ ⇒ Λ \) is initial then \( Σ ⇒ Λ \) is also initial and the claim holds by weakening.
	Otherwise, consider the rule that derives \( \prv{β}k C , Σ ⇒ Λ \).
	If the principal formula of the rule is \emph{not} \( C \) then the induction hypothesis can be applied directly to its premises and the rule re-applied to derive the desired sequent with correct bounds.
	If, however, the rule is \( \faL \) with \( C \) principal, the above argument does not work. But in this case there is \( γ < β \) and term \( t \) such that 
	\[
	  \prv{γ}k D(t) , C , Σ ⇒ Λ .
	\]
	The induction hypothesis yields
	\[
		\prv{α \nsum γ}k D(t) , Γ , Σ ⇒ Δ , Λ .
	\]
	From the inversion lemma (part \ref{oa-inversion-fa}) I know also that \( \prv{α}k Γ ⇒ Δ , D(t) \). Since \( \rk {D(t)} < \rk{ C} = k \), an application of cut yields 
	\( \prv{ α \nsum β }k Γ, Σ ⇒ Δ, Λ \).
	
	\( C = D → E \). I employ a similar argument as above but with a subtle difference in how the induction hypothesis is applied to account for the binary connectives. 
	By the previous argument I can jump directly to the case that \( C \) is principal in the derivation of \( \prv{β}k C , Σ ⇒ Λ \), for which there exist \( γ , δ < β \) and \( Λ = Λ_0 ∪ Λ_1 \) satisfying
	\begin{enumerate}
		\item \( \prv{γ}k C , Σ ⇒ Λ_0, D \).
		\item \( \prv{δ}k C , E , Σ ⇒ Λ_1 \).
	\end{enumerate}
	I start by applying the inversion lemma to my three hypotheses:
	\begin{enumerate}[resume]
		\item \( \prv{α}k D , Γ ⇒ Δ , E \).
		\item \( \prv{γ}k Σ ⇒ Λ_0, D \).
		\item \( \prv{δ}k E , Σ ⇒ Λ_1 \).
	\end{enumerate}
	Then I apply the induction hypothesis between the sequents in 3 and 5 (using ‘cut’ formula \( E \)):
	\begin{enumerate}[resume]
%		\item \( \prv{α+γ}k Γ , Σ ⇒ Δ ,Λ_0, D \).
		\item \( \prv{α \nsum δ}k D , Γ, Σ ⇒ Δ , Λ_1 \).
	\end{enumerate}
	I can now combine 6 and 3 with a (standard) cut:
	\[
		\prv{α \nsum β}k Γ , Σ ⇒ Δ ,Λ.
	\]
	The conjunction subcase is left to the reader.
	
	Case II: \( C \) is locally positive.
	This case is symmetric to the previous and left to the reader.
\end{proof}

%The observant reader will have noticed that the argument for the implication case could be simplified by applying the inversion lemma
% ------------------
\begin{figure}
	\centering
	\begin{prooftree}
%		\hypo{\prv{α}k Γ ⇒ Δ , C }
		\hypo{\prv{γ}k C , Σ ⇒ Λ_0, D}
		\infer[dashed]1[IL]{\prv{γ}k Σ ⇒ Λ_0, D}
%		\infer[dashed]2[IH]{\prv{α+γ}k Γ , Σ ⇒ Δ, Λ_0, D}
			
		\hypo{\prv{α}k Γ ⇒ Δ , C }
		\infer[dashed]1[IL]{\prv{α}k D , Γ ⇒ Δ , E }
			\hypo{\prv{δ}k C , E , Σ ⇒ Λ_1 }
			\infer[dashed]1[IL]{\prv{δ}k E , Σ ⇒ Λ_1 }
		\infer[dashed]2[IH]{\prv{α \nsum δ}k D , Γ , Σ ⇒ Δ , Λ_1 }
		\infer2[\Cut]{\prv{α \nsum β}k Γ , Σ ⇒ Δ , Λ }
	\end{prooftree}
	\caption{Illustration of the proof method in the reduction lemma for the case \( C = D → E \); IL = ‘inversion lemma’ and IH = ‘induction hypothesis’.}
	\label{f-oa-PAo}
\end{figure}
% ------------------

\begin{exercise}
	Complete the preceding proof.
\end{exercise}

\begin{exercise}
	Formulate and prove a reduction lemma for \( \HAo \) following the proof scheme above.
\end{exercise}

\begin{exercise}
	Give an alternative proof of \cref{oa-red-lem-PAo} using the proof strategy from the reduction lemma for \( \Gc \) (\cref{ce-red-lem-C}).
\end{exercise}

In the implication subcase of case II in the proof above, I used the induction hypothesis to simulate a cut on the formula \( E \)

\begin{theorem}[Reduction theorem for \( \PAo \)]\label{oa-red-thm-PAo}
	If \( \prv {α}{k+1} Γ ⇒ Δ  \) then \( \prv{ω^α}k Γ ⇒ Δ \).
\end{theorem}
\begin{proof}
	Induction on \( α \). If \( Γ ⇒ Δ \) is initial, the claim holds trivially.
	So suppose \( \prv {α}{k+1} Γ ⇒ Δ  \) is derived via a rule
	\[
	  \Infer{Γ_i ⇒ Δ_i \; \text{for } i ∈ I}[\( * \)]{ Γ ⇒ Δ }
	\]
	and for each \( i \) there is \( α_i < α  \) such that \( \prv{α_i}{k+1} Γ_i ⇒ Δ_i \).
	The induction hypothesis implies that \( \prv{ω^{α_i}}{k} Γ_i ⇒ Δ_i \) for each \( i \). So, if \( * \) is not cut, then 
	\[ \prv{ω^α}k Γ⇒ Δ \]
	obtains by re-applying the rule and observing that \( \supof{ ω^η }[η< α] ≤ ω^α \).
	Now suppose that the rule is cut, with cut formula \( C \). If \( \rk C < k \) the same argument as above applies.
	Otherwise \( \rk C = k \) and the reduction lemma is applicable, yielding
	\[
		\prv{ω^{α_0} \nsum ω^{α_1}}k Γ ⇒ Δ 
	\]
	Since \( ω^{α_0} \nsum ω^{α_1} < ω^α  \), the proof is complete.
\end{proof}

The bound in the reduction theorem can be improved fairly easily.
For the give proof strategy to work, it suffices to find an order-preserving function \( f \colon \Ord → \Ord \) such that \( f(α) ≥ \supof{ f(ξ) \nsum f(η) }[ξ,η<α] \).
An obvious candidate is \( f \colon α ↦ 2^α \) (see exercise~\ref{ex-ord-base-2}) and, indeed, \cref{oa-red-lem-PAo} can be strengthened by replacing \( ω^α \) with \( 2^α \).
Certainly, \( 2^α ≤ ω^α \) for all \( α \), so working with this bound seems a significant improvement. 
But given that for every additive principal ordinal \( α ≥ ω^ω \) in fact \( 2^α = ω^α \) (cf.\ exercise~\ref{ex-ord-cnf-2o}), the distinction between exponentiation in the two bases does little in reducing the complexity of cut elimination.

In the next section I will present a strict refinement of the cut elimination theorem in which ordinal exponentiation is directly tied to the \emph{quantifier} rank of the cut formula rather than the full rank.

Let \( ω_0^α ≔ α \) and \( ω_{k+1}^α ≔ ω^{ω_k^α} \).

\begin{theorem}[Cut elimination theorem]\label{oa-ce-PAo}
	If \( \prv{α}k Γ⇒ Δ \) then \( \prv{ω_k^α}0 Γ⇒ Δ \).
\end{theorem}
\begin{proof}
	Consequence of \cref{oa-red-thm-PAo}.
\end{proof}

\begin{exercise}
	Formulate and prove a corresponding reduction lemma and cut elimination theorem for \( \HAo \).
\end{exercise}

\begin{theorem}[Embedding theorem]
	\label{oa-embed-PA-ce}
	If \( \PA ⊢ Γ ⇒ Δ \) and this a closed sequent, then there exists \( α < ε_0 \) such that
	\[
		\PAo \prv{α}0 Γ ⇒ Δ.
	\]
	In addition, \( α \) is effectively computable from the given \( \PA \)-proof.
\end{theorem}
%
\begin{proof}
	Suppose \( \PA ⊢ Γ ⇒ Δ \).
	By the embedding lemma (\cref{oa-embed-PAo-w-bounds}) there is are \( n, k \) such that 
	\[
	  \PAo \prv{ω.n}k Γ ⇒ Δ .
	\]
	Let \( α = ω_k^{ω.n} \). Then \( α < ε_0 \) (by \cref{d-epsilon0}) and
	\[
	  \PAo \prv{α}0 Γ ⇒ Δ 
	\]
	by \cref{oa-ce-PAo}.
\end{proof}

On the basis of cut elimination, a few observations can be already made.

\begin{corollary}
	\label{PA-consis-weak}
	\( \PA \) and, hence, \( \HA \), are consistent.
\end{corollary}
%
\begin{proof}
	There can be no cut-free proof of the empty sequent.
\end{proof}

An inspection of the various proofs leading up to \cref{PA-consis-weak} can strengthen the result by clarifying what mathematical principles suffice to derive the consistency of arithmetic.
%
\begin{corollary}
	\label{PA-consis}
	Consistency of \( \PA \) can be deduced using only finitary reasoning plus the principle of transfinite induction for ordinals \( {≤} ε_0 \).
\end{corollary}
%
By ‘finitary reasoning’ I mean the ‘finite’ mathematics that can be carried out using only finite objects (such as natural numbers) and primitive recursive functions.
Examples include deciding whether one formula is a subformula of another, whether a given primitive recursive function enumerates the premises of an \( ω \)-rule (or Gödel codes of sequents) and what the concluding sequent is.
It is beyond the scope of these lecture notes to attempt to make the statement more precise, but the following proof ‘sketch’ hopefully elucidates how this could be achieved and proven.

\begin{proof}[sketch]
	Suppose there is a finite \( \PA \)-proof of the empty sequent. The embedding of \( \PA \) in \( \PAo \) (\cref{oa-embed-PAo-w-bounds}) provides an explicit number \( n < ω \) such that
	\[
		\PAo \prv{ω.n}n {} ⇒ {} .
	\]
	The existence of a cut-free proof of the empty sequent, along with the various results on which \cref{oa-ce-PAo} depends, can now be established by via finitary reasoning plus transfinite induction up to an ordinal strictly smaller than \( ε_0 \), for instance the ordinal \( ω_{n+2} \) suffices.
	
	As there can be no cut-free proof of the empty sequent, there is no derivation of the empty sequent in \( \PA \).
\end{proof}

\begin{corollary}
	If \( Γ \) is a set of \( Π^0_1 \) sentences and \( Δ \) a set of \( Σ^0_1 \) sentences, then \( \PAo ⊢ Γ ⇒ Δ \) iff there is a cut-free \( \PAo \) derivation of finite height.
\end{corollary}
%
\begin{proof}
	Exercise.
\end{proof}

% ----------------------
\section{Subtheories of Peano arithmetic}
\label{s-oa-ISigma}
% ----------------------

\tbw To cover
\begin{itemize}
	\item PRA
	\item \( \IS_n \)
\end{itemize}

The final part of this chapter will explore some important subtheories of \( \PA \) through the lens of \( ω \)-proofs.
As I will cover only classical theories, I take the opportunity to remove some (classically) definable connectives.

For the purpose of this section, the logical connectives are \( ⊥ \), \( ∧ \), \( → \) and \( ∀ \). Note, I am including implication rather than primitive negation as a matter of convenience.

The \emph{negation rank} (\emph{n-rank}) of a formula, \( \nrk F \), counts the nesting depth on the negative side of implications:
\begin{align*}
	\nrk{A} &= 0 \quad(\text{$A$ prime})
	&
	\nrk{F ∧ G} &= \max\setof{ \nrk F , \nrk G }
	\\
	\nrk{∀x F(x)} &= \nrk{F(a)}
	&
	\nrk{F → G} &= \max\setof{ \nrk F +1 , \nrk G }
\end{align*}

Compare with the quantifier hierarchy for this language fragment.

\begin{definition}
	The \( Π_{n}^P \) formulas (the \( ^P \) expresses that the predicate \( P \) is permitted, in contrast to our earlier definition of \( Π_1 \)) is the smallest set of formulas that contains
	\begin{enumerate}
		\item all prime formulas
		\item \( F ∧ G \) if \( F , G ∈ Π_{n}^P \),
		\item \( ∀x F(x) \) if \( F(a) ∈ Π_{n}^P \) and \( n > 0 \),
		\item \( F → G \) if \( F ∈ Π_{n-1}^P ∪ Π_0^P \) and \( G ∈ \Pi_{n}^P \).
	\end{enumerate}
\end{definition}

Since \( Π_0^P \) is closed under negation, there is no a priori bound on the negation depth of formulas in \( Π_n^P \).

\begin{lemma}
	Every \( Π_n^P \) formula is equivalent, over weak arithmetic, to a formula with n-rank \( n+1 \).
\end{lemma}
By \emph{weak arithmetic}, I have in mind the theory known as Robinson's \( Q \) (see~\cite[ch.~18]{LogThe}) but primitive recursive arithmetic or, equivalently, the theory known as ‘$\IS_1$’ suffices.

\begin{proof}
	Using the full logical language, express \( F ∈ Π_n^P \) in prefix normal form as
	\begin{gather}
		\label{oa-eqn-PNF}\tag{\dag}
		∀\vec x_1 ∃\vec x_2 ⋯ ∃ \vec x_{m} G
	\end{gather}
	where \( G \) is quantifier-free.
	We can choose \( m ∈ \setof{ n , n+1 } \) and \( G \) can be assumed to be \( Π_0^P \) because over weak arithmetic a disjunction \( E ∨ F \) is provably equivalent to 
	\[ 
		∃x ∃ y ( x + y = \suc \0 ∧ ( x = \0 → E ) ∧ ( x = \suc\0 → F )).
	\]
	and the leading existential quantifiers can be incorporated into the '$∃\vec x_m$' sequence of quantifiers.
	It is now straightforward to witness \eqref{oa-eqn-PNF} as an equivalent \( Π_{m}^P \)-formula whose negation rank is \( m \).
\end{proof}


I will jump the gun somewhat now.
%
A sequent is an expression \( Γ ⇒ Δ \) without free variables using the logical language isolated at the beginning of this section.
Let \( \prv{α}k Γ ⇒ Δ \) denote derivability in \( \PAo \) for such sequents in the usual way but with a more liberal cut rule bounded by negation rank:
\[
  \begin{prooftree}
	\hypo{\prv {α} k Γ ⇒ Δ , C }\hypo{ \prv{β} k C, Σ ⇒ Λ }
	\infer2[\Cut]{ \prv{γ}k Γ , Σ ⇒ Δ , Λ }
\end{prooftree}
\quad\text{for \( \nrk C < k \) and \( \max\setof{α,β} < γ \).}
\]

%The complication with adopting the above cut rule is that all transformations on proofs so far considered ‘reduce’
I assume this variation of \( \PAo \) satisfies weakening, substitution and inversion lemmas with the same bounds.
%
Given a finite sequence \( \vec A = (A_i)_{i≤k} \) of formulas, I will write \( \vec A , Γ ⇒ Δ \) for \( A_0 , …, A_n , Γ ⇒ Δ \).

%
\begin{lemma}[Refined reduction lemma]
	Suppose \( \prv{α}k Γ_i ⇒ Δ_i, C_i \) and \( \nrk {C_i} = k \) for each \( i ≤ n \). If \( \prv{β}k \vec C , Σ ⇒ Λ \), then
	\begin{gather}
		\label{oa-eqn-spec-red-lem}\tag{\dag}
		\prv{α + β} k Γ_0 , …, Γ_n , Σ ⇒ Δ_0 , …, Δ_n , Λ .
	\end{gather}
\end{lemma}
%
Applications of the refined reduction lemma, however, will also be to simulate an ordinary two-premise cut rule.
The significance of allowing multiple premises is hidden in the proof.
The multi-premise ‘cut’ can be visualised as either a generalisation of the binary cut rule:
\begin{prooftree*}
  \hypo{ Γ_0 ⇒ Δ_0, C_0 }
  \hypod 
  \hypo{ Γ_n ⇒ Δ_n, C_n }
  \hypo{ \vec C , Σ ⇒ Λ }
  \infer4[$n$-\Cut]{ Γ_0 , …, Γ_n , Σ ⇒ Δ_0 , …, Δ_n , Λ }
\end{prooftree*}
Or as sequence of binary cuts
\begin{prooftree*}
  \hypo{ Γ_n ⇒ Δ_n, C_n }
  \hypo{ Γ_1 ⇒ Δ_1, C_1 }
  \hypo{ Γ_0 ⇒ Δ_0, C_0 }
  \hypo{ \vec C , Σ ⇒ Λ }
  \infer2[\Cut]{ C_1 , …, C_{n} , Γ_0 , Σ ⇒ Δ_0 , Λ }
  \infer2[\Cut]{ }
  \ellipsis{}{ C_n, Γ_0 , …, Γ_{n-1} , Σ ⇒ Δ_0 , …, Δ_{n-1} , Λ }
  \infer2[\Cut]{ Γ_0 , …, Γ_n , Σ ⇒ Δ_0 , …, Δ_n , Λ }
\end{prooftree*}
The advantage of the former presentation is that the ‘size’ of the derivation does not depend on the order of the sequents.
This allows the application of an induction hypothesis in cases that the binary cut view grows too large.

%
\begin{proof}
	The overall structure of the proof will be recognisable as the strategy used in the proof of \cref{oa-red-lem-PAo}.
	I proceed by induction on \( β \).
	Suppose
	\begin{enumerate}
		\item \( \prv{α}k Γ_i ⇒ Δ_i, C_i \) and \( \nrk {C_i} = k \) for each \( i ≤ n \), and
		\item \( \prv{β}k \vec C , Σ ⇒ Λ \).
	\end{enumerate}
	I refer to \( \vec C \) as the \emph{cut} formulas.
	First, suppose no cut formula is principle in the final rule of assumption 2.
	If the sequent is initial, then \( Σ ⇒ Λ \) is initial and \eqref{oa-eqn-spec-red-lem} follows by weakening.
	Therefore, assume \( C_n \) is the principal formula in 2.
	There is a case distinction based on the form of \( C_n \).
	The focus will therefore be on assumption 2 above and
	\begin{gather}
		\label{oa-eqn-spec-red-lem-2}\tag{\ddag}
		\prv{α } k  Γ_n ⇒ Δ_n , C_n .
	\end{gather}

	If \( C_n = ⊥ \) or is a false equation then \eqref{oa-eqn-spec-red-lem} results from applying the inversion lemma to \eqref{oa-eqn-spec-red-lem-2}.
	If \( C_n = P s \), then \( P t ∈ Λ \) for some \( ℕ ⊨ s = t \) and \eqref{oa-eqn-spec-red-lem} also follows from \eqref{oa-eqn-spec-red-lem-2} via substitution.
	The final case is that \( C_n \) is a true equation. But it is not possible for such an atomic formula to be principal in \eqref{oa-eqn-spec-red-lem}.
	
	Moving on to the non-atomic case suppose, to begin, that \( C_n = D ∧ E \).
	From 2 I obtain \( γ < β \) and \( F ∈ \setof{ D, E} \) such that
	\begin{enumerate}[resume]
%		\label{oa-eqn-spec-red-lem}\tag{\dag}
		\item \( \prv{γ} k \vec C , F , Σ ⇒  Λ  \).
	\end{enumerate}
	Applying the inversion lemma to \eqref{oa-eqn-spec-red-lem-2} yields
	\begin{enumerate}[resume]
%		\label{oa-eqn-spec-red-lem}\tag{\dag}
		\item \( \prv{α } k  Γ_n ⇒ Δ_n , F  \).
	\end{enumerate}
	Adding this final sequent to the list of hypotheses in 1 above, and using 3 in place of 2, I can apply the induction hypothesis (as \( γ < β \)), which derives \eqref{oa-eqn-spec-red-lem}.
	
	The quantifier case, \( C_n = ∀x D(x) \) is essentially the same argument.
	From principality of \( C_n \) and the inversion lemma I know
	\begin{enumerate}[start=3,label=\arabic*'.]
		\item \( \prv{γ} k \vec C , D(s) , Σ ⇒  Λ \) for some \( γ < β \) and term \( s \).
		\item \( \prv{α } k  Γ_n ⇒ Δ_n , D(s) \).
	\end{enumerate}	
	I can then deduce \eqref{oa-eqn-spec-red-lem} from the induction hypothesis by adding 4' to the list in 1 and 3' in place of 2.
	
	The final case is involves a different in the argument.
	Suppose \( C_n = D → E \).
	Hypothesis 2 and the inversion lemma yields three derivations to work from:
	\begin{enumerate}[start=3,label=\arabic*''.]
		\item \( \prv{γ}k \vec C , E , Σ ⇒ Λ \),
		\item \( \prv{δ}k \vec C , Σ ⇒ Λ , D \),
		\item \( \prv{α}k D , Γ_n ⇒ Δ_n, E \),
	\end{enumerate}
	for \( γ, δ < β \).
	The first and third of these can be used with the induction hypothesis, obtaining as conclusion,
	\begin{enumerate}[resume,label=\arabic*''.]
		\item \( \prv{α + γ } k D , Γ_0 , …, Γ_n , Σ ⇒ Δ_0 , …, Δ_n , Λ \).
	\end{enumerate}
	To derive \eqref{oa-eqn-spec-red-lem}, I need to remove the formula \( D \) in 6'' I apply a cut against a second application of the induction hypothesis, this time using 4'' (and not expanding the list in 2):
	\begin{enumerate}[resume,label=\arabic*''.]
		\item \( \prv{ α + δ } k Γ_0 , …, Γ_n , Σ ⇒ Δ_0 , …, Δ_n , Λ , D \).
	\end{enumerate}
	As \( \nrk D < k \) a standard cut can be used between sequents 4'' and 6'', the conclusion being \eqref{oa-eqn-spec-red-lem}.
\end{proof}

As the focus is on better bounds on cut elimination, I will switch to base-$2$ exponentiation for the reduction theorem:

\begin{theorem}[Refined reduction theorem]
	Suppose \( \prv{α}{k+1} Γ ⇒ Δ \). Then \( \prv{2^α}{k} Γ ⇒ Δ \).
\end{theorem}
\begin{proof}
	This argument proceeds just as usual.
	Jumping to the main case, suppose \( \prv{α}{k+1} Γ ⇒ Δ \) is derived via cut:
	\[
		\prv{β}{k+1} Γ ⇒ Δ , C
		\qquad
		\prv{γ}{k+1} C, Σ ⇒ Λ 
	\]
	where \( β , γ < α \) and \( \nrk C ≤ k \).
	The induction hypothesis yields
	\[
		\prv{2^β}{k} Γ ⇒ Δ , C
		\qquad
		\prv{2^γ}{k} C, Σ ⇒ Λ 
	\]
	and the refined reduction lemma implies \( \prv{2^α}k Γ ⇒ Δ \).
\end{proof}

\begin{theorem}[Refined cut elimination]
	If \( \prv{α}k Γ ⇒ Δ \) then \( \prv{γ}0 Γ ⇒ Δ \) where \( γ = 2_k^α \).
\end{theorem}

\tbw

\begin{theorem}
	If \( \IS_n ⊢ A \) then \( \PAo \prv{α}0 Γ ⇒ Δ \) for some \( α < ω_{n+1} \).
\end{theorem}
%
\note{Check this.}
%
\begin{proof}[sketch]
	From \( \IS_n ⊢ {} ⇒ A \) we deduce that \( \PA \prv k {n+1} {} ⇒ A \).
	The reason is that the induction rule is only applied to formulas with n-rank \( ≤ n \) and finitary cut elimination is available in \( \PA \) to reduce the cut rank to formulas of the same n-rank as uses of induction.
	The embedding lemma of \( \PA \) into \( \PAo \) yields
	\( \PAo \prv{ω.k}{n} {}⇒ A \), so \( \PAo \prv{γ}0 {}⇒ A \) where
	\[
		γ = 2_{n}^{ω.k} .
	\]
	Recall that \( 2^{ω.k} = ω^k \), whence
	\[
		γ ≤ ω_{n}^{k} < ω_{n+1} .
	\]
\end{proof}


%---------------------------------
\chapter{Transfinite induction and proof-theoretic ordinals}
\label{c-TI-and-PTO}
%---------------------------------

The final chapter is devoted to proving the optimality of \cref{oa-embed-PA-ce}/\cref{PA-consis}.
I will show how the principle of transfinite induction can be rendered in arithmetic and show that it is precisely the ordinal \( ε_0 \) that marks the boundary between the provable and unprovable instance of transfinite induction.
%Such a characterisation of Peano arithmetic
It turns out that many interesting theories extending arithmetic (including set theories and theories of second-order arithmetic) can be characterised in such a way.
The ordinal corresponding to ‘provable instances of transfinite induction’ is one of a number of ways in which ordinals can be used to describe, delineate and compare mathematical theories.
\emph{Ordinal analysis}, in a nutshell, is the isolation and comparison of such ordinal measures.


%---------------------------------
\section{Provable transfinite induction}\label{s-oa-lower}
%---------------------------------
In the present section I will define precisely one way to assign an ordinal to a theory of arithmetic and show that under this measure the \emph{proof-theoretic ordinal} of Peano arithmetic is at least \( ε_0 \).
The following section will establish that this bound is optimal.

I begin by recalling some basic order-theory.
\begin{definition}
	Let \( ≺ \) be a relation on a non-empty set \( X \).
	\( ≺ \) is:
	\begin{itemize}
		\item \emph{well-founded} if there is no infinite \( ≺ \)-descending sequence, namely no sequence \( (x_i)_{i<ω} \) such that \( x_{i+1} ≺ x_i \) for every \( i \).
		\item a \emph{well-order} if \( ≺ \) is linear and well-founded.
	\end{itemize}
\end{definition}

\begin{example}\label{oa-ex-order-type}
	The following two orderings on natural numbers are well-orders.
	The third is well-founded but not a well-order.
	\begin{align*}
		m <_1 n \;&\text{iff}\; 0 < m < n \text{, or } n = 0 \text{ and } m ≠ 0 .
		\\
		m <_2 n \;&\text{iff}\; 
		\begin{cases}
			m < n, \text{and both are even or both odd, or}
			\\
			\text{$n$ even and \( m \) odd.}
		\end{cases}
		\\
		m <_2 n \;&\text{iff}\; m = 0 \text{ and } n ≠ 0.
	\end{align*}
\end{example}

The proof of the next lemma is left as an exercise.

\begin{lemma}
	A relation \( ≺  \) on a non-empty set \( X \) if a well-order iff every non-empty \( Y ⊆ X \) has a \( ≺ \)-least element.
\end{lemma}

Let \( ≺ \) be a well-founded ordering of \( \Nat \). I define
\begin{align*}
	\otin n &≔ \supof{ \otin m + 1 }[m ≺ n]
	\\
	\ot &≔ \supof{ \otin n + 1 }[ n ∈ \Nat ]
\end{align*}
Well-foundedness ensures the above notions are well-defined.
I call \( \otin[≺] n \) the order-type of \( n \) in \( ≺ \), and \( \ot \) the order-type of \( ≺ \).
The function \( \otin {·} \colon ℕ → \Ord \) is order-preserving: \( m ≺ n \) implies \( \otin m < \otin n \) and its range is a segment of \( \Ord \).
If \( ≺ \) is a well-order then the function is also injective, whence \( \otin {·}  \) is an order-preserving enumeration of \( ℕ \) in \( \Ord \).
%
\begin{example}
	I compute the order types of natural numbers in the three orderings from \cref{oa-ex-order-type}.
	Note, for the standard ordering on \( ℕ \),
	\begin{align*}
		\otin[<]{n} &= n \quad \text{for every \( n \)}
		\\
		\ot[<] &= \sup\setof{ n + 1 }[n ∈ \Nat] = ω .
	\end{align*}
	%
The ordering \( <_1 \) satisfies
\begin{gather*}
	\otin[<_1]{n+1} = n 
	\quad\text{and}\quad
	\otin[<_1]{0} =  ω
	\\
	\ot[<_1] = ω + 1 .
\end{gather*}
%
The ordering \( <_2 \) satisfies
\begin{align*}
	\otin[<_2]{2n} &= n 
	\\
	\otin[<_2]{2n+1} &= ω + n
	\\
	\ot[<_2] &= ω + ω .
\end{align*}
The ordering \( <_3 \) satisfies
\begin{align*}
	\otin[<_3]{0} &= 0
	\\
	\otin[<_3]{n} &= 1 \;\text{ for all $n>0$}
	\\
	\ot[<_3] &= 2.
\end{align*}
\end{example}
%
\begin{lemma}
	If \( ≺ \) is a well-founded relation on \( \Nat \) then for every \( α < \ot \) there exists \( n ∈ \Nat \) such that \( \otin n = α \).
	If \( ≺ \) is a well-ordering then \( n \) is unique.
\end{lemma}
%\begin{proof}
%	Let \( O = \setof{ \otin n }[n ∈ ℕ] \) and \( α ∈ O \).
%	If \(  \)
%%	Let \( ≺ \) be well-founded and assume, to the contrary, that \(  \)
%\end{proof}

For \( ≺ \) a primitive recursive relation on \( ℕ \) the representation theorem for arithmetic (\cref{representation-thm}) presents a \( Δ_0 \) formula \( {F_≺}(a,b) \) in the language of arithmetic (without the predicate \( P \)) such that for all \( n,m ∈ ℕ \),
\[
  \PA ⊢ {F_≺}(\nm m , \nm n) \text{ iff } m ≺ n.
\]
In what follows, I will write \( a ≺ b \) for the formula \( F_≺(a,b) \), and use \( ∀x ≺ a F(x) \) as an abbreviation for the formula \( ∀x ( x ≺ a ∧ F(x) ) \).

%
\begin{definition}
	For each primitive recursive ordering \( ≺ \) and formula \( A(x) \) define formulas:
	\begin{align*}
		\Prog {A} &≔ ∀x( ∀ y ≺ x \, A(y) → A(x) )
		\\
		\TI {A, a} &≔ \Prog {A} → ∀ y ≺ a\, A(y)
		\\
		\TI {A} &≔ ∀x\, \TI {A,x}
	\end{align*}
\end{definition}
%

If \( ≺ \) is a well-order, the formula \( \Prog A \) expresses progressiveness of the set of ordinals \( \otin n \) such that \( ℕ ⊨ A(\nm n) \).
In the case \( {≺} = {<} \) is the standard ordering on \( \Nat \), this is the same as \( A(x) \) being \emph{inductive}. 
As a result, \( \TI {A, a} \) states the principle of transfinite induction for this set restricted to the segment of ordinals \( \setof{ \otin n }[ n ≺ a ] \).


%
\begin{definition}
	Let \( \Theory{T} \) be a theory in the language \( \La \). The \emph{proof theoretic ordinal} of\, \( \Theory{T} \) is the ordinal \( \pto{\Theory T} \) defined by
	\[
	  \pto{\Theory T} = \supof{\ot }[\text{\( ≺ \) is a pr.~rec.,~well-founded  and \( \Theory{T} ⊢ \TI{P} \)}]
	\]
%	Equivalently, \( \pto{\Theory T} \) is the least primitive recursively representable ordinal for which the principal transfinite induction up to this ordinal for the predicate \( P \) is not derivable.
\end{definition}
%

%The restriction to primitive recursive orderings rather than of arbitrary complexity in the definition is because of its role in the formula \( \TI P \).
\noindent
The goal of this section is a lower bound on the proof-theoretic ordinal of Peano and Heyting arithmetic:
%
\begin{theorem}\label{pto-lower-bound}
	\( \pto \PA ≥ \pto \HA ≥ ε_0 \).
\end{theorem}
%
Unpacking \cref{pto-lower-bound}, it states that there exists a sequence of well-founded relations \( \setof{ ≺_i }_i \) such that \( \supseq \ot[≺_i] = ε_0 \) and \( \HA ⊢ \TI[≺_i] P \) for each \( i \).
A sequence of well-founded relations is not, strictly speaking, necessary as a single well-ordering can be defined of order-type \( ε_0 \) and for which transfinite induction can be proven for each proper initial segment.
I leave the proof of the next lemma as an exercise.
%
\begin{lemma}
	\label{oa-wo-e0}
	There exists a primitive recursive well-ordering of \( ℕ \) of order-type \( ε_0 \) and primitive recursive functions \( \oplus \) and \( \dot{ω} \) representing addition and exponentiation respectively in the sense that \( \oplus \colon ℕ × ℕ → ℕ \) and \( \dot{ω} \colon ℕ → ℕ \) satisfy
	\[
		\otin{ m \oplus n } = \otin m \nsum \otin n
		\qquad\text{and}\qquad
		\otin{ \dot{ω}(m) } = ω^{\otin m}
	\]
	for all \( m,n ∈ ℕ \).
\end{lemma}
%
\begin{exercise}
	\label{ex-wo-e0}
	Prove \cref{oa-wo-e0}.
	Hint: Utilise the Cantor normal form theorem and a (primitive recursive) bijection between \( ℕ \) and finite sequences of \( ℕ \).
\end{exercise}
%
\begin{exercise}
	Prove the following generalisation of \cref{oa-wo-e0}: Given a primitive recursive well-ordering of order-type \( α \) construct a primitive recursive ordering of \( ℕ \) of order-type \( ε_α \).
\end{exercise}

In the following \( ≺ \) denotes the primitive recursive well-ordering of order type \( ε_0 \) given by \cref{oa-wo-e0}.
%For \( α < ε_0 \) I will write \( ≺_α \) for the restriction of \( ≺ \) to numbers whose order-type in \( ≺ \) is \( < α \).
%That is,
%\[
%  {≺_α} ≔ \setof{ (m,n) }[ m ≺ n \text{ and } \otin[≺] n < α ]
%\]
%The relation \( ≺_α \) is not a well-order because every \( n \) for which \( \otin[≺] n ≥ α \) has the same order-type, namely \( 0 \).
%
%If \( ≺ \) is represented in arithmetic in the sense that \( \PA ⊢ \nm m ≺ \nm n \) iff \( m ≺ n \), the orderings \( ≺_α \) are represented through a formula of three variables:
%\[
%	{≺_*}(a,b,c) ≔ a ≺ b ∧ b ≺ c.
%\]
%This formula clearly has the property that
%\[
%  \PA ⊢ {≺_*}(\nm m , \nm n, \nm o) \text{ iff } \otin m ≺_{\otin o} \otin n .
%\]
%Henceforth, I will write \( a ≺_c b \) for the formula\( {≺_*}(a,b,c) \) above.
%
The proof of \cref{pto-lower-bound} relies on one lemma whose proof is rather time-consuming and will be omitted:
%
\begin{lemma}
	\label{pto-lower-bound-lem}
	For every formula \( A(a) \) in the language of arithmetic, there exists a formula \( A'(a) \) such that
	\[
		\PA ⊢ ∀x\bigl( \TI[≺] {A' , x } → \TI[≺] {A, \dot{ω}^x} \bigr).
%		\PA ⊢ \Prog[≺_{ω^α}] {A} → \Prog[≺_α] {A'}.
	\]
%	In particular, \( \PA ⊢ ∀x\bigl( \TI[≺] {A' , x } → \TI[≺] {A, \dot{ω}^x} \bigr) \).
\end{lemma}

%Viewing \( A \) as a set of ordinals,  
%\cref{pto-lower-bound-lem} expresses that there is a set \( A' ⊆ \ot \) such that for all \( α < \ot \), if transfinite induction holds for \( A' \) on the segment \( \ot \) up to \( α \) and \( A \) is progressive on the segment \( \ot \), then \( A \) contains all ordinals \( < ω^α \).
%
Although I won't present the proof, it will be useful to know how \( A' \) is constructed from \( A \).
First, I present the construction as an operation on sets of ordinals.
I write \( β ⊆ O \) as shorthand for \( (∀ξ< β)ξ ∈ O \).
Given \( O ⊆ \Ord \), define \( O' \) as the class
\[
  O' = \setof{ α }[ ∀ξ ( \, ξ ⊆ O \text{ implies } ξ + ω^α ⊆ O \, )].
\]
It is not difficult to see that \( O' \) is a segment and \( α ∈ O' \) implies \( ω^α ∈ O \), from which \( ω^{α+1} ⊆ O \) quickly follows.
%For the first observation, if \( β < α ∈ O' \) then \( ω^α ⊆ O \) (pick \( ξ = 0 \)), so \( β ∈ O \).
%And if \( α ∈ O' \) then \( ω^α ⊆ O \) (again picking \( ξ = 0 \)), so \( ω^α ∈ O \) (now picking \( ξ = ω^α \)).
%
%Looking closer at the definition, it is clear that \( O' \) is simply the largest segment of \( O \) such that \( \setof{ ω^α }[α ∈ O'] \) is a segment of \( O \).
%
Expressing the operation in the language of arithmetic provides the formula \( A' \) in \cref{pto-lower-bound-lem}:
\[
	A'(a) ≔ ∀x ( ∀ y ≺ x \, A(y) → ∀ y ≺ x \oplus \dot{ω}^a \, A(y) ).
\]

The above remarks concerning the properties of \( O \) and \( O' \) can be shown in \( \PA \) to hold for \( A \) and \( A' \). 
Thus, to prove the \namecref{pto-lower-bound-lem} it suffices to show that \( \PA ⊢ \Prog {A} → \Prog {A'} \), which involves similar argumentation.


%---------------------------------
\section{Bounding provable transfinite induction}\label{s-oa-upper}
%---------------------------------

The goal of this section is the converse to \cref{pto-lower-bound}:
%
\begin{theorem}\label{pto-upper-bound}
	\( \pto{\PA} ≤ ε_0 \).
\end{theorem}
%
The proof strategy is as follows. 
I fix an arbitrary primitive recursive well-ordering \( ≺ \) and suppose that \( \TI P \) is provable in \( \PA \). 
The embedding theorem for \( \PAo \) provides an ordinal \( α < ε_0 \) and a cut-free proof of \( \TI P \) bounded above by \( α \).
Applying the inversion lemma yields, for every \( n ∈ ℕ \),
\begin{equation}
	\PAo \prv{α}0 {\Prog P ⇒ ∀x ≺ \nm n \, Px} . \tag{\dag}\label{oa-eqn-TI}
\end{equation}
I want to infer from (\dag) that \( \otin n < ε_0 \) for every \( n \).
In fact, it will be the case that (\dag) holds only if \( \otin n ≤ α \).
%The goal of this section is to establish that from \eqref{oa-eqn-TI} we can infer that \( \ot  < ε_0 \).

To that aim I will utilise an extension of \( \PAo \), called \( \PAop \), such that \eqref{oa-eqn-TI} implies
\begin{equation}
	\PAop \prv{α}0 { ⇒ ∀x ≺ \nm n Px}. \tag{\ddag}\label{oa-eqn-TI2}
\end{equation}
%
The transfer from \eqref{oa-eqn-TI} to \eqref{oa-eqn-TI2} will depend on a cut elimination theorem for \( \PAop \).
An analysis of cut-free provability in \( \PAop \) will lead me from \eqref{oa-eqn-TI2} quite directly to \( \otin n ≤ α \) for all \( n \), i.e., \( \ot ≤ α < ε_0 \).
\medskip

I begin by introducing the extension of \( \PAo \) used in \eqref{oa-eqn-TI2}.
Henceforth, let \( ≺ \) be a fixed primitive recursive well-ordering on \( \Nat \). 
%Without loss of generality, I assume \( \ot \) is a limit ordinal.
For the sake of simplifying notation, I will write \( s^ℕ \) for the value of \( s \) in the standard model, i.e., the \( n \) such that \( ℕ ⊨ \nm n = s \).
This notation presupposes that \( s \) is closed.

%
\begin{definition}
%	Let \( ≺ \) be a well-ordering on \( \Nat \). 
%	We introduce the inference rule \( (≺) \):
	The rule \( (≺) \) comprises all instances of the inference
	\[
	  \Infer{Γ \sa Δ , P \nm n \quad\text{for every \( n ≺ s^\Nat \)}}[\( ≺ \)]{ Γ \sa Δ , Ps }
	\]
	The infinitary sequent calculus \( \PAop \) extends the axioms and rules of \( \PAo \) by the inference \( (≺) \) above. 
	The relation \( \PAop \prv{α}k Γ ⇒ Δ \) is given as in \cref{d-bound-omega-logic}.
\end{definition}

In general, the rule \( (≺) \) will have infinitely many premises like the \( ω \)-rules.
For instance, if there is an element \( m \) with order-type \( ω \), and \( M = \setof{ n ∈ ℕ }[n ≺ m] \) then one instance of the rule is
\[
  \Infer{ Γ ⇒ P \nm n \text{ for all } n ∈ M }[\( ≺ \)]{ Γ ⇒ P \nm m}
\]



The next three lemmas provide the motivation for this extension of \( \PAo \).
%
\begin{lemma}\label{oa-PAo-in-PAop}
	If \( \PAo \prv{α}k Γ ⇒ Δ \) then  \( \PAop \prv{α}k Γ ⇒ Δ \).
\end{lemma}
\begin{proof}
	Immediate.
\end{proof}
%
\begin{lemma}\label{oa-PAop-Prog}
	\( \PAop \prv{ω}0 {{}⇒ \Prog P} \).
\end{lemma}
%
\begin{proof}
	Recall that \( \Prog P = ∀ x ( ∀ y ( y ≺ x → Py ) → P x ) \). 
	Let \( k \) be the constant given by \cref{p-PAo-S1} such that \( \PAo \prv k 0 { ⇒ \nm m ≺ \nm n }\) for all \( m ≺ n \). 
	For every \( n ∈ \Nat \) I obtain the following derivation in \( \PAo \) (with implicit application of weakening) for all \( m , n ∈ ℕ \) satisfying \( m ≺ n \):
	\begin{prooftree*}
	  \axiom[\idRule]{ \prv00 P\nm m ⇒ P \nm m }
	  \subproof{\prv k0 {}⇒ \nm m \prec \nm n }
	  \infer[separation=3em]2[\impL]{\prv{k+1}0 \nm m ≺ \nm n → P \nm m  &⇒ P\nm m }
	  \infer1[\faL]{\prv{k+2}0 ∀y ≺ \nm n \, Py &⇒ P \nm m}
	\end{prooftree*}
	Continuing the derivation in \( \PAop \):
	\begin{prooftree*}
		\subproof{\prv{k+2}0 ∀y ≺ \nm n \, Py ⇒ P \nm m \text{ for all \( m ≺ n \)} }
		\infer1[\( ≺ \)]{\prv{k+3}0 ∀y ≺ \nm n \, Py ⇒ P\nm n }
		\infer1[\impR]{\prv{k+4}0 {} ⇒ ( ∀y ≺ \nm n \, Py ) → P\nm n }
		\hypo{\text{for every } n}
		\infer2[\faR]{\prv{k+5}0 {} ⇒ \Prog P }
	\end{prooftree*}
	An application of bound weakening completes the proof.
\end{proof}
%
\begin{lemma}[Refined embedding lemma]\label{oa-PAop-embedding}
	If\, \( \PA ⊢ \TI P \) then there exists \( k < ω \) such that for all \( n ∈ ℕ \), 
	\[ \PAop \prv{ω^2}k {{}⇒ ∀ y ≺ \nm n \, P y }. \]
\end{lemma}
%
\begin{proof}
	The embedding lemma for \( \PAo \) (\cref{oa-embed-PAo-w-bounds}) and inversions yields a \( k < ω \) such that for all \( n ∈ ℕ \):
	\[ 
		\PAo \prv{ω.k}k { \Prog P ⇒ ∀y ≺ \nm n \, P y } 
	\]
	\Cref{oa-PAop-Prog} and a pair of cuts completes the argument.
\end{proof}

Paired with cut elimination for \( \PAop \), treated in the next section, the \namecref{oa-PAop-embedding} above yields \eqref{oa-eqn-TI2}.
Under the assumption of cut elimination (with the same bounds as \( \PAo \)), just one lemma stands before an optimal upper bound on the proof-theoretic strength of \( \PA \).
This is the lemma below.

Since \( ≺ \) is a fixed well-ordering, for \( α < \ot \) I will write \( \cd{α} \) for the numeral \( \nm n \) such that \( α = \otin n \).
%
\begin{lemma}[Bounding lemma]
	\label{oa-bounding-lemma}
	Let \( α_1 , …, α_m , β_0, …, β_n < \ot \). If
	\[ \PAop \prv{γ}0 P\cd{α_1}, … , P\cd{α_m} ⇒ P\cd{β_0} , …, P\cd{β_n} \]
	then
	\(%begin{equation}\label{eqn-bounding-lemma}
		 \min\setof{β_0, …, β_n } ≤ \max \setof{α_1, …, α_m } + γ  . %\tag{\ast}
	\)%end{equation}
\end{lemma}
%
\begin{proof}
	Induction on \( γ \).
	If \( P\cd{α_1}, … , P\cd{α_m} ⇒ P\cd{β_0} , …, P\cd{β_n} \) is initial, then \( α_i = β_j \) for some \( i \) and \( j \) and the claim holds vacuously.
	If, however, the sequent is not initial then, as the derivation is cut-free, the final rule applied must be an instance of \( (≺) \). 
	I can assume, without loss of generality, that the principal formula is \( P \cd{β_n} \), i.e., that the inference applied is
	\begin{prooftree*}
		\hypo{ P\cd{α_1}, … , P\cd{α_m} ⇒ P\cd{β_0} , …, P\cd{β_n} , P\cd{δ} \quad\text{for all \( δ < β_n \)} }
		\infer1[\( ≺ \)]{ P\cd{α_1}, … , P\cd{α_m} ⇒ P\cd{β_0} , …, P\cd{β_n} }
	\end{prooftree*}
	For each \( δ <  β_n \) the corresponding premise has a cut-free derivation of height \( < γ \).
	That is, for every \( δ < β_n \) there exists \( γ_δ < γ \) such that 
	\[
	  \PAop \prv{γ_δ}0 P\cd{α_1}, … , P\cd{α_m} ⇒ P\cd{β_0} , …, P\cd{β_n} , P\cd{δ}.
	\]
	Let \( β = \min \setof{β_0, …, β_n } \) and \( α = \max \setof{α_1, …, α_m } \).
	The induction hypothesis implies that 
	\begin{equation}
		\label{eqn-bouding-lemma-sub}
		\text{for every \( δ < β_n \), } \min\setof{ β , δ } ≤ α + γ_δ.
	\end{equation}
	%
	Consider two cases. 
	First, suppose \( β < β_n \). 
	Choosing \( δ = β \) in \eqref{eqn-bouding-lemma-sub} yields
	\[
		β ≤ α + γ_β < α + γ,
	\]
	whereby the claim holds as desired.
	Otherwise, \( β = β_n \), and
	\begin{align*}
		β = \supof{ δ + 1 }[ δ < β ] &≤ \supof{ α + γ_δ + 1 }[ δ < β ] &&\text{by \eqref{eqn-bouding-lemma-sub}}
	  \\
	  &≤ α + \supof{ γ_δ + 1 }[ δ < β ] &&\text{continuity}
	  \\
	  &≤ α + γ.
	 \end{align*}
%	 as required.
\end{proof}

\begin{proof}[of \cref{pto-upper-bound} (assuming cut elmination)]
	Let \( ≺ \) be any primitive recursive well-order of \( ℕ \) and suppose \( \PA ⊢ \TI P \).
	Let \( α = \ot \).
	The refined embedding lemma~(\cref{oa-PAop-embedding}) provides a finite \( k \) such that for all \( n ∈ ℕ \)
	\[
		\PAop \prv{ω^2} k {}⇒ ∀ y ≺ \nm n\, Py .
	\]
	In particular, for every \( β < α \),
	\[
		\PAop \prv{ω^2} k {}⇒ P \cd {β} .
	\]
	Cut elimination for \( \PAop \) provides an ordinal \( γ < ε_0 \) such that (see next \namecref{s-oa-PAop-ce}) for every \( β < α \)
	\[
		\PAop \prv{γ} 0 {}⇒ P\cd{β} .
	\]
	The bounding lemma ensures that \( β ≤ γ \), meaning that \( \ot ≤ γ + 1 < ε_0 \).
\end{proof}

%---------------------------------
\section{Cut elimination, revisited}
\label{s-oa-PAop-ce}
%---------------------------------

What remains is to confirm cut elimination for the extended calculus \( \PAop \).
The reader can confirm that weakening and substitution remain admissible in this extension.
%
\begin{lemma}[Weakening Lemma]
	If \( \PAop \prv{α}k Γ ⇒ Δ \) and \( Γ'⇒Δ' \) is closed then \( \PAop \prv{α}k Γ' , Γ ⇒ Δ, Δ' \).
\end{lemma}
%
\begin{lemma}[Substitution Lemma]
	Let \( Γ(a) ⇒ Δ(a) \) be a sequent with \( a \) the only free variable, and let \( s \) and \( t \) be closed terms such that \( \Nat ⊨ s = t \). If \( \PAop \prv{α}k Γ(s) ⇒ Δ(s) \) then \( \PAop \prv{α}k Γ(t) ⇒ Δ(t) \).
\end{lemma}
%
%\begin{lemma}[Contraction Lemma]
%	If \( \PAop \prv{α}k A , A , Γ ⇒ Δ \) then \( \PAop \prv{α}k A , Γ ⇒ Δ \).
%	If \( \PAop \prv{α}k Γ ⇒ Δ , A , A \) then \( \PAop \prv{α}k Γ ⇒ Δ , A \).
%\end{lemma}
%
\begin{exercise}
	Prove the weakening and substitution lemmas for \( \PAop \).
\end{exercise}

Precisely the same formulation of the reduction lemma also holds, but here there are some notable changes to the proof.
%In fact, it is simple to see how 
%
I present only the ‘simple’ version of this result and leave the quantifier-relevant form for the reader.
%
\begin{lemma}[Reduction lemma]
	Suppose \( \PAop \prv{α}k Γ ⇒ Δ , C \) and \( \PAop \prv{β}k C, Γ ⇒ Λ \).
	If \( \rk C = k \) then \( \PAop \prv{α\nsum β}k Γ ⇒ Δ,Λ \).
\end{lemma}
%
\begin{proof}
	Like the inferences of \( \PAo \), the rule \( (≺) \) has the property of just one principal formula,
	namely
	\[
%		\text{If} \quad
		\Infer{Γ_i ⇒ Δ_i \;\text{for } i ∈ I}{Γ⇒Δ}
	\]
	is an instance iff there is \( F ∈ Δ \) such that
	\[
		\Infer{ Σ, Γ_i ⇒ Δ_i , Λ  \;\text{for } i ∈ I}{ Σ ⇒ Λ, F}
	\]
	is an instance for all \( Σ \) and \( Λ \).

	As such, the new rule does not affect the part of the argument where \( C \) is not principal in one of the assumptions.
	So it suffices to treat the case in which \( C = P s\) for some \( s \) and is principal in both assumptions.
	But if \( Ps \) is principal in the proof \( \prv{α}k Γ ⇒ Δ , C \) then inference deriving this sequent is either initial (whence \( P t ∈ Γ \) for \( ℕ ⊨ s = t \)) or the conclusion of \( (≺) \).
	In the latter case, however, it is not clear how to use the premises of the rule against the second assumption.
	Fortunately, though, it is not necessary because we are assuming that \( C \) is principal in the second hypothesis, \( \prv{β}k C,Γ ⇒ Λ \).
	For this to be the case, \( C , Γ ⇒ Λ \) must be an initial sequent, meaning that \( P t ∈ Λ \) such that \( ℕ ⊨ s = t \).
	The substitution lemma applied to the \emph{first} hypothesis, shows derivability of \( \prv{α}k Γ ⇒ Δ , Λ \).	
\end{proof}

\begin{exercise}
	Complete the proof of the reduction lemma.
\end{exercise}

\begin{exercise}
	Formulate and prove a quantifier-relevant formulation of the reduction lemma for \( \PAop \) following the schema of \cref{oa-red-lem-PAo-quant}.
\end{exercise}
%
\begin{theorem}[Cut elimination]
	If \( \PAop \prv{α}k Γ ⇒ Δ \) then \( \PAop \prv{ω_k^α}0 Γ ⇒ Δ  \).
\end{theorem}
%
\begin{proof}
	The proof proceeds precisely as before.
\end{proof}

%
%---------------------------------
\section{Characterisation of provable transfinite induction}
%---------------------------------

Combining the results in this chapter:
%
\begin{theorem}[Proof-theoretic characterisation theorem]
	The proof-the\-oretic ordinal of Peano and Heyting arithmetic is \( ε_0 \).
\end{theorem}
%
\begin{proof}
	As \( ε_0 ≤ \pto{\HA} ≤ \pto{\PA} \) by \cref{pto-lower-bound} and \( \pto{\PA} ≤ ε_0 \) by \cref{pto-upper-bound}.
\end{proof}
%
\begin{corollary}[Independence of transfinite induction]
	There is a primitive recursive well-ordering \( ≺ \) on \( \Nat \) and a formula \( A \) in the language of arithmetic such that \( \PA ⊬ \TI A \).
\end{corollary}
%

The following will be a consequence of \cref{oa-embed-IS-ce}, but it needs to be refined.
%
\begin{theorem}
	The proof-the\-oretic ordinal of \( \IS_n \) for \( n > 0 \) is \( ω_{n+1} \).
\end{theorem}

% Solutions
%\part{Selected solutions}
%\include{solutions/part-1}
%%%%
\chapter{Solutions to Module~\ref{module-2}}
%

%Multi-cut reduction lemma
%
%\[
%\begin{prooftree}
%	\hypod
%	\hypo{ D , Γ }
%	\infer1{ Γ ⇒ C }
%	\hypo{ \prvs {n} q C , Δ ⇒ D }
%	\hypo{ \prvs {n} q C , E , Δ ⇒ A }
%	\infer2{ \prvs {n+1} q C , Δ ⇒ A }
%	\infer3{ \prv{}{} }
%\end{prooftree}
%\]

%\include{solutions/part-3}

\backmatter


%\clearpage
\chapter{Index of conventions}
\theoremlisttype{optname}
\listtheorems{convention}

\bibliography{bib/references.bib}

\end{document}
